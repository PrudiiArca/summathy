\documentclass{article}

% --- text layout
\usepackage[parfill]{parskip}
\usepackage{geometry}
\geometry{a4paper, margin=3cm}
\usepackage{romanbar}
\usepackage{placeins} %FloatBarrier
\usepackage[utf8]{inputenc}

\usepackage{xcolor}
\definecolor{darkred}{HTML}{990000}
\definecolor{darkgreen}{HTML}{009900}

\usepackage{sectsty}
\sectionfont{\color{darkred}}
\subsectionfont{\color{darkred}}
\subsubsectionfont{\color{darkred}}
% \usepackage{algpseudocode} --> better pseudo code?
%\usepackage[pagewise]{lineno} % row numbers

% --- syncing, referencing, citing
\usepackage{pdfsync} 			% synchronization LaTeX <--> PDF
\usepackage[hidelinks]{hyperref} % hidelinks avoids red boxes around links
\usepackage{csquotes}			% enquote
\usepackage{cleveref}
\usepackage{pdfpages}

\usepackage[backend=bibtex,style=alphabetic,url=false]{biblatex}
\addbibresource{sources.bib}
% define bibliography filter like
%\defbibfilter{computability-pca}{
%	keyword=computability or keyword=pca
%}

% --- lists and tables
\usepackage[shortlabels]{enumitem}
\usepackage{tabularx}
\usepackage{tabu}
\usepackage{makecell}

% --- graphics
\usepackage{graphicx}
\usepackage{xcolor}

% --- math
\usepackage{amsmath}
\usepackage[bbgreekl]{mathbbol} % double stroke greek
\usepackage{amssymb}
\usepackage{fixmath} % whyyy? "bug": italic PI and SIGMA
\usepackage{mathtools}
\usepackage{mathrsfs} %mathscr

% --- own packages
\usepackage{../notation}
\usepackage{../diagrams}
\usepackage{../environments}

\usepackage{extarrows}

\title{$\Sum \text{math}(y)$\\The Functor of Points}
\author{Jonas Linßen}

\NewDocumentCommand{\ft}{}{\text{ft}}
\NewDocumentCommand{\fp}{}{\text{fp}}

\begin{document}
	\maketitle
	\tableofcontents

	We follow [Kapil] \url{https://www.imsc.res.in/~kapil/crypto/notes/node38.html} as well as 


	\newpage
	\section{Schemes}
	\subsection{Affine Schemes}

	Let $F=\{f_1,...,f_r\} \subseteq R[X_1,...,X_n]$ be a finite system of polynomial equations in some (commutative unital) ring $R$. Given an $R$-algebra $A$ we want to study the set of simultaneous solutions of $F$ in $A$, denoted $V(F,A)=\{(a_1,...,a_n) \in A^n \mid \forall i=1 ... r: f_i(a_1,...,a_n) = 0 \}$, which describes an abstract geometric hypersurface in $A^n$.
	Since $R$-algebra homomorphisms respect polynomial equations with coefficients in $R$, we find that this assignment is indeed functorial
	\begin{equation*}
		\begin{array}{rcl}
			\ncat{Alg}_R&\xlongrightarrow{V(F,-)}&\ncat{Set}\\
			A&\longmapsto&V(F,A)\\
			(\phi:A\rightarrow B)&\longmapsto&(\phi:V(F,A)\rightarrow V(F,B)).
		\end{array}
	\end{equation*}
	By the universal property of the polynomial ring $R[X_1,...,X_n]$ a morphism of $R$-algebras of the form $\phi:R[X_1,...,X_n]/\Gen{f_1,...,f_r} \rightarrow A$ is uniquely determined by a point $(a_1,...,a_n)\in A^n$ with the property that $f_i(a_1,...,a_n)=0$ for each $i=1...r$. Hence we find that there is a natural isomorphism 
	\begin{equation*}
		V(F,-) \isom \ncat{Alg}_R(R[X_1,...,X_n]/\Gen{f_1,...,f_r},-),
	\end{equation*}
	which shows that the functor $V(F,-)$ is in fact representable by an $R$-algebra of finite presentation (a quotient of a polynomial ring in finitely many variables by a finitely generated ideal). We use this as the reason for the following

	\begin{definition}
		%A functor, which is isomorphic to one of the form $V(F,-)$, is called an \textbf{affine $R$-scheme}. More precisely a
		An affine scheme over $R$ is a functor $X:\ncat{Alg}_R \rightarrow \ncat{Set}$, which is representable by an $R$-algebra $\mathcal{O}(X)$ called the \textbf{coordinate ring} of $X$. A \textbf{morphism of affine schemes over $R$} is a mere natural transformation. We write $\ncat{Aff}_R$ for the corresponding full subcategory of the functor category $|[\ncat{Alg}_R,\ncat{Set}|]$.
	\end{definition}
	%\footnotetext{The definition of an affine scheme can be made for arbitrary $R$-Algebras. However one runs into issues of size, since the functor category $|[\ncat{Alg}_R,\ncat{Set}|]$ is not locally small. Moreover we feel like the restriction to algebras of finite presentation is more in spirit of finite systems of equations in finitely many variables, which is our motivation to study schemes.}

	We give some examples of affine schemes.
	\begin{example}
		\vspace{-2em}
		\begin{enumerate}[$(i)$]
			\item{
				The functor 
				$\begin{array}{rcl}
					\ncat{Alg}_R & \longrightarrow & \ncat{Set}\\
					A & \longmapsto & \emptyset
				\end{array}$ is represented by the zero ring.
			}
			\item{
				The functor $\begin{array}{rcl}
					\ncat{Alg}_R & \longrightarrow & \ncat{Set}\\
					A & \longmapsto & \{*\}
				\end{array}$ is represented by $R$.
			}
			\item{
				The functor $\begin{array}{rcl}
					\ncat{Alg}_R & \xlongrightarrow{U} & \ncat{Set}\\
					A & \longmapsto & A\\
					(\phi:A\rightarrow B) & \longmapsto & (\phi:A\rightarrow B)
				\end{array}$ is represented by $R[X]$.
			}
			\item{
				The functor $\begin{array}{rcl}
					\ncat{Alg}_R & \xlongrightarrow{\A_R^n} & \ncat{Set}\\
					A & \longmapsto & A^n\\
					(\phi:A\rightarrow B) & \longmapsto & (\phi^n:A^n\rightarrow B^n)
				\end{array}$ is represented by $R[X_1,...,X_n]$.
			}
			\item{
				The functor $\begin{array}{rcl}
					\ncat{Alg}_R & \longrightarrow & \ncat{Set}\\
					A & \longmapsto & A^\times\\
					(\phi:A\rightarrow B) & \longmapsto & (\phi=\phi^\times:A^\times\rightarrow B^\times)
				\end{array}$\vspace{.5em}\\
				is represented by $R[X,Y]/(XY-1)=R[X,X^{-1}]$.
			}
			\item{
				The functor $\begin{array}{rcl}
					\ncat{Alg}_R & \xlongrightarrow{\SL_2} & \ncat{Set}\\
					A & \longmapsto & \SL_2(A)\\
					(\phi:A\rightarrow B) & \longmapsto & (\phi^{n\times n}:\SL_2(A)\rightarrow \SL_2(B))
				\end{array}$\vspace{.5em}\\
				is represented by $R[X_1,...,X_4]/(X_1X_4-X_2X_3-1)$.
			}
		\end{enumerate}
	\end{example}
	Examples (v) and (vi) possess more structure as they factor via $\ncat{Grp}$. We will study these so called \textit{affine group schemes} in a later chapter. Example (iv) is important enough to deserve its own
	\begin{definition}
		The affine scheme represented by $R[X_1,...,X_n]$ is called the \textbf{$n$-dimensional affine space} over $R$ and is denoted $\A_R^n:\ncat{Alg}_R \longrightarrow \ncat{Set}$.
	\end{definition}

	Since the category of $R$-algebras of finite presentation $\ncat{Alg}_R^\fp$ is a full subcategory of $\ncat{Alg}_R$ the restricted Yoneda embedding
	\begin{equation*}
		\begin{array}{rcl}
			(\ncat{Alg}_R^\fp)^\op & \xlongrightarrow{\fun{Y}} & [\![\ncat{Alg}_R,\ncat{Set}]\!]\\
			A & \longmapsto & \ncat{Alg}_R(A,-)\\
			(\phi:A\rightarrow B) & \longmapsto & \pre\phi
		\end{array}
	\end{equation*}
	is full and faithful and we immediately get that $\fun{Y}$ restricts to an equivalence of categories \TODO{write down assignment}
	\begin{equation*}
		(\ncat{Alg}_R^\fp)^\op \simeq \ncat{Aff}_R.
	\end{equation*}
	As such $\ncat{Aff}_R$ is essentially small and exhibits the same categorical structure as $(\ncat{Alg}_R^\fp)^\op$. Under this equivalence we obtain

	\begin{lemma}
		The category $\ncat{Aff}_R$ is finitely complete. In particular we have
		\begin{enumerate}[(i)]
			\item{
				The \textbf{terminal object} in $\ncat{Aff}_R$ is the functor represented by $R$.
			}
			\item{
				The \textbf{product} $X \times Y$ of affine schemes $X$ and $Y$ is the functor represented by $\mathcal{O}(X)\tensor_R\mathcal{O}(Y)$.
			}
			\item{
				The \textbf{pullback} of a diagram of affine schemes $X \rightarrow Y \leftarrow Z$ is the functor represented by $\mathcal{O}(X) \tensor_{\mathcal{O}(Y)} \mathcal{O}(Z)$.
			}
			\item{
				The \TODO{equalizer} of morphisms of affine schemes $f,g:X \rightarrow Y$ is the functor represented by $\mathcal{O}(Y)/\Gen{f-g}$
			}
		\end{enumerate}
	\end{lemma}

	\subsection{Projective Schemes}

	\subsection{Abstract Schemes}
\end{document}