\documentclass{article}

% --- text layout
\usepackage[parfill]{parskip}
\usepackage{geometry}
\geometry{a4paper, margin=3cm}
\usepackage{romanbar}
\usepackage{placeins} %FloatBarrier
\usepackage[utf8]{inputenc}

\usepackage{xcolor}
\definecolor{darkred}{HTML}{990000}
\definecolor{darkgreen}{HTML}{009900}

\usepackage{sectsty}
\sectionfont{\color{darkred}}
\subsectionfont{\color{darkred}}
\subsubsectionfont{\color{darkred}}
% \usepackage{algpseudocode} --> better pseudo code?
%\usepackage[pagewise]{lineno} % row numbers

% --- syncing, referencing, citing
\usepackage{pdfsync} 			% synchronization LaTeX <--> PDF
\usepackage[hidelinks]{hyperref} % hidelinks avoids red boxes around links
\usepackage{csquotes}			% enquote
\usepackage{cleveref}
\usepackage{pdfpages}

\usepackage[backend=bibtex,style=alphabetic,url=false]{biblatex}
\addbibresource{sources.bib}
% define bibliography filter like
%\defbibfilter{computability-pca}{
%	keyword=computability or keyword=pca
%}

% --- lists and tables
\usepackage[shortlabels]{enumitem}
\usepackage{tabularx}
\usepackage{tabu}
\usepackage{makecell}

% --- graphics
\usepackage{graphicx}
\usepackage{xcolor}

% --- math
\usepackage{amsmath}
\usepackage[bbgreekl]{mathbbol} % double stroke greek
\usepackage{amssymb}
\usepackage{fixmath} % whyyy? "bug": italic PI and SIGMA
\usepackage{mathtools}
\usepackage{mathrsfs} %mathscr

% --- own packages
\usepackage{../notation}
\usepackage{../diagrams}
\usepackage{../environments}

\usepackage{extarrows}

\title{$\Sum \text{math}(y)$\\The Functor of Points}
\author{Jonas Linßen}

\NewDocumentCommand{\ft}{}{\text{ft}}

\begin{document}
	\maketitle
	\tableofcontents

	We follow [Kapil] \url{https://www.imsc.res.in/~kapil/crypto/notes/node38.html} as well as 


	\newpage
	\section{Affine Schemes}

	Let $F=\{f_1,...,f_r\} \subseteq R[X_1,...,X_n]$ be a finite system of polynomial equations in some (commutative unital) ring $R$. Given an $R$-algebra $A$ we want to study the set of simultaneous solutions of $F$ in $A$, denoted $V(F,A)=\{(a_1,...,a_n) \in A^n \mid \forall i=1 ... r: f_i(a_1,...,a_n) = 0 \}$, which describes an abstract geometric hypersurface in $A^n$.
	Since $R$-algebra homomorphisms respect polynomial equations with coefficients in $R$, we find that this assignment is indeed functorial
	\begin{equation*}
		\begin{array}{rcl}
			\ncat{Alg}_R&\xlongrightarrow{V(F,-)}&\ncat{Set}\\
			A&\longmapsto&V(F,A)\\
			(\phi:A\rightarrow B)&\longmapsto&(\phi:V(F,A)\rightarrow V(F,B)).
		\end{array}
	\end{equation*}
	By the universal property of the polynomial ring $R[X_1,...,X_n]$ a morphism of $R$-algebras of the form $\phi:R[X_1,...,X_n]/\Gen{f_1,...,f_r} \rightarrow A$ is uniquely determined by a point $(a_1,...,a_n)\in A^n$ with the property that $f_i(a_1,...,a_n)=0$ for each $i=1...r$. Hence we find that there is a natural isomorphism 
	\begin{equation*}
		V(F,-) \isom \ncat{Alg}_R(R[X_1,...,X_n]/\Gen{f_1,...,f_r},-),
	\end{equation*}
	which shows that the functor $V(F,-)$ is in fact representable.

	\begin{definition}
		A functor, which is isomorphic to one of the form $V(F,-)$, is called an \textbf{affine scheme over $R$}. In other words, an affine scheme over $R$ is a functor $\ncat{Alg}_R \rightarrow \ncat{Set}$, which is representable by an $R$-algebra of finite type. A \textbf{morphism of affine schemes over $R$} is a mere natural transformation. We write $\ncat{Aff}_R$ for the subcategory of affine schemes over $R$ in the functor category $|[\ncat{Alg}_R,\ncat{Set}|]$.
	\end{definition}

	Since the category of $R$-algebras of finite type $\ncat{Alg}_R^\ft$ is a full subcategory of $\ncat{Alg}_R$ the restricted Yoneda embedding
	\begin{equation*}
		\begin{array}{rcl}
			(\ncat{Alg}_R^\ft)^\op & \xlongrightarrow{\fun{Y}} & [\![\ncat{Alg}_R,\ncat{Set}]\!]\\
			A & \longmapsto & \ncat{Alg}_R(A,-)\\
			(\phi:A\rightarrow B) & \longmapsto & \pre\phi
		\end{array}
	\end{equation*}
	is full and faithful and we immediately get that $\fun{Y}$ restricts to an equivalence of categories
	\begin{equation*}
		(\ncat{Alg}_R^\ft)^\op \simeq \ncat{Aff}_R.
	\end{equation*}
	As such $\ncat{Aff}_R$ exhibits the same categorical structure as $(\ncat{Alg}_R^\ft)^\op$, in particular it is \TODO{finitely complete and finitely cocomplete}.
\end{document}