\documentclass{article}

% --- text layout
\usepackage[parfill]{parskip}
\usepackage{geometry}
\geometry{a4paper, margin=3cm}
\usepackage{romanbar}
\usepackage{placeins} %FloatBarrier
\usepackage[utf8]{inputenc}

\usepackage{xcolor}
\definecolor{darkred}{HTML}{990000}
\definecolor{darkgreen}{HTML}{009900}

\usepackage{sectsty}
\sectionfont{\color{darkred}}
\subsectionfont{\color{darkred}}
\subsubsectionfont{\color{darkred}}
% \usepackage{algpseudocode} --> better pseudo code?
%\usepackage[pagewise]{lineno} % row numbers

% --- syncing, referencing, citing
\usepackage{pdfsync} 			% synchronization LaTeX <--> PDF
\usepackage[hidelinks]{hyperref} % hidelinks avoids red boxes around links
\usepackage{csquotes}			% enquote
\usepackage{cleveref}
\usepackage{pdfpages}

\usepackage[backend=bibtex,style=alphabetic,url=false]{biblatex}
\addbibresource{sources.bib}
% define bibliography filter like
%\defbibfilter{computability-pca}{
%	keyword=computability or keyword=pca
%}

% --- lists and tables
\usepackage[shortlabels]{enumitem}
\usepackage{tabularx}
\usepackage{tabu}
\usepackage{makecell}

% --- graphics
\usepackage{graphicx}
\usepackage{xcolor}

% --- math
\usepackage{amsmath}
\usepackage[bbgreekl]{mathbbol} % double stroke greek
\usepackage{amssymb}
\usepackage{fixmath} % whyyy? "bug": italic PI and SIGMA
\usepackage{mathtools}
\usepackage{mathrsfs} %mathscr

% --- own packages
\usepackage{../notation}
\usepackage{../diagrams}
\usepackage{../environments}

\title{
	$\Sum \text{math}(y)$\\
	Affine Varieties
}
\author{Jonas Linßen}

\begin{document}
	\maketitle
	\tableofcontents

	\newpage
	\section{Varieties}
	\subsection{Affine Varieties and Hilbert's Nullstellensatz}

	Let $K$ be a field. We denote its algebraic closure by $\overline{K}$.

	\begin{definition}[Affine Algebraic Set]
		An \textbf{affine algebraic set} $A \subseteq K^n$ is a set of the form $A = Z(F)$, where
		\begin{equation*}
			Z(F) = \{x \in K^n \mid \forall f \in F: f(x) = 0\}
		\end{equation*}
		is the \textbf{zero locus} of some subset $F \subseteq K[X_1, \dots, X_n]$.

		Defining $\{Z(F) \mid F \subseteq K[T_1, \dots, T_n]\}$ to be the closed sets we obtain the so called \textbf{Zariski-topology} on $K^n$.

		An affine algebraic set $V \subseteq K^n$ is called \textbf{affine variety}, if it is \textit{irreducible} with respect to the Zariski topology, i.e. if it cannot be written as a union of closed proper subsets.
	\end{definition}
	\begin{sketch}
		To see that the Zariski-topology is indeed a topology verify  the following equations:

		$\emptyset = Z(1)$, $K^n = Z({0})$, $\Intersection \limits_{i \in I} Z(F_i) = Z(\Union \limits_{i \in I} F_i)$, $Z(F) \union Z(G) = Z(\{f \cdot g \mid f \in F, g \in G\})$.
	\end{sketch}

	\begin{remark}
		\vspace{-1.5em}
		\begin{enumerate}[(i)]
			\item{
				By definition of the zero locus we find $Z(F) = Z(\Gen{F})$. Moreover, since $K$ and by \textit{Hilbert's Basis Theorem} $K[T_1,...,T_n]$ is noetherian, the ideal $\Gen{F} \diveq K[T_1,...,T_n]$ is generated by finitely many elements $f_1,...,f_k$ and we find
				\begin{equation*}
					Z(F) = Z(\Gen{F}) = Z(f_1,...,f_k).
				\end{equation*}
			}
			\item{
				Every point $(x_1, \dots, x_n) \in K^n$ can be represented as an affine algebraic set of the form $Z(\{(T_i - x_i) \mid i = 1, .., n\})$.
			}
		\end{enumerate}
	\end{remark}

	\begin{definition}
		Let $S \subseteq K^n$ be a subset. The \textbf{vanishing ideal} of $S$ is the ideal 
		\begin{equation*}
			I(S) = \{f \in K[X_1, \dots, X_n] \mid \forall s \in S: f(s) = 0\} \trianglelefteq K[X_1, \dots, X_n].
		\end{equation*}
	\end{definition}

	\begin{theorem}[Hilbert's Nullstellensatz]
		If $A \subseteq K^n$ is an affine algebraic set, we have
		\begin{equation*}
			Z(I(A)) = A.
		\end{equation*}

		Conversely, if $\mathfrak{a} \trianglelefteq K[X_1, \dots, X_n]$ is an ideal, then
		\begin{equation*}
			I(Z(\mathfrak{a})) = \sqrt{\mathfrak{a}},
		\end{equation*}
		where $\sqrt{\mathfrak{a}}$ denotes the \textit{radical} of $\mathfrak{a}$.
	\end{theorem}

	\begin{corollary}[IV-Correspondence]
		Let $K = \overline{K}$ be algebraically closed. Then the assignments
		\begin{equation*}
			\hspace{-.75cm}\begin{array}{rcl}
				\{\text{ideals in }K[T_1,...,T_n]\} & \longrightarrow & \{\text{subsets of }K^n\}\\
				\mathfrak{a} & \longmapsto & Z(\mathfrak{a})\\
				I(S) & \longmapsfrom & S
			\end{array}
		\end{equation*}
		restrict to order reversing bijections
		\begin{equation*}
			\begin{array}{ccc}
				\{\text{radical ideals}\} & \overset{\isom}{\longleftrightarrow} & \{\text{aff. alg. sets}\}\\
				\rotatebox[origin=c]{90}{$\subseteq$}&&\rotatebox[origin=c]{90}{$\subseteq$}\\
				\{\text{prime ideals}\} & \overset{\isom}{\longleftrightarrow} & \{\text{aff. varieties}\}\\
				\rotatebox[origin=c]{90}{$\subseteq$}&&\rotatebox[origin=c]{90}{$\subseteq$}\\
				\{\text{maximal ideals}\} & \overset{\isom}{\longleftrightarrow} & \{\text{points in }K^n\}
			\end{array}
		\end{equation*}
	\end{corollary}

	\subsection{Polynomial Morphisms}

	\begin{definition}
		Let $A \subseteq K^n$ be an affine algebraic set. 

		A \textbf{polynomial morphism} $F:A \longrightarrow K$ is a function, which is given by evaluating a polynomial $f \in K[T_1,...,T_n]$ on $A$. The set of these functions can be identified with the \textbf{coordinate ring} of $A$, defined by 
		\begin{equation*}
			K[A] = K[T_1, \dots, T_n] / I(A).
		\end{equation*}
	\end{definition}

	\begin{remark}
		\begin{enumerate}[(i)]
			\item{
				\TODO{The coordinate ring $K[A]$ is generated by the coordinate functions $x_1..,x_n$}
			}
			\item{
				For $K = \overline{K}$ the IV-correspondence yields that $A$ is an affine variety if and only if $K[A]$ is an integral domain. Moreover, given an affine variety $V$, the ideal correspondence for factor rings yields that the induced topology of $V$ can be identified with the 
			}
			\item{
				For $K=\overline{K}$ the ideal correspondence of factor rings and 
			}
		\end{enumerate}
	\end{remark}

	\begin{theorem}

	\end{theorem}
\end{document}