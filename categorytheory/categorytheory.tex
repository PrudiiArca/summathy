\documentclass{article}

% --- text layout
\usepackage[parfill]{parskip}
\usepackage{geometry}
\geometry{a4paper, margin=3cm}
\usepackage{romanbar}
\usepackage{placeins} %FloatBarrier
\usepackage[utf8]{inputenc}

\usepackage{xcolor}
\definecolor{darkred}{HTML}{990000}
\definecolor{darkgreen}{HTML}{009900}

\usepackage{sectsty}
\sectionfont{\color{darkred}}
\subsectionfont{\color{darkred}}
\subsubsectionfont{\color{darkred}}
% \usepackage{algpseudocode} --> better pseudo code?
%\usepackage[pagewise]{lineno} % row numbers

% --- syncing, referencing, citing
\usepackage{pdfsync} 			% synchronization LaTeX <--> PDF
\usepackage[hidelinks]{hyperref} % hidelinks avoids red boxes around links
\usepackage{csquotes}			% enquote
\usepackage{cleveref}
\usepackage{pdfpages}

\usepackage[backend=bibtex,style=alphabetic,url=false]{biblatex}
\addbibresource{sources.bib}
% define bibliography filter like
%\defbibfilter{computability-pca}{
%	keyword=computability or keyword=pca
%}

% --- lists and tables
\usepackage[shortlabels]{enumitem}
\usepackage{tabularx}
\usepackage{tabu}
\usepackage{makecell}

% --- graphics
\usepackage{graphicx}
\usepackage{xcolor}

% --- math
\usepackage{amsmath}
\usepackage{amssymb}
\usepackage{fixmath} % whyyy? "bug": italic PI and SIGMA
\usepackage{mathtools}
\usepackage{mathrsfs} %mathscr

% --- own packages
\usepackage{../notation}
\usepackage{../diagrams}
\usepackage{../environments}

\title{$\Sum\text{math}(y)$\\Category Theory}
\author{Jonas Linssen}

\begin{document}
	\maketitle
	\tableofcontents

	\newpage
	\section{Categories}
	\subsection{Categories}
	\subsection{Functors and Natural Transformations}

	\begin{lemma}
		\TODO{full + faithful implies conservative}
	\end{lemma}

	\begin{lemma}
		\TODO{$\fun{K}$ is esssurj iff $\pre\fun{K}$ is conservative}
	\end{lemma}

	\newpage
	\subsection{Examples}

	\partitle{Concrete Categories}[concretecat]
	\partitle{Functor Categories}[funcat]
	\partitle{Comma Categories and Slice Categories}[commacat]

	\begin{definition}
		Let $\fun{F}:\cat{A} --> \cat{C}$ and $\fun{G}:\cat{B}-->\cat{C}$ be two functors. The \textbf{comma category} $\fun{F} \downarrow \fun{G}$ consists of\\

		\begin{tabular}{rl}
			Objects: & triples $(A,f,B)$, where $A \in \Ob \cat{A}$, $B \in \Ob \cat{B}$ and $f:\fun{F}A --> \fun{F}B$\\
			Morphisms: & pairs $(a,b)$, where $a:A --> A'$ in $\cat{A}$, $b:B-->B'$ in $\cat{B}$ satisfying
			%\begin{equation*}
				\begin{diagram}
					\twobytwo
						{\fun{F}A}{\fun{F}A'}
						{\fun{G}B}{\fun{G}B'}
					\arrow{nw}{ne}{\fun{F}a}[above]
					\arrow{sw}{se}{\fun{G}b}[below]
					\arrow{nw}{sw}{f}[left]
					\arrow{ne}{se}{f'}[right]
				\end{diagram}
			%\end{equation*}
		\end{tabular}

		This construction yields canonical functors
		\begin{equation*}
			\begin{array}{rcl}
				\fun{F}\downarrow\fun{G} & \xrightarrow{\fun{U}_\cat{A}} & \cat{A}\\
				(A,f,B) & \longmapsto & A\\
				(a,b) & \longmapsto & a
			\end{array}
			\hspace{1cm}\text{and}\hspace{1cm}
			\begin{array}{rcl}
				\fun{F}\downarrow\fun{G} & \xrightarrow{\fun{U}_\cat{B}} & \cat{B}\\
				(A,f,B) & \longmapsto & B\\
				(a,b) & \longmapsto & b 
			\end{array}
		\end{equation*}
		as well as
		\begin{equation*}
			\begin{array}{rcl}
				\fun{F}\downarrow\fun{G} & \xrightarrow{\dom} & \cat{C}\\
				(A,f,B) & \longmapsto & \fun{F}A\\
				(a,b) & \longmapsto & \fun{F}a
			\end{array}
			\hspace{1cm}\text{and}\hspace{1cm}
			\begin{array}{rcl}
				\fun{F}\downarrow\fun{G} & \xrightarrow{\cod} & \cat{C}\\
				(A,f,B) & \longmapsto & \fun{G}B\\
				(a,b) & \longmapsto & \fun{G}b
			\end{array}
		\end{equation*}
		and by construction $\dom = \fun{FU}_\cat{A}$ and $\cod = \fun{GU}_\cat{B}$.
	\end{definition}

	\newpage
	\subsection{Representables and the Yoneda Lemma}

	\begin{definition}
		Let $\cat{C}$ be a category. 

		A contravariant functor $\fun{F}:\cat{C}^\op --> \ncat{Set}$, which is isomorphic to a hom-functor $\cat{C}(-,C):\cat{C}-->\ncat{Set}$, is said to be \textbf{representable}. The object $C$ is called a \textbf{representing object} of $\fun{F}$.

		Similarly, a covariant functor $\fun{F}:\cat{C} --> \ncat{Set}$, which is isomorphic to a hom-functor $\cat{C}(C,-):\cat{C}-->\ncat{Set}$ is called \textbf{representable} and $C$ is a \textbf{representing object} of $\fun{F}$.
	\end{definition}

	\begin{theorem}[Yoneda Lemma]
		Let $\cat{C}$ be a category. The mutually inverse assignments
		\begin{align*}
			\begin{array}{ccl}
				|[\cat{C^\op,\ncat{Set}}|](\cat{C}(-,C),\fun{F}) &\isom& \fun{F}C
				\vspace{.5em}\\
				\alpha=(\alpha_{C'})_{C'} & \mapsto & \alpha_C(1_C)
				\vspace{.5em}\\
				\left(\begin{array}{rcl}
					\cat{C}(C',C) &\rightarrow& \fun{F}C'\\
				 	f & \mapsto & \fun{F}f(x)
				\end{array}\right)_{C'} & \mapsfrom & x
			\end{array}
		\end{align*}
		exhibit isomorphisms natural in $C$ and $F$. Similarly, the mutually inverse assignments
		\begin{align*}
			\begin{array}{ccl}
				|[\cat{C,\ncat{Set}}|](\cat{C}(C,-),\fun{F}) &\isom& \fun{F}C
				\vspace{.5em}\\
				\alpha=(\alpha_{C'})_{C'} & \mapsto & \alpha_C(1_C)
				\vspace{.5em}\\
				\left(\begin{array}{rcl}
				 	\cat{C}(C,C') &\rightarrow& \fun{F}C'\\
				 	f & \mapsto & \fun{F}f(x)
				\end{array}\right)_{C'} & \mapsfrom & x
			\end{array}
		\end{align*}
		exhibit isomorphisms natural in $C$ and $F$.
	\end{theorem}

	\begin{corollary}
		The (covariant) \textbf{Yoneda embedding}
		\begin{align*}
			\begin{array}{rcl}
				\cat{C} & \xrightarrow{\fun{Y}} & |[\cat{C}^\op,\ncat{Set}|]\\
				C & \longmapsto & \cat{C}(-,C)\\
				{[f:C\rightarrow C']} & \longmapsto & {[f\post:\cat{C}(-,C)\Rightarrow\cat{C}(-,C')]}
			\end{array}
		\end{align*}
		is full and faithful. In particular $\fun{Y}$ is conservative, which implies that any two representing objects of a functor are isomorphic.
	\end{corollary}

	\newpage
	\subsection{Adjoint Functors and Kan Extensions}

	\begin{definition}
		A functor $\fun{R}:\cat{D} --> \cat{C}$ \textbf{admits a left adjoint} and \textbf{is a right adjoint}, if it satisfies one of the following equivalent conditions:
		\begin{enumerate}[(i)]
			\item{
				There is a functor $\fun{L}:\cat{C} --> \cat{D}$ together with natural transformations $\eta: 1_\cat{C} ==> \fun{RL}$ and $\epsilon: \fun{LR} ==> 1_\cat{D}$, called \textbf{unit} and \textbf{counit}, which satisfy the \textit{triangle identities}
				% \begin{equation*}
				% 	\begin{diagram}
				% 		\twobytwo
				% 			{\fun{L}}{\fun{LRL}}
				% 			{}{\fun{L}}
				% 		\arrow[equals]{nw}{se}{}
				% 		\arrow[implies]{nw}{ne}{\fun{L}\eta}[above]
				% 		\arrow[implies]{ne}{se}{\epsilon\fun{L}}[right]
				% 	\end{diagram}
				% 	\hspace{1cm}\text{and}\hspace{1cm}
				% 	\begin{diagram}
				% 		\twobytwo
				% 			{\fun{R}}{\fun{RLR}}
				% 			{}{\fun{R}}
				% 		\arrow[equals]{nw}{se}{}
				% 		\arrow[implies]{nw}{ne}{\eta\fun{R}}[above]
				% 		\arrow[implies]{ne}{se}{\fun{R}\epsilon}[right]
				% 	\end{diagram}\hspace{.25cm}.
				% \end{equation*}
				\begin{equation*}
					\begin{diagram}
						\twobytwo
							{\fun{L}}{\fun{LRL}}
							{}{\fun{L}}

						\arrow[implies]{nw}{ne}{\fun{L}\eta}[above]
						\arrow[implies]{ne}{se}{\epsilon\fun{L}}[right]
						\arrow[equals]{nw}{se}{}
					\end{diagram}
					\hspace{1cm}
					\text{resp.}
					\hspace{1cm}
					\begin{diagram}
						\fourbyone{\cat{C}}{\cat{D}}{\cat{C}}{\cat{D}}
						\arrow{ww}{w}{\fun{L}}[ontop]
						\arrow{w}{e}{\fun{R}}[ontop]
						\arrow{e}{ee}{\fun{L}}[ontop]
						\arrow[out=45,in=135,equals]{ww}{e}{1_\cat{C}}[above]
						\arrow[out=-45,in=-135,equals]{w}{ee}{1_\cat{D}}[below]
						\node at (-1,.5) {$\Downarrow \eta$};
						\node at (1,-.5) {$\Downarrow \epsilon$};
					\end{diagram}
					% =
					% \begin{diagram}
					% 	\twobyone{\cat{C}}{\cat{D}}
					% 	\arrow[bend left]{w}{e}{\fun{L}}[above]
					% 	\arrow[bend right]{w}{e}{\fun{L}}[below]
					% 	\node at (0,0) {$||$};
					% \end{diagram}
					=
					\begin{diagram}
						\twobyone{\cat{C}}{\cat{D}}
						\arrow{w}{e}{\fun{L}}[above]
					\end{diagram}
				\end{equation*}
				and
				\begin{equation*}
					\begin{diagram}
						\twobytwo
							{\fun{R}}{\fun{RLR}}
							{}{\fun{R}}

						\arrow[implies]{nw}{ne}{\eta\fun{R}}[above]
						\arrow[implies]{ne}{se}{\fun{R}\epsilon}[right]
						\arrow[equals]{nw}{se}{}
					\end{diagram}
					\hspace{1cm}
					\text{resp.}
					\hspace{1cm}
					\begin{diagram}
						\fourbyone{\cat{D}}{\cat{C}}{\cat{D}}{\cat{C}}
						\arrow{ww}{w}{\fun{R}}[ontop]
						\arrow{w}{e}{\fun{L}}[ontop]
						\arrow{e}{ee}{\fun{R}}[ontop]
						\arrow[out=45,in=135,equals]{w}{ee}{}
						\arrow[out=-45,in=-135,equals]{ww}{e}{}
						\node at (1,.5) {$\Downarrow \eta$};
						\node at (-.8,-.5) {$\Downarrow \epsilon$};
					\end{diagram}
					% =
					% \begin{diagram}
					% 	\twobyone{\cat{D}}{\cat{C}}
					% 	\arrow[bend left]{w}{e}{\fun{R}}[above]
					% 	\arrow[bend right]{w}{e}{\fun{R}}[below]
					% 	\node at (0,0) {$||$};
					% \end{diagram}
					=
					\begin{diagram}
						\twobyone{\cat{D}}{\cat{C}}
						\arrow{w}{e}{\fun{R}}[above]
					\end{diagram}
				\end{equation*}
			}
			\item{
				\begin{minipage}[t]{\linewidth-4cm}
					For every $C$ in $\cat{C}$ the comma category $C \downarrow \fun{R}$ has an initial object $(C,\eta_C,\fun{L}C)$.\vspace{.5em}

					This is to say there is the following universal property: for every $C$ in $\cat{C}$ there is an object $\fun{L}C$ in $\cat{D}$ and a morphism $\eta_C:C-->\fun{RL}C$ such that for any $D$ in $\cat{D}$ and morphism $f:C-->\fun{R}D$ there exists a unique extension $\widehat{f}:\fun{L}C-->D$ in $\cat{D}$ such that the diagram on the right commutes in $\cat{C}$.
				\end{minipage}
				\begin{minipage}[t]{4cm}
					\vspace{-.5cm}
					\begin{equation*}
						\begin{diagram}
							\twobytwo
								{\fun{RL}C}{\fun{R}D}
								{C}{}
							\arrow{sw}{nw}{\eta_C}[left]
							\arrow{sw}{ne}{f}[below right]
							\arrow[dashed]{nw}{ne}{\exists!\;\fun{R}(\widehat{f})}[above]
						\end{diagram}
					\end{equation*}
				\end{minipage}
			}
			\item{
				There is a functor $\fun{L}:\cat{C} --> \cat{D}$ and isomorphisms $\phi_{CD}:\cat{D}(\fun{L}C,D) \isom \cat{C}(C,\fun{R}D):\psi_{CD}$ natural in $C$ and $D$. Writing $\eta_C:C-->\fun{RL}C$ for the morphism corresponding to $1_{\fun{L}C}$ and $\epsilon_D:\fun{LR}D --> D$ for the morphism corresponding to $1_{\fun{R}D}$ this isomorphism is given by
				\begin{equation*}
					\begin{array}{rcl}
						\cat{D}(\fun{L}C,D) &\isom& \cat{C}(C,\fun{R}D)\\
						g & \mapsto & \fun{R}g \circ \eta_C\\
						\epsilon_D \circ \fun{L}f & \mapsfrom & f
					\end{array}
				\end{equation*}
			}
			\item{
				For every $C$ in $\cat{C}$ the functor $\cat{C}(C,\fun{R}-):\cat{D} --> \ncat{Set}$ is representable by an object $\fun{L}C$ in $\cat{D}$.
			}
		\end{enumerate}
	\end{definition}
	\begin{sketch}
		\begin{enumerate}[leftmargin=1.5cm]
			\item[(i)$\Rightarrow$(ii)]{
				The following commuting diagrams demonstrate existence and uniqueness of $\widehat{f}$.
				\begin{equation*}
					\begin{diagram}
						\threebytwo[wide]
							{\fun{RL}C}{\fun{RLR}D}{\fun{RD}}
							{C}{\fun{R}D}{}

						\arrow{sw}{nw}{\eta_C}[left]
						\arrow{sw}{s}{f}[below]
						\arrow{s}{n}{\eta_{\fun{R}D}}[ontop]
						\arrow[equals]{s}{ne}{}
						\arrow{nw}{n}{\fun{RL}f}[above]
						\arrow{n}{ne}{\fun{R}\epsilon_D}[above]
					\end{diagram}
					\hspace{1cm}
					\begin{diagram}
						\twobythree[wide]
							{\fun{LC}}{D}
							{\fun{LRL}C}{\fun{LR}D}
							{\fun{L}C}{}

						\arrow{sw}{w}{\fun{L}\eta_C}[right]
						\arrow{w}{nw}{\epsilon_{\fun{L}C}}[right]
						\arrow{w}{e}{\fun{LR}\widehat{f}}[above]
						\arrow{sw}{e}{\fun{L}f}[below right]
						\arrow{e}{ne}{\epsilon_D}[right]
						\arrow{nw}{ne}{\widehat{f}}[above]
						\arrow[equals,out=135,in=-135]{sw}{nw}{}
					\end{diagram}
				\end{equation*}
			}
			\item[(ii)$\Rightarrow$(iii)]{
				\begin{minipage}[t]{\linewidth-4cm}
					Given $c:C'-->C$ the initial objects $(C,\eta_C,\fun{L}C)$ and $(C',\eta_{C'},\fun{L}C')$ yield a unique morphism
					\begin{equation*}
						\fun{L}c:\fun{L}C' \xrightarrow{\widehat{\eta_Cc}} \fun{L}C
					\end{equation*}
					making the diagram on the right commute. By uniqueness this assignment is functorial.
				\end{minipage}
				\begin{minipage}[t]{4cm}
					\vspace{-.5cm}
					\begin{equation*}
						\begin{diagram}
							\twobytwo[wide]
								{\fun{RL}C'}{\fun{RL}C}
								{C'}{C}
							\arrow[dashed]{nw}{ne}{\exists!\fun{R}(\fun{L}c)}[above]
							\arrow{sw}{se}{c}[below]
							\arrow{sw}{nw}{\eta_{C'}}[left]
							\arrow{se}{ne}{\eta_C}[right]
						\end{diagram}
					\end{equation*}
				\end{minipage}

				Moreover every initial object $(C,\eta_C,\fun{L}C)$ provides a bijective function 
				\begin{equation*}
					\begin{array}{rcl}
						\psi_{CD}:\cat{C}(C,\fun{R}D) &\rightarrow& \cat{D}(\fun{L}C,D)\\
						f&\mapsto&\widehat{f}
					\end{array}
				\end{equation*}
				with $\eta_C$ corresponding to $\widehat{\eta_C} = 1_{\fun{L}C}$ and inverse given by
				\begin{equation*}
					\begin{array}{rcl}
						\phi_{CD}:\cat{D}(\fun{L}C,D) &\rightarrow& \cat{C}(C,\fun{R}D) \\
						g&\mapsto&\fun{R}g\circ\eta_C
					\end{array}
				\end{equation*}
				which is by construction natural in $D$.

				Writing $\epsilon_D = \widehat{1_{\fun{R}D}}$ we note that by definition $\fun{R}\epsilon_D \circ \eta_{\fun{R}D} = 1_{\fun{R}D}$ and so by uniqueness of inverses

				% the commutativity of the diagram
				% \begin{equation*}
				% 	\begin{diagram}
				% 		\threebytwo[verywide]
				% 			{\fun{RL}C}{\fun{RLRL}C}{\fun{RL}C}
				% 			{C}{\fun{RL}C}{}
				% 		\arrow{nw}{n}{\fun{RL}\eta_C}[above]
				% 		\arrow{n}{ne}{\fun{R}\epsilon_{\fun{L}C}}[above]
				% 		\arrow{sw}{s}{\eta_C}[below]
				% 		\arrow[equals]{s}{ne}{\fun{R}1_{\fun{L}C}}[below right]
				% 		\arrow{sw}{nw}{\eta_C}[left]
				% 		\arrow{s}{n}{\eta_{\fun{RL}C}}[left]
				% 	\end{diagram}
				% \end{equation*}
				% and the injectivity of $\phi_{CD}$ imply $\epsilon_{\fun{L}C}\circ \fun{L}\eta_C = 1_{\fun{L}C}$.

				For naturality in $C$ note that given $c:C'-->C$ the initial element in $C'\downarrow\fun{R}$ gives rise to
				\begin{equation*}
					\begin{diagram}
						\threebytwo[verywide]
							{\fun{RL}C'}{\fun{RL}C}{\fun{R}D}
							{C'}{C}{}

						\arrow[bend left]{nw}{ne}{\fun{R}\widehat{\widehat{fc}}}[above]
						%\arrow{nw}{n}{\fun{R}\widehat{\widehat{t}\,}}[above]
						\arrow{nw}{n}{\fun{RL}c}[above]
						\arrow{n}{ne}{\fun{R}\widehat{f}}[above]
						\arrow{sw}{s}{c}[below]
						\arrow{s}{ne}{f}[below right]
						\arrow{sw}{nw}{\eta_{C'}}[left]
						\arrow{s}{n}{\eta_C}[left]
						%\arrow{sw}{n}{t}[ontop]
					\end{diagram}
				\end{equation*}
			}
		\end{enumerate}
	\end{sketch}

	\begin{definition}
		Dually a functor $\fun{L}:\cat{C} --> \cat{D}$ \textbf{admits a right adjoint} and \textbf{is a left adjoint}, if it satisfies one of the following equivalent conditions:
		\begin{enumerate}[(i)]
			\item{
				There is a functor $\fun{R}:\cat{D} --> \cat{C}$ together with natural transformations $\eta: 1_\cat{C} ==> \fun{RL}$ and $\epsilon: \fun{LR} ==> 1_\cat{D}$, called \textbf{unit} and \textbf{counit}, which satisfy the \textit{triangle identities}.
			}
			\item{
				\begin{minipage}[t]{\linewidth-4cm}
					For every $D$ in $\cat{D}$ the comma category $\fun{L} \downarrow D$ has a terminal object $(\fun{R}D,\epsilon_D,D)$.\vspace{.5em}

					This is to say there is the following universal property: for every $D$ in $\cat{D}$ there is an object $\fun{R}D$ in $\cat{C}$ and a morphism $\epsilon_D:\fun{LR}D-->D$ such that for any $C$ in $\cat{C}$ and morphism $f:\fun{L}C-->D$ there exists a unique lift $\widehat{f}:C --> \fun{R}D$ in $\cat{C}$ such that the diagram on the right commutes in $\cat{D}$.
				\end{minipage}
				\begin{minipage}[t]{4cm}
					\vspace{-.5cm}
					\begin{equation*}
						\begin{diagram}
							\twobytwo
								{\fun{L}C}{\fun{LR}D}
								{}{D}
							\arrow{ne}{se}{\eta_C}[right]
							\arrow{nw}{se}{f}[below left]
							\arrow[dashed]{nw}{ne}{\exists!\;\fun{L}(\widehat{f})}[above]
						\end{diagram}
					\end{equation*}
				\end{minipage}
			}
			\item{
				There is a functor $\fun{R}:\cat{D} --> \cat{L}$ and isomorphisms $\varphi_{CD}:\cat{D}(\fun{L}C,D) \isom \cat{C}(C,\fun{R}D)$ natural in $C$ and $D$.
			}
			\item{
				For every $D$ in $\cat{D}$ the functor $\cat{D}(\fun{L}-,D):\cat{C}^\op --> \ncat{Set}$ is corepresentable by an object $\fun{R}D$ in $\cat{C}$.
			}
		\end{enumerate}
	\end{definition}

	\begin{lemma}[Properties of Adjoint Functors]
		\TODO{Left/Right adjoints to a given functor are unique up to \TODO{unique} iso.}

		\TODO{Adjoint functors compose.}
	\end{lemma}

	\begin{lemma}[Adjoint Equivalences]
		Any equivalence of categories can be turned into an adjoint equivalence.

		Conversely, let $\fun{L}: \cat{C} <-> \cat{D}: \fun{R}$ be a pair of adjoint functors. Then
		\begin{enumerate}[(i)]
			\item{
				$\fun{R}$ is faithful / $\fun{L}$ is faithful, if and only if $\varepsilon$ is epi / $\eta$ is mono.
			}
			\item{
				$\fun{R}$ is full / $\fun{L}$ is full, if and only if $\varepsilon$ is split mono / $\eta$ is split epi.
			}
		\end{enumerate}
		In particular, $\fun{L} \dashv \fun{R}$ is an adjoint equivalence, if and only if both $\fun{L}$ and $\fun{R}$ are full and faithful.
	\end{lemma}

	\begin{definition}
		The functor $\fun{F}:\cat{A} --> \cat{C}$ admits a \textbf{left Kan extension along} the functor $\fun{K}:\cat{A} --> \cat{B}$, if it satisfies one of the following equivalent conditions.
		\begin{enumerate}[(i)]
			\item{
				There is a functor $\fun{L}:\cat{B}--> \cat{C}$ and a natural transformation $\lambda: \fun{F} ==> \fun{LK}$ such that the composite
				\begin{align*}
					|[\cat{B},\cat{C}|](\fun{L},\fun{G}) \xrightarrow{-\circ \fun{K}} |[\cat{A},\cat{C}|](\fun{LK},\fun{GK}) \xrightarrow{\pre\lambda}|[\cat{A},\cat{C}|](\fun{F},\fun{GK})
				\end{align*}
				is an isomorphism natural in $\fun{G} \in |[\cat{B},\cat{C}|]$.

				\TODO{universal property}
			}
			\item{
				The functor $|[\cat{A},\cat{C}|](F,\pre\fun{K}-):|[\cat{B},\cat{C}|] --> \ncat{Set}$ is represented by a functor $\fun{L} \in |[\cat{B},\cat{C}|]$.
			}
		\end{enumerate}
		The functor $\fun{L}$ is commonly denoted by $\Lan_\fun{K}\fun{F}$.
	\end{definition}

	\begin{lemma}[Properties of Kan Extensions]
		\TODO{left/right Kan extensions are unique up to unique iso}

		\TODO{left/right adjoints preserve left/right Kan extensions}

		\TODO{left/right Kan extensions compose}
	\end{lemma}

	\begin{lemma}
		\begin{minipage}{\linewidth-5cm}
			Let $\cat{A},\cat{B}$ be small categories, $\fun{K}:\cat{A} --> \cat{B}$ be a functor and $\cat{C}$ be another category.\\

			Then every functor $\fun{F}:\cat{A} --> \cat{C}$ admits a left/right Kan extension along $\fun{K}$, if and only if the functor $\pre\fun{K}:|[\cat{B},\cat{C}|] --> |[\cat{A},\cat{C}|]$ has a left/right adjoint.
		\end{minipage}
		\begin{minipage}{5cm}
			\begin{equation*}
				\begin{diagram}
					\node (w) at (-1.5,0) {$[\![\cat{B},\cat{C}]\!]$};
					\node (e) at (1.5,0) {$[\![\cat{A},\cat{C}]\!]$};

					\arrow[bend right]{e}{w}{\Lan_\fun{K}}[above]
					\arrow{w}{e}{\pre\fun{K}}[ontop]
					\arrow[bend left]{e}{w}{\Ran_\fun{K}}[below]

					\node at (0,.35) {$\bot$};
					\node at (0,-.3) {$\bot$};
				\end{diagram}
			\end{equation*}
		\end{minipage}
	\end{lemma}

	% \section{2-Categories}
	% \subsection{2-Categories, 2-Functors, 2-Natural Transformations}
	% \subsection{Adjoints and Kan Extensions}
	\newpage
	\section{Limits and Colimits}
	\subsection{Limits and Colimits}

	\begin{definition}
		\TODO{limits as Ranext}
	\end{definition}

	\begin{lemma}
		\TODO{conservative functors reflect all limits/colimits they preserve}
	\end{lemma}

	\subsection{Examples}

	\begin{proposition}
		\TODO{Limits/Colimits in functor categories are computed pointwise}
	\end{proposition}

	\begin{theorem}[Density Theorem]
		Every presheaf $P$ is colimit of representables
	\end{theorem}

	\begin{proposition}
		Suppose that $\cat{C}$ has limits of shape $\cat{I}$ and $\fun{F}:\cat{C} --> \cat{D}$ preserves them.

		Then $\cat{D} \downarrow \fun{F}$ has limits of shape $\cat{I}$ and $\fun{U}: \cat{D} \downarrow \fun{F} --> \cat{D}$ creates them.
	\end{proposition}

	\subsection{Limits vs. Colimits}
	\subsection{Adjoint Functor vs. Limits and Colimits}

	\begin{proposition}[RAPL + LAPC]
		\TODO{todo via RAPRan + LAPLan and via Yoneda}
	\end{proposition}

	\begin{theorem}[General Adjoint Functor Theorem]
		Let $\cat{D}$ be a complete category. Then $\fun{R}: \cat{D} --> \cat{C}$ has a left adjoint, if and only if $\fun{R}$ is continuous and for every $C$ in $\cat{C}$ the comma category $C \downarrow \fun{R}$ has a weakly initial set.

		Dually, if $\cat{C}$ is a cocomplete category, then $\fun{L}: \cat{C} --> \cat{D}$ has a right adjoint, if and only if $\fun{L}$ is cocontinuous and for every $D$ in $\cat{D}$ the comma category $\fun{L} \downarrow D$ has a weakly terminal set.
	\end{theorem}

	\begin{theorem}[Special Adjoint Functor Theorem]
		Let $\cat{D}$ be a complete category. \TODO{todo}
	\end{theorem}

	\newpage
	\section{Density and Locally Presentable Categories}
	\subsection{Pointwise Kan Extensions}

	\begin{definition}
		Suppose the functor $\fun{F}:\cat{A} --> \cat{C}$ admits a right Kan extension along $\fun{K}:\cat{A} --> \cat{B}$. $\Ran_\fun{K}\fun{F}$ is said to be a \textbf{pointwise right Kan extension}, if it satisfies one of the following equivalent conditions.
		\begin{enumerate}[(i)]
			\item{
				$\Ran_\fun{K}\fun{F}$ is preserved by representables.
				%For every $C$ in $\cat{C}$ the hom-functor $\cat{C}(C,-): \cat{C} --> \ncat{Set}$ preserves $\Ran_\fun{K}\fun{F}$.
			}
			\item{
				$\Ran_\fun{K}\fun{F}$ can be computed pointwise via
				\begin{align*}
					\Ran_\fun{K}\fun{F}(B) = \lim_{B \downarrow \fun{K}} \fun{FU}_{B \downarrow \fun{K}}
				\end{align*}
			}
			\item{
				For every $B$ in $\cat{B}$ the functor \TODO{???} is represented by \TODO{???}
			}
		\end{enumerate}
	\end{definition}

	\begin{lemma}
		Let $\cat{A},\cat{B}$ be small categories and $\cat{C}$ be complete/cocomplete. Then for all $\fun{K}:\cat{A} --> \cat{B}$ and $\fun{F}:\cat{A}-->\cat{C}$ the right Kan extension $\Ran_\fun{K}\fun{F}$/left Kan extension $\Lan_\fun{K}\fun{F}$ exists and is pointwise.
	\end{lemma}

	\subsection{Density}

	\subsection{Locally Presentable Categories}

	\newpage
	\section{Monads}
\end{document}