\documentclass{article}

% --- text layout
\usepackage[parfill]{parskip}
\usepackage{geometry}
\geometry{a4paper, margin=3cm}
\usepackage{romanbar}
\usepackage{placeins} %FloatBarrier
\usepackage[utf8]{inputenc}

\usepackage{xcolor}
\definecolor{darkred}{HTML}{990000}
\definecolor{darkgreen}{HTML}{009900}

\usepackage{sectsty}
\sectionfont{\color{darkred}}
\subsectionfont{\color{darkred}}
\subsubsectionfont{\color{darkred}}
% \usepackage{algpseudocode} --> better pseudo code?
%\usepackage[pagewise]{lineno} % row numbers

% --- syncing, referencing, citing
\usepackage{pdfsync} 			% synchronization LaTeX <--> PDF
\usepackage[hidelinks]{hyperref} % hidelinks avoids red boxes around links
\usepackage{csquotes}			% enquote
\usepackage{cleveref}
\usepackage{pdfpages}

\usepackage[backend=bibtex,style=alphabetic,url=false]{biblatex}
\addbibresource{sources.bib}
% define bibliography filter like
%\defbibfilter{computability-pca}{
%	keyword=computability or keyword=pca
%}

% --- lists and tables
\usepackage[shortlabels]{enumitem}
\usepackage{tabularx}
\usepackage{tabu}
\usepackage{makecell}

% --- graphics
\usepackage{graphicx}
\usepackage{xcolor}

% --- math
\usepackage{amsmath}
\usepackage{amssymb}
\usepackage{fixmath} % whyyy? "bug": italic PI and SIGMA
\usepackage{mathtools}
\usepackage{mathrsfs} %mathscr

% --- own packages
\usepackage{../notation}
\usepackage{../diagrams}
\usepackage{../environments}

\title{$\Sum \text{math}(y)$\\Topological Spaces}
\author{Jonas Linssen}

\begin{document}

\maketitle
\tableofcontents

\newpage
\section{Topological Spaces}
\subsection{Topological Spaces}

\begin{definition}
	A \textbf{topological space} $(X,\mathcal{O})$ is a set $X$ together with a family of sets $\mathcal{O} \subseteq \mathcal{P}(X)$ called \textbf{topology}, which satisfies the following axioms. Sets contained in $\mathcal{O}$ are called \textbf{open} in $X$.
	\begin{enumerate}[$\bullet$]
		\item{
			Both $\emptyset$ and $X$ are open.
		}
		\item{
			Finite intersections of open sets are open.
		}
		\item{
			Arbitrary unions of open sets are open.
		}
	\end{enumerate}
	The relative complement $X \setminus O$ of an open set $O$ is called \textbf{closed} in $X$. By de'Morgan one may define a topology as the set of relative complements of a family $\mathcal{C} \subseteq \mathcal{P}(X)$ of closed sets, which satisfies the following axioms.
	\begin{enumerate}[$\bullet$]
		\item{
			Both $\emptyset$ and $X$ are closed.
		}
		\item{
			Arbitrary intersections of closed sets are closed.
		}
		\item{
			Finite unions of closed sets are closed.
		}
	\end{enumerate}

	A subset $N$ in $X$ is called a \textbf{neighborhood} of a given point $x$, if there is a open set $U$, which satisfies $x \in U \subseteq N$. The set of all neighborhoods of $x$ is the \textbf{neighborhood filter} of $x$ and denoted by $\mathcal{N}(x)$.
\end{definition}

\begin{example}\vspace{-1.5em}
	\begin{enumerate}[(i)]
		\item{
			Any set $X$ can be equipped with two canonical topologies. The \textbf{discrete topology} on $X$ is given by $\mathcal{O}_\text{disc} = \mathcal{P}(X)$. The \textbf{chaotic topology} on $X$ is the topology given by $\mathcal{O}_\text{chaot} = \{\emptyset, X\}$.

			In fact any topology $\mathcal{O}$ on $X$ satisfies $\mathcal{O}_\text{chaot} \subseteq \mathcal{O} \subseteq \mathcal{O}_\text{disc}$. Hence the discrete topology is the \textbf{finest topology} and the chaotic topology the \textbf{coarsest topology} on $X$.
		}
		\item{
			The \textbf{empty space} is the unique topological space without any points $(\emptyset,\{\emptyset\})$.

			The \textbf{singleton space} $(\{*\}, \{\emptyset,\{*\}\})$, is the unique topological space with exactly one point.

			On a two element set $\{x,y\}$ there are precisely three different topologies (up to switching the roles of $x$ and $y$):
			\begin{enumerate}[$\bullet$]
				\item{
					The discrete topology $\{\emptyset, \{x\}, \{y\}, \{x,y\}\}$. We will denote this space by $\{\circ, \circ\}$ or $\{\bullet, \bullet\}$.
				}
				\item{
					The topology $\{\emptyset, \{x\}, \{x,y\}\}$. This defines the so called \textbf{Sierpinski Space}, which we will denote with $\{\circ, \bullet\}$.
				}
				\item{
					The chaotic topology $\{\emptyset, \{x,y\}\}$. We will denote the corresponding space with $\{*, *\}$.
				}
			\end{enumerate}
		}
		\item{
			Given a topological space $(X,\mathcal{O}_X)$ any subset $M \subseteq X$ can be equipped with the \textbf{subspace topology}, which is given by $\mathcal{O}_X \intersection M = \{O \intersection M \mid O \in \mathcal{O}_X\}$. Equivalently, the closed sets are given by $\mathcal{C}_X \intersection M = \{C \intersection M \mid C \in \mathcal{C}_X\}$.
		}
	\end{enumerate}
\end{example}

 

\begin{definition}
	Let $(X, \mathcal{O})$ be a topological space. 

	A subset $\mathcal{B}$ of $\mathcal{P}(X)$ is a \textbf{basis} of the topology $\mathcal{O}$, if any open set can be obtained as an arbitrary union of sets in $\mathcal{B}$.

	A subset $\mathcal{S}$ of $\mathcal{P}(X)$ is a \textbf{subbasis} of the topology $\mathcal{O}$, if any open set can be obtained as an arbitrary union of finite intersections of sets in $\mathcal{S}$.

	On $X$ one can define the \textbf{topology generated by} $\mathcal{S}$ as the smallest topology on $X$, which contains $\mathcal{S}$, or equivalently as the (unique) topology on $X$, which has $\mathcal{S}$ as a subbasis. Concretely
	\begin{equation*}
		\mathcal{O}(\mathcal{S}) := \Intersection \limits_{\substack{\mathcal{S} \subseteq \mathcal{T} \subseteq \mathcal{P}(X)\\\mathcal{T} \text{ topology}}} \mathcal{T} = \{\Union \limits_{i \in I} \Intersection \limits_{j=1}^{n_i} S_{ij} \mid I \text{ arbitrary, } n_i \in \N,\, S_{ij} \in \mathcal{S}\}
	\end{equation*}

	A \textbf{neighborhood basis} of a given point $x$ is a set $\mathcal{B}$ of neighborhoods, such that every neighborhood $N$ of $x$ has a subset $B \subseteq N$ in $\mathcal{B}$.
\end{definition}

\begin{definition}
	Let $(X, \mathcal{O})$ be a topological space and $M$ be a subset.

	The \textbf{interior} of $M$ is given by 
	\begin{equation*}
		\interior M := \Union \limits_{\substack{O \subseteq M\\O \text{ open}}} O = \{x \mid M \text{ is neighborhood of } x\}.
	\end{equation*}

	The \textbf{closure} of $M$ is given by 
	\begin{equation*}
		\closure M := \Intersection \limits_{\substack{M \subseteq A\\A \text{ closed}}} A = \{x \mid \text{every neighborhood } N \text{ of } x \text{ satisfies } M \intersection N \neq \emptyset\}.
	\end{equation*}

	Clearly $\interior M \subseteq \closure M$, so the \textbf{boundary} of $M$ can be defined by $\partial M = \boundary M := \closure M \setminus \interior M$.

	The set $M$ is \textbf{dense in} $X$, if it satisfies $\closure M = X$.
\end{definition}

\begin{lemma}[Properties of Interior and Closure]
	Let $(X,\mathcal{O})$ be a topological space and $M,N$ be subsets of $X$.
	\begin{enumerate}[(i)]
		\item{
			$M$ is open, if and only if $\interior M = M$.\\
			$M$ is closed, if and only if $\closure M = M$
		}
		\item{
			The following identities hold.
			\begin{equation*}
				\begin{array}{ccc}
					\interior (M \intersection N) = \interior M \intersection \interior N&&
					\closure(M \union N) = \closure M \union \closure N\\
					X \setminus \interior M = \closure (X \setminus M)&&
					X \setminus \closure M = \interior (X \setminus M)\\
					\interior M = \interior \closure M = \interior \interior M&&
					\closure M = \closure \interior M = \closure \closure M
				\end{array}
			\end{equation*}
		}
		\item{
			If $M \subseteq N$, then $\interior M \subseteq \interior N$ and $\closure M \subseteq \closure N$.\\
			If $\interior M \subseteq N \subseteq \closure M$, then $\interior M = \interior N$ and $\closure M = \closure N$.
		}
	\end{enumerate}
\end{lemma}
\begin{sketch}
	\begin{enumerate}[(i)]
		\item{
			by definition.
		}
		\item{
			by definition, De'Morgan and the observation that for any open set $O$ the inclusion $O \subseteq \closure M$ implies $O \subseteq M$.
		}
		\item{
			by definition and using the third identities in (ii).
		}
		\vspace{-2em}
	\end{enumerate}
\end{sketch}

\begin{definition}
	Let $(X,\mathcal{O}_X)$ and $(Y,\mathcal{O}_Y)$ be topological spaces. A function $f:X-->Y$ is said to be \textbf{continuous}, if it satisfies one of the following equivalent conditions.
	\begin{enumerate}[(i)]
		\item{
			Preimages of open sets are open.
		}
		\item{
			Preimages of closed sets are open.
		}
		\item{
			For every point $x$ in $X$ $f$ is \textbf{continuous at} $x$, i.e. for any neighborhood $N$ of $f(x)$ in $Y$, the preimage $f^{-1}(N)$ is a neighborhood of $x$.
		}
		\item{
			For some (and thus all) basis $\mathcal{B}$ of the topology $\mathcal{O}_Y$ preimages of elements of $\mathcal{B}$ are open.
		}
		\item{
			For some (and thus all) subbasis $\mathcal{S}$ of the topology $\mathcal{O}_Y$ preimages of elements of $\mathcal{S}$ are open.
		}
	\end{enumerate}
	Clearly continouity is preserved by composition and the identity functions are continuous. Thus we get a category $\ncat{Top}$ of topological spaces and continuous functions.
\end{definition}

\begin{sketch}
	(i) $<==>$ (ii), since preimages commute with relative complements\\
	(i) $<==>$ (iii), by definition of neighborhood and the observation that an open set is a neighborhood of all of its points\\
	(i) $==>$ (iv) $==>$ (v), because a basis is in particular a subbasis\\
	(v) $==>$ (i), since preimages commute with arbitrary unions and arbitrary intersections
\end{sketch}

In the following we will commonly refer to \textit{continuous functions} synonymously as \textit{continuous maps} and \textit{morphisms (of topological spaces)}.

\begin{definition}
	Let $f:(X, \mathcal{O}_X) --> (Y, \mathcal{O}_Y)$ be a morphism in $\ncat{Top}$. It is said to be
	\begin{enumerate}[$\bullet$]
		\item{
			\textbf{open}, if the image of any open set in $X$ is open in $Y$.
		}
		\item{
			\textbf{closed}, if the image of any closed set in $X$ is closed in $Y$.
		}
		\item{
			a \textbf{homeomorphism}, if it satisfies one of the following equivalent conditions (using the fact that preimages under $f$ are images under $f^{-1}$).
			\begin{enumerate}[(i)]
				\item{
					$f$ is an isomorphism in $\ncat{Top}$.
				}
				\item{
					$f$ is a (continuous) bijection and open.
				}
				\item{
					$f$ is a (continuous) bijection and closed.
				}
			\end{enumerate}
		}
	\end{enumerate}
\end{definition}

\TODO{open does not imply closed and vice versa}

\begin{lemma}[Properties of $\ncat{Top}$]
	In $\ncat{Top}$ the following holds.
	\begin{enumerate}[(i)]
		\item{
			The forgetful functor $\fun{U}:\ncat{Top} --> \ncat{Set}$ has left adjoint / right adjoint functors
			\begin{equation*}
				\begin{array}{rcl}
					\ncat{Set} & \xrightarrow{(-)_\text{disc}} & \ncat{Top}\\
					X & \longmapsto & (X,\mathcal{O}_\text{disc})\\
					(f:X \rightarrow Y) & \longmapsto & (f:X \rightarrow Y)
				\end{array}
				\hspace{.5cm}\text{resp.}\hspace{.5cm}
				\begin{array}{rcl}
					\ncat{Set} & \xrightarrow{(-)_\text{chaot}} & \ncat{Top}\\
					X & \longmapsto & (X,\mathcal{O}_\text{chaot})\\
					(f:X \rightarrow Y) & \longmapsto & (f:X \rightarrow Y)
				\end{array}.
			\end{equation*}
			% \begin{equation*}
			% 	\begin{diagram}
			% 		\twobyone[wide]{\ncat{Set}}{\ncat{Top}}
			% 		\arrow[out=50,in=130]{w}{e}{(-)_\text{disc}}[above]
			% 		\arrow{e}{w}{\fun{U}}[ontop]
			% 		\arrow[out=-50,in=-130]{w}{e}{(-)_\text{chaot}}[below]
			% 		\node at (0,.4) {$\bot$};
			% 		\node at (0,-.35) {$\bot$};
			% 	\end{diagram}
			% \end{equation*}
		}
		\item{
			$\ncat{Top}$ is complete and cocomplete.
		}
		\item{
			A morphism of topological spaces $f:(X,\mathcal{O}_X)-->(Y,\mathcal{O}_Y)$ is mono/epi if and only if it is injective/surjective.
		}
	\end{enumerate}
\end{lemma}
\begin{sketch}
	\begin{enumerate}[(i)]
		\item{
			The described functors satisfy the universal properties
			\begin{equation*}
				\begin{diagram}
					\twobytwo[verywide]
					{X}{}
					{\fun{U}(X_\text{disc})}{\fun{U}(T)}
					\arrow[equals]{nw}{sw}{}
					\arrow{nw}{se}{f}[above right]
					\arrow[dashed]{sw}{se}{\fun{U}(f)}[below]
				\end{diagram}
				\hspace{.5cm}\text{resp.}\hspace{.5cm}
				\begin{diagram}
					\twobytwo[verywide]
					{\fun{U}(T)}{\fun{U}(X_\text{chaot})}
					{}{X}
					\arrow{nw}{se}{f}[below left]
					\arrow[equals]{ne}{se}{}
					\arrow[dashed]{nw}{ne}{\fun{U}(f)}[above]
				\end{diagram}
			\end{equation*}
		}
		\item{
			From (i) we know that, if a limit / colimit in $\ncat{Top}$ exists, its underlying set has to be the corresponding limit / colimit in $\ncat{Set}$. It thus suffices to give the corresponding topologies and to verify that the functions of the underlying universal properties are continuous.

			For a limit given by projection functions $\pi_i:\lim_{i \in I} X_i --> X_i$ use the topology generated by all the preimages $\pi_i^{-1}(U_i)$ with $U_i \in \mathcal{O}_{X_i}$. This is the coarsest topology on $\lim_{i \in I} X_i$ making all projections continuous.

			For a colimit given by inclusion functions $\iota_i: X_i --> \colim_{i \in I} X_i$ use the topology generated by all the sets \TODO{$\iota_i(V_i)$}, where $V_i = \iota_i^{-1}(M) \in \mathcal{O}_{X_i}$ for some $M \subseteq \colim_{i \in I} X_i$. This is the finest topology on $\colim_{i \in I} X_i$, which makes all the inclusions continuous.
		}
	\end{enumerate}
\end{sketch}

\begin{example}\vspace{-1.5em}
	\begin{enumerate}[(i)]
		\item{
			The terminal object in $\ncat{Top}$ is the \textbf{singleton space} $\{*\}$, which is commonly written as $*$. The initial object in $\ncat{Top}$ is the \textbf{empty space} $\emptyset$.
		}
		\item{
			The topology on the product $\Product \limits_{i\in I} X_i$ of topological spaces $(X_i, \mathcal{O}_{X_i})$ is generated by the sets $U_i \times \Product \limits_{j \neq i} X_j$ for open sets $U_i \subseteq X_i$. In particular the sets $U_{i_1} \times \dots \times U_{i_n} \times \Product \limits_{j \notin \{i_1, \dots, i_n\}} X_j$ form a basis of the topology. \TODO{pri(U) = Xi for all but finitely many, contains and contained in cube}
		}
		\item{
			The coproduct / disjoint union $\Coproduct \limits_{i \in I} X_i$ of topological spaces $(X_i,\mathcal{O}_{X_i})$ has the open sets $U_i$ of $X_i$ as basis.
		}
	\end{enumerate}
\end{example}

\begin{definition}
	A morphism $f:(X,\mathcal{O}_X)-->(Y,\mathcal{O}_Y)$ of topological spaces is an \textbf{embedding of a subspace} if it satisfies one of the following equivalent conditions.
	\begin{enumerate}[(i)]
		\item{
			$X$ is homeomorphic to its image $f(X)$, when the latter is equipped with the subspace topology. In particular $f$ is injective.
		}
		\item{
			$f$ is a \textit{regular monomorphism}, i.e. an equalizer.
		}
		\item{
			\TODO{pullback}
		}
		\item{
			For any topological space $(W,\mathcal{O}_W)$ and function $k:W-->X$ it holds that $k$ is continuous if and only if $fk$ is continuous.
		}
	\end{enumerate}
\end{definition}

\begin{definition}
	A morphism $f:(X,\mathcal{O}_X)-->(Y,\mathcal{O}_Y)$ of topological spaces is a \textbf{quotient map} if it satisfies one of the following equivalent conditions.
	\begin{enumerate}[(i)]
		% \item{
		% 	$X$ is homeomorphic to its image $f(X)$, when the latter is equipped with the subspace topology. In particular $f$ is injective.
		% }
		\item{
			$f$ is surjective and a subset $V$ in $Y$ is open if and only if $f^{-1}(V)$ is open in $X$.
		}
		\item{
			$f$ is a \textit{regular epimorphism}, i.e. a coequalizer. Concretely, $Y$ is homeomorphic to the \textbf{quotient} $X/R$ of $X$ by some equivalence relation $R \subseteq X \times X$, equipped with the subspace topology, defined via
			\begin{equation*}
				X/R := \coeq(\begin{diagram}
					\twobyone{R}{X}
					\arrow[higher]{w}{e}{\pi_1}[above]
					\arrow[lower]{w}{e}{\pi_2}[below]
				\end{diagram}).
			\end{equation*}
		}
		\item{
			\TODO{pushout?}
		}
		\item{
			For any topological space $(Z,\mathcal{O}_Z)$ and function $g:Y-->Z$ it holds that $g$ is continuous if and only if $gf$ is continuous.
		}
	\end{enumerate}
\end{definition}

\begin{remark}
	It is important to note that quotient maps are in general \underline{not} open maps. \TODO{counterexample}
\end{remark}

\begin{definition}
	\TODO{gluing datum} gluing construction
\end{definition}

\TODO{examples: Pullbacks, Pushouts}

\subsection{Separation Axioms}

\begin{definition}
	Two points in a topological space are \textbf{topologically distinguishable} if they have distinct neighborhood filters, or, more concise, if there is an open/closed set, which contains precisely one of both points. A topological space, is \textbf{Kolmogorov} or \textbf{T0}, if any two distinct points are topologically distinguishable.
\end{definition}

\begin{lemma}[Properties of Kolmogorov spaces]\vspace{-1.5em}
	\begin{enumerate}[(i)]
		\item{
			Subspaces of Kolmogorov spaces and arbitrary products of Kolmogorov spaces are Kolmogorov. In particular the category $\ncat{Kol}$ of Kolmogorov spaces and continuous maps is \textit{complete} and the \textit{full embedding} $\ncat{Kol} \longrightincl \ncat{Top}$ is \textit{continuous} (preserves limits).
		}
		\item{
			The Sierpinski space $\{\circ, \bullet\}$ is a \textit{coseparating object}, i.e. for any pair of continuous functions $f_1,f_2:(X,\mathcal{O}_X) --> (Y,\mathcal{O}_Y)$ between Kolmogorov spaces, it holds that $f_1 \neq f_2$, if and only if there is a morphism $s:(Y,\mathcal{O}_Y) --> \{\circ,\bullet\}$ satisfying $sf_1 \neq sf_2$.
		}
		\item{
			$\ncat{Kol}$ is a \textit{(full) reflective subcategory} of $\ncat{Top}$. The \textit{reflector} $\ncat{Top} --> \ncat{Kol}$ sends a topological space $(X,\mathcal{O}_X)$ to its \textbf{Kolmogorov quotient}.
		}
	\end{enumerate}
\end{lemma}

\begin{definition}
	A topological space $(X, \mathcal{O})$ is \textbf{Frechét} or \textbf{T1}, if it satisfies one of the following equivalent conditions.
	\begin{enumerate}[(i)]
		\item{
			For any two points $x_1,x_2$ there are (open) neighborhoods $U_1$ and $U_2$ of $x_1,x_2$ respectively, such that $x_1 \notin U_2$ and $x_2 \notin U_1$.
		}
		\item{
			Every singleton set $\{x\}$ is closed in $X$.
		}
	\end{enumerate}
\end{definition}

\begin{lemma}[Properties of Frechét spaces]\vspace{-1.5em}
	\begin{enumerate}[(i)]
		\item{
			Subspaces of Frechét spaces and arbitrary products of Frechét spaces are Frechét. In particular the category $\ncat{Fre}$ of Frechét spaces and continuous maps is \textit{complete} and the \textit{full embedding} $\ncat{Fre} \longrightincl \ncat{Top}$ is \textit{continuous} (preserves limits).
		}
		\item{
			\TODO{coseperating set / object? see https://mathoverflow.net/a/300982}
		}
		\item{
			$\ncat{Fre}$ is a \textit{reflective subcategory} of $\ncat{Top}$.
		}
		\item{
			Any finite Frechét space is discrete.
		}
	\end{enumerate}
\end{lemma}
\begin{sketch}
	\begin{enumerate}[(i)]
		\item{
			By definition.
		}
		\item{
			\TODO{wtf}
		}
		\item{
			\TODO{todo}
		}
		\item{
			Note that the complement of any one element set can be obtained as finite union of closed sets.
		}
	\end{enumerate}
\end{sketch}

\begin{definition}
	A topological space $(X, \mathcal{O})$ is \textbf{Hausdorff} or \textbf{T2}, if it satisfies one of the following equivalent conditions.
	\begin{enumerate}[(i)]
		\item{
			For any two points $x_1,x_2$ there are (open) neighborhoods $U_1$ and $U_2$ of $x_1,x_2$ respectively, such that $U_1$ and $U_2$ are disjoint.
		}
		\item{
			The diagonal $\Delta_X \subseteq X \times X$ is closed.
		}
	\end{enumerate}
\end{definition}
\begin{sketch}
	Denote $U = X \times X \setminus \Delta_X$.

	(i) $==>$ (ii), since $U = \Union \limits_{x_1 \neq x_2} U_1 \times U_2$ is open.\\
	(ii) $==>$ (i), because for $x_1 \neq x_2$ we have $x_1 \in U_1 := \text{pr}_1^{-1}(U)$, $x_2 \in U_2 := \text{pr}_2^{-1}(U)$ and $U_1 \intersection U_2 = \emptyset$.
\end{sketch}

\begin{lemma}[Properties of Hausdorff Spaces]\vspace{-1.5em}
	\begin{enumerate}[(i)]
		\item{
			Subspaces of Hausdorff spaces and arbitrary products of Hausdorff spaces are Hausdorff. In particular the category $\ncat{Haus}$ if Hausdorff spaces and continuous maps is complete and the full embedding $\ncat{Haus} \longrightincl \ncat{Top}$ is \textit{continuous} (preserves limits).
		}
		\item{
			\TODO{coseperating set / object? see https://mathoverflow.net/a/300982}
		}
		\item{
			$\ncat{Haus}$ is a \textit{(full) reflective subcategory} of $\ncat{Top}$. The \textit{reflector} $\ncat{Haus} --> \ncat{Top}$ sends a topological space $(X,\mathcal{O}_X)$ to its \textbf{Hausdorff quotient} or \textbf{Hausdorffication}.
		}
		\item{
			\TODO{epimorphisms, see https://arxiv.org/pdf/1810.00778.pdf}
		}
	\end{enumerate}
\end{lemma}
\begin{sketch}
	\begin{enumerate}[(i)]
		\item{
			The assertion about subspaces is by definition. For Products note that for $x \neq y$ there is some index $i$ with $x_i \neq y_i$ and use Hausdorffness of the corresponding space.
		}
		\item{
			\TODO{whut}
		}
		\item{
			Let $(X,\mathcal{O}_X)$ be a topological space. Denote by $R$ the smallest equivalence relation on $X$, such that $X/R$ is Hausdorff, i.e. 
			$$R = \Intersection \{S \subseteq X \times X \mid S \text{ equivalence relation and } X/S \text{ Hausdorff}\}.$$
			By definition $X/R$ is Hausdorff. 

			Given an Hausdorff space $(Y,\mathcal{O}_Y)$ and continuos function $f:X-->Y$ consider the equivalence relation $S$ given by $(x_1,x_2) \in S <==> f(x_1) = f(x_2)$. Note that by construction $X/S$ is Hausdorff. Hence $R \subseteq S$ and we get the diagram
			\begin{equation*}
				\begin{diagram}
					\threebythree
						{R}{X}{X/R}
						{S}{X}{X/S}
						{}{Y}{}

					\arrow[higher]{nw}{n}{}
					\arrow[lower]{nw}{n}{}
					\arrow[epi]{n}{ne}{\pi_R}[above]

					\arrow[higher]{w}{c}{}
					\arrow[lower]{w}{c}{}
					\arrow[epi]{c}{e}{\pi_S}[above]

					\arrow[incl]{nw}{w}{}
					\arrow[equals]{n}{c}{}
					\arrow{c}{s}{f}[left]
					\arrow[dashed]{ne}{e}{\widetilde{i}}[right]
					\arrow[dashed]{e}{s}{\widetilde{f}}[below right]
				\end{diagram}
			\end{equation*}
			where $\widetilde{f}$ is induced by the universal property of $X/S$ applied to $f$ and $i$ is induced by the universal property of $X/R$ applied to $\pi_S$. The composition $\widetilde{f}\widetilde{i}$ then gives the desired factorization of $f$, which is unique by $\pi_R$ being epi.
		}
	\end{enumerate}
\end{sketch}

\begin{definition}
	A topological space $(X,\mathcal{O})$ is \textbf{normal} (not necessarily Hausdorff) if for any two disjoint closed subsets $C_0,C_1$ there are disjoint open sets $U_0,U_1$ containing $C_0$ respectively $C_1$.

	A space, which is normal and Frechét (T1), is by definition Hausdorff. In this case we call the space \textbf{normal Hausdorff} or \textbf{T4}.
\end{definition}

\begin{theorem}[Tietze-Urysohn Extension Theorem]
	\begin{minipage}{\textwidth-4cm}
	\vspace{-1em}
		Let $(X,\mathcal{O})$ be a normal space and $C$ be a closed subset. Then any continuous map $f:C-->\R$ has a continuous extension $\widehat{f}:X-->\R$ making the diagram on the right commute.
	\end{minipage}
	\begin{minipage}{4cm}
	\vspace{-1.5em}
		\begin{equation*}
			\begin{diagram}
				\twobytwo
					{X}{\R}
					{C}{}
				\arrow[dashed]{nw}{ne}{\exists \widehat{f}}[above]
				\arrow[incl]{sw}{nw}{}
				\arrow{sw}{ne}{f}[below right]
			\end{diagram}
		\end{equation*}
	\end{minipage}
\end{theorem}
\begin{proof}\TODO{A short proof of the Tietze Urysohn Extension theorem, Mark Mandelkern}

	\TODO{By reparametrization} it suffices to consider the case $f:C --> [0,1]$.

	For any rational $r \in \Q$ we can define the closed set
	\begin{equation*}
		C_r = \{x \in C \mid f(x) \leq r\}.
	\end{equation*}
	The strategy of this proof is to give a countable collection of closed subsets $X_r \subseteq X$ (for $r \in \Q$) with the properties
	\begin{equation}\label{TietzeUrysohnEq1}
		\begin{array}{rl}
			X_r = \emptyset & \text{ for } r < 0\\
			X_r = X & \text{ for } r \geq 1\\
			X_r \subseteq \interior(X_s) & \text{ for }r<s \text{ in } \Q \intersection [0,1)\\
			X_r \intersection C = C_r & \text{ for } r \in [0,1],
		\end{array}
	\end{equation}
	and extend $f$ gradually in the sense that $X_r$ satisfies \TODO{$X_r = \{x \in X \mid \widehat{f}(x) \leq r\}$.}\\

	In the following we will consider pairs $(r,s) \in \Q^2$ with $0 \leq r < s < 1$ and will abreviate them by writing $rs$. To construction of the sets $X_r$ will require us to temporarily fix an indexing $(r_n,s_n)_{n \in \N}$ of the countable set of such pairs.

	For $s \in \Q \intersection (0,1)$ we define the open set
	\begin{equation*}
		U_s = X \setminus \{x \in C \mid f(x) \geq s\}.
	\end{equation*}
	We now inductively construct closed sets $H_{rs}$, which satisfy
	\begin{align*}
		C_r \subseteq \interior(H_{rs}) \subseteq H_{rs} \subseteq U_s\\
		H_{rs} \subseteq \interior(H_{tu}) \text{ for }r<t,s<u.
	\end{align*}

	For the purpose of induction we will temporarily denote the set $H_{rs}$ by $H_n$, given $r=r_n$ and $s=s_n$. The construction requires  some bookkeeping, so we define the following sets of indices
	\begin{align*}
		J_n &= \{j \in \N \mid j < n, r_j < r_n, s_j < s_n\}\\
		K_n &= \{k \in \N \mid k < n, r_k > r_n, s_k > s_n\}.
	\end{align*}

	For $n=1$ set \TODO{TODO}.

	For $n>1$ suppose that for $m<n$ we have constructed closed sets $H_m$, which satisfy
	\begin{align*}
		C_{r_m} \subseteq \interior H_m \subseteq H_m \subseteq U_{s_m}\\
		H_j \subseteq \interior H_m \hspace{0.7em}\text{ for } j\in J_m.
	\end{align*}
	Since $X$ is normal we find a closed set $H_n$, which satisfies
	\begin{equation*}
		C_{r_n} \subseteq C_{r_n} \union \Union \limits_{j \in J_n}H_j \subseteq \interior (H_n) \subseteq H_n \subseteq U_{s_n} \intersection \Intersection \limits_{k \in K_n} \interior (H_k) \subseteq U_{s_n}.
	\end{equation*}

	We can now define for fixed $r \in \Q \intersection [0,1)$
	\begin{equation*}
		X_r = \Intersection \limits_{rs} H_{rs},
	\end{equation*}
	and $X_r = \emptyset$ for $r<0$ and $X_r = X$ for $r \geq 1$. We have to check, that these sets satisfy the requirements (\ref{TietzeUrysohnEq1}). Indeed, given $0 \leq r < t < s < 0$ we have
	\begin{equation*}
		X_r \subseteq H_{rt} \subseteq \interior(H_{ts}) \subseteq H_{ts} \subseteq \Intersection \limits_{su} H_{su} = X_s
	\end{equation*}
	and
	\begin{equation*}
		C_r \subseteq C \intersection X_r = C \intersection \Intersection \limits_{rs} H_{rs} \subseteq C \intersection \Intersection \limits_{rs} U_s = C_r.
	\end{equation*}

	Finally we can define 
	\begin{equation*}
		\widehat{f}:X \longrightarrow [0,1],\; \widehat{f}(x) = \inf \{r \mid x \in X_r\}.
	\end{equation*}
	Since for $x \in C$ it holds that $f(x) = \inf \{r \mid x \in C_r = X_r \intersection C\}$, we have $\widehat{f}\vert_C = f$. It remains to show that $\widehat{f}$ is continous. Indeed, given an open interval $(a,b) \subseteq \R$ we have
	\begin{equation*}
		\widehat{f}^{-1}((a,b)) = \Union \limits_{a<r<s<b}(\interior(X_s) \setminus X_r),
	\end{equation*}
	which is open.
\end{proof}

\begin{corollary}[Urysohn's lemma]
	A topological space $(X,\mathcal{O})$ is normal, if and only if for any two disjoint closed subsets $C_0,C_1$ there exists an \textbf{Urysohn function}, i.e. a continuous function $u:X --> [0,1]$ satisfying $u\vert_{C_0} = 0$ and $u\vert_{C_1} = 1$.
\end{corollary}
\begin{sketch}
	$<==$ Given $C_1,C_2$ and a corresponding Urysohn function $u$, consider the preimages $u^{-1}([0,\frac{1}{3}))$ and $u^{-1}((\frac{2}{3},1])$.

	$==>$ Define $u = \widehat{f}$ to be a Tietze-Urysohn extension of the locally constant map $f:C_0 \sqcup C_1 --> [0,1]$ induced by the constant maps $C_0 --> \{0\}$ and $C_1 --> \{1\}$.
\end{sketch}

\TODO{
	char of normal space via insertion (Katetov) %https://dml.cz/bitstream/handle/10338.dmlcz/119148/CommentatMathUnivCarolRetro_41-2000-1_13.pdf
}

	% A normal space is \textbf{hereditarily normal} or \textbf{completely normal}, if every subspace is normal aswell.

	% A normal space is \textbf{perfectly normal}, if for every pair of disjoint closed sets $C_0, C_1$ there is a continuous map $f:X --> [0,1]$ satisfying $f^{-1}(\{0\}) = C_0$ and $f^{-1}(\{1\}) = C_1$.\\

	%  Hence we call $(X,\mathcal{O})$
	% \begin{enumerate}[$\bullet$]
	% 	\item{
	% 		\textbf{normal Hausdorff} or \textbf{T4}
	% 	}
	% 	\item{
	% 		\textbf{hereditarily normal Hausdorff}, \textbf{completely normal Hausdorff} or \textbf{T5}
	% 	}
	% 	\item{
	% 		\textbf{perfectly normal Hausdorff} or \textbf{T6}
	% 	}
	% \end{enumerate}
	% if it is Frechét (T1) and normal / hereditarily normal / perfectly normal.

\subsection{Connectivity}

\begin{definition}
	A topological space $(X, \mathcal{O})$ is \textbf{connected}, if it satisfies one of the following equivalent conditions.
	\begin{enumerate}[(i)]
		\item{
			For any decomposition $U_1 \union U_2$ of $X$ into open,  disjoint subsets either $U_1$ or $U_2$ are empty.
		}
		\item{
			$\emptyset$ and $X$ are the only subsets of $X$, which are both open and closed.
		}
		\item{
			Any continuous map into a discrete space is constant.
		}
	\end{enumerate}
\end{definition}

\begin{lemma}[Properties of Connectivity]
	\TODO{todo}
\end{lemma}

\begin{definition}
	Let $(X, \mathcal{O})$ be a topological space. The \textbf{connected component} of $x$ is the maximal connected neighborhood of $x$ i.e. given by the identity
	\begin{equation*}
		[x]_{\pi_0} = \Union \limits_{\substack{x \in N \subseteq X\\N\text{ connected}}} N.
	\end{equation*}

	$X$ is \textbf{totally disconnected}, if for any $x$ in $X$ the identity $[x]_{\pi_0} = \{x\}$ holds.
\end{definition}

\begin{remark}
	Connected components need \underline{not} be both open and closed. \TODO{counterexample}
\end{remark}

\begin{lemma}
	\TODO{Toptd reflective subcat of Top}
\end{lemma}

\begin{definition}
	A topological space $(X, \mathcal{O})$ is \textbf{path-connected}, if for every two points $x_0, x_1$ in $X$ there is a continuous function $\gamma: [0,1] --> X$ satisfying $\gamma(0) = x_0$ and $\gamma(1) = x_1$.
\end{definition}

\subsection{Compactness}

\begin{definition}
	Let $(X,\mathcal{O})$ be a topological space and $Y$ be a subset. A family $(U_i)_i$ of open subsets of $X$ is an \textbf{open cover} of $Y$, if $Y \subseteq \smash{\Union \limits_{i \in I} U_i}$. A \textbf{subcover} of $(U_i)_i$ is a subfamily $(V_j)_j$ of $(U_i)_i$, which itself is a cover.
\end{definition}

\begin{definition}
	A topological space $(X,\mathcal{O})$ is \textbf{quasicompact}, if it satisfies one of the following equivalent conditions.
	\begin{enumerate}[(i)]
		\item{
			Every open cover has a finite subcover.
		}
		\item{
			Every family of closed subsets, where any finite subfamily has nonempty intersection, has nonempty intersection.
		}
		\item{
			\TODO{every net has convergent subnet, every filter has convergent refinement, every ultrafilter converges to at least one point}
		}
	\end{enumerate}
\end{definition}

\begin{theorem}[Alexander's Subbase Theorem]
	A topological space $(X,\mathcal{O})$ is quasicompact, if and only if for some (and thus all) subbasis $\mathcal{S}$ any cover of $X$, which consists of elements of $\mathcal{S}$, has a finite subcover.
\end{theorem}

\begin{sketch}
	\enquote{$==>$} is obvious. For \enquote{$<==$} use contradiction.\\
	\textit{Assume} $(X,\mathcal{O})$ is \underline{not} quasicompact.
	\begin{tab}[1.3cm]
		Use \textit{Zorn's lemma} on the set $\mathcal{U} = \{(U_i)_i \mid (U_i)_i \text{ covers } X \text{ but no finite subcover}\}$, which is nonempty and partially ordered wrt. subcovers. Obtain maximal element $\mathcal{U}_0$. \TODO{finish}
	\end{tab}
\end{sketch}

\begin{corollary}[Tychonoff's Theorem]
	Let $(X_i, \mathcal{O}_i)_{i\in I}$ be a family of quasicompact spaces. Then $\Product \limits_{i \in I} X_i$ is quasicompact.
\end{corollary}

\TODO{one point compactification}

\subsection{Size}
\TODO{first countable, second countable, separable, combinatorial dimension}

\subsection{Topological Groups}

\begin{definition}
	A \textbf{topological group} is a group object in $\ncat{Top}$. Concretely it is a group $G$, equipped with a topology $\mathcal{O}_G$, which satisfies one of the following equivalent conditions:
	\begin{enumerate}[(i)]
		\item{
			The multiplication map $m:G \times G --> G, (g,h) \mapsto gh$ and inverting map $i: G --> G, g \mapsto g^{-1}$ are continuous.
		}
		\item{
			The shear map $s: G \times G --> G \times G, (g,h) \mapsto (gh, h)$ are continuous.
		}
	\end{enumerate}
\end{definition}

\subsection{Local Properties}
\subsection{Topological Manifolds}
\subsection{Covering Spaces}
\subsection{Homotopy}

\begin{definition}
	A \textbf{homotopy} between continuous maps $f_0,f_1: (X, \mathcal{O}_X) --> (Y,\mathcal{O}_Y)$ is a continuous map $h:[0,1]\times X --> Y$, satisfying that $f_0 = h|_{\{0\}\times X}$ and $f_1 = h|_{\{1\}\times X}$. In this case $f_0$ and $f_1$ are called \textbf{homotopic}, which is denoted by $f_0 \underset{h}{\sim} f_1$ or simply $f_0 \sim f_1$.

	For arbitrary objects $X,Y$ in $\ncat{Top}$, homotopy defines an equivalence relation on $\ncat{Top}(X,Y)$ as
	\begin{equation*}
		\begin{array}{rcl}
			f_0 \sim f_0 & \text{via} & (t,x) \mapsto f(x)\vspace{.5em}\\
			f_0 \underset{h}{\sim} f_1 \implies f_1 \sim f_0 & \text{via} &(t,x) \mapsto h(1-t,x)\vspace{-.25em}\\
			f_0 \underset{h}{\sim} f_1 \underset{h'}{\sim} f_2 \implies f_0 \sim f_2 & \text{via} &(t,x) \mapsto \left\{\begin{array}{cl} 
				h(2t,x) & t \in [0,\frac{1}{2}]\\
				h'(2t-1,x) & t \in [\frac{1}{2},1]
			\end{array}\right.
		\end{array}
	\end{equation*}
	Moreover this equivalence relation is compatible with composition in the sense that, given maps $s:W-->X$, $f_0,f_1: X --> Y$ and $t:Y-->Z$, any homotopy $f_0 \underset{h}{\sim} f_1$ extends to a homotopy $tf_0s \sim tf_1s$, by taking the composite
	\begin{equation*}
		\begin{diagram}
			\fourbyone[verywide]
				{[0,1]\times W}{[0,1] \times X}{Y}{Z}
			\arrow{ww}{w}{[0,1]\times s}[above]
			\arrow{w}{e}{h}[above]
			\arrow{e}{ee}{t}[above]
		\end{diagram}.
	\end{equation*}
	Hence, by considering morphisms of topological spaces just up to homotopy, one obtains the \textbf{homotopy category} $\ncat{hTop}$, which comes with the canonical functor
	\begin{equation*}
		\begin{array}{rcl}
			\ncat{Top} & \longrightarrow & \ncat{hTop}\\
			(X,\mathcal{O}_X) & \mapsto & (X,\mathcal{O}_X)\\
			f & \mapsto & [f]_\sim
		\end{array}.
	\end{equation*}
\end{definition}

\begin{definition}
	A continuous map $f:X-->Y$ is a \textbf{homotopy equivalence}, if $[f]_\sim: X --> Y$ is an isomorphism in $\ncat{hTop}$. Explicitly this means that there exists a morphism $g:Y --> X$ in $\ncat{Top}$ such that $gf \sim 1_X$ and $fg \sim 1_Y$. Topological spaces $X$ and $Y$ are said to be \textbf{homotopy equivalent}, which will be denoted by $X \simeq Y$, if there is a homotopy equivalence between them.

	A topological space $X$ is \textbf{contractible}, if it is homotopy equivalent to the singleton space $\{*\}$. In particular $X$ is a nonempty space.

	A subspace $M \subseteq X$ of a topological space $X$ is a \textbf{weak}/\textbf{strong deformation retract} of $X$, if the inclusion $m:M \longrightincl X$ has a retract $r:X --> M$ (i.e. satisfying $rm = 1_M$) such that $mr \smash{\underset{h}{\sim}} 1_X$ with $h(t,m) \in M$ / $h(t,m) = m$ for all $m$ in $M$.
\end{definition}

\begin{example}\vspace{-2em}
	\begin{enumerate}
		\item{
			$\R^n$, $\D^n$, $\B^n$, $[0,1]^n$ are contractible spaces
		}
		\item{
			$\S^n$ is \underline{not} contractible.
		}
	\end{enumerate}
\end{example}

\begin{definition}
	A functor $\fun{F}:\ncat{Top} --> \cat{C}$ into some category $\cat{C}$ is said to be \textbf{homotopy invariant}, if it satisfies one of the following equivalent conditions.
	\begin{enumerate}[(i)]
		\item{
			$F$ factors through $\ncat{hTop}$.
		}
		\item{
			For any pair of continuous maps $f_0,f_1$ in $\ncat{Top}$ it holds that $f_0 \sim f_1$ implies $Ff_0 = Ff_1$.
		}
		\item{
			For any topological space $X$ he projection $\pi_X:[0,1]\times X --> X$ induces an isomorphism $\fun{F}\pi_X: \fun{F}([0,1]\times X) \xrightarrow{\isom} \fun{F}X$
		}
	\end{enumerate}
\end{definition}

\newpage
\section{Homology}
\subsection{Generalized Homology}

\begin{definition}
	Let $\ncat{Top}^\rightincl$ denote the category, whose objects are pairs $(X,A)$, where $X$ is a topological space and $A$ is a subspace of $X$, and whose morphisms $f:(X,A) --> (Y,B)$ are given by continuous maps $f:X-->Y$ with $f(A) \subseteq B$.

	There is a canonical restriction functor
	\begin{equation*}
		\begin{array}{rcl}
			\ncat{Top}^\rightincl & \overset{R}{\longrightarrow} & \ncat{Top}^\rightincl\\
			(X,A) & \longmapsto & (A,\emptyset)\\
			(f:(X,A)\rightarrow(Y,B)) & \longmapsto & (f|_A:(A,\emptyset)\rightarrow(B,\emptyset))
		\end{array}
	\end{equation*}
\end{definition}

\TODO{Arr(Top) is complete and cocomplete. Topincl is reflective subcat. homotopy}

\begin{remark}
	Let $\cat{A}$ be an abelian category and regard $\Z$ as a discrete category. Recall that the category $\cat{A}^\Z \isom |[\Z,\cat{A}|]$ has all limits and colimits which $\cat{A}$ has and is again an abelian category. In particular there is a notion of complex and exact sequences: A sequence $(A_n)_n --> (B_n)_n --> (C_n)_n$ is complex/exact in $(B_n)_n$, if and only if for all $n \in \Z$ the sequence $A_n --> B_n --> C_n$ is complex/exact in $B_n$.
	
	On $\cat{A}^\Z$ there are for any $k \in \Z$ canonical shifting functors
	\begin{equation*}
		\begin{array}{rcl}
			\cat{A}^\Z & \overset{S_{k}}{\longrightarrow} & \cat{A}^\Z\\
			(A_n)_n & \longmapsto & (A_{n+k})_n\\
			(f_n)_n & \longmapsto & (f_{n+k})_n
		\end{array}
	\end{equation*}
\end{remark}

\begin{definition}
	A (\textbf{generalized, unreduced}) \textbf{homology theory} on $\ncat{Top}^\rightincl$ is a functor $H: \ncat{Top}^\rightincl --> \cat{A}^\Z$ together with a natural transformation $\partial: H ==> S_{-1} \circ H \circ R$, the so called \textbf{boundary operator} or \textbf{connecting homomorphism}, which satisfy the following axioms.
	\begin{enumerate}[$\bullet$]
		\item{
			\textit{homotopy invariance}
			\TODO{how to make precise}
		}
		\item{
			\textit{exactness}\\
			For any object $(X,A)$ in $\ncat{Top}^\rightincl$ the sequence
			\begin{equation*}
				\begin{diagram}
					\fivebyone[verywide]
						{...\;H_{n+1}(X,A)}
						{H_n(A,\emptyset)}
						{H_n(X,\emptyset)}
						{H_n(X,A)}
						{H_{n-1}(A,\emptyset)\;...}

					\arrow{ww}{w}{\partial_{n+1}}[above]
					\arrow{w}{c}{H_n(i)}[above]
					\arrow{c}{e}{H_n(j)}[above]
					\arrow{e}{ee}{\partial_n}[above]
				\end{diagram}
			\end{equation*}
				is exact, where $i: (A,\emptyset) \longrightincl (X,\emptyset)$ and $j: (X,\emptyset) \longrightincl (X,A)$ are the inclusion maps.

		}
		\item{
			\textit{excission}\\
			For every object $(X,A)$ in $\ncat{Top}^\rightincl$ and subset $U \subseteq \closure U \subseteq \interior A$, the inclusion $(X\setminus U, A \setminus U) \subseteq (X, A)$ induces an isomorphism $$H(X\setminus U, A \setminus U) \isom H(X,A).$$
		}
	\end{enumerate}
	One might further assume
	\begin{enumerate}[$\bullet$]
		\item{
			\textit{additivity / wedge axiom}\\
			$H$ preserves coproducts. Specifically, for any family $(X_i,A_i)_i$ in $\ncat{Top}^\rightincl$ the canonical morphism
			\begin{equation*}
				\bigoplus \limits_{i \in I}H(X_i,A_i) \longrightarrow H(\Coproduct \limits_{i \in I} (X_i,Y_i))
			\end{equation*}
			is an isomorphism.
		}
		\item{
			\textit{ordinarity / dimension}\\
			For all $n \neq 0$ it holds that $H_n(*,\emptyset) = 0$. 
		}
	\end{enumerate}
\end{definition}

\begin{example}
	\TODO{zero functor, other example?, simplicial homology?}
\end{example}

\begin{proposition}[Mayer Vitoris]
	Let $(H,\partial)$ be a homology theory. Let $X$ be a topological space with open cover $U,V$. \TODO{finish}
\end{proposition}

\begin{definition}
	Let $(H,\partial)$ be a homology theory. The \textbf{reduced homology} is the homology theory obtained by taking $\tilde{H}(X,\emptyset) = \ker(H(X,\emptyset) --> H(*,\emptyset))$ and $\tilde{H}(X,A) = H(X,A)$ otherwise.
\end{definition}



\end{document}






