\documentclass{article}

% --- text layout
\usepackage[parfill]{parskip}
\usepackage{geometry}
\geometry{a4paper, margin=3cm}
\usepackage{romanbar}
\usepackage{placeins} %FloatBarrier
\usepackage[utf8]{inputenc}

\usepackage{xcolor}
\definecolor{darkred}{HTML}{990000}
\definecolor{darkgreen}{HTML}{009900}

\usepackage{sectsty}
\sectionfont{\color{darkred}}
\subsectionfont{\color{darkred}}
\subsubsectionfont{\color{darkred}}
% \usepackage{algpseudocode} --> better pseudo code?
%\usepackage[pagewise]{lineno} % row numbers

% --- syncing, referencing, citing
\usepackage{pdfsync} 			% synchronization LaTeX <--> PDF
\usepackage[hidelinks]{hyperref} % hidelinks avoids red boxes around links
\usepackage{csquotes}			% enquote
\usepackage{cleveref}
\usepackage{pdfpages}

\usepackage[backend=bibtex,style=alphabetic,url=false]{biblatex}
\addbibresource{sources.bib}
% define bibliography filter like
%\defbibfilter{computability-pca}{
%	keyword=computability or keyword=pca
%}

% --- lists and tables
\usepackage[shortlabels]{enumitem}
\usepackage{tabularx}
\usepackage{tabu}
\usepackage{makecell}

% --- graphics
\usepackage{graphicx}
\usepackage{xcolor}

% --- math
\usepackage{amsmath}
\usepackage[bbgreekl]{mathbbol} % double stroke greek
\usepackage{amssymb}
\usepackage{fixmath} % whyyy? "bug": italic PI and SIGMA
\usepackage{mathtools}
\usepackage{mathrsfs} %mathscr

% --- own packages
\usepackage{../notation}
\usepackage{../diagrams}
\usepackage{../environments}

\title{$\Sum \text{math}(y)$\\Topological Spaces}
\author{Jonas Linssen}

\begin{document}

\maketitle
\tableofcontents

\newpage
\section{Topological Spaces}
\subsection{Topological Spaces}

\begin{definition}
	A \textbf{topological space} $(X,\mathcal{O})$ is a set $X$ together with a family of sets $\mathcal{O} \subseteq \mathcal{P}(X)$ called \textbf{topology}, which satisfies the following axioms. Sets contained in $\mathcal{O}$ are called \textbf{open} in $X$.
	\begin{enumerate}[$\bullet$]
		\item{
			Both $\emptyset$ and $X$ are open.
		}
		\item{
			Finite intersections of open sets are open.
		}
		\item{
			Arbitrary unions of open sets are open.
		}
	\end{enumerate}
	The relative complement $X \setminus O$ of an open set $O$ is called \textbf{closed} in $X$. By de'Morgan one may define a topology as the set of relative complements of a family $\mathcal{C} \subseteq \mathcal{P}(X)$ of closed sets, which satisfies the following axioms.
	\begin{enumerate}[$\bullet$]
		\item{
			Both $\emptyset$ and $X$ are closed.
		}
		\item{
			Arbitrary intersections of closed sets are closed.
		}
		\item{
			Finite unions of closed sets are closed.
		}
	\end{enumerate}

	A subset $N$ in $X$ is called a \textbf{neighborhood} of a given point $x$, if there is a open set $U$, which satisfies $x \in U \subseteq N$. The set of all neighborhoods of $x$ is the \textbf{neighborhood filter} of $x$ and denoted by $\mathcal{N}(x)$.
\end{definition}

Any set can be equipped with two canonical topologies. The \textbf{discrete topology} on $X$ is given by $\mathcal{O}_\text{disc} = \mathcal{P}(X)$. The \textbf{chaotic topology} on $X$ is the topology given by $\mathcal{O}_\text{chaot} = \{\emptyset, X\}$. 

\begin{definition}
	Let $(X, \mathcal{O})$ be a topological space. 

	A subset $\mathcal{B}$ of $\mathcal{P}(X)$ is a \textbf{basis} of the topology $\mathcal{O}$, if any open set can be obtained as an arbitrary union of sets in $\mathcal{B}$.

	A subset $\mathcal{S}$ of $\mathcal{P}(X)$ is a \textbf{subbasis} of the topology $\mathcal{O}$, if any open set can be obtained as an arbitrary union of finite intersections of sets in $\mathcal{S}$.

	On $X$ one can define the \textbf{topology generated by} $\mathcal{S}$ as the smallest topology on $X$, which contains $\mathcal{S}$, or equivalently as the (unique) topology on $X$, which has $\mathcal{S}$ as a subbasis. Concretely
	\begin{equation*}
		\mathcal{O}(\mathcal{S}) := \Intersection \limits_{\substack{\mathcal{S} \subseteq \mathcal{T} \subseteq \mathcal{P}(X)\\\mathcal{T} \text{ topology}}} \mathcal{T} = \{\Union \limits_{i \in I} \Intersection \limits_{j=1}^{n_i} S_{ij} \mid I \text{ arbitrary, } n_i \in \N,\, S_{ij} \in \mathcal{S}\}
	\end{equation*}

	A \textbf{neighborhood basis} of a given point $x$ is a set $\mathcal{B}$ of neighborhoods, such that every neighborhood $N$ of $x$ has a subset $B \subseteq N$ in $\mathcal{B}$.
\end{definition}

\begin{definition}
	Let $(X, \mathcal{O})$ be a topological space and $M$ be a subset.

	The \textbf{interior} of $M$ is given by 
	\begin{equation*}
		\interior M := \Union \limits_{\substack{O \subseteq M\\O \text{ open}}} O = \{x \mid M \text{ is neighborhood of } x\}.
	\end{equation*}

	The \textbf{closure} of $M$ is given by 
	\begin{equation*}
		\closure M := \Intersection \limits_{\substack{M \subseteq A\\A \text{ closed}}} A = \{x \mid \text{every neighborhood } N \text{ of } x \text{ satisfies } M \intersection N \neq \emptyset\}.
	\end{equation*}

	Clearly $\interior M \subseteq \closure M$, so the \textbf{boundary} of $M$ can be defined by $\partial M = \boundary M := \closure M \setminus \interior M$.

	The set $M$ is \textbf{dense in} $X$, if it satisfies $\closure M = X$.
\end{definition}

\begin{lemma}[Properties of Interior and Closure]
	Let $(X,\mathcal{O})$ be a topological space and $M,N$ be subsets of $X$.
	\begin{enumerate}[(i)]
		\item{
			$M$ is open, if and only if $\interior M = M$.\\
			$M$ is closed, if and only if $\closure M = M$
		}
		\item{
			The following identities hold.
			\begin{equation*}
				\begin{array}{ccc}
					\interior (M \intersection N) = \interior M \intersection \interior N&&
					\closure(M \union N) = \closure M \union \closure N\\
					X \setminus \interior M = \closure (X \setminus M)&&
					X \setminus \closure M = \interior (X \setminus M)\\
					\interior M = \interior \closure M = \interior \interior M&&
					\closure M = \closure \interior M = \closure \closure M
				\end{array}
			\end{equation*}
		}
		\item{
			If $M \subseteq N$, then $\interior M \subseteq \interior N$ and $\closure M \subseteq \closure N$.\\
			If $\interior M \subseteq N \subseteq \closure M$, then $\interior M = \interior N$ and $\closure M = \closure N$.
		}
	\end{enumerate}
\end{lemma}
\begin{sketch}
	\begin{enumerate}[(i)]
		\item{
			by definition.
		}
		\item{
			by definition, De'Morgan and the observation that for any open set $O$ the inclusion $O \subseteq \closure M$ implies $O \subseteq M$.
		}
		\item{
			by definition and using the third identities in (ii).
		}
		\vspace{-2em}
	\end{enumerate}
\end{sketch}

\begin{definition}
	Let $(X,\mathcal{O}_X)$ and $(Y,\mathcal{O}_Y)$ be topological spaces. A function $f:X-->Y$ is said to be \textbf{continuous}, if it satisfies one of the following equivalent conditions.
	\begin{enumerate}[(i)]
		\item{
			Preimages of open sets are open.
		}
		\item{
			Preimages of closed sets are open.
		}
		\item{
			For every point $x$ in $X$ $f$ is \textbf{continuous at} $x$, i.e. for any neighborhood $N$ of $f(x)$ in $Y$, the preimage $f^{-1}(N)$ is a neighborhood of $x$.
		}
		\item{
			For some (and thus all) basis $\mathcal{B}$ of the topology $\mathcal{O}_Y$ preimages of elements of $\mathcal{B}$ are open.
		}
		\item{
			For some (and thus all) subbasis $\mathcal{S}$ of the topology $\mathcal{O}_Y$ preimages of elements of $\mathcal{S}$ are open.
		}
	\end{enumerate}
	Clearly continouity is preserved by composition and the identity functions are continuous. Thus we get a category $\ncat{Top}$ of topological spaces and continuous functions.
\end{definition}

\begin{sketch}
	(i) $<==>$ (ii), since preimages commute with relative complements\\
	(i) $<==>$ (iii), by definition of neighborhood and the observation that an open set is a neighborhood of all of its points\\
	(i) $==>$ (iv) $==>$ (v), because a basis is in particular a subbasis\\
	(v) $==>$ (i), since preimages commute with arbitrary unions and arbitrary intersections
\end{sketch}

\TODO{morphism vs cont function}

\begin{definition}
	Let $f:(X, \mathcal{O}_X) --> (Y, \mathcal{O}_Y)$ be a morphism in $\ncat{Top}$. It is said to be
	\begin{enumerate}[$\bullet$]
		\item{
			\textbf{open}, if the image of any open set in $X$ is open in $Y$.
		}
		\item{
			\textbf{closed}, if the image of any closed set in $X$ is closed in $Y$.
		}
		\item{
			a \textbf{homeomorphism}, if it satisfies one of the following equivalent conditions.
			\begin{enumerate}[(i)]
				\item{
					$f$ is an isomorphism in $\ncat{Top}$.
				}
				\item{
					$f$ is a (continuous) bijection and open.
				}
				\item{
					$f$ is a (continuous) bijection and closed.
				}
			\end{enumerate}
		}
	\end{enumerate}
\end{definition}

\begin{lemma}[Properties of $\ncat{Top}$]
	In $\ncat{Top}$ the following holds.
	\begin{enumerate}[(i)]
		\item{
			A morphism of topological spaces $f:(X,\mathcal{O}_X)-->(Y,\mathcal{O}_Y)$ is mono/epi if and only if it is injective/surjective.
		}
		\item{
			The forgetful functor $\fun{U}:\ncat{Top} --> \ncat{Set}$ is left adjoint / right adjoint to the functors
			\begin{equation*}
				\begin{array}{rcl}
					\ncat{Set} & \xrightarrow{(-)_\text{disc}} & \ncat{Top}\\
					X & \longmapsto & X_\text{disc}\\
					(f:X \rightarrow Y) & \longmapsto & (f:X \rightarrow Y)
				\end{array}
				\hspace{.5cm}\text{resp.}\hspace{.5cm}
				\begin{array}{rcl}
					\ncat{Set} & \xrightarrow{(-)_\text{chaot}} & \ncat{Top}\\
					X & \longmapsto & X_\text{chaot}\\
					(f:X \rightarrow Y) & \longmapsto & (f:X \rightarrow Y)
				\end{array}.
			\end{equation*}
			% \begin{equation*}
			% 	\begin{diagram}
			% 		\twobyone[wide]{\ncat{Set}}{\ncat{Top}}
			% 		\arrow[out=50,in=130]{w}{e}{(-)_\text{disc}}[above]
			% 		\arrow{e}{w}{\fun{U}}[ontop]
			% 		\arrow[out=-50,in=-130]{w}{e}{(-)_\text{chaot}}[below]
			% 		\node at (0,.4) {$\bot$};
			% 		\node at (0,-.35) {$\bot$};
			% 	\end{diagram}
			% \end{equation*}
		}
		\item{
			$\ncat{Top}$ is complete and cocomplete.
		}
	\end{enumerate}
\end{lemma}
\TODO{examples: Pullbacks, Equalizers/subspaces, Products, Coproducts, Pushouts, Coequealizers}

\subsection{Separation Axioms, Connectivity}

\begin{definition}
	A topological space $(X, \mathcal{O})$ is \textbf{Frechét} or \textbf{T1}, if it satisfies one of the following equivalent conditions.
	\begin{enumerate}[(i)]
		\item{
			For any two points $x_1,x_2$ there are (open) neighborhoods $U_1$ and $U_2$ of $x_1,x_2$ respectively, such that $x_1 \notin U_2$ and $x_2 \notin U_1$.
		}
		\item{
			Every singleton set $\{x\}$ is closed in $X$.
		}
	\end{enumerate}
\end{definition}

\begin{definition}
	A topological space $(X, \mathcal{O})$ is \textbf{Hausdorff} or \textbf{T2}, if it satisfies one of the following equivalent conditions.
	\begin{enumerate}[(i)]
		\item{
			For any two points $x_1,x_2$ there are (open) neighborhoods $U_1$ and $U_2$ of $x_1,x_2$ respectively, such that $U_1$ and $U_2$ are disjoint.
		}
		\item{
			The diagonal $\Delta_X \subseteq X \times X$ is closed.
		}
	\end{enumerate}
\end{definition}
\begin{sketch}
	Denote $U = X \times X \setminus \Delta_X$.

	(i) $==>$ (ii), since $U = \Union \limits_{x_1 \neq x_2} U_1 \times U_2$ is open.\\
	(ii) $==>$ (i), because for $x_1 \neq x_2$ we have $x_1 \in U_1 := \text{pr}_1^{-1}(U)$, $x_2 \in U_2 := \text{pr}_2^{-1}(U)$ and $U_1 \intersection U_2 = \emptyset$.
\end{sketch}

\begin{definition}
	A topological space $(X, \mathcal{O})$ is \textbf{connected}, if it satisfies one of the following equivalent conditions.
	\begin{enumerate}[(i)]
		\item{
			For any decomposition $U_1 \union U_2$ of $X$ into open,  disjoint subsets either $U_1$ or $U_2$ are empty.
		}
		\item{
			$\emptyset$ and $X$ are the only subsets of $X$, which are both open and closed.
		}
		\item{
			Any continuous map into a discrete space is constant.
		}
	\end{enumerate}
\end{definition}

\begin{lemma}[Properties of Connectivity]
	\TODO{todo}
\end{lemma}

\begin{definition}
	Let $(X, \mathcal{O})$ be a topological space. The \textbf{connected component} of $x$ is the maximal connected neighborhood of $x$ i.e. given by the identity
	\begin{equation*}
		[x]_{\pi_0} = \Union \limits_{\substack{x \in N \subseteq X\\N\text{ connected}}} N.
	\end{equation*}

	$X$ is \textbf{totally disconnected}, if for any $x$ in $X$ the identity $[x]_{\pi_0} = \{x\}$ holds.
\end{definition}

\begin{lemma}
	\TODO{Toptd reflective subcat of Top}
\end{lemma}

\begin{definition}
	A topological space $(X, \mathcal{O})$ is \textbf{path-connected}, if for every two points $x_0, x_1$ in $X$ there is a continuous function $\gamma: \I^1 --> X$ satisfying $\gamma(0) = x_0$ and $\gamma(1) = x_1$.
\end{definition}

\subsection{Compactness}

\begin{definition}
	Let $(X,\mathcal{O})$ be a topological space and $Y$ be a subset. A family $(U_i)_i$ of open subsets of $X$ is an \textbf{open cover} of $Y$, if $Y \subseteq \smash{\Union \limits_{i \in I} U_i}$. A \textbf{subcover} of $(U_i)_i$ is a subfamily $(V_j)_j$ of $(U_i)_i$, which itself is a cover.
\end{definition}

\begin{definition}
	A topological space $(X,\mathcal{O})$ is \textbf{quasicompact}, if it satisfies one of the following equivalent conditions.
	\begin{enumerate}[(i)]
		\item{
			Every open cover has a finite subcover.
		}
		\item{
			Every family of closed subsets, where any finite subfamily has nonempty intersection, has nonempty intersection.
		}
		\item{
			\TODO{every net has convergent subnet, every filter has convergent refinement, every ultrafilter converges to at least one point}
		}
	\end{enumerate}
\end{definition}

\begin{theorem}[Alexander's Subbase Theorem]
	A topological space $(X,\mathcal{O})$ is quasicompact, if and only if for some (and thus all) subbasis $\mathcal{S}$ any cover of $X$, which consists of elements of $\mathcal{S}$, has a finite subcover.
\end{theorem}

\begin{sketch}
	\enquote{$==>$} is obvious. For \enquote{$<==$} use contradiction.\\
	\textit{Assume} $(X,\mathcal{O})$ is \underline{not} quasicompact.
	\begin{tab}[1.3cm]
		Use \textit{Zorn's lemma} on the following nonempty set, which is partially ordered wrt. subcovers $\mathcal{U} = \{(U_i)_i \mid (U_i)_i \text{ covers } X \text{ and has no finite subcover}\}$. Obtain maximal element $\mathcal{U}_0$. \TODO{finish}
	\end{tab}
\end{sketch}

\begin{corollary}[Tychonoff's Theorem]
	Let $(X_i, \mathcal{O}_i)_{i\in I}$ be a family of quasicompact spaces. Then $\Product \limits_{i \in I} X_i$ is quasicompact.
\end{corollary}

\subsection{Size}
\TODO{first countable, second countable, separable, combinatorial dimension}

\subsection{Topological Groups}

\begin{definition}
	A \textbf{topological group} is a group object in $\ncat{Top}$. Concretely it is a group $G$, equipped with a topology $\mathcal{O}_G$, which satisfies one of the following equivalent conditions:
	\begin{enumerate}[(i)]
		\item{
			The multiplication map $m:G \times G --> G, (g,h) \mapsto gh$ and inverting map $inv: G --> G, g \mapsto g^{-1}$ are continuous.
		}
		\item{
			The shear map $s: G \times G --> G \times G, (g,h) \mapsto (gh, h)$ are continuous.
		}
	\end{enumerate}
\end{definition}

\subsection{Local Properties}
\subsection{Topological Manifolds}
\subsection{Covering Spaces}
\subsection{Homotopy}

\newpage
\section{Homology}
\subsection{Generalized Homology}

\begin{definition}
	Let $\ncat{Top}^\rightincl$ denote the category, whose objects are pairs $(X,A)$, where $X$ is a topological space and $A$ is a subspace of $X$, and whose morphisms $f:(X,A) --> (Y,B)$ are given by continuous maps $f:X-->Y$ with $f(A) \subseteq B$.

	There is a canonical restriction functor
	\begin{equation*}
		\begin{array}{rcl}
			\ncat{Top}^\rightincl & \overset{R}{\longrightarrow} & \ncat{Top}^\rightincl\\
			(X,A) & \longmapsto & (A,\emptyset)\\
			(f:(X,A)\rightarrow(Y,B)) & \longmapsto & (f|_A:(A,\emptyset)\rightarrow(B,\emptyset))
		\end{array}
	\end{equation*}
\end{definition}

\begin{remark}
	Let $\cat{A}$ be an abelian category and regard $\Z$ as a discrete category. Recall that the category $\cat{A}^\Z \isom |[\Z,\cat{A}|]$ has all limits and colimits which $\cat{A}$ has and is again an abelian category. In particular there is a notion of complex and exact sequences: A sequence $(A_n)_n --> (B_n)_n --> (C_n)_n$ is complex/exact in $(B_n)_n$, if and only if for all $n \in \Z$ the sequence $A_n --> B_n --> C_n$ is complex/exact in $B_n$.
	
	On $\cat{A}^\Z$ there are for any $k \in \Z$ canonical shifting functors
	\begin{equation*}
		\begin{array}{rcl}
			\cat{A}^\Z & \overset{S_{k}}{\longrightarrow} & \cat{A}^\Z\\
			(A_n)_n & \longmapsto & (A_{n+k})_n\\
			(f_n)_n & \longmapsto & (f_{n+k})_n
		\end{array}
	\end{equation*}
\end{remark}

\begin{definition}
	A (\textbf{generalized, unreduced}) \textbf{homology theory} on $\ncat{Top}^\rightincl$ is a functor $H: \ncat{Top}^\rightincl --> \cat{A}^\Z$ together with a natural transformation $\partial: H ==> S_{-1} \circ H \circ R$, the so called \textbf{boundary operator} or \textbf{connecting homomorphism}, which satisfy the following axioms.
	\begin{enumerate}[$\bullet$]
		\item{
			\textit{homotopy invariance}
			\TODO{how to make precise}
		}
		\item{
			\textit{exactness}\\
			For any object $(X,A)$ in $\ncat{Top}^\rightincl$ the sequence
			\begin{equation*}
				\begin{diagram}
					\fivebyone[verywide]
						{...\;H_{n+1}(X,A)}
						{H_n(A,\emptyset)}
						{H_n(X,\emptyset)}
						{H_n(X,A)}
						{H_{n-1}(A,\emptyset)\;...}

					\arrow{ww}{w}{\partial_{n+1}}[above]
					\arrow{w}{c}{H_n(i)}[above]
					\arrow{c}{e}{H_n(j)}[above]
					\arrow{e}{ee}{\partial_n}[above]
				\end{diagram}
			\end{equation*}
				is exact, where $i: (A,\emptyset) \longrightincl (X,\emptyset)$ and $j: (X,\emptyset) \longrightincl (X,A)$ are the inclusion maps.

		}
		\item{
			\textit{excission}\\
			For every object $(X,A)$ in $\ncat{Top}^\rightincl$ and subset $U \subseteq \closure U \subseteq \interior A$, the inclusion $(X\setminus U, A \setminus U) \subseteq (X, A)$ induces an isomorphism $$H(X\setminus U, A \setminus U) \isom H(X,A).$$
		}
	\end{enumerate}
	One might further assume
	\begin{enumerate}[$\bullet$]
		\item{
			\textit{additivity / wedge axiom}\\
			$H$ preserves coproducts. Specifically, for any family $(X_i,A_i)_i$ in $\ncat{Top}^\rightincl$ the canonical morphism
			\begin{equation*}
				\bigoplus \limits_{i \in I}H(X_i,A_i) \longrightarrow H(\Coproduct \limits_{i \in I} (X_i,Y_i))
			\end{equation*}
			is an isomorphism.
		}
		\item{
			\textit{ordinarity / dimension}\\
			For all $n \neq 0$ it holds that $H_n(*,\emptyset) = 0$. 
		}
	\end{enumerate}
\end{definition}



\end{document}






