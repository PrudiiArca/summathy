\documentclass{article}

% --- text layout
\usepackage[parfill]{parskip}
\usepackage{geometry}
\geometry{a4paper, margin=3cm}
\usepackage{romanbar}
\usepackage{placeins} %FloatBarrier
\usepackage[utf8]{inputenc}

\usepackage{xcolor}
\definecolor{darkred}{HTML}{990000}
\definecolor{darkgreen}{HTML}{009900}

\usepackage{sectsty}
\sectionfont{\color{darkred}}
\subsectionfont{\color{darkred}}
\subsubsectionfont{\color{darkred}}
% \usepackage{algpseudocode} --> better pseudo code?
%\usepackage[pagewise]{lineno} % row numbers

% --- syncing, referencing, citing
\usepackage{pdfsync} 			% synchronization LaTeX <--> PDF
\usepackage[hidelinks]{hyperref} % hidelinks avoids red boxes around links
\usepackage{csquotes}			% enquote
\usepackage{cleveref}
\usepackage{pdfpages}

\usepackage[backend=bibtex,style=alphabetic,url=false]{biblatex}
\addbibresource{sources.bib}
% define bibliography filter like
%\defbibfilter{computability-pca}{
%	keyword=computability or keyword=pca
%}

% --- lists and tables
\usepackage[shortlabels]{enumitem}
\usepackage{tabularx}
\usepackage{tabu}
\usepackage{makecell}

% --- graphics
\usepackage{graphicx}
\usepackage{xcolor}

% --- math
\usepackage{amsmath}
\usepackage{amssymb}
\usepackage{fixmath} % whyyy? "bug": italic PI and SIGMA
\usepackage{mathtools}
\usepackage{mathrsfs} %mathscr

% --- own packages
\usepackage{../notation}
\usepackage{../diagrams}
\usepackage{../environments}

\title{
	$\Sum \text{math}(y)$\\
	Rings
}
\author{Jonas Linßen}

\begin{document}
	\maketitle
	\tableofcontents

	\newpage
	\section{Rings}
	\subsection{Rings}
	\begin{definition}
		A \textbf{(commutative, unital) ring} is a set $R$ together with a \textbf{addition} $+: R \times R --> R$ and a \textbf{multiplication} $\cdot: R \times R --> R$ such that 
		\begin{enumerate}
			\item{
				$(R,+)$ is an \textit{abelian group}, i.e. $+$ is associative and commutative, any element $a$ has a unique additive inverse $-a$ and there is an additive identity element $0$. 
			}
			\item{
				$(R,\cdot)$ is a commutative \textit{monoid}, i.e. $\cdot$ is associative and commutative and there is a multiplicative identity element $1$.
			}
			\item{
				the following \textit{distributive law} holds for any $a,b,c \in R$:
				\begin{align*}
					a \cdot (b+c) &= a \cdot b + a \cdot c
				\end{align*}
			}
		\end{enumerate}

		If $0 = 1$ the ring necessessarily becomes $R = \{0\}$, which is called the \textbf{zero ring} or \textbf{trivial ring}.
	\end{definition}
	%\footnotetext{Some authors drop the requirement of a multiplicative identity and mark its presence by saying unital ring.}
	%\footnotetext{For binary operators like $+$ it is common to use infix notation $a+b$ instead of $+(a,b)$. Moreover we use the convention \enquote{multiplication before addition}.}
	\TODO{examples: Z, End(A), subring}

	\TODO{infix notation, dropping $\cdot$}

	\begin{definition}
		A \textbf{morphism of rings} is a function $f:R --> S$ between rings $R$ and $S$, which satisfies
		\begin{equation*}
			f(ab + c) = f(a)f(b) + f(c) \;\;\;\;\text{and}\;\;\;\; f(1_R) = 1_S.
		\end{equation*}
		As the identity function $1_R: R --> R$ is a morphism of rings for any ring $R$ and morphisms of rings are closed under composition, we get a category $\ncat{CRing}$.
	\end{definition}

	\begin{lemma}[Properties of $\ncat{CRing}$]
		In $\ncat{CRing}$ the following holds.
		\begin{enumerate}[(i)]
			\item{
				A morphism of rings $f:R-->S$ is a morphism of the underlying \textit{abelian groups} $(R,+)$ and $(S,+)$, i.e. there is a forgetful functor $\fun{U}:\ncat{CRing} --> \ncat{Ab}$. In particular it holds that
				\begin{equation*}
					f(0) = 0 \;\;\;\;\text{and}\;\;\;\; f(-a) = -f(a).
				\end{equation*}
			}
			\item{
				A morphism of rings $f:R --> S$ is epi if and only if it is surjective.
			}
			\item{
				A morphism of rings $f:R --> S$ is mono if and only if it is injective. Using (i) this is equivalent to the equation $\ker f := \{r \in R \mid f(r) = 0\} = \{0\}$.
			}
			\item{
				The trivial ring $1$ is terminal and the ring of integers $\Z$ is initial.
			}
			\item{
				$\ncat{CRing}$ has arbitrary products, where the ring structure on the underlying set $\Product \limits_{i \in I} R_i$ is defined componentwise. It has equalizers
				\begin{equation*}
					\eq(
					\begin{diagram}
						\twobyone{R}{S}
						\arrow[higher]{w}{e}{f_1}[above]
						\arrow[lower]{w}{e}{f_1}[below]
					\end{diagram}
					)
				\end{equation*}
				given by the subring $\{r \in R \mid f_1(r) = f_2(r)\}$ and its inclusion. Thus $\ncat{CRing}$ is \textit{complete}.
			}
		\end{enumerate}
	\end{lemma}
	\begin{sketch}
		\begin{enumerate}[(i)]
			\item{
				By definition.
			}
			\item{
				Clearly surjective implies epi. \TODO{converse?}
			}
			\item{
				Injective implies mono. \TODO{converse?}
			}
			\item{
				The first statement holds by definition. For the second note that $n = 1 + ... + 1$ in $\Z$ and use a morphism of rings has to preserve $1$.
			}
			\item{
				 In both cases it suffices to check that the unique morphism into $\Product \limits_{i \in I} R_i$, respectively $\eq(f_1,f_2)$ of the universal property in $\ncat{Set}$ is a morphism of rings.
			}
		\end{enumerate}\vspace{-2em}
	\end{sketch}

	Consider the binary product $R^2$ of a ring $R$ with itself. The two elements $(1,0)$ and $(0,1)$ are clearly nonzero, but satisfy $(1,0)\cdot(0,1) = (0,0)$.

	\begin{definition}
		A \textbf{zero divisor} in a ring $R$ is an element $a$, such that there exists a nonzero element $b$ such that $ab = 0$. A \textbf{nilpotent element} $a$ satisfies that $a^n = 0$ for some $n \in \N$. In particular nilpotent elements are zero divisors.
		An element, which is not a zero divisor, is sometimes called \textbf{regular}. 

		A ring, which does not possess any zero divisors besides $0$, is called an \textbf{integral domain}.
	\end{definition}

	Note that $0$ is a zero divisor unless the ring is the trivial ring.

	\begin{definition}
		A \textbf{unit} in a ring $R$ is an element $a$, for which there exists a \textbf{multiplicative inverse}, i.e. an element $a^{-1}$ satisfying $aa^{-1} = 1 \;{\color{gray}(= a^{-1}a)}$.
		The subset of all units in $R$ will be denoted by $R^\times$. 

		A ring, in which every element except $0$ is a unit, is a \textbf{field}.
	\end{definition}

	\begin{lemma}[Properties of Units and Zero Divisors]
		\vspace{-1.5em}\begin{enumerate}[(i)]
			\item{
				No zero divisor is a unit and vice versa.
			}
			\item{
				Let $f:R --> S$ be a morphism of rings. If $a$ is a zero divisor / unit in $R$, then $f(a)$ is a zero divisor / unit in $S$.
			}
			\item{
				Let $f: K --> R$ be a morphism of rings, where $K$ is a field. Then $f$ is mono.
			}
			\item{
				For any nilpotent element $r$ in some ring $R$, the element $1-r$ is a unit.
			}
		\end{enumerate}
	\end{lemma}
	\begin{sketch}
		\begin{enumerate}[(i)]
			\item{
				Holds for the trivial ring by definition. In a nontrivial ring use contradiction.
			}
			\item{
				By definition.
			}
			\item{
				Follows from (ii) via $\ker f = \{0\}$.
			}
			\item{
				Given $r^n = 0$ the inverse of $1-r$ is given by $\smash{\Sum \limits_{k=0}^{n-1}r^k}$.
			}
		\end{enumerate}\vspace{-2em}
	\end{sketch}

	\newpage
	\subsection{Ideals, Factor Rings and Spectrum}
	\begin{definition}
		Let $R$ be a ring. An \textbf{ideal} $\ideal{a}$ is an \textit{abelian subgroup} $\ideal{a} \leq_\ncat{Ab} R$, which is closed under multiplication with $R$. We write $\ideal{a} \diveq R$.

		The \textbf{ideal generated by} a set of elements $X$ is given by $\Gen{X} = \Intersection \limits_{X \subseteq \ideal{a} \diveq R} \ideal{a}$.
	\end{definition}

	\begin{lemma}[Properties of Ideals]
		\vspace{-1.5em}\begin{enumerate}[(i)]
			\item{
				Let $R$ be a ring and $\ideal{a},\ideal{b} \diveq R$ ideals. Then
				\begin{align*}
					\ideal{ab} &= \Gen{ab \mid a \in \ideal{a}, b \in \ideal{b}}\\
					\ideal{a} \intersection \ideal{b} &= \{r \mid r \in \ideal{a}, r \in \ideal{b}\}\\
					\ideal{a} + \ideal{b} &= \{a + b \mid a \in \ideal{a}, b \in \ideal{b}\}
				\end{align*}
				are ideals and it holds that 
				\begin{equation*}
					\ideal{ab} \subseteq \ideal{a} \intersection \ideal{b} \subseteq \ideal{a}, \ideal{b} \subseteq \ideal{a} + \ideal{b}.
				\end{equation*}
				If $\ideal{a}$ and $\ideal{b}$ are \textit{comaximal}, i.e. $\ideal{a} + \ideal{b} = R$, the equation $\ideal{ab} = \ideal{a} \intersection \ideal{b}$ holds.
			}
			\item{
				Let $f:R --> S$ be a morphism of rings and $\ideal{b} \diveq S$ be an ideal in $S$.

				Then $\ker f \diveq R$ and $f^{-1}(\ideal{b}) \diveq R$ are ideals in $R$.
			}
			\item{
				Let $f:R -->> S$ be an epimorphism of rings and $\ideal{a} \diveq R$ be an ideal in $R$.

				Then $f(\ideal{a}) \diveq S$ is an ideal in $S$.
			}
		\end{enumerate}
	\end{lemma}
	\begin{sketch}
		\begin{enumerate}[(i)]
			\item{
				The ideal property and inclusions hold by definition. For the equation assuming comaximality note that $1 = a_0+b_0$ for some $a_0 \in \ideal{a}, b_0 \in \ideal{b}$. Given $r \in \ideal{a}\intersection\ideal{b}$ we then have $r = r\cdot 1 = r\cdot(a_0+b_0) = a_0r + rb_0 \in \ideal{ab}$.
			}
			\item{
				By definition.
			}
			\item{
				By definition.
			}
		\end{enumerate}\vspace{-2em}
	\end{sketch}

	\begin{definition}
		Let $R$ be a ring and $\ideal{a} \diveq R$ be an ideal in $R$. As $\ideal{a} \leq_\ncat{Ab} R$ is an abelian subgroup we can construct the factor group $R/\ideal{a}$, with addition given by
		\begin{equation*}
			(x + \ideal{a}) + (y + \ideal{a}) = (x+y) + \ideal{a}.
		\end{equation*}
		Furthermore, since $\ideal{a}$ is closed under multiplication with $R$, the canonic multiplication
		\begin{equation*}
			(x + \ideal{a}) \cdot (y + \ideal{a}) = (x \cdot y) + \ideal{a}
		\end{equation*}
		is well defined, turning $R / \ideal{a}$ into a \textbf{factor ring} (or \textbf{quotient ring} or \textbf{residue class ring}). This construction yields a canonic \textbf{projection morphism} $\pi: R -->> R/\ideal{a}$, which is epi and satisfies $\ker \pi = \ideal{a}$.
	\end{definition}

	\begin{lemma}[Properties of the Factor Ring]
		Let $R$ be a ring and $\ideal{a} \diveq R$ be an ideal.
		\begin{enumerate}[(i)]
			\item{
				Given parallel morphisms of rings $f_1,f_2:R-->S$ define $\ideal{a} = \Gen{s \in S \mid s=f_1(r)=f_2(r) \text{ for some }r \in R}_S$. Then\vspace{-1em}
				\begin{equation*}
					S/\ideal{a}=\coeq(
					\begin{diagram}
						\twobyone{R}{S}
						\arrow[higher]{w}{e}{f_1}[above]
						\arrow[lower]{w}{e}{f_2}[below]
					\end{diagram}
					)
				\end{equation*}
			}
		\end{enumerate}
	\end{lemma}

	\begin{theorem}[Isomorphism Theorem]
		Let $f:R --> S$ be a morphism of rings such that $\ideal{a} \diveq R$ is an ideal in $R$, which satisfies $\ideal{a} \subseteq \ker f$. Then the following universal property holds.
		\begin{equation*}
			\begin{diagram}
				\threebytwo[small]
					{R}{}{S}
					{}{R/\ideal{a}}{}

				\arrow{nw}{ne}{f}[above]
				\arrow[epi]{nw}{s}{\pi}[below left]
				\arrow[dashed]{s}{ne}{\exists!\tilde{f}}[below right]
			\end{diagram}
		\end{equation*}
		In particular, for $\ideal{a} = \ker{f}$ we get a epi-mono-factorization (i.e. $\tilde f$ is mono) and 
		\begin{equation*}
			R / {\ker f} \isom \Im f.
		\end{equation*}
	\end{theorem}
	\begin{sketch}
		Check that 
		\begin{equation*}
			\tilde{f}:
			\begin{array}{rrcl}
					R/\ideal{a}	&\longrightarrow	&S\\
					x+\ideal{a}	&\longmapsto 		&f(x)
			\end{array}
		\end{equation*}
		is a well defined morphism of rings. Uniqueness follows from $\pi$ being epi.
	\end{sketch}

	\begin{theorem}[Chinese Remainder Theorem]
		Let $R$ be a ring and $\ideal{a}_1, ..., \ideal{a}_n \diveq R$ be a ideals in $R$ such that $\ideal{a}_i + \ideal{a}_j = R$ for all $i \neq j$. Then
		\begin{equation*}
			R / {\Intersection \limits_{i=1...n} \ideal{a}_i} \isom R/\ideal{a}_1 \times ... \times R/\ideal{a}_n.
		\end{equation*}
	\end{theorem}
	\begin{sketch}
		The morphism
		\begin{equation*}
			f:
			\begin{array}{rrcl}
					R	&\longrightarrow	&R/\ideal{a}_1 \times ... \times R/\ideal{a}_n\\
					r	&\longmapsto 		&(r+\ideal{a}_1, ..., r + \ideal{a}_n)
			\end{array}
		\end{equation*}
		satisfies $\Intersection \limits_{i=1...n} \ideal{a}_i = \ker f$ and thus by the isomorphism theorem there is an injective morphism of rings in place of the claimed isomorphism.

		For surjectivity consider $(r_1 + \ideal{a}_1, ..., r_n + \ideal{a}_n) = e_1f(r_1) + ... + e_nf(r_n)$, where $e_i$ denotes the \textit{standard unit vector} $(0,...,1+\ideal{a}_i,...,0)$. Suppose for all $i$ there is $c_i \in R$ with $e_i = f(c_i)$. Then
		\begin{align*}
			f(c_1r_1 + ... + c_nr_n) &= f(c_1)f(r_1) + ... + f(c_n)f(r_n)\\
			&=e_1f(r_1) + ... + e_nf(r_n)\\
			&=(r_1 + \ideal{a}_1, ..., r_n + \ideal{a}_n).
		\end{align*}
		By pairwise comaximality we have $a_{ji} \in \ideal{a}_i$ and $a_{ij} \in \ideal{a}_j$ with $a_{ij} + a_{ji} = 1$ for all $i\neq j$. Setting $c_i = \Product\limits_{j\neq i}a_{ij}$ we obtain $c_i \in \ideal{a}_j$, i.e. $c_i + \ideal{a}_j = 0 + \ideal{a}_j$, for all $j \neq i$ and 
		\begin{equation*}
			c_i + \ideal{a}_i = \Product \limits_{j\neq i}(a_{ij}+\ideal{a}_i) = \Product \limits_{j \neq i}(a_{ij} + a_{ji} + \ideal{a}_i) = \Product \limits_{j \neq i}(1 + \ideal{a}_i) = 1 + \ideal{a}_i.\vspace{-2.8em}
		\end{equation*}
	\end{sketch}

	\begin{definition}
		Let $R$ be a ring. A \textbf{prime ideal} is an ideal $\ideal{p}$, which satisfies one of the following equivalent conditions.
		\begin{enumerate}[(i)]
			\item{
				The factor ring $R/\ideal{p}$ is an integral domain.
			}
			\item{
				It holds that $ab \in \ideal{p}$ implies that $a \in \ideal{p}$ or $b \in \ideal{p}$.
			}
			\item{
				For ideals $\ideal{a},\ideal{b} \diveq R$ it holds that $\ideal{a}\ideal{b} \subseteq \ideal{p}$ implies $\ideal{a} \subseteq \ideal{p}$ or $\ideal{b} \subseteq \ideal{p}$.
			}
			% \item{
			% 	\TODO{For ideals $\ideal{a},\ideal{b} \diveq R$ it holds that $\ideal{a} \intersection \ideal{b} \subseteq \ideal{p}$ implies $\ideal{a} \subseteq \ideal{p}$ or $\ideal{b} \subseteq \ideal{p}$.}
			% }
			\item{
				The set $R \setminus \ideal{p}$ is a multiplicative system, i.e. contains $1$ and is closed under multiplication.
			}
		\end{enumerate}
		The set of all prime ideals is called the \textbf{spectrum} of $R$ and is denoted by $\Spec R$.
	\end{definition}
	\begin{sketch}
		(i)$<==>$(ii)$<==>$(iv) just reformulations.\\
		(iii)$==>$(ii) consider the principal ideals $\Gen{a},\Gen{b}$.\\
		\TODO{todo}
	\end{sketch}

	\begin{lemma}[Properties of Prime Ideals]
		\TODO{ R integral domain then (0) prime, preimage of primes are prime, any prime contains minimal prime}
	\end{lemma}

	\begin{definition}
		Let $R$ be a ring. A \textbf{maximal ideal} is an ideal $\ideal{m} \diveq R$, which satisfies one of the following equivalent conditions.
		\begin{enumerate}[(i)]
			\item{
				The factor ring $R/\ideal{m}$ is a field.
			}
			\item{
				The ideal $\ideal{m}$ is maximal in $\{\ideal{a} \divneq R\}$ with respect to inclusion.
			}
			\item{
				The ideal $\ideal{m}$ is maximal in $\{\ideal{p} \divneq R \mid \ideal{p} \text{ prime}\}$ with respect to inclusion.
			}
		\end{enumerate}
		By definition any maximal ideal is prime. The set of all maximal ideals is denoted by $\mSpec R$.
	\end{definition}

	\begin{lemma}[Properties of Maximal Ideals]
		\TODO{ any ideal is in maximal ideal}
	\end{lemma}

	\begin{definition}
		Let $R$ be a ring and $\ideal{a} \diveq R$ be an ideal. 

		The \textbf{radical} of $\ideal{a}$ is the ideal 
		\begin{equation*}
			\sqrt{\ideal{a}} = \{x \in R \mid \exists n \in \N: x^n \in \ideal{a}\} = \Intersection \limits_{\ideal{a} \subseteq \ideal{p} \in \Spec R} \ideal{p}.
		\end{equation*}
		An ideal $\ideal{a}$, which satisfies $\ideal{a} = \sqrt{\ideal{a}}$ is said to be \textbf{radical}.

		The \textbf{nilradical} of $R$ is the radical ideal
		\begin{align*}
			\nrad R &= \{x \in R \mid \exists n \in \N: x^n = 0\} = \Intersection \limits_{\ideal{p} \in \Spec R} \ideal{p}\\
			& \overset{?}{=} \{x \in R \mid \exists n \in \N: 1 - x^n \in R^\times\}.
		\end{align*}

		The \textbf{Jacobson radical} of $R$ is the radical ideal
		\begin{equation*}
			\jrad R = \Intersection \limits_{\ideal{m} \in \mSpec R} \ideal{m} = \{x \in R \mid \forall r \in R: 1 - rx \in R^\times\}.
		\end{equation*}
	\end{definition}

	\begin{lemma}[Properties of Radical Ideals]
		Let $R$ be a ring and $\ideal{a},\ideal{b} \diveq R$ be two ideals in R.
		\begin{enumerate}[(i)]
			\item{
				For any $n \in \N$ we have $\sqrt{\sqrt{\ideal{a}}} = \sqrt{\ideal{a}} = \sqrt{\ideal{a}^n}$.
			}
			\item{
				It holds that $\ideal{a} \subseteq \sqrt{\ideal{a}}$ and that $\ideal{a} \subseteq \ideal{b}$ implies $\sqrt{\ideal{a}} \subseteq \sqrt{\ideal{b}}$.
			}
			\item{
				We have $\sqrt{\ideal{a} \intersection \ideal{b}} = \sqrt{\ideal{a}} \intersection \sqrt{\ideal{b}}$
			}
			\item{
				The equation $\sqrt{\ideal{a}} = \sqrt{\ideal{b}}$ is equivalent to having $n,m \in \N$ such that $\ideal{a}^m \subseteq \sqrt{\ideal{b}}$ and $\ideal{b}^n \subseteq \sqrt{\ideal{a}}$.
			}
			\item{
				Any prime ideal $\ideal{p}$ is radical, i.e. satisfies $\sqrt{\ideal{p}} = \ideal{p}$.
			}
		\end{enumerate}
	\end{lemma}

	\begin{theorem}[Correspondence Theorem]
		Let $R$ be a ring and $\ideal{a} \diveq R$ be an ideal. The canonic projection $\pi: R -->> R/\ideal{a}$ exhibits bijections
		\begin{equation*}
			\begin{array}{rcl}
				\{\text{ideals containing }\ideal{a}\} &\overset{\sim}{\longrightarrow} &\{\text{ideals in } R/\ideal{a}\}\\
				\{\text{radical ideals containing }\ideal{a}\} &\overset{\sim}{\longrightarrow} &\{\text{radical ideals in } R/\ideal{a}\}\\
				\{\text{prime ideals containing }\ideal{a}\} &\overset{\sim}{\longrightarrow} &\{\text{prime ideals in } R/\ideal{a}\}\\
				\{\text{maximal ideals containing }\ideal{a}\} &\overset{\sim}{\longrightarrow} &\{\text{maximal ideals in } R/\ideal{a}\}\\\\
				\ideal{b} &\longmapsto &\pi(\ideal{b})\\
				\pi^{-1}(\ideal{b}) &\reflectbox{$\longmapsto$} &\ideal{b}
			\end{array}
		\end{equation*}
	\end{theorem}

	\newpage
	\subsection{Localization and Local Rings}

	\begin{definition}
		Let $S$ be a multiplicative system in a ring $R$, i.e. contain $1$ and be closed under multiplication. The relation
		\begin{align*}
			(r_1,s_1) \sim (r_2,s_2) <==> \exists f\in S: fr_1s_2 = fr_2s_1
		\end{align*}
		defines an equivalence relation on $R \times S$. We write $\dfrac{r}{s}$ for the equivalence class of $[(r,s)]$. By setting
		\begin{equation*}
			\dfrac{r_1}{s_1} + \dfrac{r_2}{s_2} = \dfrac{r_1s_2 + r_2s_1}{s_1s_2} \hspace{1cm}\text{and}\hspace{1cm} \dfrac{r_1}{s_1} \cdot \dfrac{r_2}{s_2} = \dfrac{r_1r_2}{s_1s_2}
		\end{equation*}
		we obtain a ring structure on $R \times S / \sim$. This is the so called \textbf{localization} $S^{-1}R$ of $R$ by $S$.
	\end{definition}

	\TODO{remark on factor $f$}

	\begin{theorem}[Correspondence Theorem]
		Let $R$ be a ring and $S$ be a multiplicative system in $R$. The canonic morphism $\iota_S: R --> S^{-1}R$ exhibits bijections
		\TODO{todo}
		\begin{equation*}
			\begin{array}{rcl}
				\{\text{ideals }\ideal{a}\text{ with } \forall s \in S, r \in R \setminus \ideal{a}: sr \notin \ideal{a}\} &\overset{\sim}{\longrightarrow} &\{\text{ideals in } S^{-1}R\}\\
				\{\text{radical ideals }\ideal{r}\text{ with } \forall s \in S, r \in R \setminus \ideal{r}: sr \notin \ideal{r}\} &\overset{\sim}{\longrightarrow} &\{\text{radical ideals in } S^{-1}R\}\\
				\{\text{prime ideals }\ideal{p}\text{ with } \forall s \in S, r \in R \setminus \ideal{p}: sr \notin \ideal{p}\} &\overset{\sim}{\longrightarrow} &\{\text{prime ideals in } S^{-1}R\}\\
				\{\text{maximal ideals }\ideal{m}\text{ with } \forall s \in S, r \in R \setminus \ideal{m}: sr \notin \ideal{m}\} &\overset{\sim}{\longrightarrow} &\{\text{maximal ideals in } S^{-1}R\}\\\\
				\ideal{a} &\longmapsto &S^{-1}\ideal{a} = \Gen{\iota_S(\ideal{a})}_{S^{-1}R}\\
				\iota_S^{-1}(\ideal{b}) &\reflectbox{$\longmapsto$} &\ideal{b}
			\end{array}
		\end{equation*}
	\end{theorem}

	\begin{definition}
		A ring $R$ is said to be \textbf{local}, if it satisfies one of the following equivalent contitions.
		\begin{enumerate}[(i)]
			\item{
				$R$ has exactly one maximal ideal $\ideal{m}_R$.
			}
			\item{
				The complement $R \setminus R^\times$ is an ideal.
			}
			\item{
				$R$ is nontrivial and for any $r \in R$ it holds that $r$ or $1-r$ are invertible.
			}
		\end{enumerate}

		Recall that in general preimages of maximal ideals need not be maximal. A morphism $f:R-->R'$ between local rings is a \textbf{local morphism}, if it satisfies $f^{-1}(\ideal{m}_{R'}) = \ideal{m}_R$ or equivalently that $f(r)$ is invertible if and only if $r$ is invertible.
	\end{definition}

	\begin{corollary}
		Let $\ideal{p}$ be a prime ideal in some ring $R$. Recall that by definition the complement $S = R\setminus \ideal{p}$ is a multiplicative system. The localization $R_\ideal{p} = S^{-1}R$ is a local ring with maximal ideal $\ideal{m}_{R_\ideal{p}} = S^{-1}\ideal{p}$.
	\end{corollary}

	\newpage
	\subsection{Factorial Rings and Principal Ideal Domains}

	\begin{definition}
		Let $R$ be a ring. An element $a$ \textbf{divides} another element $r$ (or is a \textbf{divisor} of $r$), if there is an element $b$ such that $ab = r$. In this case we write $a \mid r$.

		Two elements $a$ and $b$ are \textbf{associate}, if there is a unit $u$ such that $au = b$. Being associate defines an equivalence relation on $R$ and we denote the equivalence class corresponding to $a$ by $aR^\times$. 
	\end{definition}

	\begin{lemma}
		\begin{enumerate}[(i)]
			\item{
				If $a \mid r$ and $a$ is a zero divisor then $r$ is a zero divisor. If $a$ is not a zero divisor then the corresponding factor $b$ with $ab = r$ is unique.
			}
			\item{
				\TODO{all zero divisors are associate? one zero divisor then all associate zero divisors?}
			}
			\item{
				If $R$ is an integral domain, then divisibility defines a partial? order on the set $R/_\sim$.
			}
		\end{enumerate}
	\end{lemma}

	\begin{definition}
		Let $R$ be a ring. 

		An element $r$ is \textbf{irreducible}, if it is not a unit and $ab = r$ implies that either $a$ or $b$ are a unit. Note that the first condition makes this an exclusive or and the second condition can be reformulated as any divisor is associate to $r$.

		An \TODO{nonzero} element $p$ is \textbf{prime}, if it is not a unit and for any two elements $a,b$ the statement $p \mid ab$ implies $p \mid a$ or $p \mid b$.
	\end{definition}

	\begin{lemma}
		Let $R$ be a ring. The following statements hold.
		\begin{enumerate}[(i)]
			\item{
				Irreducible elements are not zero divisors and are inparticular nonzero.
			}
			\item{
				In an integral domain any prime element is irreducible
			}
		\end{enumerate}
	\end{lemma}
	\begin{proof}
		\begin{enumerate}[(i)]
			\item{
				If $R$ is trivial $0$ is a unit.
			}
			\item{

			}
		\end{enumerate}
	\end{proof}

	\begin{definition}
		Let $R$ be a ring and $I$ be a system of representatives
		An integral domain $R$ is a \textbf{factorial ring} (or unique factorization domain), if every element has a unique decomposition into irreducible elements
	\end{definition}

	\begin{definition}
		Let $R$ be a ring. A \textbf{principal ideal} is an ideal, which is generated by one element. A \textbf{principal ideal domain} is an integral domain, in which every ideal is a principal ideal.
	\end{definition}

	\begin{theorem}[Fundamental Theorem of Arithmetic]
		Let $R$ be a principal ideal domain. Then $R$ is a factorial ring, i.e. every element has a (in some sense) unique decomposition into irreducible elements.
	\end{theorem}

	\subsection{Dedekind Domains?}
	\subsection{Noetherian and Artinian Rings}

	\newpage
	\section{Algebras}
	\subsection{Polynomial Rings}
	\subsection{Algebras}

	\begin{definition}
		Let $R$ be a ring. A \textbf{$R$-algebra} is a ring $A$ together with a morphism of rings $\varphi_A:R-->A$ called the \textbf{structure morphism}. In more explicit terms $A$ is a ring  together with a scalar multiplication $\cdot: R \times A --> A$, such that the multiplication $\cdot: A \times A --> A$ is $R$-bilinear, i.e. satisfies
		\begin{equation*}
			 (r\cdot a)\cdot b = r\cdot(a\cdot b) = a \cdot (r \cdot b).
		\end{equation*}
		The structure morphism can be recovered from this description by considering
		\begin{equation*}
			\begin{array}{rcl}
				R & \longrightarrow & A\\
				r & \longmapsto & r\cdot 1_A.
			\end{array}
		\end{equation*}
		\begin{minipage}{\textwidth-4cm}
			A morphism of $R$-algebras is a morphism of rings $f:A-->B$ compatible with the structure morphisms in the sense that the diagram on the right commutes. Explicitly $f$ has to satisfy the additional requirement that
			\begin{equation*}
				f(ra) = rf(a).
			\end{equation*}
		\end{minipage}
		\begin{minipage}{4cm}
			\begin{equation*}
				\begin{diagram}
					\threebytwo[narrow]
						{}{R}{}
						{A}{}{B}

					\arrow{n}{sw}{\varphi_A}[above left]
					\arrow{n}{se}{\varphi_B}[above right]
					\arrow{sw}{se}{f}[below]
				\end{diagram}
			\end{equation*}
		\end{minipage}
		By construction the category of $R$-algebras, denoted $\ncat{Alg}_R$, is the comma category $R \downarrow \ncat{CRing}$.
	\end{definition}

	\section{Topological Rings?}
	\subsection{Adic Topology}
\end{document}