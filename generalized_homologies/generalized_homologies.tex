\documentclass{article}

% --- text layout
\usepackage[parfill]{parskip}
\usepackage{geometry}
\geometry{a4paper, margin=3cm}
\usepackage{romanbar}
\usepackage{placeins} %FloatBarrier
\usepackage[utf8]{inputenc}

\usepackage{xcolor}
\definecolor{darkred}{HTML}{990000}
\definecolor{darkgreen}{HTML}{009900}

\usepackage{sectsty}
\sectionfont{\color{darkred}}
\subsectionfont{\color{darkred}}
\subsubsectionfont{\color{darkred}}
% \usepackage{algpseudocode} --> better pseudo code?
%\usepackage[pagewise]{lineno} % row numbers

% --- syncing, referencing, citing
\usepackage{pdfsync} 			% synchronization LaTeX <--> PDF
\usepackage[hidelinks]{hyperref} % hidelinks avoids red boxes around links
\usepackage{csquotes}			% enquote
\usepackage{cleveref}
\usepackage{pdfpages}

\usepackage[backend=bibtex,style=alphabetic,url=false]{biblatex}
\addbibresource{sources.bib}
% define bibliography filter like
%\defbibfilter{computability-pca}{
%	keyword=computability or keyword=pca
%}

% --- lists and tables
\usepackage[shortlabels]{enumitem}
\usepackage{tabularx}
\usepackage{tabu}
\usepackage{makecell}

% --- graphics
\usepackage{graphicx}
\usepackage{xcolor}

% --- math
\usepackage{amsmath}
\usepackage[bbgreekl]{mathbbol} % double stroke greek
\usepackage{amssymb}
\usepackage{fixmath} % whyyy? "bug": italic PI and SIGMA
\usepackage{mathtools}
\usepackage{mathrsfs} %mathscr

% --- own packages
\usepackage{../notation}
\usepackage{../diagrams}
\usepackage{../environments}

\title{$\Sum\text{math}(y)$\\Generalized Homology Theories}
\author{Jonas Linssen}

\begin{document}
	\maketitle
	\tableofcontents

	\newpage
	\section{Generalized Homology}
	\subsection{Definition and Exact Sequences}

	\begin{definition}
		A \textbf{generalized} (\textbf{unreduced, relative}) \textbf{homology theory} on $\ncat{Top}^2$ is a functor $H_\bullet: \ncat{Top}^2 --> \ncat{Mod}_R^{\Z\text{gr}}$ together with a natural transformation $\partial_\bullet: H_\bullet ==> H_{\bullet-1} \circ R$, the so called \textbf{boundary operator} or \textbf{connecting homomorphism}, which satisfy the following axioms.
	\begin{enumerate}[$\bullet$]
		\item{
			\textit{homotopy invariance}\\
			$H_\bullet$ factors via $\ncat{hTop}^2$.
		}
		\item{
			\textit{exactness}\\
			For any object $(X,A)$ in $\ncat{Top}^\rightincl$ the sequence
			\begin{equation*}
				...
				\begin{diagram}
					\fivebyone[verywide]
						{H_{\bullet+1}(X,A)}
						{H_\bullet(A,\emptyset)}
						{H_\bullet(X,\emptyset)}
						{H_\bullet(X,A)}
						{H_{\bullet-1}(A,\emptyset)}

					\arrow{ww}{w}{\partial_{\bullet+1}}[above]
					\arrow{w}{c}{H_\bullet(i)}[above]
					\arrow{c}{e}{H_\bullet(j)}[above]
					\arrow{e}{ee}{\partial_\bullet}[above]
				\end{diagram}
				...
			\end{equation*}
				is exact, where $i: (A,\emptyset) \longrightincl (X,\emptyset)$ and $j: (X,\emptyset) \longrightincl (X,A)$ are the inclusion maps.

		}
		\item{
			\textit{excission}\\
			For every object $(X,A)$ in $\ncat{Top}^\rightincl$ and subset $U \subseteq \closure U \subseteq \interior A$, the inclusion $(X\setminus U, A \setminus U) \subseteq (X, A)$ induces an isomorphism $$H_\bullet(X\setminus U, A \setminus U) \isom H_\bullet(X,A).$$
		}
	\end{enumerate}
	One might further assume
	\begin{enumerate}[$\bullet$]
		\item{
			\textit{additivity}\\
			$H_\bullet$ preserves coproducts. Specifically, for any family $(X_i,A_i)_i$ in $\ncat{Top}^\rightincl$ the canonical morphism
			\begin{equation*}
				\bigoplus \limits_{i \in I}H_\bullet(X_i,A_i) \longrightarrow H_\bullet(\Coproduct \limits_{i \in I} (X_i,Y_i))
			\end{equation*}
			is an isomorphism.
		}
		\item{
			\textit{ordinarity / dimension}\\
			For all $n \neq 0$ it holds that $H_n(*,\emptyset) = 0$. 
		}
	\end{enumerate}
	\end{definition}

	\begin{lemma}[First Calculations]
		Let $(H_\bullet, \partial_\bullet)$ be a generalized homology theory.
		\begin{enumerate}[(i)]
			\item{
				If $f:(X_1,A_1) \xrightarrow{\simeq} (X_2,A_2)$ is a homotopy equivalence, then $H_\bullet(X_1,A_1) \isom H_\bullet(X_2,A_2)$.
			}
			\item{
				If $A \subseteq X$ is a homotopy equivalence, then $H_\bullet(X,A) = 0$, in particular $H_\bullet(X,X) = 0$.
			}
			\item{
				If $H_\bullet$ is additive, it holds that $H_\bullet(X_\text{disc}) = \Directsum \limits_X M_\bullet$.
			}
		\end{enumerate}
	\end{lemma}

	By (i) of the previous lemma we have $H_\bullet(*) \isom H_\bullet(\D^n) \isom H_\bullet(\R^n)$ for all $n \in \N$.\\
	Similarly $\H_\bullet(\S^{n-1}) \isom H_\bullet(\D^n \setminus \{0\})$.

	\begin{proposition}[Mayer Vitoris]
		Let $(H_\bullet,\partial_\bullet)$ be a generalized homology theory and $B \subseteq A_1,A_2 \subseteq X$ in $\ncat{Top}$, such that $X \subseteq \interior(A_1) \union \interior(A_2)$. Then the following sequence is exact.
		\begin{equation*}
			\begin{diagram}
				\threebythree[hyperwide]
					{}{}{...H_{\bullet+1}(X,B)}
					{H_\bullet(A_1 \intersection A_2,B)}{H_\bullet(A_1,B)\directsum H_\bullet(A_2,B)}{H_\bullet(X,B)}
					{H_{\bullet-1}(A_1 \intersection A_2, B)...}{}{}

				\arrow{w}{c}{H_\bullet(i_1)\directsum H_\bullet(i_2)}[above]
				\arrow{c}{e}{H_\bullet(j_1) - H_\bullet(j_2)}[above]
				
				\draw[->,rounded corners] (ne.east) -| node[auto,pos=.7]{$\partial_{\bullet+1}$}($(ne)+(1.5,-.75)$) -| ($(w)+(-2,0)$) |- (w.west);
				\draw[->,rounded corners] (e.east) -| node[auto,pos=.7]{$\partial_\bullet$}($(e)+(1.5,-.75)$) -| ($(sw)+(-2,0)$) |- (sw.west);
			\end{diagram}
		\end{equation*}
	\end{proposition}

	\begin{proposition}[Triple Sequence]
		Let $(H_\bullet,\partial_\bullet)$ be a generalized homology theory and $B \subseteq A \subseteq X$ in $\ncat{Top}$. Given
		\begin{equation*}
			\Delta_\bullet :=
			\begin{diagram}
				\threebyone[wide]
					{H_\bullet(X,A)}{H_{\bullet-1}(A)}{H_{\bullet-1}(A,B)}
				\arrow{w}{c}{\partial_\bullet}[above]
				\arrow{c}{e}{}
			\end{diagram}
		\end{equation*}
		the following sequence is exact.
		\begin{equation*}
			...
			\begin{diagram}
				\fivebyone[verywide]
					{H_{\bullet+1}(X,A)}{H_\bullet(A,B)}{H_\bullet(X,B)}{H_\bullet(X,A)}{H_{\bullet-1}(A,B)}

				\arrow{ww}{w}{\Delta_{\bullet+1}}[above]
				\arrow{w}{c}{}[above]
				\arrow{c}{e}{}[above]
				\arrow{e}{ee}{\Delta_\bullet}[above]
			\end{diagram}
			...
		\end{equation*}
	\end{proposition}

	\newpage
	\subsection{Reduced Homology}

	\begin{lemma}
		Let $(H_\bullet, \partial_\bullet)$ be a generalized homology theory and $x$ be a point in a topological space $X$. Then for any $n \in \Z$ the sequence
		\begin{equation*}
			\begin{diagram}
				\threebyone[wide]
					{H_n(\{x\})}{H_n(X)}{H_n(X,\{x\})}
				\arrow{w}{c}{}
				\arrow{c}{e}{}
			\end{diagram}
		\end{equation*}
		is split exact with section $H_n(c):H_n(X) --> H_n(\{x\})$, where $c: X --> \{x\}$ denotes the constant map.

		In particular it holds that $H_\bullet(X,\{x_1\}) \isom H_\bullet(X,\{x_2\})$ and we get $H_\bullet(X) \isom H_\bullet(*) \oplus H_\bullet(X,*)$.
	\end{lemma}

	\begin{definition}
		The \textbf{module of coefficients} of a generalized homology theory $(H_\bullet, \partial_\bullet)$ is the graded module $M_\bullet := H_\bullet(*)$.

		The \textbf{reduced homology} of $H_\bullet$ is the functor \TODO{involve kernel}
		\begin{equation*}
			\begin{array}{rcl}
				\ncat{Top}_* & \xrightarrow{\widetilde{H}_\bullet} & \ncat{Mod}_R^{\Z\text{gr}}\\
				(X,x_0) & \longmapsto & H_\bullet(X,\{x_0\})\\
				f:(X,x_0)\rightarrow (Y,y_0) & \longmapsto & H_\bullet(f)
			\end{array}
		\end{equation*}
	\end{definition}

	\begin{remark}
		\TODO{mapping cone and reduced suspension}
	\end{remark}

	\begin{lemma}
		Let $(H_\bullet,\partial_\bullet)$ be a generalized homology theory.
		\begin{enumerate}[(i)]
			\item{
				$\widetilde{H}_\bullet$ is homotopy invariant, i.e. factors via $\ncat{hTop}_{*}$.
			}
			\item{
				Given $i_A:A \longrightincl X$ in $\ncat{Top}$ and $x_0 \in A$ it holds that $H_\bullet(X,A) \isom \widetilde{H}_\bullet(\Cone(i_A), x_0)$.
			}
			\item{
				Given $x_0 \in A$, the long exact sequence of $(X,A)$ induces a long exact sequence.
				\begin{equation*}
					\begin{diagram}
						\threebythree[ultrawide]
							{}{}{...\widetilde{H}_{\bullet+1}(\Cone(i_A),x_0)}
							{\widetilde{H}_\bullet(A,x_0)}{\widetilde{H}_\bullet(X,x_0)}{\widetilde{H}_\bullet(\Cone(i_A),x_0)}
							{\widetilde{H}_{\bullet-1}(A,x_0)...}{}{}

						\arrow{w}{c}{}
						\arrow{c}{e}{}

						\draw[->,rounded corners] (ne.east) -| node[auto,pos=.7]{$\partial_{\bullet+1}$}($(ne)+(2,-.75)$) -| ($(w)+(-1.5,0)$) |- (w.west);
						\draw[->,rounded corners] (e.east) -| node[auto,pos=.7]{$\partial_\bullet$}($(e)+(2,-.75)$) -| ($(sw)+(-1.5,0)$) |- (sw.west);
					\end{diagram}
				\end{equation*}
			}
			\item{
				\TODO{suspension}
			}
		\end{enumerate}
	\end{lemma}

	\begin{lemma}
		\TODO{Reduced Mayer Vitoris}
	\end{lemma}

	\subsection{Examples and Applications}

	\begin{lemma}[Homology of the Sphere]
		\TODO{Sphere}
	\end{lemma}

	\begin{theorem}[Invariance of Dimension]
		Let $U \subseteq \R^m$ and $V \subseteq \R^n$ be open subsets. The existence of a homeomorphism $f:U-->V$ implies $m = n$.
	\end{theorem}

	\begin{lemma}[Homology of the Oriented Surfaces]
		\TODO{Torus and higher genus}
	\end{lemma}

	\section{Simplicial Homology}
	\subsection{Simplicial Spaces}
	\subsection{Simplicial Homology}
	\subsection{Examples and Applications}
\end{document}