\documentclass{article}

% --- text layout
\usepackage[parfill]{parskip}
\usepackage{geometry}
\geometry{a4paper, margin=3cm}
\usepackage{romanbar}
\usepackage{placeins} %FloatBarrier
\usepackage[utf8]{inputenc}

\usepackage{xcolor}
\definecolor{darkred}{HTML}{990000}
\definecolor{darkgreen}{HTML}{009900}

\usepackage{sectsty}
\sectionfont{\color{darkred}}
\subsectionfont{\color{darkred}}
\subsubsectionfont{\color{darkred}}
% \usepackage{algpseudocode} --> better pseudo code?
%\usepackage[pagewise]{lineno} % row numbers

% --- syncing, referencing, citing
\usepackage{pdfsync} 			% synchronization LaTeX <--> PDF
\usepackage[hidelinks]{hyperref} % hidelinks avoids red boxes around links
\usepackage{csquotes}			% enquote
\usepackage{cleveref}
\usepackage{pdfpages}

\usepackage[backend=bibtex,style=alphabetic,url=false]{biblatex}
\addbibresource{sources.bib}
% define bibliography filter like
%\defbibfilter{computability-pca}{
%	keyword=computability or keyword=pca
%}

% --- lists and tables
\usepackage[shortlabels]{enumitem}
\usepackage{tabularx}
\usepackage{tabu}
\usepackage{makecell}

% --- graphics
\usepackage{graphicx}
\usepackage{xcolor}

% --- math
\usepackage{amsmath}
\usepackage{amssymb}
\usepackage{fixmath} % whyyy? "bug": italic PI and SIGMA
\usepackage{mathtools}
\usepackage{mathrsfs} %mathscr

% --- own packages
\usepackage{../notation}
\usepackage{../diagrams}
\usepackage{../environments}

\title{$\Sum\text{math}(y)$\\Generalized Homology Theories}
\author{Jonas Linssen}

\begin{document}
	\maketitle
	\tableofcontents

	\newpage
	\setcounter{section}{-1}
	\section{Prerequisites}
	\subsection{Homotopy}

	\begin{definition}
		A \textbf{homotopy} between morphisms $f_0,f_1: (X, \mathcal{O}_X) --> (Y,\mathcal{O}_Y)$ in $\ncat{Top}$ is a continuous map 
		\begin{align*}
			h:[0,1]\times X --> Y,
		\end{align*}
		satisfying that $f_0 = h|_{\{0\}\times X}$ and $f_1 = h|_{\{1\}\times X}$. In this case $f_0$ and $f_1$ are called \textbf{homotopic}, which is denoted by $f_0 h_\sim f_1$ or simply $f_0 \sim f_1$.

		For arbitrary objects $X,Y$ in $\ncat{Top}$, homotopy defines an equivalence relation on $\ncat{Top}(X,Y)$ as
		\begin{equation*}
			\begin{array}{rcl}
				f_0 \sim f_0 & \text{via} & (t,x) \mapsto f(x)\vspace{.5em}\\
				f_0 \underset{h}{\sim} f_1 \implies f_1 \sim f_0 & \text{via} &(t,x) \mapsto h(1-t,x)\vspace{-.25em}\\
				f_0 \underset{h}{\sim} f_1 \underset{h'}{\sim} f_2 \implies f_0 \sim f_2 & \text{via} &(t,x) \mapsto \left\{\begin{array}{cl} 
					h(2t,x) & t \in [0,\frac{1}{2}]\\
					h'(2t-1,x) & t \in [\frac{1}{2},1]
				\end{array}\right.
			\end{array}
		\end{equation*}
		Moreover this equivalence relation is compatible with composition in the sense that, given maps $s:W-->X$, $f_0,f_1: X --> Y$ and $t:Y-->Z$, any homotopy $f_0 h_\sim f_1$ extends to a homotopy $tf_0s \sim tf_1s$, by taking the composite
		\begin{equation*}
			\begin{diagram}
				\fourbyone[verywide]
					{[0,1]\times W}{[0,1] \times X}{Y}{Z}
				\arrow{ww}{w}{[0,1]\times s}[above]
				\arrow{w}{e}{h}[above]
				\arrow{e}{ee}{t}[above]
			\end{diagram}.
		\end{equation*}
		Hence, by considering morphisms of topological spaces just up to homotopy, one obtains the \textbf{homotopy category} $\ncat{hTop}$, which comes with the canonical functor
		\begin{equation*}
			\begin{array}{rcl}
				\ncat{Top} & \longrightarrow & \ncat{hTop}\\
				(X,\mathcal{O}_X) & \mapsto & (X,\mathcal{O}_X)\\
				f & \mapsto & [f]_\sim
			\end{array}.
		\end{equation*}
	\end{definition}

	\begin{definition}
		A continuous map $f:X-->Y$ is a \textbf{homotopy equivalence}, if $[f]_\sim: X --> Y$ is an isomorphism in $\ncat{hTop}$. Explicitly this means that there exists a morphism $g:Y --> X$ in $\ncat{Top}$ such that $gf \sim 1_X$ and $fg \sim 1_Y$. Topological spaces $X$ and $Y$ are said to be \textbf{homotopy equivalent}, which will be denoted by $X \simeq Y$, if there is a homotopy equivalence between them.

		A topological space $X$ is \textbf{contractible}, if it is homotopy equivalent to the singleton space $\{*\}$. In particular $X$ is a nonempty space.
	\end{definition}

	For example the canonical projection $[0,1] \times X \xrightarrow{\pi_X} X$ defines a homotopy equivalence with homotopy inverse given by the homotopy class of any of the following maps ($t \in [0,1]$).
	\begin{equation*}
		\begin{array}{rcc}
			\iota_t:X &\longrightarrow &[0,1] \times X,\\
			 x & \longmapsto &(t,x)
		\end{array}
	\end{equation*}

	\begin{definition}
		A functor $\fun{F}:\ncat{Top} --> \cat{C}$ into some category $\cat{C}$ is said to be \textbf{homotopy invariant}, if it satisfies one of the following equivalent conditions.
		\begin{enumerate}[(i)]
			\item{
				$\fun{F}$ factors through $\ncat{hTop}$.
			}
			\item{
				For any pair of continuous maps $f_0,f_1$ in $\ncat{Top}$ it holds that $f_0 \sim f_1$ implies $\fun{F}f_0 = \fun{F}f_1$.
			}
			\item{
				For any topological space $X$ he projection $\pi_X:[0,1]\times X --> X$ induces an isomorphism $\fun{F}\pi_X: \fun{F}([0,1]\times X) \xrightarrow{\isom} \fun{F}X$.
			}
			\item{
				For any topological space $X$ and $s,t \in [0,1]$ the inclusions $\iota_s, \iota_t: X --> [0,1]\times X$ satisfy $\fun{F}\iota_s = \fun{F}\iota_t$.
			}
		\end{enumerate}
	\end{definition}
	\begin{sketch}
		(i) $\Leftrightarrow$ (ii): by definition.

		(i) $\Rightarrow$ (iii):  $[\pi_X]_\sim$ is iso in $\ncat{hTop}$, so $\fun{F}\pi_X$ is an isomorphism.

		(iii) $\Rightarrow$ (iv): by uniqueness of inverses.

		(iv) $\Rightarrow$ (ii): Given $f_0 \sim_h f_1$ we have $f_i = h\iota_i$, which yields $\fun{F}f_0 = \fun{F}(h)\fun{F}\iota_0 = \fun{F}h\fun{F}\iota_1 = \fun{F}f_1$.

	\end{sketch}

	\newpage
	\subsection{Deformation Retracts}
	\TODO{Kammeyer section 2.4}

	\begin{definition}
		A morphism $i:A --> X$ of topological spaces is a \textbf{deformation retract}, if there is a retraction $r:X --> A$ and a homotopy $h:ir \sim 1_X$, which satisfies $h(t,i(a)) = a$ for all $a$ in $A$. Note that $i$ being split mono (and thus regular mono) makes it an embedding of a subspace.

		The morphism $i$ is a \textbf{neighborhood deformation retract}, if there exists a neighborhood $U$ of $i(A)$, such that the restricted map $i:A-->U$ is a deformation retract.

		In the case that $i$ is the canonical inclusion we also say that $A$ is a (neighborhood-) deformation retract.

		We abbreviate closed neighborhood deformation retracts, i.e. those where $i(A)$ is a closed subspace, by \textbf{CNDR}.
	\end{definition}

	\begin{lemma}
		Deformation retracts / neighborhood deformation retracts / CNDRs are stable under pushouts.
	\end{lemma}
	\begin{sketch}
		\begin{minipage}[t]{\linewidth-4cm}
			Given an embedding retraction pair $(i,r):A-->X$, a morphism $f:A --> B$ and a pushout square as depicted on the right, we find that the pushout $i'$ of $i$ has a retraction $r'$ given by the universal property of the pushout $Y$. If $h:ir \sim 1_X$ is a homotopy turning $i$ into a deformation retract, we seek to construct a homotopy $k:i'r' \sim 1_Y$ with $k(t,i'(b)) = b$ for all $b$ in $B$.
		\end{minipage}
		\begin{minipage}[t]{4cm}
			\vspace{-1.5em}
			\begin{equation*}
				\begin{diagram}
					\twobytwo[small]
						{A}{B}
						{X}{Y}

					\arrow{nw}{ne}{f}[above]
					\arrow{sw}{se}{f'}[below]
					\arrow{nw}{sw}{i}[left]
					\arrow{ne}{se}{i'}[right]

					\node at ($(se)+(-.4,.4)$) {$\secorner$};
				\end{diagram}
			\end{equation*}
		\end{minipage}

		\begin{minipage}[t]{5.5cm}
			\vspace{-1em}
			\begin{equation*}
				\begin{diagram}
					\threebythree
						{I \times A}{I \times B}{B}
						{I \times X}{I \times Y}{}
						{X}{}{Y}

					\arrow{nw}{n}{I \times f}[above]
					\arrow{w}{c}{I \times f'}[below]
					\arrow{nw}{w}{I \times i}[left]
					\arrow{n}{c}{I \times i'}[right]

					\arrow{n}{ne}{\pi_B}[above]
					\arrow{ne}{se}{i'}[right]
					\arrow{w}{sw}{h}[left]
					\arrow{sw}{se}{f'}[below]

					\arrow[dashed]{c}{se}{k}[above right]

					\node at ($(c)+(-.4,.4)$) {$\secorner$};
				\end{diagram}
			\end{equation*}
		\end{minipage}
		\begin{minipage}[t]{\linewidth-5.5cm}
			\vspace{1em}
			Since $I = [0,1]$ is compact, $I \times -$ preserves colimits and from the diagram on the left we obtain a morphism $k: I \times Y --> Y$, which by construction satisfies $k(t,i'(b)) = i'(b)$. We have to check that $k(0,y) = i'r'(y)$ for all $y$ in $Y$. This can by deduced by using the homotopy property of $h$ to show that both $Y \xrightarrow{c_0 \times Y} I \times Y \xrightarrow{k} Y$ and $i'r': Y --> Y$ satisfy the universal property of the pushout $Y$. A similar argument shows $k(1,y) = y$ for all $y \in Y$.
		\end{minipage}

		\begin{minipage}[t]{\linewidth-4.5cm}
			For $i:A --> X$ being a neighborhood deformation retract with $i(A) \subseteq U \subseteq X$, note that we can factor the pushout as depicted on the right and that $V$ is by definition an open subspace of $Y$, of which $B$ is a deformation retract.

			\vspace{1em}
			Finally $i'(B)$ is closed in $Y = (X \sqcup B)/\sim$, since the preimage ${f'}^{-1}(i'(B)) \sqcup B = A \sqcup B$ is closed in $X \sqcup B$.
		\end{minipage}
		\begin{minipage}[t]{4.5cm}
			\vspace{-1cm}
			\begin{equation*}
				\begin{diagram}
					\twobythree
						{A}{B}
						{U}{V}
						{X}{Y}

					\arrow{nw}{ne}{f}[above]
					\arrow{w}{e}{f\vert_U^V}[below]
					\arrow{sw}{se}{f'}[below]

					\arrow{nw}{w}{i\vert^U}[right]
					\arrow{w}{sw}{\subseteq}[right]
					\arrow[bend right]{nw}{sw}{i}[left]

					\arrow{ne}{e}{i'\vert^V}[left]
					\arrow{e}{se}{\subseteq}[left]
					\arrow[bend left]{ne}{se}{i'}[right]

					\node at ($(e)+(-.3,.3)$) {$\secorner$};
					\node at ($(se)+(-.3,.3)$) {$\secorner$};
				\end{diagram}
			\end{equation*}
		\end{minipage}
	\end{sketch}

	\begin{lemma}
		The functor $I \times -$ preserves CNDRs, i.e. f $i:A \rightarrow X$ is a CNDR, then $I \times i: I \times A \rightarrow I \times X$ is a CNDR.
	\end{lemma}
	\begin{sketch}
		For $i(A) \subseteq U \subseteq X$ and retraction $r: U --> A$ we obtain an open neighborhood $I\times i(I \times A) \subseteq I \times U \subseteq I \times X$ and a retraction $I \times r: I \times U --> I \times A$.

		Given the homotopy $h:ir \sim 1_X$ it suffices to verify that
		\begin{equation*}
			\begin{diagram}
				\threebyone[verywide]
					{k:I \times I \times U}{I \times I \times U}{I \times U}

				\arrow{w}{c}{\tau \times U}[above]
				\arrow{c}{e}{I \times h}[above]
			\end{diagram}
		\end{equation*}
		satisfies the required properties of a homotopy $k:(I \times ir) --> 1_{I \times U}$, where $\tau: X \times X --> X \times X$ denotes the transposition isomorphism. Finally $I \times i(I \times A)$ is closed in $I \times X$.
	\end{sketch}

	\begin{lemma}
		If $i:A --> X$ is a CNDR, then the canonical morphism $I \times A \union \{1\} \times X --> I \times X$ is a deformation retract.
	\end{lemma}
	\begin{sketch}
		Let $A \subseteq U \subseteq X$ and consider the diagram
		\begin{equation*}
			\begin{diagram}
				\threebythree[verywide]
					{\{1\}\times A}{\{1\}\times X}{I \times X}
					{I \times A}{I \times A \union \{1\} \times X}{}
					{I \times U}{}{I \times X}

				\arrow{nw}{n}{}
				\arrow{n}{ne}{}
				\arrow{w}{c}{}
				\arrow{nw}{w}{}
				\arrow{w}{sw}{}
				\arrow{n}{c}{}
				\arrow[equals]{ne}{se}{}
				\arrow{sw}{se}{}
				\arrow[dashed]{c}{se}{}

				\node at ($(c)+(-.5,.5)$) {$\secorner$};
			\end{diagram}
		\end{equation*}
	\end{sketch}

	\newpage
	\subsection{Mapping Cone, Suspension, Mapping Telescope}

	The following constructions will be used in the following text.

	\begin{definition}
		Let $f:X-->Y$ be a morphism of topological spaces. The \textbf{mapping cone $\Cone f$ of $f$} is the quotient space defined on $([0,1] \times X) \sqcup Y$ by the equivalence relation generated by $(0,x) \sim (0, x')$ and $(1,x) \sim f(x)$.
	\end{definition}

	\begin{lemma}[Properties of the Mapping Cone]
		TODO
	\end{lemma}

	\begin{definition}\vspace{.5em}
		\begin{minipage}{\linewidth-4cm}
			The \textbf{suspension} of a topological space $X$ is the topological space given by the pushout
			\begin{equation*}
				\begin{diagram}
					\twobytwo[verywide]
						{\{-1,1\}\times X}{[-1,1]\times X}
						{\{-1,1\}}{\Susp X}

					\arrow{nw}{ne}{}
					\arrow{sw}{se}{}
					\arrow{nw}{sw}{}
					\arrow{ne}{se}{}
					
					\node at ($(se)+(-.4,.4)$) {$\secorner$};
				\end{diagram}
			\end{equation*}
		\end{minipage}
		\begin{minipage}{4cm}
			\begin{center}
				\begin{tikzpicture}
					\draw[gray] (0,1) ellipse (.5cm and .25cm);
					\draw[gray] (0,-1) ellipse (.5cm and .25cm);
					\draw[gray] (-.5,1) -- (-.5,-1);
					\draw[gray] (.5,1) -- (.5,-1);

					\node[gray] at (1.5,1) {$[-1,1]\times X$};

					\draw (0,0) ellipse (.5cm and .25cm);
					\draw (-.5,0) -- (0,1) -- (.5,0) -- (0,-1) -- cycle;

					\node at (1.5,0) {$\Susp X$};
				\end{tikzpicture}
			\end{center}
		\end{minipage}
		\begin{minipage}{\linewidth-4cm}
			The \textbf{reduced suspension} of a pointed topological space $(X,x_0)$ is the pointed topological space given by the pushout
			\begin{equation*}
				\begin{diagram}
					\twobytwo[hyperwide]
						{(\{-1,1\}\times X)\sqcup([-1,1]\times \{x_0\})}{[-1,1]\times X}
						{*}{\Susp_{*}(X,x_0)}

					\arrow{nw}{ne}{}
					\arrow{sw}{se}{}
					\arrow{nw}{sw}{}
					\arrow{ne}{se}{}
					
					\node at ($(se)+(-.75,.5)$) {$\secorner$};
				\end{diagram}
			\end{equation*}
		\end{minipage}
		\begin{minipage}{4cm}
			\begin{center}
				\begin{tikzpicture}
					\draw (0,0) ellipse (.5cm and .25cm);
					\draw (-1,0) circle (.5cm);
					\draw[thick] (-.725,0) circle (.775cm);
					\draw (-.5,0) circle (1cm);

					\fill (0,-.25) circle (.05cm);

					\node at (1.75,0) {$\Susp_{*}(X,x_0)$};
				\end{tikzpicture}
			\end{center}
		\end{minipage}

		These constructions give functors $\Susp: \ncat{Top} --> \ncat{Top}$ and $\Susp_{*}: \ncat{Top}_{*} --> \ncat{Top}_{*}$.
	\end{definition}

	\begin{lemma}[Properties of Suspension]
		There is a canonical isomorphism $\Susp \S^n \isom \S^{n+1}$.

		\TODO{suspension vs quotients vs deformation retracts?}
	\end{lemma}
	\begin{sketch}
		The universal morphism induced by the continous maps
		\begin{equation*}
			\begin{array}{rcl}
				\{-1,1\} & \longrightarrow & \S^{n+1} \subseteq \R^{n+2}\\
				s & \longmapsto & (0,s)
			\end{array}
			\hspace{1cm}\text{and}\hspace{1cm}
			\begin{array}{rcl}
				{[-1,1]} \times \S^n & \longrightarrow & \S^{n+1} \subseteq \R^{n+2}\\
				(t,x) & \longmapsto & (\sqrt{1-t^2}\cdot x,t)
			\end{array}
		\end{equation*}
		is a continous bijection between the quasicompact space $\Susp X$ and the hausdorff space $\S^{n+1}$ and hence a homeomorphism.
	\end{sketch}

	\begin{definition}
		\begin{minipage}{\linewidth-5cm}
			Let $X_0 \xrightarrow{i_0} X_1 \xrightarrow{i_1} ... --> X = \colim \limits_{n \in \N} X_i$ be an linearly filtered diagram of topological spaces.

			\vspace{.5em}
			We inductively obtain a sequence $T_0 \subseteq T_1 \subseteq ...$ of topological spaces by letting $T_0 = [0,1] \times X_0$ and considering the pushouts
			\begin{equation*}
				\begin{diagram}
					\twobytwo[verywide]
						{X_n}{T_n}
						{[n+1,n+2]\times X_{n+1}}{T_{n+1},}

					\arrow{nw}{ne}{\scalarproduct{c_{n+1}}{1_{X_n}}}[above]
					\arrow{sw}{se}{}
					\arrow{nw}{sw}{\scalarproduct{c_{n+1}}{i_n}}[left]
					\arrow{ne}{se}{}

					\node at ($(se)+(-.4,.4)$) {\secorner};
					\node[gray] at ($(ne)+(1.65,0)$) {$\subseteq [0,n+1] \times X_n$};
				\end{diagram}
			\end{equation*}
			where $c_{n+1}$ denotes the obvious constant functions. 

			\vspace{.5em}
			The \textbf{mapping telescope} of the sequence $(X_n,i_n)_n$ is defined as
			\begin{equation*}
				\Tel X = \colim \limits_{n \in \N} T_n.
			\end{equation*}

			\vspace{.5em}
			Note that the mapping telescope admits the canonical arrow
			\begin{equation*}
				\begin{array}{rcl}
					\Tel X &\longrightarrow &X = \colim \limits_{n \in \N_0} X_n\\
					{[(t,x_n)]} & \longmapsto & \iota_n(x_n)
				\end{array}
			\end{equation*}
			%which is welldefined for all $t = n+1$, since $\iota_n(x_n) = \iota_{n+1}(i_n(x_n))$.
			induced by the universal property of $\Tel X$. It is in general \underline{not} an isomorphism.
		\end{minipage}
		\begin{minipage}{5cm}
			\begin{center}
				% \begin{tikzpicture}
				% 	\draw[gray] (0,-1.5) -- (0,1.5) -- (2.5,1.5) -- (2.5,-1.5) -- cycle;

				% 	\draw (0,-.5) -- (0,.5) -- (.5,.5) -- (.5,-.5) -- cycle;
				% 	\draw (1,-1) -- (1,1) -- (1.5,1) -- (1.5,-1) -- cycle;
				% 	\draw (2,-1.5) -- (2,1.5) -- (2.5,1.5) -- (2.5,-1.5) -- cycle;

				% 	\draw[very thick] (.5,.5) -- (.5,-.5);
				% 	\draw[very thick] (1,.5) -- (1,-.5);
				% 	\draw[very thick] (1.5,1) -- (1.5,-1);
				% 	\draw[very thick] (2,1) -- (2,-1);
				% 	%\draw[very thick] (2.5,1.5) -- (2.5,-1.5);

				% 	\draw[->] (.5,0) -- node[above]{$i_0$} (1,0);
				% 	\draw[->] (1.5,0) -- node[above]{$i_1$} (2,0);

				% 	\node[gray] at (.8,1.8) {$[0,3] \times X_2$};
				% \end{tikzpicture}
				\begin{tikzpicture}
					\draw[gray] (0,-1.5) -- (0,1.5) -- (3,1.5) -- (3,-1.5) -- cycle;

					\draw (0,-.5) -- (0,.5) -- (1,.5) -- (1,-.5) -- cycle;
					\draw (1,-1) -- (1,1) -- (2,1) -- (2,-1) -- cycle;
					\draw (2,-1.5) -- (2,1.5) -- (3,1.5) -- (3,-1.5) -- cycle;

					%\draw[very thick] (.5,.5) -- (.5,-.5);
					\draw[very thick] (1,.5) -- (1,-.5);
					%\draw[very thick] (1.5,1) -- (1.5,-1);
					\draw[very thick] (2,1) -- (2,-1);
					%\draw[very thick] (2.5,1.5) -- (2.5,-1.5);

					% \draw[->] (.5,0) -- node[above]{$i_0$} (1,0);
					% \draw[->] (1.5,0) -- node[above]{$i_1$} (2,0);

					\node[gray] at (.8,1.8) {$[0,3] \times X_2$};
					\node at (0,-1.8) {$0$};
					\node at (1,-1.8) {$1$};
					\node at (2,-1.8) {$2$};
					\node at (3,-1.8) {$3$};
					\node at (3.5,0) {$T_3$};
					\node at (.5,0) {$X_0$};
					\node at (1.5,0) {$X_1$};
					\node at (2.5,0) {$X_2$};
				\end{tikzpicture}
			\end{center}
		\end{minipage}
	\end{definition}

	\begin{lemma}[Properties of the Mapping Telescope]
		If $X_0 \subseteq X_1 \subseteq ... \subseteq X = \colim \limits_{n \in \N_0} X_n$ is a sequence of CNDRs, then the canonical morphism $$\Tel X --> X$$ is a homotopy equivalence.
	\end{lemma}
	\begin{sketch}
		Factor the morphism as $\Tel X \xrightarrow{i} [0,\infty) \times X \xrightarrow{\pi_X} X$ and note that, since $\pi_X$ is a homotopy equivalence, it suffices to show that $i$ is a homotopy equivalence.
	\end{sketch}

	\newpage
	\section{Generalized Homology}
	\subsection{Definition and Basic Properties}

	\begin{definition}
		A \textbf{generalized} (\textbf{unreduced, relative}) \textbf{homology theory} on $\ncat{Top}^2$ is a functor $H_\bullet: \ncat{Top}^2 --> \ncat{Mod}_R^{\Z\text{gr}}$ together with a natural transformation $\partial_\bullet: H_\bullet ==> H_{\bullet-1} \circ R$, the so called \textbf{boundary operator} or \textbf{connecting homomorphism}, which satisfy the following axioms.
	\begin{enumerate}[$\bullet$]
		\item{
			\textit{homotopy invariance}\\
			$H_\bullet$ factors via $\ncat{hTop}^2$.
		}
		\item{
			\textit{exactness}\\
			For any object $(X,A)$ in $\ncat{Top}^2$ the sequence
			\begin{equation*}
				...
				\begin{diagram}
					\fivebyone[verywide]
						{H_{\bullet+1}(X,A)}
						{H_\bullet(A,\emptyset)}
						{H_\bullet(X,\emptyset)}
						{H_\bullet(X,A)}
						{H_{\bullet-1}(A,\emptyset)}

					\arrow{ww}{w}{\partial_{\bullet+1}}[above]
					\arrow{w}{c}{H_\bullet(i)}[above]
					\arrow{c}{e}{H_\bullet(j)}[above]
					\arrow{e}{ee}{\partial_\bullet}[above]
				\end{diagram}
				...
			\end{equation*}
				is exact, where $i: (A,\emptyset) \longrightincl (X,\emptyset)$ and $j: (X,\emptyset) \longrightincl (X,A)$ are the inclusion maps.

		}
		\item{
			\textit{excission}\\
			For every object $(X,A)$ in $\ncat{Top}^2$ and subset $U \subseteq \closure U \subseteq \interior A$, the inclusion $(X\setminus U, A \setminus U) \subseteq (X, A)$ induces an isomorphism $$H_\bullet(X\setminus U, A \setminus U) \isom H_\bullet(X,A).$$
		}
	\end{enumerate}
	One might further assume
	\begin{enumerate}[$\bullet$]
		\item{
			\textit{additivity}\\
			$H_\bullet$ preserves coproducts. Specifically, for any family $(X_i,A_i)_i$ in $\ncat{Top}^2$ the canonical morphism
			\begin{equation*}
				\bigoplus \limits_{i \in I}H_\bullet(X_i,A_i) \longrightarrow H_\bullet(\Coproduct \limits_{i \in I} (X_i,Y_i))
			\end{equation*}
			is an isomorphism.
		}
		\item{
			\textit{dimension}\\
			For all $n \neq 0$ it holds that $H_n(*,\emptyset) = 0$.

			In this case we call $(H_\bullet,\partial_\bullet)$ an \textbf{ordinary} (\textbf{unreduced, relative}) \textbf{homology theory}.
		}
	\end{enumerate}
	\end{definition}

	In the following we will abbreviate $H_\bullet(X,\emptyset)$ to $H_\bullet(X)$.

	\begin{lemma}[Homology vs Homotopy]
		Let $(H_\bullet, \partial_\bullet)$ be a generalized homology theory.
		\begin{enumerate}[(i)]
			\item{
				If $f:(X_1,A_1) \xrightarrow{\simeq} (X_2,A_2)$ is a homotopy equivalence, then $H_\bullet(X_1,A_1) \isom H_\bullet(X_2,A_2)$.

				In particular for $n \in \N$ we have
				\begin{equation*}
					H_\bullet(*) \isom H_\bullet(\D^n) \isom H_\bullet(\R^n) \hspace{1cm}\text{and}\hspace{1cm} H_\bullet(\S^{n-1}) \isom H_\bullet(\D^n \setminus \{0\}).
				\end{equation*}
			}
			\item{
				If $A \subseteq X$ is a homotopy equivalence, then $H_\bullet(X,A) \isom 0$, in particular $H_\bullet(X,X) \isom 0$.
			}
			\item{
				If $A \subseteq X$ is a deformation retract of $U \subseteq X$, then the canonical morphism $H_\bullet(X,A) --> H_\bullet(X,U)$ is an isomorphism.
			}
			% \item{
			% 	If $H_\bullet$ is additive, it holds that $H_\bullet(X_\text{disc}) = \Directsum \limits_X H_\bullet(*)$.
			% }
		\end{enumerate}
	\end{lemma}
	\begin{proof}
		\begin{enumerate}[(i)]
			\item{
				By homotopy invariance $f$ becomes an isomorphism in $\ncat{hTop}$, which is then mapped to a isomorphism in homology.
			}
			\item{
				Consider the long exact sequence
				\begin{equation*}
					\begin{diagram}
						\fivebyone[verywide]
							{H_{\bullet}(A)}{H_{\bullet}(X)}{H_\bullet(X,A)}{H_{\bullet-1}(A)}{H_{\bullet-1}(X)}
						\arrow{ww}{w}{}
						\arrow{w}{c}{}
						\arrow{c}{e}{\partial_\bullet}[above]
						\arrow{e}{ee}{}
					\end{diagram}
				\end{equation*}
				and note that as in (i) we have that $H_\bullet(A) --> H_\bullet(X)$ is an isomorphism. Hence by exactness we have $H_\bullet(X,A) \isom 0$.
			}
			\item{
				Consider the following diagram and apply the 5-lemma.
				\begin{equation*}
					\begin{diagram}
						\fivebytwo[verywide]
							{H_\bullet(A)}{H_\bullet(X)}{H_\bullet(X,A)}{H_{\bullet-1}(A)}{H_{\bullet-1}(X)}
							{H_\bullet(U)}{H_\bullet(X)}{H_\bullet(X,U)}{H_{\bullet-1}(U)}{H_{\bullet-1}(X)}

						\arrow{nww}{nw}{}
						\arrow{nw}{n}{}
						\arrow{n}{ne}{\partial_\bullet}[above]
						\arrow{ne}{nee}{}

						\arrow{sww}{sw}{}
						\arrow{sw}{s}{}
						\arrow{s}{se}{\partial_\bullet}[below]
						\arrow{se}{see}{}

						\arrow{nww}{sww}{\isom}[left]
						\arrow[equals]{nw}{sw}{}
						\arrow{n}{s}{}
						\arrow{ne}{se}{\isom}[left]
						\arrow[equals]{nee}{see}{}
					\end{diagram}
				\end{equation*}
			}
		\end{enumerate}
	\end{proof}

	\begin{lemma}[Homology of Discrete Spaces]
		Let $(H_\bullet, \partial_\bullet)$ be a generalized additive homology theory. Then there is a canonical isomorphism
		\begin{equation*}
			H_\bullet(X_\text{disc}) \isom \Oplus_X H_\bullet(*).
		\end{equation*}
		Furthermore, given a map $f:X --> Y$ its homology can be calculated via the induced map
		\begin{equation*}
			\begin{array}{rcl}
				\Oplus_X H_\bullet(*) & \longrightarrow & \Oplus_Y H_\bullet(*)\vspace{.5em}\\(m_x)_x & \longmapsto & \Big(\Sum \limits_{x \in f^{-1}(y)}m_x\Big)_y
			\end{array}
		\end{equation*}
	\end{lemma}
	\begin{proof}
		Note that $X_\text{disc} \isom \Coproduct_X *$, hence $H_\bullet(X_\text{disc}) \isom H_\bullet(\Coproduct_X *) \isom \Oplus_X H_\bullet(*)$.

		By construction the assignment $x \mapsto f(x)$ induces the identity $1_{H_\bullet(*)}$. Hence $f$ yields an induced map
		\begin{equation*}
			\begin{array}{rcl}
				\Oplus_{f^{-1}(\{y\})} H_\bullet(*) &\longrightarrow& H_\bullet(*)\\
				(m_x)_x & \longmapsto & \Sum \limits_{x \in f^{-1}(y)} m_x.
			\end{array}
		\end{equation*}
		Using $\Oplus \limits_{y\in Y} \Oplus \limits_{f^{-1}(\{y\})} * \isom \Oplus_X *$ yields the desired result. 
	\end{proof}

	\begin{proposition}[Triple Sequence]
		Let $(H_\bullet,\partial_\bullet)$ be a generalized homology theory and $B \subseteq A \subseteq X$ in $\ncat{Top}$. Given
		\begin{equation*}
			\Delta_\bullet :=
			\begin{diagram}
				\threebyone[wide]
					{H_\bullet(X,A)}{H_{\bullet-1}(A)}{H_{\bullet-1}(A,B)}
				\arrow{w}{c}{\partial_\bullet}[above]
				\arrow{c}{e}{}
			\end{diagram}
		\end{equation*}
		the following sequence is exact and natural in $(A,B)$.
		\begin{equation*}
			...
			\begin{diagram}
				\fivebyone[verywide]
					{H_{\bullet+1}(X,A)}{H_\bullet(A,B)}{H_\bullet(X,B)}{H_\bullet(X,A)}{H_{\bullet-1}(A,B)}

				\arrow{ww}{w}{\Delta_{\bullet+1}}[above]
				\arrow{w}{c}{}[above]
				\arrow{c}{e}{}[above]
				\arrow{e}{ee}{\Delta_\bullet}[above]
			\end{diagram}
			...
		\end{equation*}
	\end{proposition}
	\begin{sketch}
		Consider the following diagram obtained by braiding the long exact sequences of \textcolor{red}{$(X,A)$}, \textcolor{darkgreen}{$(A,B)$} and \textcolor{blue}{$(X,B)$}.
		\begin{equation*}
			\hspace{-.5cm}
			\begin{diagram}
				\matrix[objects,wide]{
					\node[] (a1) {\ensuremath{H_{\bullet+1}(X,A)}};\&\&
					\node[] (a2) {\ensuremath{H_\bullet(A,B)}};\&\&
					\node[] (a3) {\ensuremath{H_{\bullet-1}(B)}};\&\&
					\node[] (a4) {\ensuremath{H_{\bullet-1}(X)}};\\\&
					\node[] (b1) {\ensuremath{H_\bullet(A)}};\&\&
					\node[] (b2) {\ensuremath{H_\bullet(X,B)}};\&\&
					\node[] (b3) {\ensuremath{H_{\bullet-1}(A)}};\&\\
					\node[] (c1) {\ensuremath{H_\bullet(B)}};\&\&
					\node[] (c2) {\ensuremath{H_\bullet(X)}};\&\&
					\node[] (c3) {\ensuremath{H_\bullet(X,A)}};\&\&
					\node[] (c4) {\ensuremath{H_{\bullet-1}(A,B)}};\\
				};

				\arrow[red]{a1}{b1}{\partial_{\bullet+1}}[above right]
				\arrow[red]{b1}{c2}{}
				\arrow[red,bend right]{c2}{c3}{}
				\arrow[red]{c3}{b3}{\partial_\bullet}[below right]
				\arrow[red]{b3}{a4}{}

				\arrow[darkgreen]{c1}{b1}{}
				\arrow[darkgreen]{b1}{a2}{}
				\arrow[darkgreen,bend left]{a2}{a3}{\partial_\bullet}[above]
				\arrow[darkgreen]{a3}{b3}{}
				\arrow[darkgreen]{b3}{c4}{}

				\arrow[blue,bend right]{c1}{c2}{}
				\arrow[blue]{c2}{b2}{}
				\arrow[blue]{b2}{a3}{\partial_\bullet}[above left]
				\arrow[blue,bend left]{a3}{a4}{}

				\arrow[bend left]{a1}{a2}{\Delta_{\bullet+1}}[above]
				\arrow{a2}{b2}{H_\bullet(i)}[above right]
				\arrow{b2}{c3}{H_\bullet(j)}[below left]
				\arrow[bend right]{c3}{c4}{\Delta_\bullet}[below]
			\end{diagram}
		\end{equation*}

		Complex at $H_\bullet(A,B)$ and $H_\bullet(X,A)$ follow immediately from the commutativity of the diagram.

		Complex at $H_\bullet(X,B)$: The composite $H_\bullet(A,B) --> H_\bullet(X,B) --> H_\bullet(X,A)$ factors through $H_\bullet(A,A) = 0$.

		Exactness at $H_\bullet(A,B)$: Consider $x \in \ker H_\bullet(i)$. By the following diagram chase we obtain $\Delta_{\bullet+1}(w) = x$.
		\begin{equation*}
			\begin{diagram}
				\matrix[objects,small]{
					\node[] (a1) {\ensuremath{}};\&\&
					\node[] (a2) {\ensuremath{x}};\&\&
					\node[] (a3) {\ensuremath{0}};\\\&
					\node[] (b1) {\ensuremath{y}};\&\&
					\node[] (b2) {\ensuremath{0}};\&\\
					\node[] (c1) {\ensuremath{u}};\&\&
					\node[] (c2) {\ensuremath{z}};\&\&
					\node[] (c3) {\ensuremath{}};\\
				};
				\node (v) at ($(b1)+(-.5,-.3)$) {\ensuremath{v}};

				\arrow[mapsto,bend left]{a2}{a3}{}
				\arrow[mapsto]{a2}{b2}{}
				\arrow[mapsto]{b1}{a2}{}
				\arrow[mapsto]{b1}{c2}{}
				\arrow[mapsto]{c2}{b2}{}
				\arrow[mapsto,bend right]{c1}{c2}{}
				\arrow[mapsto]{c1}{v}{}
				\arrow[mapsto]{v}{c2}{}
			\end{diagram}
			\text{ and }
			\begin{diagram}
				\matrix[objects,small]{
					\node[] (a1) {\ensuremath{w}};\&\&
					\node[] (a2) {\ensuremath{x}};\\\&
					\node[] (b1) {\ensuremath{y-v}};\&\\
					\node[] (c1) {\ensuremath{}};\&\&
					\node[] (c2) {\ensuremath{0}};\\
				};

				\arrow[mapsto,bend left]{a1}{a2}{}
				\arrow[mapsto]{a1}{b1}{}
				\arrow[mapsto]{b1}{c2}{}
				\arrow[mapsto]{b1}{a2}{}
			\end{diagram}
			\hspace{3cm}
		\end{equation*}

		Exactness at $H_\bullet(X,B)$: Let $x \in \ker H_\bullet(j)$ and by the following diagram chase we find $H_\bullet(i)(z+r) = x$.
		\begin{equation*}
			\begin{diagram}
				\matrix[objects,small]{
					\node[] (a1) {\ensuremath{z}};\&\&
					\node[] (a2) {\ensuremath{y}};\&\&
					\node[] (a3) {\ensuremath{}};\\\&
					\node[] (b1) {\ensuremath{x}};\&\&
					\node[] (b2) {\ensuremath{0}};\&\\
					\node[] (c1) {\ensuremath{}};\&\&
					\node[] (c2) {\ensuremath{0}};\&\&
					\node[] (c3) {\ensuremath{}};\\
				};
				\node (u) at ($(b1)+(-.3,.3)$) {\ensuremath{u}};

				\arrow[mapsto,bend left]{a1}{a2}{}
				\arrow[mapsto]{a2}{b2}{}
				\arrow[mapsto]{b1}{a2}{}
				\arrow[mapsto]{b1}{c2}{}
				\arrow[mapsto]{c2}{b2}{}
				\arrow[mapsto]{a1}{u}{}
				\arrow[mapsto]{u}{a2}{}
			\end{diagram}
			\hspace{-1cm}
			\text{ and }
			\hspace{-1cm}
			\begin{diagram}
				\matrix[objects,small]{
					\node[] (a1) {\ensuremath{}};\&\&
					\node[] (a2) {\ensuremath{r}};\&\&
					\node[] (a3) {\ensuremath{0}};\\\&
					\node[] (b1) {\ensuremath{w}};\&\&
					\node[] (b2) {\ensuremath{x-u}};\&\\
					\node[] (c1) {\ensuremath{}};\&\&
					\node[] (c2) {\ensuremath{v}};\&\&
					\node[] (c3) {\ensuremath{0}};\\
				};

				\arrow[mapsto]{b2}{c3}{}
				\arrow[mapsto]{c2}{b2}{}
				\arrow[mapsto,bend right]{c2}{c3}{}
				\arrow[mapsto]{b1}{c2}{}
				\arrow[mapsto]{b1}{a2}{}
				\arrow[mapsto]{b2}{a3}{}
				\arrow[mapsto]{a2}{b2}{}
			\end{diagram}
		\end{equation*}

		Exactness at $H_\bullet(X,A)$: Suppose $x \in \ker \Delta_\bullet$. By the following diagram chase we obtain $H_\bullet(j)(r+u) = x$.
		\begin{equation*}
			\begin{diagram}
				\matrix[objects,small]{
					\node[] (a1) {\ensuremath{}};\&\&
					\node[] (a2) {\ensuremath{z}};\&\&
					\node[] (a3) {\ensuremath{0}};\\\&
					\node[] (b1) {\ensuremath{u}};\&\&
					\node[] (b2) {\ensuremath{y}};\&\\
					\node[] (c1) {\ensuremath{}};\&\&
					\node[] (c2) {\ensuremath{x}};\&\&
					\node[] (c3) {\ensuremath{0}};\\
				};
				\node (v) at ($(c2)+(-.3,.3)$) {\ensuremath{v}};

				\arrow[mapsto,bend left]{a2}{a3}{}
				\arrow[mapsto]{a2}{b2}{}
				\arrow[mapsto]{b1}{a2}{}
				\arrow[mapsto]{b2}{a3}{}
				\arrow[mapsto,bend right]{c2}{c3}{}
				\arrow[mapsto]{c2}{b2}{}
				\arrow[mapsto]{b2}{c3}{}
				\arrow[mapsto]{b1}{v}{}
				\arrow[mapsto]{v}{b2}{}
			\end{diagram}
			\text{ and }
			\begin{diagram}
				\matrix[objects,small]{
					\node[] (a1) {\ensuremath{}};\&\&
					\node[] (a2) {\ensuremath{}};\&\&
					\node[] (a3) {\ensuremath{}};\\\&
					\node[] (b1) {\ensuremath{r}};\&\&
					\node[] (b2) {\ensuremath{0}};\&\\
					\node[] (c1) {\ensuremath{w}};\&\&
					\node[] (c2) {\ensuremath{x-v}};\&\&
					\node[] (c3) {\ensuremath{}};\\
				};

				\arrow[mapsto,bend right]{c1}{c2}{}
				\arrow[mapsto]{c2}{b2}{}
				\arrow[mapsto]{c1}{b1}{}
				\arrow[mapsto]{b1}{c2}{}
			\end{diagram}
			\vspace{-2em}
		\end{equation*}
	\end{sketch}

	\begin{proposition}[Mayer Vietoris]
		Let $(H_\bullet,\partial_\bullet)$ be a generalized homology theory and $B \subseteq A_1,A_2 \subseteq X$ in $\ncat{Top}$, such that $X \subseteq \interior(A_1) \union \interior(A_2)$.

		Name the canonical morphisms temporarily like
		\begin{equation*}
			\begin{diagram}
				\threebytwo[verywide]
					{(A_1 \intersection A_2, B)}{(A_1,B)}{(A_1,A_1\intersection A_2)}
					{(A_2,B)}{(X,B)}{(X,A_2)}

					\arrow{nw}{n}{i_1}[above]
					\arrow{nw}{sw}{i_2}[left]
					\arrow{n}{s}{j_1}[right]
					\arrow{sw}{s}{j_2}[below]
					\arrow{s}{se}{b}[below]
					\arrow{ne}{se}{e}[right]
					\arrow{n}{ne}{\widetilde{b}}[above]
			\end{diagram}
		\end{equation*}

		\TODO{By excission} $H_\bullet(e):H_\bullet(A_1,A_1 \intersection A_2) --> H_\bullet(X,A_2)$ is an isomorphism. Now given the composite morphism
		\begin{align*}
			\nu_\bullet := H_\bullet(X,B) \xrightarrow{H_\bullet(b)} H_\bullet(X,A_2) \xrightarrow{H_\bullet(e)^{-1}} H_\bullet(A_1, A_1 \intersection A_2) \xrightarrow{\Delta_\bullet^{(1,12,B)}} H_{\bullet-1} (A_1 \intersection A_2, B)
		\end{align*}
		the following sequence is exact.
		\begin{equation*}
			\begin{diagram}
				\threebythree[hyperwide]
					{}{}{...H_{\bullet+1}(X,B)}
					{H_\bullet(A_1 \intersection A_2,B)}{H_\bullet(A_1,B)\directsum H_\bullet(A_2,B)}{H_\bullet(X,B)}
					{H_{\bullet-1}(A_1 \intersection A_2, B)...}{}{}

				\arrow{w}{c}{H_\bullet(i_1)\directsum H_\bullet(i_2)}[above]
				\arrow{c}{e}{H_\bullet(j_1) - H_\bullet(j_2)}[above]
				
				\draw[->,rounded corners] (ne.east) -| node[auto,pos=.7]{$\nu_{\bullet+1}$}($(ne)+(1.5,-.75)$) -| ($(w)+(-2,0)$) |- (w.west);
				\draw[->,rounded corners] (e.east) -| node[auto,pos=.7]{$\nu_\bullet$}($(e)+(1.5,-.75)$) -| ($(sw)+(-2,0)$) |- (sw.west);
			\end{diagram}
		\end{equation*}
	\end{proposition}
	\begin{sketch}
		Consider the following diagram with exact rows given by the exact sequences of the triples $(A_1,A_1 \intersection A_2, B)$ and $(X,A_2,B)$.
		\begin{equation*}
			\hspace{-2cm}
			\begin{diagram}
				\fivebytwo[ultrawide]
					{H_{\bullet+1}(A_1,B)}{H_{\bullet+1}(A_1,A_1\intersection A_2)}{H_\bullet(A_1 \intersection A_2,B)}{H_\bullet(A_1,B)}{H_\bullet(A_1,A_1 \intersection A_2)}
					{H_{\bullet+1}(X,B)}{H_{\bullet+1}(X,A_2)}{H_\bullet(A_2,B)}{H_\bullet(X,B)}{H_\bullet(X,A_2)}

				\node[gray] (c) at ($0.5*(s)+0.5*(ne)$) {$\Oplus$};
				\arrow[gray,lower]{c}{s}{}
				\arrow[gray,lower]{c}{ne}{}
				\arrow[gray,higher]{s}{c}{}
				\arrow[gray,higher]{ne}{c}{}
				\arrow[dashed,gray]{n}{c}{}

				\arrow{nww}{nw}{H_{\bullet+1}(\widetilde{b})}[above]
				\arrow{nw}{n}{\Delta_{\bullet+1}^{(1,12,B)}}[above]
				\arrow{n}{ne}{H_\bullet(i_1)}[above]
				\arrow{ne}{nee}{H_\bullet(\widetilde{b})}[above]

				\arrow{sww}{sw}{H_{\bullet+1}(b)}[below]
				\arrow{sw}{s}{\Delta_{\bullet+1}^{(X,2,B)}}[below]
				\arrow{s}{se}{H_\bullet(j_2)}[below]
				\arrow{se}{see}{H_\bullet(b)}[below]

				\arrow{nww}{sww}{H_{\bullet+1}(j_1)}[left]
				\arrow{nw}{sw}{\hspace{.8cm}\isom\;\;H_{\bullet+1}(e)}
				\arrow{n}{s}{H_\bullet(i_2)}[right]
				\arrow{ne}{se}{H_\bullet(j_1)}[right]
				\arrow{nee}{see}{\hspace{.5cm}\isom\;\;H_\bullet(e)}
			\end{diagram}
		\end{equation*}

		Complex at $H_\bullet(A_1 \intersection A_2, B)$, using exactness of rows:
		\begin{align*}
			H_\bullet(i_1)\nu_{\bullet+1} &= H_\bullet(i_1)\Delta_{\bullet+1}^{(1,12,B)}H_{\bullet+1}(e)^{-1}H_{\bullet+1}(b) = 0H_{\bullet+1}(e)^{-1}H_{\bullet+1}(b) = 0\\
			H_\bullet(i_2)\nu_{\bullet+1} &= H_\bullet(i_2)\Delta_{\bullet+1}^{(1,12,B)}H_{\bullet+1}(e)^{-1}H_{\bullet+1}(b) = \Delta_{\bullet+1}^{(X,2,B)}H_\bullet(b) = 0
		\end{align*}

		Complex at $H_\bullet(A_1,B) \oplus H_\bullet(A_2,B)$:
		\begin{equation*}
			H_\bullet(j_1)\pi_1(H_\bullet(i_1)\oplus H_\bullet(i_2)) = H_\bullet(j_1)H_\bullet(i_1) = H_\bullet(j_2)H_\bullet(i_2) = H_\bullet(j_2)\pi_2(H_\bullet(i_1)\oplus H_\bullet(i_2))
		\end{equation*}

		Complex at $H_\bullet(X,B)$, using exactness of rows:
		\begin{align*}
			\nu_\bullet H_\bullet(j_1)\pi_1 &= \Delta_\bullet H_\bullet(e)^{-1}H_\bullet(b)H_\bullet(j_1)\pi_1 = \Delta_\bullet H_\bullet(\widetilde{b})\pi_1 = 0\pi_1 = 0\\
			\nu_\bullet H_\bullet(j_2)\pi_2 &= \Delta_\bullet H_\bullet(e)^{-1}H_\bullet(b)H_\bullet(j_2)\pi_2 = \Delta_\bullet H_\bullet(e)^{-1}0\pi_2 = 0
		\end{align*}

		Exactness at $H_\bullet(A_1\intersection A_2,B)$:
		\begin{tab}
			Let $x \in \ker H_\bullet(i_1) \intersection \ker H_\bullet(i_2)$. By exactness of the upper row and $H_{\bullet+1}(e)$ being an isomorphism we find $y \in H_{\bullet+1}(X,A_2)$ with $\Delta_{\bullet+1}^{(1,12,B)}H_{\bullet+1}(e)^{-1}(y) = x$. By assumption on $x$, commutativity of the corresponding square and exactness of the lower row we find $y \in \ker \Delta_{\bullet+1}^{(X,2,B)} = \im H_{\bullet+1}(b)$, so there is a $z \in H_{\bullet+1}(X,B)$ with $\nu_{\bullet+1}(z) = x$.
		\end{tab}

		Exactness at $H_\bullet(A_1,B) \oplus H_\bullet(A_2,B)$:
		\begin{tab}
			Let $(x_1,x_2) \in H_\bullet(A_1,B) \oplus H_\bullet(A_2,B)$ with $H_\bullet(i_1)(x_1) - H_\bullet(i_2)(x_2) = 0$. Then $0 = H_\bullet(b)H_\bullet(i_1)(x_1) - H_\bullet(b)H_\bullet(i_2)(x_2) = H_\bullet(b)H_\bullet(i_1)(x_1) = H_\bullet(e) H_\bullet(\widetilde{b})(x_1)$, so $x_1 \in \ker H_\bullet(\widetilde{b}) = \im H_\bullet(i_1)$ and we find $y \in H_\bullet(A_1 \intersection A_2, B)$ with $H_\bullet(i_1)(y) = x_1$. Furthermore $z = H_\bullet(i_2)(y)$ satisfies $H_\bullet(j_2)(z) = H_\bullet(j_2)(x_2)$, so $x_2-z \in \ker H_\bullet(j_2) = \im \Delta_{\bullet+1}^{(X,2,B)} = \im (\Delta_{\bullet+1}H_{\bullet+1}(e))$ and we find a corresponding $w \in H_{\bullet+1}(A_1,A_1\intersection A_2)$. Let $u = \Delta_{\bullet+1}^{(1,12,B)}(w)$ and note that $H_\bullet(i_1)(u+y) = 0+x_1$ and $H_\bullet(i_2)(u+y) = x_2 - z + z$.
		\end{tab}

		Exactness at $H_\bullet(X,B)$:
		\begin{tab}
			Let $x \in \ker \nu_\bullet = \ker \Delta_\bullet^{(1,12,B)}H_\bullet(e)^{-1}H_\bullet(b)$. Now either  $x \in \ker H_\bullet(b) = \im H_\bullet(j_2)$ and we find $z_2 \in H_\bullet(A_2,B)$ with $x = -H_\bullet(i_2)(z_2)$ or $y = H_\bullet(e)^{-1}H_\bullet(b)(x) \in \ker \Delta_\bullet^{(1,12,B)} = \im H_\bullet(\widetilde{b})$. In the latter case there is a $z_1 \in H_\bullet(A_1,B)$ such that $H_\bullet(e)^{-1}H_\bullet(b)H_\bullet(j_1)(z_1) = H_\bullet(\widetilde{b}) = y$. Hence $x$ can be represented as $x = H_\bullet(j_1)(z_1) - H_\bullet(j_2)(z_2)$.
		\end{tab}
		\TODO{diagram chase as diagrams}
	\end{sketch}

	\begin{proposition}[Homology of CNDRs]
		Let $(H_\bullet, \partial_\bullet)$ be a generalized homology theory and $(X,A)$ a pair of spaces such that $A$ is a CNDR of $X$.

		\begin{enumerate}[(i)]
			\item{
				The \textit{collapse map} $(X,A) --> (X/A,A/A)$ induces an isomorphism
				\begin{equation*}
					H_\bullet(X,A) \isom H_\bullet(X/A,A/A).
				\end{equation*}
			}
			\item{
				\begin{minipage}[t]{\linewidth-4cm}
					Recall that a pushout as depicted on the right makes $B$ a CNDR of $Y$. The map corresponding map $(X,A)-->(Y,B)$ induces an isomorphism
					\begin{equation*}
						H_\bullet(X,A) \isom H_\bullet(Y,B).
					\end{equation*}
				\end{minipage}
				\begin{minipage}[t]{4cm}
					\vspace{-1.5em}
					\begin{equation*}
						\begin{diagram}
							\twobytwo[small]
								{A}{B}
								{X}{Y}
							\arrow{nw}{ne}{}[above]
							\arrow{sw}{se}{}[below]
							\arrow{nw}{sw}{}
							\arrow{ne}{se}{}

							\node at ($(se)+(-.4,.4)$) {$\secorner$};
						\end{diagram}
					\end{equation*}
				\end{minipage}
			}
		\end{enumerate}
	\end{proposition}
	\begin{proof}
		\begin{enumerate}[(i)]
			\item{
				Let $A \subseteq U \subseteq X$ be an open neighborhood, which deformation retracts to $A$. As such, the triple exact sequence gives us isomorphisms $H_\bullet(X,A) \isom H_\bullet(X,U)$. Further, by excission we have an isomorphism $H_\bullet(X\setminus A,U \setminus A) \isom H_\bullet(X,U)$. Finally we note that there are canonical homeomorphisms $X \setminus A \isom (X / A) \setminus (A/A)$ and $U \setminus A \isom (U/A)\setminus(A/A)$. Putting things together the commutativity of the following diagram shows the claim.
				\begin{equation*}
					\hspace{-.2cm}
					\begin{diagram}
						\threebytwo[hyperwide]
							{H_\bullet(X,A)}{H_\bullet(X,U)}{H_\bullet(X\setminus A, U \setminus A)}
							{H_\bullet(X/A,A/A)}{H_\bullet(X/A,U/A)}{H_\bullet((X/A)\setminus(A/A), (U/A)\setminus(A/A))}

						\arrow{nw}{n}{\isom}[above]
						\arrow{sw}{s}{\isom}[below]

						\arrow{ne}{n}{\isom}[above]
						\arrow{se}{s}{\isom}[below]

						\arrow{nw}{sw}{}
						\arrow{n}{s}{}
						\arrow{ne}{se}{\isom}[right]
					\end{diagram}
				\end{equation*}
			}
			\item{
				\begin{minipage}[t]{\linewidth-4cm}
					Let $A \subseteq U \subseteq X$ and $B \subseteq V \subseteq Y$ be open neighborhoods of $A$ and $B$, which deformation retract to $A$ and $B$ respectively and fit into pushout squares as depicted on the right. \TODO{By construction of pushouts} in $\ncat{Top}$ we have that $X \setminus A \isom Y \setminus B$ and $U \setminus A \isom V \setminus B$. Thus we have obtain an isomorphism $H_\bullet(X\setminus A, U \setminus A) \isom H_\bullet(Y \setminus B, V \setminus B)$. Similar to (i) the commutativity of the following diagram gives the desired result.
				\end{minipage}
				\begin{minipage}[t]{4cm}
					\vspace{-1.5em}
					\begin{equation*}
						\begin{diagram}
							\twobythree[small]
								{A}{B}
								{U}{V}
								{X}{Y}
							\arrow{nw}{ne}{f}[above]
							\arrow{w}{e}{f'\vert_U^V}[below]
							\arrow{sw}{se}{f'}[below]

							\arrow{nw}{w}{}
							\arrow{w}{sw}{}
							\arrow{ne}{e}{}
							\arrow{e}{se}{}

							\node at ($(e)+(-.4,.4)$) {$\secorner$};
							\node at ($(se)+(-.4,.4)$) {$\secorner$};
						\end{diagram}
					\end{equation*}
				\end{minipage}
				\begin{equation*}
					\begin{diagram}
						\threebytwo[ultrawide]
							{H_\bullet(X,A)}{H_\bullet(X,U)}{H_\bullet(X \setminus A, U \setminus A)}
							{H_\bullet(Y,B)}{H_\bullet(Y,V)}{H_\bullet(Y \setminus B, V \setminus B)}

						\arrow{nw}{n}{\isom}[above]
						\arrow{sw}{s}{\isom}[below]

						\arrow{ne}{n}{\isom}[above]
						\arrow{se}{s}{\isom}[below]

						\arrow{nw}{sw}{}
						\arrow{n}{s}{}
						\arrow{ne}{se}{\isom}[right]
					\end{diagram}
				\end{equation*}
			}
		\end{enumerate}
	\end{proof}


	% \item{
	% 	Let $A \subseteq X_0 \subseteq X_1 \subseteq ... \subseteq X$ be a chain of CNDRs. Then there is a \TODO{canonical?} isomorphism
	% 	\begin{equation*}
	% 		\colim \limits_{i \in I} H_\bullet(X,A) \isom H_\bullet(\Tel X, A).
	% 	\end{equation*}
	% }

	% \item{
	% 	Use the following decomposition of the nonnegative real line
	% 	\begin{center}
	% 	\begin{tikzpicture}
	% 		% \node at (-4,.5) {$[0,1) \sqcup \Coproduct \limits_{n \in \N}(2(n-1)-\tfrac{1}{4},2(n-1)+\tfrac{1}{4})$};
	% 		% \node at (-.5,-.5) {$A_2$};
	% 		% \node at (-1,0) {$A_1 \intersection A_2$};

	% 		\node (x0) at (0,.5){};
	% 		\node (y0) at (0,0){};
	% 		\node (z0) at (0,-.5){};

	% 		\node at (6.2,.5){...};
	% 		\node at (6.2,0){...};
	% 		\node at (6.2,-.5){...};

	% 		\draw[|-] (x0) -- ($(x0)+(1,0)$);
	% 		\draw[|-] ($(x0)+(1,0)$) -- ($(x0)+(2,0)$);
	% 		\draw[|-] ($(x0)+(2,0)$) -- ($(x0)+(3,0)$);
	% 		\draw[|-] ($(x0)+(3,0)$) -- ($(x0)+(4,0)$);
	% 		\draw[|-] ($(x0)+(4,0)$) -- ($(x0)+(5,0)$);
	% 		\draw[|-] ($(x0)+(5,0)$) -- ($(x0)+(6,0)$);
	% 		\draw[|-] (y0) -- ($(y0)+(1,0)$);
	% 		\draw[|-] ($(y0)+(1,0)$) -- ($(y0)+(2,0)$);
	% 		\draw[|-] ($(y0)+(2,0)$) -- ($(y0)+(3,0)$);
	% 		\draw[|-] ($(y0)+(3,0)$) -- ($(y0)+(4,0)$);
	% 		\draw[|-] ($(y0)+(4,0)$) -- ($(y0)+(5,0)$);
	% 		\draw[|-] ($(y0)+(5,0)$) -- ($(y0)+(6,0)$);
	% 		\draw[|-] (z0) -- ($(z0)+(1,0)$);
	% 		\draw[|-] ($(z0)+(1,0)$) -- ($(z0)+(2,0)$);
	% 		\draw[|-] ($(z0)+(2,0)$) -- ($(z0)+(3,0)$);
	% 		\draw[|-] ($(z0)+(3,0)$) -- ($(z0)+(4,0)$);
	% 		\draw[|-] ($(z0)+(4,0)$) -- ($(z0)+(5,0)$);
	% 		\draw[|-] ($(z0)+(5,0)$) -- ($(z0)+(6,0)$);

			
	% 		\draw[very thick,red] (x0) -- ($(x0)+(1.25,0)$);
	% 		\draw[very thick,red] ($(x0)+(1.75,0)$) -- ($(x0)+(3.25,0)$);
	% 		\draw[very thick,red] ($(x0)+(3.75,0)$) -- ($(x0)+(5.25,0)$);
	% 		\draw[very thick,red] ($(x0)+(5.75,0)$) -- ($(x0)+(6,0)$);

	% 		\draw[very thick,darkgreen] ($(y0)+(.75,0)$) -- ($(y0)+(1.25,0)$);
	% 		\draw[very thick,darkgreen] ($(y0)+(1.75,0)$) -- ($(y0)+(2.25,0)$);
	% 		\draw[very thick,darkgreen] ($(y0)+(2.75,0)$) -- ($(y0)+(3.25,0)$);
	% 		\draw[very thick,darkgreen] ($(y0)+(3.75,0)$) -- ($(y0)+(4.25,0)$);
	% 		\draw[very thick,darkgreen] ($(y0)+(4.75,0)$) -- ($(y0)+(5.25,0)$);
	% 		\draw[very thick,darkgreen] ($(y0)+(5.75,0)$) -- ($(y0)+(6,0)$);

	% 		\draw[very thick,blue] ($(z0)+(0.75,0)$) -- ($(z0)+(2.25,0)$);
	% 		\draw[very thick,blue] ($(z0)+(2.75,0)$) -- ($(z0)+(4.25,0)$);
	% 		\draw[very thick,blue] ($(z0)+(4.75,0)$) -- ($(z0)+(6,0)$);		
	% 	\end{tikzpicture}
	% 	\end{center}
	% 	to obtain an decomposition of $\Tel X$
	% }

	\newpage
	\subsection{Reduced Homology}

	\begin{lemma}
		Let $(H_\bullet, \partial_\bullet)$ be a generalized homology theory and $x$ be a point in a topological space $X$. Then
		\begin{equation*}
			H_\bullet(X) \isom H_\bullet(\{x\}) \directsum H_\bullet(X,\{x\}) \isom H_\bullet(*) \oplus H_\bullet(X,\{x\}).
		\end{equation*}
	\end{lemma}
	\begin{proof}
		The constant map $r:X -->> \{x\}$ and the embedding $i:\{x\} \longrightincl X$ form an embedding retraction pair. Hence, by functoriality of $H_\bullet$, $H_\bullet(i)$ is split mono. For $n \in \Z$ the exact sequence
		\begin{equation*}
			\begin{diagram}
				\fivebyone[verywide]
					{H_n(\{x\})}{H_n(X)}{H_n(X,\{x\})}{H_{n-1}(\{x\})}{H_{n-1}(X)}

				\arrow[mono]{ww}{w}{H_n(i)}[above]
				\arrow[]{w}{c}{}[above]
				\arrow[]{c}{e}{\partial_n}[above]
				\arrow[mono]{e}{ee}{H_{n-1}(i)}[above]
			\end{diagram}
		\end{equation*}
		yields $\partial_n = 0$, so $\ker \partial_n = H_n(X,\{x\})$ and we obtain the split exact sequence
		\begin{equation*}
			\begin{diagram}
				\fivebyone[wide]
					{0}{H_n(\{x\})}{H_n(X)}{H_n(X,\{x\})}{0,}

				\arrow{ww}{w}{}
				\arrow{w}{c}{H_n(i)}[above]
				\arrow{c}{e}{}[above]
				\arrow{e}{ee}{}
				\arrow[bend left]{c}{w}{H_n(r)}[below]
			\end{diagram}
		\end{equation*}
		from which the desired isomorphism can be deduced.
	\end{proof}

	\begin{definition}
		The \textbf{module of coefficients} of a generalized homology theory $(H_\bullet, \partial_\bullet)$ is the graded module $M_\bullet := H_\bullet(*)$.

		The \textbf{reduced homology} of $H_\bullet$ is the functor \TODO{involve kernel}
		\begin{equation*}
			\begin{array}{rcl}
				\ncat{Top}_* & \xrightarrow{\widetilde{H}_\bullet} & \ncat{Mod}_R^{\Z\text{gr}}\\
				(X,x_0) & \longmapsto & H_\bullet(X,\{x_0\})\\
				f:(X,x_0)\rightarrow (Y,y_0) & \longmapsto & H_\bullet(f)
			\end{array}
		\end{equation*}
	\end{definition}

	\begin{remark}
		Let $(H_\bullet, \partial_\bullet)$ be an ordinary homology theory and $x$ be a point in a topological space $X$. For $n \neq 0$ there is by construction an isomorphism
		\begin{equation*}
			H_n(X) \isom \widetilde{H}_n(X,x).
		\end{equation*}
	\end{remark}

	\begin{lemma}[Reduced Mayer Vitoris]
		Let $(H_\bullet,\partial_\bullet)$ be a generalized homology theory and $A_1,A_2 \subseteq X$ in $\ncat{Top}$, such that $X \subseteq \interior(A_1) \union \interior(A_2)$ and $x_0 \in A_1 \intersection A_2$. Then the following sequence is exact.
		\begin{equation*}
			\begin{diagram}
				\threebythree[hyperwide]
					{}{}{...\widetilde{H}_{\bullet+1}(X,x_0)}
					{\widetilde{H}_\bullet(A_1 \intersection A_2,x_0)}{\widetilde{H}_\bullet(A_1,x_0)\directsum \widetilde{H}_\bullet(A_2,x_0)}{\widetilde{H}_\bullet(X,x_0)}
					{\widetilde{H}_{\bullet-1}(A_1 \intersection A_2,x_0)...}{}{}

				\arrow{w}{c}{H_\bullet(i_1)\directsum H_\bullet(i_2)}[above]
				\arrow{c}{e}{H_\bullet(j_1) - H_\bullet(j_2)}[above]
				
				\draw[->,rounded corners] (ne.east) -| node[auto,pos=.7]{$\nu_{\bullet+1}$}($(ne)+(1.5,-.75)$) -| ($(w)+(-2,0)$) |- (w.west);
				\draw[->,rounded corners] (e.east) -| node[auto,pos=.7]{$\nu_\bullet$}($(e)+(1.5,-.75)$) -| ($(sw)+(-2,0)$) |- (sw.west);
			\end{diagram}
		\end{equation*}
	\end{lemma}
	\begin{sketch}
		Follows immediately from unreduced Mayer Vietories.
	\end{sketch}

	\begin{lemma}
		Let $(H_\bullet,\partial_\bullet)$ be a generalized homology theory.
		\begin{enumerate}[(i)]
			\item{
				There are natural isomorphisms (called \textbf{suspension morphisms})
				\begin{equation*}
					\sigma_\bullet: \widetilde{H}_{\bullet}(X,x_0) \isom \widetilde{H}_{\bullet+1}(\Susp X, x_0).
				\end{equation*}
			}
			\item{
				$\widetilde{H}_\bullet$ is homotopy invariant, i.e. factors via $\ncat{hTop}_{*}$.
			}
			\item{
				Given $i_A:A \longrightincl X$ in $\ncat{Top}$ and $x_0 \in A$ it holds that $H_\bullet(X,A) \isom \widetilde{H}_\bullet(\Cone(i_A), x_0)$. Moreover the following sequence is exact.
				\begin{equation*}
					\begin{diagram}
						% \threebythree[ultrawide]
						% 	{}{}{...\widetilde{H}_{\bullet+1}(\Cone(i_A),x_0)}
						% 	{\widetilde{H}_\bullet(A,x_0)}{\widetilde{H}_\bullet(X,x_0)}{\widetilde{H}_\bullet(\Cone(i_A),x_0)}
						% 	{\widetilde{H}_{\bullet-1}(A,x_0)...}{}{}

						\threebyone[verywide]
							{\widetilde{H}_\bullet(A,x_0)}{\widetilde{H}_\bullet(X,x_0)}{\widetilde{H}_\bullet(\Cone(i_A),x_0).}

						\arrow{w}{c}{}
						\arrow{c}{e}{}

						%\draw[->,rounded corners] (ne.east) -| node[auto,pos=.7]{${\partial}_{\bullet+1}$}($(ne)+(2,-.75)$) -| ($(w)+(-1.5,0)$) |- (w.west);
						%\draw[->,rounded corners] (e.east) -| node[auto,pos=.7]{${\partial}_\bullet$}($(e)+(2,-.75)$) -| ($(sw)+(-1.5,0)$) |- (sw.west);
					\end{diagram}
				\end{equation*}
			}
		\end{enumerate}
		If $H_\bullet$ is additive we further have
		\begin{enumerate}
			\item[(iv)]{
				Let $(X_i,x_i)_{i \in I}$ be a family \textit{wellpointed} spaces (i.e. each $x_i$ has a neighborhood, which deformation retracts to it). Then
				\begin{equation*}
					\widetilde{H}_\bullet(\Join \limits_{i\in I}X_i, x_0) \isom \Directsum \limits_{i \in I}\widetilde{H}_\bullet(X_i, x_i).
				\end{equation*}
			}
		\end{enumerate}
	\end{lemma}
	\begin{proof}
		\begin{enumerate}[(i)]
			\item{
				Denote the points $x_1 = [\{1\},x_0]$ and $x_2 = [\{-1\},x_0]$ in $\Susp X$ and consider the decomposition of $\Susp X$ into $A_1 = \Susp(X) \setminus \{x_1\}$ and $A_2 = \Susp(X) \setminus \{x_2\}$. Note that by construction $A_1 \simeq * \simeq A_2$ and $A_1 \intersection A_2 \simeq X$. Hence Reduced Mayer Vietoris gives an isomorphism
				\begin{equation*}
					\begin{diagram}
						\fourbyone[verywide]
							{0 \oplus 0}{\widetilde{H}_{\bullet+1}(\Susp X, x_0)}{\widetilde{H}_\bullet(X)}{0 \oplus 0}

						\arrow{ww}{w}{}
						\arrow{w}{e}{\nu_{\bullet+1}}[above]
						\arrow{e}{ee}{}
					\end{diagram}
				\end{equation*}
			}
			\item{
				By definition.
			}
			\item{

			}
			\item{
				Let $U_i$ be a neighborhood of $x_i$, which deformation retracts to $x_i$. We calculate
				\begin{equation*}
					\begin{array}{rcl}
						H_\bullet(\Join \limits_{i \in I}X_i,x_{*}) &\underset{\text{homotopy}}{\isom}& H_\bullet(\Join \limits_{i \in I} X_i, U_i)\\[1em]
						&\underset{\text{excission}}{\isom}& H_\bullet((\Join \limits_{i \in I}X_i)\setminus\{x_{*}\},(\Join \limits_{i \in I} U_i) \setminus \{x_{*}\})\\[1em]
						&=& H_\bullet(\Coproduct \limits_{i \in I}X_i \setminus\{x_i\},\Coproduct \limits_{i \in I}U_i \setminus\{x_i\})\\[1em]
						&\underset{\text{additivity}}{\isom}& \Directsum \limits_{i \in I}H_\bullet(X_i \setminus \{x_i\}, U_i \setminus \{x_i\})\\[1em]
						&\underset{\text{excission}}{\isom}& \Directsum \limits_{i \in I}H_\bullet(X_i, U_i) \underset{\text{homotopy}}{\isom} \Directsum \limits_{i \in I} H_\bullet(X_i,x_i).
					\end{array}
				\end{equation*}
			}
		\end{enumerate}
	\end{proof}

	\begin{remark}
		The properties of reduced homology stated in the previous lemma can be used as axioms for generalized reduced homology theories.
	\end{remark}

	\newpage
	\subsection{Examples and Applications}

	\begin{lemma}[Homology of the Sphere]
		Let $(H_\bullet, \partial_\bullet)$ be a generalized homology theory satisfying additivity and $n \in \N$. There are isomorphisms
		\begin{equation*}
			\widetilde{H}_\bullet(\S^n,*) \isom H_\bullet(\D^n,\S^{n-1}) \isom \widetilde{H}_{\bullet-1}(\S^{n-1},*),
		\end{equation*}
		which imply
		\begin{equation*}
			H_\bullet(\S^n) \isom M_\bullet \oplus M_{\bullet-n} \hspace{1cm}\text{and}\hspace{1cm} \widetilde{H}_\bullet(\S^n,*) \isom M_{\bullet-n}.
		\end{equation*}
	\end{lemma}
	\begin{proof}
		Consider the canoncial decomposition $\S^n = \S^n_{+} \union \S^n_{-}$ into upper and lower halfs and let $x_0$ denote the north pole in $\S^n$. Note that by homotopy invariance $\widetilde{H}_\bullet(\S^n_{+},x_0) \isom \widetilde{H}_\bullet(\D^n,0) \isom 0$. Hence by the long exact sequence of $\widetilde{H}_\bullet$ we have
		\begin{equation*}
			\begin{diagram}
				\fourbyone[ultrawide]
					{0\isom\widetilde{H}_\bullet(\S^n_{+},x_0)}{\widetilde{H}_\bullet(\S^n,x_0)}{H_\bullet(\S^n,\S^n_{+})}{\widetilde{H}_{\bullet-1}(\S^n_{+},x_0)\isom0}
				\arrow{ww}{w}{}
				\arrow{w}{e}{}
				\arrow{e}{ee}{\partial_\bullet}[above]
			\end{diagram},
		\end{equation*}
		which gives an isomorphism $\widetilde{H}_\bullet(\S^n,x_0) \isom H_\bullet(\S^n,\S^n_{+})$. By applying excision of $x_0$ and using homotopy invariance we get a chain of isomorphisms
		\begin{equation*}
			\widetilde{H}_\bullet(\S^n,x_0) \isom H_\bullet(\S^n,\S^n_{+}) \isom H_\bullet(\S^n \setminus \{x_0\}, \S^n_{+} \setminus \{x_0\}) \isom H_\bullet(\D^n,\S^{n-1}).
		\end{equation*}
		Denote by $x_1$ the north pole in $\S^{n-1}$. Again the long exact sequence of $\widetilde{H}_\bullet$ gives
		\begin{equation*}
			\begin{diagram}
				\fourbyone[ultrawide]
					{0 \isom \widetilde{H}_\bullet(\D^n,x_1)}{H_\bullet(\D^n,\S^{n-1})}{\widetilde{H}_{\bullet-1}(\S^{n-1},x_1)}{\widetilde{H}_{\bullet-1}(\D^n,x_1) \isom 0}
				\arrow{ww}{w}{}
				\arrow{w}{e}{\partial_\bullet}[above]
				\arrow{e}{ee}{}
			\end{diagram}
		\end{equation*}
		and we get an isomorphism $H_\bullet(\D^n,\S^{n-1}) \isom \widetilde{H}_{\bullet-1}(\S^{n-1})$. We can now explicitly calculate the homology of $\S^n$ (with $n \in \N_0$ now) by induction.
		\begin{enumerate}
			\item[(IB)]{
				For $n=0$ we have by additivity
				\begin{equation*}
					H_\bullet(\S^0) = H_\bullet(* \sqcup *) \isom H_\bullet(*) \oplus H_\bullet(*) = M_\bullet \oplus M_\bullet
				\end{equation*}
				and further \TODO{why}
				\begin{equation*}
					\widetilde{H}_\bullet(\S^n,*) = M_\bullet.
				\end{equation*}
			}
			\item[(IH)]{
				For $k<n$ it holds that $\widetilde{H}_\bullet(\S^k,*) \isom M_{\bullet-k}$.
			}
			\item[(IS)]{
				For $n$ we calculate
				\begin{equation*}
					\widetilde{H}_\bullet(\S^n,*) \isom \widetilde{H}_{\bullet-1}(\S^{n-1},*) \underset{(IH)}{\isom} M_{\bullet-1-(n-1)} = M_{\bullet-n}
				\end{equation*}
				and finally
				\begin{equation*}
					H_\bullet(\S^n) \isom M_\bullet \oplus H_\bullet(\S^n,*) \isom M_\bullet \oplus M_{\bullet-n}.
				\end{equation*}
			}\vspace{-3.5em}
		\end{enumerate}
	\end{proof}

	\begin{theorem}[Invariance of Dimension]
		Let $(H_\bullet, \partial_\bullet)$ be an ordinary additive homology theory, ie. satisfying additivity and dimension.

		Let $U \subseteq \R^m$ and $V \subseteq \R^n$ be open subsets. If there is a homeomorphism $f:U-->V$, then $m = n$.
	\end{theorem}
	\begin{proof}
		Note that for a point $x \in U$ we can calculate by homotopy invariance
		\begin{equation*}
			H_\bullet(U,U\setminus\{x\}) \isom H_\bullet(\B^m(x,r),\B^m(x,r)\setminus\{x\}) \isom H_\bullet(\D^m,\S^{m-1}) \isom \widetilde{H}_\bullet(\S^m) \isom M_{\bullet-m}
		\end{equation*}
		and similarly for $y \in V$. By the dimension axiom we then have that 
		\begin{equation*}
			M_{\bullet-m} \isom H_\bullet(U,U\setminus\{x\}) \isom H_\bullet(V,V\setminus\{f(x)\}) \isom M_{\bullet-n}
		\end{equation*}
		implies $m=n$.
	\end{proof}

	\begin{theorem}[Brouwers Fixed Point Theorem]
		Let $(H_\bullet, \partial_\bullet)$ be a nontrivial ordinary additive homology theory.

		For any $n\in \N$ $\D^n$ does not retract onto it's boundary $\S^{n-1}$. This is equivalent to the statement that every continuous map $f:\D^n --> \D^n$ has a fixed point.
	\end{theorem}
	\begin{sketch}
		\textit{Assume} there is a retraction $r: \D^n --> \S^{n-1}$ of the embedding $i:\S^{n-1} --> \D^n$.
		\begin{tab}[1.3cm]
			Then $\id_{M_0} = \id_{\widetilde{H}_{n-1}(\S^{n-1})} = \widetilde{H}_{n-1}(\id_{\S^{n-1}}) = \widetilde{H}_{n-1}(ri)$ factorizes over $\widetilde{H}_{n-1}(\D^n,p) = 0$, where $p$ is a point on $\S^{n-1}$. A contradiction.
		\end{tab}

		If $f:\D^n --> \D^n$ has no fixed point, we can define a continous map 
		\begin{equation*}
			\begin{array}{rccl}
				r:&\D^n & \longrightarrow & \S^{n-1}\\
				&p & \longmapsto & f(p) + \sigma(p)(p - f(p)),
			\end{array}
		\end{equation*}
		where $\sigma:\D^n --> [0,\infty)$ is the continous function calculating the intersection of the ray from $f(p)$ to $p$ with the boundary, ie. satisfying $\Vert f(p) + \sigma(p)(p - f(p))\Vert = 1$. Clearly this defines a retraction of $\S^n$.

		Conversely, suppose there is a retraction $r:\D^n --> \S^{n-1}$ and consider 
		\begin{equation*}
			\begin{diagram}
				\fourbyone
					{\D^n}{\S^{n-1}}{\S^{n-1}}{\D^n,}

				\arrow{ww}{w}{r}[above]
				\arrow{w}{e}{a}[above]
				\arrow{e}{ee}{i}[above]
			\end{diagram}
		\end{equation*}
		where $a: \S^{n-1} --> \S^{n-1}$ is the antipodal map. This map has no fixed points.
	\end{sketch}

	\begin{lemma}[Homology of Wedge Sums of Circles]
		\TODO{Wedge sums of circles}
	\end{lemma}

	\begin{lemma}[Homology of the Oriented Surfaces]
		\TODO{Torus and higher genus}
	\end{lemma}

	\newpage
	\subsection{Degree and Further Applications}

	To abbreviate following statements, we make the following nonstandard definition.

	\begin{definition}
		Let $R$ be a (commutative unital) ring. An $R$-\textbf{homology theory} is an ordinary additive homology theory, whose module of coefficients is $M_0 = R$.
	\end{definition}

	Recall that any morphism of $R$-modules $f:R --> R$ is given by multiplication with $r = f(1)$.

	\begin{definition}
		Let $(H_\bullet,\partial_\bullet)$ be an $R$-homology theory, let $n \in \N$ and fix an isomorphism $H_n(\S^n) \isom R$. The \textbf{$R$-degree} of a morphism $f:\S^n --> \S^n$ in $\ncat{Top}$ is the element $\deg_R(f) := \tilde{f}(1)$ in $R$ for the morphism $\tilde f$ defined by the composition
		\begin{equation*}
			R \isom H_n(\S^n) \xrightarrow{H_n(f)} {H}_n(\S^n)\isom R
		\end{equation*}

		By construction $\deg_R(gf) = \deg_R(g)\deg_R(f)$ and $\deg_R(\id) = 1$ in $R$. Thus we have a morphism of monoids
		\begin{equation*}
			\deg_R: \End_\ncat{Top}(\S^n) \longrightarrow (R,\cdot).
		\end{equation*}
	\end{definition}

	\begin{lemma}
		Let $(H,\partial_\bullet)$ be an $R$-homology theory. Fix an isomorphism $H_n(\S^n) \isom R$ for each $n \geq 1$.

		For $n = 0$ all morphisms have $R$-degree $\pm 1$ or $0$ and for $n \geq 1$ \TODO{at least} all values in $\chi(\Z) \subseteq R$ can appear. 
	\end{lemma}
	\begin{sketch}
		For $n \geq 1$ we use induction on $n$.
		\begin{enumerate}
			\item[(IB)]{
				Consider $\S^1 \subseteq \C$.

				For $m = 0$ denote $c_1: \S^1 --> \{1\} --> \S^1$ and note $H_1(c_1) = 0$.

				For $m > 0$ define the map 
				\begin{equation*}
					\begin{array}{rccl}
						f_m: &\S^1 &\longrightarrow& \S^1\\
						&z &\longmapsto& z^m
					\end{array},
				\end{equation*}
				\TODO{finish with Hannes-Kammeyer}
				% note that the covering space 
				% \begin{equation*}
				% 	S_m = \{(\cos (\tfrac{t}{m}), \sin (\tfrac{t}{m}), \tfrac{t}{2\pi}) \mid t \in [0,2\pi m]\} / \sim
				% \end{equation*}
				% with $(1,0,0) \sim (1,0,m)$ is $\S^1$ and that its projection $p:S_m --> \S^1$ is $f_m$. Consider the decomposition of $\S^1$ into $A_1 = \S^1 \setminus \{1\} \simeq *$ and $A_2 = \S^1 \setminus \{-1\} \simeq *$. The preimages under $p$ give a decomposition of $S_m$ into sets $B_1$ and $B_2$ and we find $B_1,B_2 \simeq \Coproduct \limits_{i=1}^m *$. Mayer Vietories yields a diagram
				% \begin{equation*}
				% 	\begin{diagram}
				% 		\fourbytwo[ultrawide]
				% 			{0}{H_1(S_m)}{H_0(B_1 \intersection B_2)}{H_0(B_1) \oplus H_0(B_2)}
				% 			{0}{H_1(\S^1)}{H_0(A_1 \intersection A_2)}{H_0(A_1) \oplus H_0(A_2)}

				% 		\arrow{nww}{nw}{}
				% 		\arrow{nw}{ne}{\nu_1}[above]
				% 		\arrow{ne}{nee}{}

				% 		\arrow{sww}{sw}{}
				% 		\arrow{sw}{se}{\nu_1}[below]
				% 		\arrow{se}{see}{}

				% 		\arrow[equals]{nww}{sww}{}
				% 		\arrow{nw}{sw}{H_1(p)}[ontop]
				% 		\arrow{ne}{se}{H_1(p|_{B_1 \intersection B_2})}[ontop]
				% 		\arrow{nee}{see}{H_1(p|B_1) \oplus H_1(p|B_2)}[ontop]
				% 	\end{diagram}
				% \end{equation*}
				% Using homotopy invariance we can translate this into a diagram, where the right square is given by homologies of discrete spaces.
				% \begin{equation*}
				% 	\begin{diagram}
				% 		\fourbytwo[ultrawide]
				% 			{0}{R}{\Oplus \limits_{k=1}^m(R \oplus R)}{\Oplus \limits_{i=1}^m R \oplus \Oplus \limits_{j=1}^m R}
				% 			{0}{R}{R \oplus R}{R \oplus R}

				% 		\arrow{nww}{nw}{}
				% 		\arrow{nw}{ne}{u}[above]
				% 		\arrow{ne}{nee}{a}[above]

				% 		\arrow{sww}{sw}{}
				% 		\arrow{sw}{se}{v}[below]
				% 		\arrow{se}{see}{b}[below]

				% 		\arrow[equals]{nww}{sww}{}
				% 		\arrow{nw}{sw}{s}[left]
				% 		\arrow{ne}{se}{t}[right]
				% 		\arrow{nee}{see}{}
				% 	\end{diagram}
				% \end{equation*}
				% By construction of Mayer Vietories and the homology of maps between discrete spaces we have
				% \begin{align*}
				% 	a(y_1^1,y_1^2,...,y_m^1,y_m^2) &= (y_1^1+y_1^2,...,y_m^1+y_m^2,y_1^1+y_m^2,y_2^1+y_1^2,...,y_m^1+y_{m-1}^2)\\
				% 	b(w_1,w_2) &= (w_1+w_2,w_1+w_2)\\
				% 	t(y_1^1,y_1^2,...,y_m^1,y_m^2) &= y_1^1+...+y_m^1,y_1^2 + ... + y_m^2)
				% \end{align*}
				% From exactness of the rows, we get that
				% \begin{align*}
				% 	u(x) &= (\alpha x, -\alpha x, ..., \alpha x, -\alpha x)\\
				% 	v(z) &= (\beta z,-\beta z)
				% \end{align*}
				% for some values $\alpha,\beta \in R$ and, since $(1,-1,...,1,-1)$ resp. $(1,-1)$ are in the kernel, that $\alpha$ and $\beta$ are units in $R$. Moreover, because \TODO{why the hell}, we have $\alpha = \beta$. By definition $\deg_R(f_m) = s(1)$ and we get by commutativity of the diagram
				% \begin{equation*}
				% 	(\alpha \deg_R(f_m), -\alpha \deg_R(f_m)) = v(s(1)) = t(u(1)) = (\alpha m,-\alpha m),
				% \end{equation*}
				% from which we deduce $\deg_R(f_m) = m$.

				For $m=-1$ consider the complex conjugate $f:\S^1 --> \S^1, z \mapsto \overline{z}$ and reuse the decomposition $A_1,A_2$. As for $m > 0$ we obtain a diagram
				\begin{equation*}
					\begin{diagram}
						\fourbytwo[wide]
							{0}{R}{R \oplus R}{R \oplus R}
							{0}{R}{R \oplus R}{R \oplus R}

						\arrow{nww}{nw}{}
						\arrow{nw}{ne}{v}[above]
						\arrow{ne}{nee}{b}[above]

						\arrow{sww}{sw}{}
						\arrow{sw}{se}{v}[below]
						\arrow{se}{see}{b}[below]

						\arrow[equals]{nww}{sww}{}
						\arrow{nw}{sw}{s'}[left]
						\arrow{ne}{se}{t'}[right]
						\arrow{nee}{see}{}
					\end{diagram}
				\end{equation*}
				where
				\begin{equation*}
					t'(y_1,y_2) = (y_2,y_1),
				\end{equation*}
				which then leads to $(\alpha \deg_R(f), -\alpha \deg_R(f)) = v(s'(1)) = t'(v(1)) = (-\alpha,\alpha)$. Hence $\deg_R(f) = -1$.
			}
			\item[(IH)]{
				For $n$ and $m \in \chi(\Z) \subseteq R$ there is a map $f:\S^n --> \S^n$ with $\deg_R(f) = m$.
			}
			\item[(IS)]{
				Consider the following diagram, where $\kappa: \Susp \S^n --> \S^{n+1}$ is the homeomorphism constructed in lemma \TODO{cite}.
				\begin{equation*}
					\begin{diagram}
						\threebytwo[ultrawide]
							{H_{n+1}(\S^{n+1})}{H_{n+1}(\Susp \S^n)}{H_n(\S^n)}
							{H_{n+1}(\S^{n+1})}{H_{n+1}(\Susp \S^n)}{H_n(\S^n)}

						\arrow{n}{nw}{\begin{array}{c}\isom\\\kappa\end{array}}
						\arrow{s}{sw}{\begin{array}{c}\isom\\\kappa\end{array}}

						\arrow{n}{ne}{\begin{array}{c}\isom\\\nu_{n+1}\end{array}}
						\arrow{s}{se}{\begin{array}{c}\isom\\\nu_{n+1}\end{array}}

						\arrow{nw}{sw}{H_{n+1}(\kappa \Susp(f) \kappa^{-1})}[ontop]
						\arrow{n}{s}{H_{n+1}(\Susp f)}[ontop]
						\arrow{ne}{se}{f}[right]
					\end{diagram}
				\end{equation*}
				By commutativity of the diagram the degree of the left morphism is given by the image of $1 \in R \isom H_{n+1}(\S^{n+1})$ along the boundary. Note that the horizontal morphisms of $R$-modules $R \isom H_{n+1}(\S^{n+1}) --> H_n(\S^n) \isom R$ are given by multiplication with some invertible element $a \in R$. This gives $\deg_R(\kappa\Susp(f)\kappa^{-1}) = a^{-1}\deg_R(f)a = \deg_R(f)$.
			}
		\end{enumerate}
	\end{sketch}

	\begin{theorem}[\TODO{Projective Fixed Point Theorem}]
		Let $(H_\bullet, \partial_\bullet)$ be an $R$-homology theory with $\characteristic R \notin \{1,2\}$.

		Then for $n \in \N$ it holds that for every function $f: \S^{2n} --> \S^{2n}$ there is an $x_0 \in \S^{2n}$ with $f(x_0) \in \{-x_0,x_0\}$. \TODO{In particular, every map $\widetilde{f}:\R\P^{2n} --> \R\P^{2n}$ has a fixed point.}
	\end{theorem}
	\begin{proof}
		\begin{assume}
			there is no $x$ with $f(x) \in \{-x,x\}$.
				
			We find that
			\begin{equation*}
				\begin{array}{rccl}
					h: &[0,1] \times \S^{2n} & \longrightarrow & \S^{2n}\\
					&(t,x) & \longmapsto & \dfrac{(1-t)x + tf(x)}{\Norm{(1-t)x + tf(x)}}\vspace{.5em}\\
					k: &[0,1] \times \S^{2n} & \longrightarrow & \S^{2n}\\
					&(t,x) & \longmapsto & \dfrac{(1-t)f(x) + t(-x)}{\Norm{(1-t)f(x) + t(-x)}}
				\end{array}
			\end{equation*}
			give homotopies $\id \sim_h f \sim_k -\id$. Hence
			\begin{equation*}
				1 = \deg_R(\id) = \deg_R(-\id) = (-1)^{2n+1} = -1.
			\end{equation*}
			A contradiction. \vspace{-2em}
		\end{assume} 
	\end{proof}

	\begin{theorem}[Hairy Ball Theorem]
		Let $(H_\bullet, \partial_\bullet)$ be an $R$-homology theory with $\characteristic R \notin \{1,2\}$.

		Then every continuous vector field $v:\S^{2n} --> \R^{2n+1}$ (meaning $\scalarproduct{x}{v(x)} = 0$ for all $x \in \S^{2n}$) vanishes, i.e. there is a point $x_0 \in \S^{2n}$ with $v(x_0) = 0$.
	\end{theorem}
	\begin{proof}
		\textit{Assume} otherwise and define the continuous function
		\begin{tab}[1.3cm]
		\begin{equation*}
			\begin{array}{rccl}
				f: & \S^{2n} & \longrightarrow & \S^{2n}\\
				& x & \longmapsto & \dfrac{v(x)}{\Norm{v(x)}}
			\end{array}.
		\end{equation*}
		By the Projective Fixed Point Theorem there is some $x_0 \in \S^{2n}$ with $f(x_0) = \pm x_0$. But then
		\begin{equation*}
			0 = \dfrac{\scalarproduct{v(x_0)}{x_0}}{\Norm{v(x_0)}} = \scalarproduct{f(x_0)}{x_0} = \scalarproduct{\pm x_0}{x_0} = \pm 1.
		\end{equation*}
		A contradiction.\vspace{-2em}
		\end{tab}
	\end{proof}

	\begin{theorem}[Fundamental Theorem of Algebra]
		Let $(H_\bullet, \partial_\bullet)$ be an $R$-homology theory with $\characteristic R = 0$. Then every nonconstant polynomial in $\C[X]$ has a zero.
	\end{theorem}
	\begin{proof}
		\begin{assume}
			there is a polynomial $p(X) = X^n + a_{n-1}X^{n-1} + ... + a_0$ without a zero and consider it as a function $p:\C --> \C$.
			For any $t \in \R$ the map $f_t$ is nullhomotopic \TODO{via $h_t$}, where
			\begin{equation*}
				\begin{array}{rccc}
					f_t: &\S^1 & \longrightarrow & \S^1\\
					&z & \longmapsto & \dfrac{p(tz)}{\Norm{p(tz)}}
				\end{array}
				\hspace{1cm}
				\begin{array}{rccc}
					h_t: &[0,1] \times \S^1 & \longrightarrow & \S^1\\
					 &(s,z) & \longmapsto & \dfrac{p(stz)}{\Norm{p(stz)}}
				\end{array},
				\hspace{1.5cm}
			\end{equation*}
			so $\deg f_t = 0$. Choose $r>0$ so large that $\norm{z}^n > \norm{a_{n-1}z^{n-1} + ... + a_0}$. Consider the maps
			\begin{equation*}
				p_t(z) = z^n + t(a_{n-1}z^{n-1} + ... + a_0),
			\end{equation*}
			which in turn give maps
			\begin{equation*}
				\begin{array}{rccc}
					g_t: & \S^1 & \longrightarrow & \S^1\\
					& z & \longmapsto & \dfrac{p_t(r z)}{\norm{p_t(rz)}}
				\end{array}.
			\end{equation*}
			Now $f_r = g_1 \sim g_0 = (z \mapsto z^n)$. Hence
			\begin{equation*}
				0 = \deg_R(f_r) = \deg_R(z \mapsto z^n) = n.
			\end{equation*}
			A contradiction.\vspace{-2em}
		\end{assume}
	\end{proof}

	\newpage
	\subsection{Local Degree and Homological Orientation}

	\begin{lemma}
		Let $(H_\bullet,\partial_\bullet)$ be an $R$-homology theory.

		Let $X$ be an $n$-dimensional topological manifold and $x$ be a point in $X$. Then there is an isomorphism $H_n(X,X\setminus \{x\}) \isom R$.
	\end{lemma}
	\begin{proof}
		By definition of a manifold there is a chart $\phi_x: U_x \isom \R^n$ with $\phi_x(x) = 0$. By excission (of the complement of $\phi_x^{-1}(\D^n)$) we get a chain of isomorphisms
		\begin{equation*}
			H_\bullet(X,X\setminus\{x\}) \isom H_\bullet(\phi_x^{-1}(\D^n),\phi_x^{-1}(\D^n\setminus \{0\})) \isom H_\bullet(\D^n,\D^n\setminus \{0\}) \isom H_\bullet(\D^n,\S^{n-1}) \isom M_{\bullet-n}.
		\end{equation*}
		In particular we have $H_n(X,X\setminus\{x\}) \isom R$. 
	\end{proof}

	\begin{definition}
		Let $(H_\bullet, \partial_\bullet)$ be a 
	\end{definition}

	\newpage
	\subsection{CW-complexes and Cellular Homology}

	\begin{definition}
		\begin{minipage}{.7\textwidth}
			A (\textbf{relative}) \textbf{CW-structure} on a pair $(X,A)$ of spaces is a \textit{filtration}
			\begin{equation*}
				A \subseteq X_0 \subseteq X_1 \subseteq ... \subseteq \colim \limits_{n \in \N_0} X_n = X
			\end{equation*}
			of $X$, such that for every $n \in \N_0$ there is an indexing set $I_n$ and for each $i \in I_n$
			\vspace{.5em}
			\begin{enumerate}[--]
				\item{
					an \textbf{attaching map} $\chi_i:\S^{n-1} --> X_{n-1}$
				}
				\item{
					a \textbf{characteristic map} $\widetilde{\chi}_i: \D^n --> X_n$
				}
			\end{enumerate}
			\vspace{.5em}
			which make the diagram on the right a pushout diagram.\\
			The $X_n$ are called the \textbf{$n$-skeleton} of $X$.\vspace{.5em}

			A (\textbf{relative}) \textbf{CW-complex} is a pair of spaces $(X,A)$, which admits a CW-structure.
		\end{minipage}
		\begin{minipage}{.3\textwidth}
			\begin{equation*}
				\begin{diagram}
					\twobytwo[wide]
						{\Coproduct \limits_{i \in I_n} \S^{n-1}}{X_{n-1}}
						{\Coproduct \limits_{i \in I_n} \D^n}{X_n}

					\arrow{nw}{ne}{\coproduct \chi_i}[above]
					\arrow{sw}{se}{\coproduct \widetilde{\chi_i}}[below]
					\arrow{nw}{sw}{}
					\arrow{ne}{se}{}

					\node at ($(se)+(-.4,.4)$) {$\secorner$};
				\end{diagram}
			\end{equation*}
			\vspace{1em}
			\begin{equation*}
				\S^{-1} := \emptyset,\hspace{1em} X_{-1} := A
			\end{equation*}
		\end{minipage}

		The \textbf{dimension} of a CW-complex \TODO{with filtration $(X_n), I_n$} is $\dim(X,A) = \sup \{n \mid I_n \neq \emptyset\}$. The CW-complex is said to be \textbf{locally finite}, if the indexing set $I_n$ is finite for all $n$, and is \textbf{finite}, if it is locally finite and has finite dimension, i.e. $\Union\limits_{n \in \N_0}I_n$ is finite.
	\end{definition}

	\TODO{examples $\S^n,M_k(n),\R\P^n,\R\P^\infty, \C\P^n, \C\P^\infty$}

	\begin{lemma}[Properties of CW-complexes]
		Let $(X,A)$ be a relative CW-complex with a given CW-structure. For any $n \in \N_0$ the following properties hold. 
		\begin{enumerate}[(i)]
			\item{
				We have $X_n \setminus X_{n-1} \isom \Coproduct \limits_{i \in I_n} (\D^n \setminus \S^{n-1}) = \Coproduct \limits_{i \in I_n} \B^n$ and in particular $I_n \isom \pi_0(X_n \setminus X_{n-1})$.
			}
			\item{
				$X_{n-1}$ is a CNDR of $X_n$ and $X_n / X_{n-1} \isom \Join \limits_{i \in I} \S^n$.

				\TODO{characteristic map induces homeom see Kammeyer eq (6.1)}
			}
			\item{
				The composition $(\Tel(X),A) --> ([-1,\infty)\times X,[-1,\infty)\times A) --> (X,A)$ is a homotopy equivalence.
			}
		\end{enumerate}
	\end{lemma}

	\begin{corollary}[Homology of CW-complexes]
		Let $(H_\bullet, \partial_\bullet)$ be an ordinary homology theory and $(X,A)$ be a relative CW-complex with given CW-structure.
		\begin{enumerate}[(i)]
			\item{
				We have $H_\bullet(X_n,X_{n-1}) \isom \Directsum \limits_{i \in I_n}M_{\bullet-n}$.% and in particular
				% \begin{equation*}
				% 	H_\bullet(X_n,X_{n-1}) = \left \{ \begin{array}{ll}
				% 		\Directsum \limits_{i \in I} M_0 & k = n\\ 0 & k \neq n
				% 	\end{array}\right.
				% \end{equation*}
			}
			\item{
				It holds that $H_\bullet(X,A) \isom H_\bullet(\Tel (X), A) \isom \colim \limits_{n \in \N_0} H_\bullet(X_n,A)$.
			}
		\end{enumerate}
	\end{corollary}

	\TODO{cellular homology see kammeyer}

	\begin{definition}
		Let $(H_\bullet, \partial_\bullet)$ be an $R$-homology theory and $(X,A)$ be a relative CW-complex with filtration $(X_n)_n$. 

		The \textbf{cellular chain complex} on $(X,A)$ associated with $(H_\bullet, \partial_\bullet)$ is given by
		\begin{equation*}
			\begin{diagram}
				\threebyone[verywide]
					{...C_{n+1}^{H_\bullet}(X,A)}{C_n^{H_\bullet}(X,A)}{C_{n-1}^{H_\bullet}(X,A)...}
				\arrow{w}{c}{\Delta_{n+1}}[above]
				\arrow{c}{e}{\Delta_n}[above]
			\end{diagram}
		\end{equation*}
		where $C_n^{H_\bullet} := H_n(X_n,X_{n-1})$ and 
		\begin{equation*}
			\Delta_{n+1} = 
			\begin{diagram}
				\threebyone[verywide]
					{H_{n+1}(X_{n+1},X_{n})}{H_{n}(X_{n})}{H_n(X_n,X_{n-1})}
				\arrow{w}{c}{\partial_n}[above]
				\arrow{c}{e}{}
			\end{diagram}
		\end{equation*}
		is the morphism from the triple exact sequence. Note that this indeed turns $(C_{*}^{H_\bullet}, \Delta_{*})$ into a chain complex. \TODO{remove C notation?}

		The \textbf{cellular homology} of $(X,A)$ with respect to $(H_\bullet, \partial_\bullet)$ is the homology of its cellular chain complex and denoted by $H_\bullet^\text{CW}(X,A)$.
	\end{definition}

	\begin{theorem}[Cellular Homology]
		Let $(H_\bullet, \partial_\bullet)$ be an $R$-homology theory and $(X,A)$ be a relative CW-complex. Then there are isomorphisms
		\begin{equation*}
			H_\bullet^\text{CW}(X,A) \isom H_\bullet(X,A),
		\end{equation*}
		which are natural \TODO{with respect to \textit{cellular maps}} and \TODO{compatible with the boundary morphisms}. Moreover on $(*,\emptyset)$ this isomorphism is the identity on $H_\bullet(*)$.
	\end{theorem}
	\begin{proof}
		Recall that for $n \in \N_0$ we have $\smash{H_\bullet(X_n,X_{n-1}) \isom \Oplus \limits_{i \in I_n} M_{\bullet-n}}$ and consider the triple exact sequence for $A \subseteq X_{n-1} \subseteq X_n$.
		\begin{equation*}
			\begin{diagram}
				\fourbyone[ultrawide]
					{...H_{\bullet+1}(X_n,X_{n-1})}{H_\bullet(X_{n-1},A)}{H_\bullet(X_n,A)}{H_\bullet(X_n,X_{n-1})...}

				\arrow{ww}{w}{\Delta_{\bullet+1}}[above]
				\arrow{w}{e}{}
				\arrow{e}{ee}{}
			\end{diagram}
		\end{equation*}
		Hence, for $k \notin \{n,n-1\}$ we get an isomorphism $H_k(X_{n-1},A) \isom H_k(X_n,A)$. In particular for $k > n$ we have
		\begin{equation*}
			H_k(X_n,A) \isom H_k(X_{n-1},A) \isom ... \isom H_k(X_{-1},A) = H_k(A,A) = 0
		\end{equation*}
		and for $k < n$ we have for any $m > n$
		\begin{equation*}
			H_k(X_n,A) \isom H_k(X_{n+1},A) \isom ... \isom H_k(X_m,A).
		\end{equation*}
		This yields for any $k$ and arbitrary $m > k$ \TODO{correct?}
		\begin{equation*}
			H_k(X,A) \isom \colim \limits_{n \in \N_0} H_k(X_n,A) \isom \colim \limits_{n > k} H_k(X_n,A) \isom H_k(X_m,A).
		\end{equation*}
		
		Now consider the following diagram. Note that the vertical composition is part of the triple exact sequence of $(X_n,X_{n-1},A)$, and that the lower diagonal is an excerpt of the triple exact sequence of $(X_{n-1},X_{n-2},A)$. Furthermore, note that the two triangles commute by definition. By exactness of the vertical and the diagonal sequence and the previous calculations we obtain that the two canonical morphisms are injective, as indicated.
		\begin{equation*}
			\begin{diagram}
				\threebyfive[hyperwide]
					{}{H_n(X_{n-1},A)=0}{}
					{}{H_n(X_n,A)}{}
					{H_{n+1}(X_{n+1},X_n)}{H_n(X_n,X_{n-1})}{H_{n-1}(X_{n-1},X_{n-2})}
					{}{H_{n-1}(X_{n-1},A)}{}
					{0=H_{n-1}(X_{n-2},A)}{}{}

				\arrow{w}{c}{\vspace{1em}\Delta_{n+1}^{(n+1,n,n-1)}}[below]
				\arrow{c}{e}{\Delta_{n}^{(n,n-1,n-2)}}[above]

				\arrow{w}{n}{\Delta_{n+1}^{(n+1,n,A)}}[above left]
				\arrow[incl]{s}{e}{}
				\arrow{ssw}{s}{}

				\arrow{nn}{n}{}
				\arrow[incl]{n}{c}{i}[right]
				\arrow{c}{s}{\Delta_n^{(n,n-1,A)}}[left]
			\end{diagram}
		\end{equation*}
		Note that $\coker \Delta_{n+1}^{(n+1,n,A)} = H_n(X_n,A) / \im \Delta_{n+1}^{(n+1,n,A)}$ and futhermore exactness yields $H_n(X_n,A) \isom i(H_n(X_n,A)) = \ker \Delta_n^{(n,n-1,A)}$. This gives an isomorphism
		\begin{align*}
			H_n^{CW}(X,A) &= \ker \Delta_n^{(n,n-1,n-2)}/\im \Delta_{n+1}^{(n+1,n,n-1)}\\ 
			&= \ker \Delta_n^{(n,n-1,A)} / \im i\circ \Delta_{n+1}^{(n+1,n,A)}\\
			&\isom H_n(X_n,A) / \im \Delta_{n+1}^{(n+1,n,A)}\\
			&= \coker \Delta_{n+1}^{(n+1,n,A)}.
		\end{align*}
		Finally, in the following excerpt of the triple exact sequence of $(X_{n+1},X_n,A)$ the rightmost homology group is $0$ by previous calculations, making the arrow $j$ surjective as indicated.
		\begin{equation*}
			\begin{diagram}
				\fourbyone[ultrawide]
					{...H_{n+1}(X_{n+1},X_n)}{H_n(X_n,A)}{H_n(X_{n+1},A)}{H_n(X_{n+1},X_n)...}

				\arrow{ww}{w}{\Delta_{n+1}^{(n+1,n,A)}}[above]
				\arrow[epi]{w}{e}{j}[above]
				\arrow{e}{ee}{}
			\end{diagram}
		\end{equation*}
		Exactness of this sequence, the isomorphism theorem and previous calculations now yield.
		\begin{align*}
			H_n^{CW}(X,A) \isom \coker \Delta_{n+1}^{(n+1,n,A)} &= H_n(X_n,A) / \im \Delta_{n+1}^{(n+1,n,A)}\\
			&\isom H_n(X_n,A)/\ker j\\
			&\isom H_n(X_n,A)\\
			&\isom H_n(X,A).
		\end{align*}
	\end{proof}

	\begin{corollary}
		Let $(H_\bullet, \partial_\bullet)$ be an $R$-homology theory and $(X,A)$ be a relative CW-complex.
		\begin{enumerate}[(i)]
			\item{
				For $k>\dim(X,A)$ it holds that $H_k(X,A) =0$.
			}
			\item{
				If $(X,A)$ is locally finite then for all $k$ the $R$-module $H_k(X,A)$ is finitely generated.
			}
		\end{enumerate}
	\end{corollary}
	\begin{sketch}
		For (i) note that $H_k(X,A)$ is a quotient of $C_k(X,A)$, which is finitely generated.\\
		For (ii) note that in this case $C_k(X,A) = 0$ for $k > \dim(X,A)$.
	\end{sketch}

	\TODO{uniqueness}

	\begin{remark}
		Let $R$ be a integral domain and $M$ be an $R$-module. The \textbf{rank} of $M$ is defined as 
		\begin{equation*}
			\rk M = \sup\{n \in \N_0 \mid \text{ ex } R^n \longrightincl M\}.
		\end{equation*}

		\TODO{https://mathoverflow.net/questions/153659/different-definitions-of-the-rank-of-a-module?noredirect=1\&lq=1}
	\end{remark}

	\begin{definition}
		Let $R$ be an integral domain and $(H_\bullet, \partial_\bullet)$ be an $R$-homology theory.

		The \textbf{$n$th $R$-Betti number} of a pair of spaces $(X,A)$ is defined to be
		\begin{equation*}
			b_n^R (X,A) = \rk H_n(X,A).
		\end{equation*}

		Suppose $b_n^R(X,A)$ is finite for all $n$ and there is an $N$ such that $b_n^R(X,A) = 0$ for all $n > N$. In this case the \textbf{$R$-Euler characteristic} is given by
		\begin{equation*}
			\chi_R(X,A) = \sum \limits_{n \in \N_0} (-1)^n b_n^R(X,A).
		\end{equation*}
	\end{definition}

	\begin{proposition}[Euler Characteristic of Finite CW-complexes]
		Let $R$ be a integral domain and $(H_\bullet, \partial_\bullet)$ be an $R$-homology theory.

		If $(X,A)$ is a finite relative CW-complex, then
		\begin{equation*}
			\chi_R(X,A) = \sum \limits_{n \in \N_0}(-1)^n \#I_n = \sum \limits_{n \in \N_0} (-1)^n \rk C_n(X,A).
		\end{equation*}
	\end{proposition}


	\newpage
	\section{Singular Homology}
	\subsection{Simplicial Spaces}

	Recall the \textbf{simplex category} $\Delta$ consisting of the total ordered sets $[n] = \{0,...,n\}$ and order preserving maps $[m] --> [n]$. In what comes the following two kinds of morphisms play a special role:

	\begin{minipage}{.65\textwidth}
		The \textbf{$i$-th boundary} of the $n$-th simplex is the map
		\begin{equation*}
			\delta_i^n: [n-1] \longrightarrow [n],\; \delta_i^n(k) = \left\{\begin{array}{ll}
				k & k<i\\
				k+1 & k\geq i
			\end{array}\right..
		\end{equation*}
	\end{minipage}
	\begin{minipage}{.35\textwidth}
		\vspace{.8cm}
		\begin{center}
		\begin{tikzpicture}
			% \node (x) at (-1.5,1.25) {$[1]$};
			% \node (y) at (1.5,1.25) {$[2]$};
			% \draw[->] (x) -- (y);
			%
			\node (a0) at (-2,0) {$0$};
			\node (a1) at (-1,0) {$1$};
			\draw[->] (a0) -- (a1);

			\node at (0,.25) {$\delta^2_1$};
			\node at (0,0) {$\longmapsto$};

			\node (b0) at (1,-.4) {$0$};
			\node[gray] (b1) at (2,-.4) {$1$};
			\node (b2) at (1.5,.4) {$2$};
			\draw[->,gray] (b0) -- (b1);
			\draw[->] (b0) -- (b2);
			\draw[->,gray] (b1) -- (b2);
		\end{tikzpicture}
		\end{center}
	\end{minipage}
	\begin{minipage}{.65\textwidth}
		The \textbf{$i$-th degeneration} of the $n$-th simplex is the map
		\begin{equation*}
			\sigma_i^n: [n] \longrightarrow [n-1],\; \sigma_i^n(k) = \left\{\begin{array}{ll}
				k & k\leq i\\
				k-1 & k > i
			\end{array}\right..
		\end{equation*}
	\end{minipage}
	\begin{minipage}{.35\textwidth}
		\vspace{.6cm}
		\begin{center}
		\begin{tikzpicture}
			\node (a0) at (1,-.4) {$0$};
			\node (a1) at (2,-.4) {$1$};
			\node (a2) at (1.5,.4) {$2$};
			\draw[->] (a0) -- (a1);
			\draw[->] (a0) -- (a2);
			\draw[->,thick] (a1) -- (a2);

			\node at (0,.25) {$\sigma^3_1$};
			\node at (0,0) {$\longmapsto$};

			\node (b0) at (-2.2,-.6) {$0$};
			\node (b1) at (-.8,-.6) {$1$};
			\node (b2) at (-1.5,0) {$2$};
			\node (b3) at (-1.5,.8) {$3$};
			\draw[->] (b0) -- (b1);
			\draw[->] (b0) -- (b2);
			\draw[->] (b0) -- (b3);
			\draw[->] (b1) -- (b2);
			\draw[->] (b1) -- (b3);
			\draw[->,thick] (b2) -- (b3);
		\end{tikzpicture}
		\end{center}
	\end{minipage}

	\begin{lemma}
		Given $n \in \N$ and $i \leq j \leq n$ the boundary morphisms satisfy the equality
		\begin{equation*}
			\delta_{j+1}^{n+1}\delta_i^{n+1} = \delta_i^{n+1}\delta_j^{n}.
		\end{equation*}
	\end{lemma}
	\begin{sketch}
		If $i=j$ consider
		\begin{equation*}
			\hspace{-3em}
			\begin{tikzpicture}[diagram]
				\matrix[matrix of nodes,
		ampersand replacement=\&,column sep=1em,row sep=1em]{
					\node (m-0-0) {\ensuremath{\bullet}}; \&
					\node (m-0-1) {\ensuremath{\bullet}}; \&
					\node (m-0-2) {\ensuremath{\bullet}}; \&
					\node (m-0-3) {\ensuremath{\bullet}}; \&
					\node (m-0-4) {\ensuremath{\bullet}}; \&
					\node (m-0-5) {}; \&
					\node (m-0-6) {}; \\
	%
					\node (m-1-0) {\ensuremath{\bullet}}; \&
					\node (m-1-1) {\ensuremath{\bullet}}; \&
					\node (m-1-2) {\ensuremath{\bullet}}; \&
					\node (m-1-3) {\ensuremath{\bullet}}; \&
					\node (m-1-4) {\ensuremath{\bullet}}; \&
					\node (m-1-5) {\ensuremath{\bullet}}; \&
					\node (m-1-6) {}; \\
	%
					\node (m-2-0) {\ensuremath{\bullet}}; \&
					\node (m-2-1) {\ensuremath{\bullet}}; \&
					\node (m-2-2) {\ensuremath{\bullet}}; \&
					\node (m-2-3) {\ensuremath{\bullet}}; \&
					\node (m-2-4) {\ensuremath{\bullet}}; \&
					\node (m-2-5) {\ensuremath{\bullet}}; \&
					\node (m-2-6) {\ensuremath{\bullet}}; \\
				};
				\node at ($(m-0-2)+(0,1em)$) {\ensuremath{i}};
				\node at ($.5*(m-0-0) + .5*(m-1-0) + (-2em,0)$) {\ensuremath{\delta_i^n}};
				\node at ($.5*(m-1-0) + .5*(m-2-0) + (-2em,0)$) {\ensuremath{\delta_{i+1}^{n+1}}};
				\draw[->,thick] (m-0-0) -- (m-1-0) -- (m-2-0);
				\draw[->,thick] (m-0-1) -- (m-1-1) -- (m-2-1);
				\draw[->,thick] (m-0-2) -- (m-1-3) -- (m-2-4);
				\draw[->,thick] (m-0-3) -- (m-1-4) -- (m-2-5);
				\draw[->,thick] (m-0-4) -- (m-1-5) -- (m-2-6);
				\draw[->,gray] (m-1-2) -- (m-2-2);
			\end{tikzpicture}
			=
			\begin{tikzpicture}[diagram]
				\matrix[matrix of nodes,
		ampersand replacement=\&,column sep=1em,row sep=1em]{
					\node (m-0-0) {\ensuremath{\bullet}}; \&
					\node (m-0-1) {\ensuremath{\bullet}}; \&
					\node (m-0-2) {\ensuremath{\bullet}}; \&
					\node (m-0-3) {\ensuremath{\bullet}}; \&
					\node (m-0-4) {\ensuremath{\bullet}}; \&
					\node (m-0-5) {}; \&
					\node (m-0-6) {}; \\
	%
					\node (m-1-0) {\ensuremath{\bullet}}; \&
					\node (m-1-1) {\ensuremath{\bullet}}; \&
					\node (m-1-2) {\ensuremath{\bullet}}; \&
					\node (m-1-3) {\ensuremath{\bullet}}; \&
					\node (m-1-4) {\ensuremath{\bullet}}; \&
					\node (m-1-5) {\ensuremath{\bullet}}; \&
					\node (m-1-6) {}; \\
	%
					\node (m-2-0) {\ensuremath{\bullet}}; \&
					\node (m-2-1) {\ensuremath{\bullet}}; \&
					\node (m-2-2) {\ensuremath{\bullet}}; \&
					\node (m-2-3) {\ensuremath{\bullet}}; \&
					\node (m-2-4) {\ensuremath{\bullet}}; \&
					\node (m-2-5) {\ensuremath{\bullet}}; \&
					\node (m-2-6) {\ensuremath{\bullet}}; \\
				};
				\node at ($(m-0-2)+(0,1em)$) {\ensuremath{i}};
				\node at ($.5*(m-0-0) + .5*(m-1-0) + (-2em,0)$) {\ensuremath{\delta_i^n}};
				\node at ($.5*(m-1-0) + .5*(m-2-0) + (-2em,0)$) {\ensuremath{\delta_{i}^{n+1}}};
				\draw[->,thick] (m-0-0) -- (m-1-0) -- (m-2-0);
				\draw[->,thick] (m-0-1) -- (m-1-1) -- (m-2-1);
				\draw[->,thick] (m-0-2) -- (m-1-3) -- (m-2-4);
				\draw[->,thick] (m-0-3) -- (m-1-4) -- (m-2-5);
				\draw[->,thick] (m-0-4) -- (m-1-5) -- (m-2-6);
				\draw[->,gray] (m-1-2) -- (m-2-3);
			\end{tikzpicture}
		\end{equation*}
		If $i < j$ consider
		\begin{equation*}
			\hspace{-3em}
			\begin{tikzpicture}[diagram]
				\matrix[matrix of nodes,
		ampersand replacement=\&,column sep=1em,row sep=1em]{
					\node (m-0-0) {\ensuremath{\bullet}}; \&
					\node (m-0-1) {\ensuremath{\bullet}}; \&
					\node (m-0-2) {\ensuremath{\bullet}}; \&
					\node (m-0-3) {\ensuremath{\bullet}}; \&
					\node (m-0-4) {\ensuremath{\bullet}}; \&
					\node (m-0-5) {\ensuremath{\bullet}}; \&
					\node (m-0-6) {}; \&
					\node (m-0-7) {}; \\
	%
					\node (m-1-0) {\ensuremath{\bullet}}; \&
					\node (m-1-1) {\ensuremath{\bullet}}; \&
					\node (m-1-2) {\ensuremath{\bullet}}; \&
					\node (m-1-3) {\ensuremath{\bullet}}; \&
					\node (m-1-4) {\ensuremath{\bullet}}; \&
					\node (m-1-5) {\ensuremath{\bullet}}; \&
					\node (m-1-6) {\ensuremath{\bullet}}; \&
					\node (m-1-7) {}; \\
	%
					\node (m-2-0) {\ensuremath{\bullet}}; \&
					\node (m-2-1) {\ensuremath{\bullet}}; \&
					\node (m-2-2) {\ensuremath{\bullet}}; \&
					\node (m-2-3) {\ensuremath{\bullet}}; \&
					\node (m-2-4) {\ensuremath{\bullet}}; \&
					\node (m-2-5) {\ensuremath{\bullet}}; \&
					\node (m-2-6) {\ensuremath{\bullet}}; \&
					\node (m-2-7) {\ensuremath{\bullet}}; \\
				};
				\node at ($(m-0-2)+(0,1em)$) {\ensuremath{i}};
				\node at ($(m-0-3)+(0,1em)$) {\ensuremath{j}};
				\node at ($(m-0-4)+(0,1em)$) {\ensuremath{j+1}};
				\node at ($.5*(m-0-0) + .5*(m-1-0) + (-2em,0)$) {\ensuremath{\delta_i^n}};
				\node at ($.5*(m-1-0) + .5*(m-2-0) + (-2em,0)$) {\ensuremath{\delta_{j+1}^{n+1}}};
				\draw[->,thick] (m-0-0) -- (m-1-0) -- (m-2-0);
				\draw[->,thick] (m-0-1) -- (m-1-1) -- (m-2-1);
				\draw[->,thick] (m-0-2) -- (m-1-3) -- (m-2-3);
				\draw[->,thick] (m-0-3) -- (m-1-4) -- (m-2-5);
				\draw[->,thick] (m-0-4) -- (m-1-5) -- (m-2-6);
				\draw[->,thick] (m-0-5) -- (m-1-6) -- (m-2-7);
				\draw[->,gray] (m-1-2) -- (m-2-2);
			\end{tikzpicture}
			=
			\begin{tikzpicture}[diagram]
				\matrix[matrix of nodes,
		ampersand replacement=\&,column sep=1em,row sep=1em]{
					\node (m-0-0) {\ensuremath{\bullet}}; \&
					\node (m-0-1) {\ensuremath{\bullet}}; \&
					\node (m-0-2) {\ensuremath{\bullet}}; \&
					\node (m-0-3) {\ensuremath{\bullet}}; \&
					\node (m-0-4) {\ensuremath{\bullet}}; \&
					\node (m-0-5) {\ensuremath{\bullet}}; \&
					\node (m-0-6) {}; \&
					\node (m-0-7) {}; \\
	%
					\node (m-1-0) {\ensuremath{\bullet}}; \&
					\node (m-1-1) {\ensuremath{\bullet}}; \&
					\node (m-1-2) {\ensuremath{\bullet}}; \&
					\node (m-1-3) {\ensuremath{\bullet}}; \&
					\node (m-1-4) {\ensuremath{\bullet}}; \&
					\node (m-1-5) {\ensuremath{\bullet}}; \&
					\node (m-1-6) {\ensuremath{\bullet}}; \&
					\node (m-1-7) {}; \\
	%
					\node (m-2-0) {\ensuremath{\bullet}}; \&
					\node (m-2-1) {\ensuremath{\bullet}}; \&
					\node (m-2-2) {\ensuremath{\bullet}}; \&
					\node (m-2-3) {\ensuremath{\bullet}}; \&
					\node (m-2-4) {\ensuremath{\bullet}}; \&
					\node (m-2-5) {\ensuremath{\bullet}}; \&
					\node (m-2-6) {\ensuremath{\bullet}}; \&
					\node (m-2-7) {\ensuremath{\bullet}}; \\
				};
				\node at ($(m-0-2)+(0,1em)$) {\ensuremath{i}};
				\node at ($(m-0-3)+(0,1em)$) {\ensuremath{j}};
				\node at ($(m-0-4)+(0,1em)$) {\ensuremath{j+1}};
				\node at ($.5*(m-0-0) + .5*(m-1-0) + (-2em,0)$) {\ensuremath{\delta_j^n}};
				\node at ($.5*(m-1-0) + .5*(m-2-0) + (-2em,0)$) {\ensuremath{\delta_{i}^{n+1}}};
				\draw[->,thick] (m-0-0) -- (m-1-0) -- (m-2-0);
				\draw[->,thick] (m-0-1) -- (m-1-1) -- (m-2-1);
				\draw[->,thick] (m-0-2) -- (m-1-2) -- (m-2-3);
				\draw[->,thick] (m-0-3) -- (m-1-4) -- (m-2-5);
				\draw[->,thick] (m-0-4) -- (m-1-5) -- (m-2-6);
				\draw[->,thick] (m-0-5) -- (m-1-6) -- (m-2-7);
				\draw[->,gray] (m-1-3) -- (m-2-4);
			\end{tikzpicture}
			\vspace{-2.4em}
		\end{equation*}
	\end{sketch}

	Now recall that a \textbf{simplicial object} on a category $\cat{C}$ is a just a presheaf $\Delta^{op} --> \cat{C}$ and one denotes by $\ncat{s}\cat{C} = |[\Delta^{op},\cat{C}|]$ the category of simplicial objects. In the case of $\cat{C} = \ncat{Set}$ or $\cat{C} = \ncat{Top}$ simplicial objects are usually called \textbf{simplicial sets} or \textbf{simplicial spaces} respectively.

	%Denote by $\R^\infty$ the real vector space of countable infinite dimension, with standard basis $e_0,e_1,...$.

	\begin{definition}
		The \textbf{geometric realization} of $\Delta$ is the functor
		\begin{equation*}
			\begin{array}{rcl}
				\Delta & \overset{\mathbb{\Delta}}{\longrightarrow} & \ncat{Top}\\
				{[n]} & \longmapsto & \mathbb{\Delta}^n := \convex \{e_0,...,e_n\} \subseteq \R^{n+1}\\
				{[\rho:[m]\rightarrow[n]]} & \longmapsto & {\left[\begin{array}{ccc}\mathbb{\Delta}^m &\rightarrow& \mathbb{\Delta}^n\\e_i &:\mapsto& e_{\rho(i)}\end{array}\right]}
			\end{array}
		\end{equation*}
		where $e_0,...,e_n$ denote the standard basis vectors and
		\begin{equation*}
			\smash{\convex \{e_0,...,e_n\} = \{(t_0,...,t_n) \mid 0 \leq t_i \text{ and } \Sum \limits_{i=0}^n t_i = 1\}}.
		\end{equation*}
		The action on morphisms is defined by linear (and hence continuous) extension.
	\end{definition}

	%Note that the geometric $n$-simplex $\mathbb{\Delta}^n$ is inherently $n$-dimensional, as it lies completely in the hyperplane spanned by the unit vectors. Thus it is common to consider it as subset of $\R^n$.

	\begin{definition}
		Let $X$ be a convex subset of a real vector space $V$. Given an ordered family of points $x_0,...,x_n$ there is a unique linear map $\R^{n+1} --> V$ induced by the assignment $e_i \mapsto x_i$. The \textbf{linear $n$-simplex} $[x_0,...,x_n]$ with \textbf{vertices} $x_0,...,x_n$ is defined to be the restriction of this map to $\mathbb{\Delta}^n \longrightarrow X$. It is common to denot the linear subsimplex, obtained by ommitting $x_i$ by $[x_0,...,\widehat{x_i},...,x_n]$.

		It is important to stress that the order of the vertices plays a crucial role here, since
		\begin{equation*}
			\im [x_0,...,x_n] = \convex \{x_0,...,x_n\} = \im [x_n,...,x_0],
		\end{equation*}
		but $[x_0,...,x_n] \neq [x_n,...,x_0]$.
	\end{definition}

	\TODO{linear sing functor}

	\TODO{remark on different parameterizations needed?}
	% \begin{remark}
	% 	The \textit{convex hull} $\{0,e_0,...,e_n\}$ in $\R^n$ lends itself to two different kinds of parameterizations, namely
	% 	\begin{equation*}
	% 		\smash{\convex \{e_0,...,e_n\} = \{(t_0,...,t_n) \mid 0 \leq t_i \text{ and } \Sum \limits_{i=0}^n t_i = 1\}}
	% 	\end{equation*}
	% 	and
	% 	\begin{equation*}
	% 		\convex \{e_0,...,e_n\} = \{(\varphi_1,...,\varphi_n) \mid 0 \leq \varphi_i \leq 1\}.
	% 	\end{equation*}
	% 	We can convert between both parametrizations by letting
	% 	\begin{equation*}
	% 		\begin{array}{rcl}
	% 			(t_0,\,...\,,t_n) & \longmapsto & (t_0,\,t_0+t_1,\,...\,,\,\Sum \limits_{i=0}^{n-1} t_i)\\
	% 			(\varphi_1,\, \varphi_2 - \varphi_1,\,...\,,\,\varphi_n-\varphi_{n-1},\,1-\varphi_n) & \reflectbox{$\longmapsto$} & (\varphi_1,\, ...\,,\, \varphi_n)
	% 		\end{array}
	% 	\end{equation*}
	% 	\TODO{coordinates and their $d_i^n, s_i^n$}
	% \end{remark}
	\begin{lemma}[Subdivision of the Prism]
		The \textbf{prism} $[0,1] \times \mathbb{\Delta}^n \subseteq \R^{n+1}$ admits a subdivision into $n+1$ linear $n+1$-simplices in such a way that pairwise their intersection is a $k$-simplex for some $k \leq n$.
	\end{lemma}
	\begin{sketch}
		\begin{minipage}[t]{\linewidth-3cm}
			The inclusions $\iota_0,\iota_1: \mathbb{\Delta}^n --> [0,1] \times \mathbb{\Delta}^n$ yield two linear $n$-simplices $[v_0,...,v_n]$ and $[w_0,...,w_n]$ in the prism. Consider the family of linear $(n+1)$-simplices of the form
			\begin{equation*}
				\rho_i = [v_0,...,v_i,w_i,...,w_n]
			\end{equation*}
			for $i \in \{0,...,n\}$. We find for $i < j$
			\begin{equation*}
				\im \rho_i \intersection \im \rho_j = \convex \{v_0,...,v_i,w_{j+1}...,w_n\}.
			\end{equation*}
			
			\TODO{why is every point contained in the union?}
		\end{minipage}
		\begin{minipage}[t]{3cm}
			\textcolor{white}{.}
			\begin{center}
			\begin{tikzpicture}
				\draw[line cap=round] (-.5,-1) -- (-.5,1) -- (0,1.5) -- (0,-.5) -- (-.5,-1) -- (.5,-1) -- (.5,1) -- (0,1.5) -- (0,-.5) -- (.5,-1);
				\draw[line cap=round,densely dotted] (-.5,1) -- (.5,1);

				\draw[line cap=round,thick,densely dotted,darkgreen] (-.5,-1) -- (.5,1);
				\draw[line cap=round,thick,darkgreen] (0,1.5) -- (-.5,-1) -- (.5,-1) -- (0,1.5);
				\draw[line cap=round,thick,darkgreen] (0,1.5) -- (.5,1) -- (.5,-1);

				\node at (-.5,-1.25) {$v_0$};
				\node at (.5,-1.25) {$v_1$};
				\node at (-.8,1) {$w_0$};
				\node at (.8,1) {$w_1$};
				\node at (0,1.75) {$w_2$};

				% \draw[rounded corners=2mm,thick,densely dotted, darkgreen] (-.5,-1) -- (.5,1);
				% \draw[rounded corners=.25mm,thick,darkgreen] (0,1.5) -- (-.5,-1) -- (.5,-1) -- (0,1.5) -- (.5,1) -- (.5,-1);
			\end{tikzpicture}
			\end{center}
		\end{minipage}
	\end{sketch}

	\begin{definition}[Barycentric Subdivision]
		The \textbf{barycenter} of the standard $n$-simplex $\mathbb{\Delta}n$ is the point $b_n = (\frac{1}{n+1},...,\frac{1}{n+1}) \in \R^{n+1}$.
	\end{definition}

	\newpage
	\subsection{Construction of Singular Homology}

	Let $M_\bullet$ be a $\Z$-graded $R$-module. In this section we explicitly construct a generalized homology theory with $M_\bullet$ as its module of coefficients. 

	We start by defining a functor $H:\ncat{Top} --> \ncat{Ab}^{\Z\text{gr}}$ as the composite
	\begin{equation*}
		\begin{diagram}
			\fivebyone[wide]
				{\ncat{Top}}{\ncat{sSet}}{\ncat{sAb}}{\ncat{Ch}}{\ncat{Ab}^{\Z\text{gr}}}

			\arrow{ww}{w}{\ncat{Top}(\mathbb{\Delta},-)}[above]
			\arrow{w}{c}{\Z[-]_{*}}[above]
			\arrow{c}{e}{D_\bullet}[above]
			\arrow{e}{ee}{H_{*}}[above]
		\end{diagram},
	\end{equation*}
	of the following functors.

	The first functor is the \TODO{somewhat} canonical way to turn a topological space into a simplicial set.
	\begin{equation*}
		\begin{array}{rcl}
			\ncat{Top} & \xrightarrow{\ncat{Top}(\mathbb{\Delta},-)} & \ncat{sSet}\\
			X & \longmapsto & \ncat{Top}(\mathbb{\Delta}-,X)\\
			{[X\overset{f}{\rightarrow} Y]} & \longmapsto & {[\ncat{Top}(\mathbb{\Delta}-,X) \overset{f_{*}}{\Rightarrow} \ncat{Top}(\mathbb{\Delta}-,Y)]}
		\end{array}
	\end{equation*}

	\begin{remark}
		\vspace{-1.5em}
		\begin{enumerate}[(i)]
			\item{
				By construction we have the identifications
				\begin{align*}
					\ncat{Top}(\mathbb{\Delta}_0,X) &\isom \ncat{Top}(*,X) = \{\text{points in }X\}\\
					\ncat{Top}(\mathbb{\Delta}_1,X) &\isom \ncat{Top}([0,1],X) = \{\text{paths in }X\}\\
					\ncat{Top}(\mathbb{\Delta}_2,X) &\isom \ncat{Top}(\S^1,X) = \{\text{loops in }X\text{ (without basepoint)}\}.
				\end{align*}
			}
			\item{
				An element $\sigma$ in $\ncat{Top}(\mathbb{\Delta}^n,X)$ is called a \textbf{singular $n$-simplex in $X$}.

				By definition, the image of an $n$-simplex $\sigma$ in $X$ is a quasicompact subset of $X$, since $\mathbb{\Delta}^n$ is quasicompact. Moreover, given a map $f:[m] --> [n]$ the image of $f_{*}(\sigma)=f\sigma$ is a quasicompact subset of the image of $\sigma$. \TODO{why?}
			}
		\end{enumerate}
	\end{remark}

	The \textit{free abelian group functor} $\Z[-]: \ncat{Set} --> \ncat{Ab}$ extends by postcomposition to a functor
	\begin{equation*}
		\begin{array}{rcl}
			\ncat{sSet} & \xrightarrow{\Z[-]_*} & \ncat{sAb}\\
			{[\Delta^{op}\rightarrow\ncat{Set}]} & \longmapsto & {[\Delta^{op}\rightarrow\ncat{Set}\xrightarrow{\Z[-]}\ncat{Ab}]}\\
			{[\alpha:g_1\Rightarrow g_2]} & \longmapsto& {[\Z[\alpha]:\Z[g_1] \Rightarrow \Z[g_2]]}
		\end{array}
	\end{equation*}

	The next functor $D_\bullet$ is a bit more involed, as it has to turn a simplicial object $\fun{A}:\Delta^{op} --> \ncat{Ab}$ into a chain complex. First recall that the boundary morphisms $\delta_i^n:[n-1]-->[n]$ in $\Delta$ give us morphisms $d_i^n:= \fun{A}(\delta_i^n):\fun{A}([n])-->\fun{A}([n-1])$. Let $D_\bullet(\fun{A})$ be the chain complex given by
	% \begin{equation*}
	% 	C_n(\fun{A}) = \left\{\begin{array}{ll} \fun{A}([n]) & n \geq 0\\ 0 & \text{else}\end{array}\right. \hspace{1cm} \partial_n = \left\{\begin{array}{ll}\Sum \limits_{i=0}^n(-1)^i d_i^n & n > 0\\ 0 & \text{else} \end{array}\right.
	% \end{equation*}
	\begin{equation*}
		\begin{diagram}
			\fivebyone[wide]
				{...\fun{A}([2])}{\fun{A}([1])}{\fun{A}([0])}{0}{0...}

			\arrow{ww}{w}{\gamma_2}[above]
			\arrow{w}{c}{\gamma_1}[above]
			\arrow{c}{e}{\gamma_0}[above]
			\arrow{e}{ee}{0}[above]
		\end{diagram},
	\end{equation*}
	where
	\begin{equation*}
		\gamma_n = \Sum \limits_{i=0}^n(-1)^i d_i^n
	\end{equation*}
	for $n \in \N$. By definition of a natural transformation $\beta:\fun{A} ==> \fun{B}$ the components $\beta_{[n]}$ make the diagram
	\begin{equation*}
		\begin{diagram}
			\twobytwo[large]
				{\fun{A}([n])}{\fun{A}([n-1])}
				{\fun{A}([n])}{\fun{A}([n-1])}

			\arrow{nw}{ne}{\fun{A}(\delta_i^n)}[above]
			\arrow{sw}{se}{\fun{B}(\delta_i^n)}[below]

			\arrow{nw}{sw}{\beta_{[n]}}[left]
			\arrow{ne}{se}{\beta_{[n-1]}}[right]
		\end{diagram}
		\hspace{1cm}\text{and thus}\hspace{1cm}
		\begin{diagram}
			\twobytwo[large]
				{\fun{A}([n])}{\fun{A}([n-1])}
				{\fun{A}([n])}{\fun{A}([n-1])}

			\arrow{nw}{ne}{\gamma_n}[above]
			\arrow{sw}{se}{\gamma_n}[below]

			\arrow{nw}{sw}{\beta_{[n]}}[left]
			\arrow{ne}{se}{\beta_{[n-1]}}[right]
		\end{diagram}
	\end{equation*}
	commute, so $\fun{D}(\beta)$ is just the extension of $\beta$ by zeroes in negative degrees.

	To show that this indeed defines a chain complex we use that by Lemma \TODO{cite} the identity
	\begin{equation*}
		d_j^nd_i^{n+1} = d_i^nd_{j+1}^{n+1}
	\end{equation*}
	holds for $i \leq j \leq n$. With this we compute
	\begin{align*}
		\gamma_n\gamma_{n+1} &= \Sum \limits_{i=0}^n(-1)^i d_i^n \left(\Sum \limits_{j=0}^{n+1}(-1)^j d_j^{n+1}\right)
		= \Sum \limits_{i=0}^n \Sum \limits_{j=0}^{n+1}(-1)^{i+j}d_i^nd_j^{n+1}\\
		&=\sum\limits_{i=0}^n\Sum\limits_{j=0}^{i}(-1)^{j+i}d_i^nd_j^{n+1} + \Sum\limits_{i=0}^n\Sum\limits_{j=i+1}^{n+1}(-1)^{i+j}d_i^nd_j^{n+1}\\
		&=\sum\limits_{i=0}^n\Sum\limits_{j=0}^{i}(-1)^{j+i}d_i^nd_j^{n+1} + \Sum\limits_{i=0}^n\Sum\limits_{j=i}^{n}(-1)^{i+j+1}d_i^nd_{j+1}^{n+1}\\
		&=\sum\limits_{i=0}^n\Sum\limits_{j=0}^{i}(-1)^{j+i}d_i^nd_j^{n+1} - \Sum\limits_{i=0}^n\Sum\limits_{j=i}^{n}(-1)^{i+j}d_j^nd_i^{n+1}\\
		&=\sum\limits_{i=0}^n\Sum\limits_{j=0}^{i}(-1)^{j+i}d_i^nd_j^{n+1} - \sum\limits_{j=0}^n\Sum\limits_{i=j}^{n}(-1)^{j+i}d_i^nd_j^{n+1} = 0
	\end{align*}
	Hence we have constructed the \textit{Dold-Kan functor}
	\begin{equation*}
		\begin{array}{rcl}
			\ncat{sAb} & \xrightarrow{D_\bullet} & \ncat{Ch}\\
			{[\fun{A}:\ncat{Ab}^{op}\rightarrow\ncat{Set}]} & \longmapsto & {D_\bullet(\fun{A})}\\
			{[\beta:\fun{A}\Rightarrow \fun{B}]} & \longmapsto& {[\beta:D_\bullet(\fun{A})\rightarrow D_\bullet(\fun{B})].}
		\end{array}
	\end{equation*}

	As a side note, this functor indeed gives an equivalence of categories $D_\bullet: \ncat{sAb} \simeq \ncat{Ch}^{+}$ of simplicial abelian groups and the full subcategory $\ncat{Ch}^{+}$ of chain complexes concentrated in nonnegative degree. \TODO{see ncatlab Dold Kan correspondence}

	\begin{definition}
		The \textbf{(absolute) singular chain complex functor} is the functor given by the composition
		\begin{equation*}
			C_\bullet^\text{sing}:=
			\begin{diagram}
				\fourbyone[wide]
					{\ncat{Top}}{\ncat{sSet}}{\ncat{sAb}}{\ncat{Ch}}

				\arrow{ww}{w}{\ncat{Top}(\mathbb{\Delta},X)}[above]
				\arrow{w}{e}{\Z[-]_{*}}[above]
				\arrow{e}{ee}{D_\bullet}[above]
			\end{diagram}
		\end{equation*}
		and the \textbf{(absolute) singular homology functor} is the corresponding homology functor, obtained by postcomposing with the usual homology functor $H_{*}: \ncat{Ch} --> \ncat{Ab}^{\Z gr}$, i.e. explicitly
		\begin{equation*}
			H^\text{sing} := H_{*}\circ C_\bullet^\text{sing}: \ncat{Top} \longrightarrow \ncat{Ab}^{\Z gr}.
		\end{equation*}

		For given $n \in \N_0$ an element $c$ of $C_n^\text{sing}(X)$ is called a \textbf{$n$-cycle} of $X$ and is explicitly given by a linear combination
		\begin{equation*}
			c = \Sum \limits_{\sigma \in \ncat{Top}(\mathbb{\Delta}^n,X)} c_\sigma \cdot \sigma,
		\end{equation*}
		where $c_\sigma$ are coefficients in $\Z$ and the basis vectors $e_\sigma$ are simply written as $\sigma$. 

		An $n$-cycle is a \textbf{$(n+1)$-boundary}, if it lies in the image of $\gamma_{n+1}:C_{n+1}^\text{sing}(X)-->C_n^\text{sing}(X)$.

		An $n$-cycle $c$, i.e. an element of $\Z^{\oplus\;\ncat{Top}(\mathbb{\Delta}^n,X)}$, determines a subset of $X$, given by the union of the images of all singular $n$-simplices contributing to $c$. This set, which we will call the \textbf{support} $\supp c$ of $c$, is a finite union of quasicompact subsets and hence again quasicompact.
		% By construction we can identify an $n$-cycle $c$, i.e. an element of $\Z^{\oplus\;\ncat{sing}(X)([n])}$, with a function $c: \ncat{sing}(X)([n]) --> \Z$ with finite support, such that
		% \begin{equation*}
		% 	c = \sum \limits_{s \in \ncat{sing}(X)([n])} c(s)e_s.
		% \end{equation*}
		% Moreover $c$ defines a subset of $X$ given by the union of the images of all $n$-simplices contributing to $c$. As a finite union of quasicompact subsets this set is again quasicompact.

		%Moreover $c$ gives rise to a \TODO{unique} simplicial complex \TODO{$\mathbb{\Delta}(c)$}, which induces a map $\widetilde{c}:\mathbb{\Delta}(c) --> X$, whose image is given by the finite union $\Union \limits_{a\\b} s(\mathbb{\Delta}n)$ and is thus quasicompact.
	\end{definition}

	Now we extend this to a homology functor on pairs of spaces. Note that for any pair of spaces $(X,A)$ the sequence of chain complexes
	\begin{align*}
		0 --> C_\bullet^\text{sing}(A) --> C_\bullet^\text{sing}(X) --> C_\bullet^\text{sing}(X)/C_\bullet^\text{sing}(A) --> 0
	\end{align*}
	is exact and functorial in $(X,A)$. Hence we set
	\begin{equation*}
		C_\bullet^\text{sing}(X,A) := C_\bullet^\text{sing}(X)/C_\bullet^\text{sing}(A).
	\end{equation*}

	\begin{definition}
		The \textbf{(relative) singular chain complex functor} is the functor
		\begin{equation*}
			\begin{array}{rcl}
				\ncat{Top}^2 & \xrightarrow{C_\bullet^\text{sing}(-,-)} & \ncat{Ch}\\
				(X,A) & \longmapsto & C_\bullet^\text{sing}(X,A) = C_\bullet^\text{sing}(X)/C_\bullet^\text{sing}(A)\\
				{[f:(X,A)\rightarrow(Y,B)]} & \longmapsto & {[\Z[f_{*}]:C_\bullet^\text{sing}(X)/C_\bullet^\text{sing}(A) \rightarrow C_\bullet^\text{sing}(Y) / C_\bullet^\text{sing}(Y)]}
			\end{array}
		\end{equation*}
		Postcomposing with the homology functor $H_{*}$ yields the \textbf{(relative) singular homology functor with coefficients in $M$}
		\begin{align*}
			H_\bullet^\text{sing}(-,-): \ncat{Top}^2 --> \ncat{Ab}^{\Z gr}.
		\end{align*}
		The corresponding boundary operator is given by the boundary map induced by the short exact sequences of the form
		\begin{align*}
			0 --> C_\bullet^\text{sing}(A) --> C_\bullet^\text{sing}(X) --> C_\bullet^\text{sing}(X,A) --> 0.
		\end{align*}
	\end{definition}

	\begin{lemma}
		Relative singular homology satisfies the dimension axiom. To be specific, $H_\bullet^\text{sing}(*) = \Z[0]$.
	\end{lemma}

	\begin{lemma}
		Relative singular homology satisfies the exactness and the additivity axiom.
	\end{lemma}
	\begin{sketch}
		The exactness axiom is built in by construction.

		Let $(X_i,Y_i)$ be a family of pairs of spaces and $(X,A) = \Coproduct \limits_{i \in I}(X_i,A_i)$. Then 
		\begin{equation*}
			\ncat{Top}(\mathbb{\Delta}-,X) \isom \Coproduct \limits_{i \in I} \ncat{Top}(\mathbb{\Delta}-,X_i),
		\end{equation*}
		\TODO{since the $\mathbb{\Delta}^n$ are connected}. As $\Z[-]_{*}$ is a left adjoint, we get 
		\begin{equation*}
			\Z[\ncat{Top}(\mathbb{\Delta}-,X)] \isom \Z\left[\Coproduct \limits_{i \in I} \ncat{Top}(\mathbb{\Delta}-,X_i)\right] \isom \Oplus \limits_{i \in I} \Z[\ncat{Top}(\mathbb{\Delta}-,X_i)].
		\end{equation*}
		Moreover the Dold-Kan functor preserves colimits, so we have
		\begin{equation*}
			C_\bullet^\text{sing}(X) \isom \Oplus \limits_{i \in I} C_\bullet^\text{sing}(X_i) \hspace{1cm}\text{and likewise}\hspace{1cm} C_\bullet^\text{sing}(A) \isom \Oplus \limits_{i \in I} C^\text{sing}(A_i).
		\end{equation*}
		Since colimits commute we obtain
		\begin{equation*}
			C_\bullet^\text{sing}(X,A) \isom \Oplus \limits_{i \in I} C_\bullet^\text{sing}(X_i,A_i).
		\end{equation*}
		Finally, $H_{*}$ is additive, so we find
		\begin{equation*}
			H_\bullet^\text{sing}(X,A) \isom \Oplus \limits_{i \in I} H_\bullet^\text{sing}(X_i,A_i).\vspace{-3em}
		\end{equation*}
	\end{sketch}

	\begin{proposition}
		Relative singular homology is homotopy invariant.
	\end{proposition}
	\begin{proof}
		It suffices to show \TODO{why? (see Kammeyer)} that $H_\bullet^\text{sing}(-)$ is homotopy invariant. This will be done by showing that a homotopy $f_0 \sim_h f_1$ between maps $f_0,f_1:X-->Y$ induces a homotopy of chains $k_\bullet: C_\bullet^\text{sing}(X) \leadsto C_{\bullet+1}^\text{sing}(Y)$.

		Let $\theta: \mathbb{\Delta}^n --> X$ be a singular $n$-simplex. Recall the the linear $(n+1)$-simplices $\rho_j = [v_0,...,v_j,w_j,...,w_n]: \mathbb{\Delta}^{n+1} --> \I \times \mathbb{\Delta}^n$ used in subdivision of the prism. From these we obtain a family  of singular $(n+1)$-simplices $\sigma_j:\mathbb{\Delta}^{n+1} --> Y$ by taking the composite
		\begin{equation*}
			\begin{diagram}
				\fourbyone[wide]
					{\mathbb{\Delta}^{n+1}}{\I \times \mathbb{\Delta}^n}{\I \times X}{Y.}

				\arrow{ww}{w}{\rho_i}[above]
				\arrow{w}{e}{\I \times \theta}[above]
				\arrow{e}{ee}{h}[above]
			\end{diagram}
		\end{equation*}
		We note that
		\begin{equation*}
			d_i^{n+1}(\sigma_j) = \left\{
			\begin{array}{ll}
				\sigma_j\vert_{[v_0,...,\widehat{v_i},...,v_j,w_j,...,w_n]} & i \leq j\\
				\sigma_j\vert_{[v_0,...,v_j,w_j,...,\widehat{w_{i-1}},...,w_n]} & i > j
			\end{array}
			\right.
		\end{equation*}
		and further
		\begin{align*}
			d_0^{n+1}(\sigma_n) &= \sigma_n \vert_{[\widehat{v_0},w_0,...,w_n]} = h \circ (\{1\} \times \theta) = f_1 \circ \theta\\
			d_{n+1}^{n+1}(\sigma_0) &= \sigma_0 \vert_{[v_0,...,v_n,\widehat{w_n}]} = h \circ (\{0\} \times \theta) = f_0 \circ \theta\\
		\end{align*}

		Define the map $k_n: C_n^\text{sing}(X) --> C_{n+1}^\text{sing}(Y)$ by linear extension of the assignment
		\begin{equation*}
			k_n(\theta) = \Sum \limits_{j=0}^n (-1)^j \cdot \sigma_j.
		\end{equation*}
		For $n=0$ we calculate
		\begin{equation*}
			\gamma_1(k_0(\theta)) = \gamma_1(\sigma_0) = d_0^1(\sigma_0) - d_1^1(\sigma_0) = \sigma\vert_{[\widehat{v_0},w_0]} - \sigma\vert_{[v_0,\widehat{w_0}]} = f_1 \circ \theta - f_0 \circ \theta
		\end{equation*}
		and for $n>0$ we obtain
		\begin{align*}
			\gamma_{n+1}k_n(\theta) &= \Sum \limits_{i=0}^{n+1}\Sum \limits_{j=0}^n (-1)^{i+j} d_i^{n+1}(\sigma_j)\\
			&= \Sum \limits_{0 \leq i \leq j \leq n}(-1)^{i+j}d_i^{n+1}(\sigma_j) + \Sum \limits_{0 \leq j < i \leq n+1} (-1)^{i+j}d_i^{n+1}(\sigma_j)\\
			&= \Sum \limits_{0 \leq i \leq j \leq n}(-1)^{i+j}d_i^{n+1}(\sigma_j) + \Sum \limits_{1 \leq j < i \leq n+1} (-1)^{i+j}d_i^{n+1}(\sigma_j)\\
			&= \Sum \limits_{i \leq j} (-1)^{i+j}\sigma_j\vert_{[v_0,...,\widehat{v_i},...,v_j,w_j,...,w_n]} + \Sum \limits_{i > j}(-1)^{i+j}\sigma_j\vert_{[v_0,...,v_j,w_j,...,\widehat{w_{i-1}},...,w_n]}
		\end{align*}
	\end{proof}

	\begin{theorem}
		Relative singular homology satisfies excission.
	\end{theorem}

	\newpage
	\subsection{Special Properties of Singular Homology}

	\begin{lemma}
		Let $X$ be a topological space. Then there is a natural isomorphism
		\begin{equation*}
			H_0^\text{sing}(X) \isom \Z[\largepi_0(X)]
		\end{equation*}
		where $\largepi_0(X)$ denotes the set of path components of $X$.
	\end{lemma}
	\begin{sketch}
		Use the identifications from remark \TODO{cite} and that $\Z[-]$ is a left adjoint to calculate
		\begin{align*}
			H_0^\text{sing}(X) &\isom \coker \left(
				\begin{diagram}
					\twobyone[ultrawide]
						{\Z[\ncat{Top}([0,1],X)]}{\Z[\ncat{Top}(*,X)]}
					\arrow{w}{e}{\gamma(1) - \gamma(0)}[gray,above]
				\end{diagram}
			 \right)\\
			 &= \coeq \left(
				\begin{diagram}
					\twobyone[ultrawide]
						{\Z[\ncat{Top}([0,1],X)]}{\Z[\ncat{Top}(*,X)]}
					\arrow[higher]{w}{e}{\gamma(1)}[gray,above]
					\arrow[lower]{w}{e}{\gamma(0)}[gray,below]
				\end{diagram}
			 \right)\\
			 &\isom \Z\left [\coeq \left(
 				\begin{diagram}
 					\twobyone[verywide]
 						{\ncat{Top}([0,1],X)}{\ncat{Top}(*,X)}
 					\arrow[higher]{w}{e}{}
 					\arrow[lower]{w}{e}{}
 				\end{diagram}
 			 \right)\right]\\
 			 &\isom \Z[\largepi_0(X)].
		\end{align*}
		{\vspace{-2em}}
	\end{sketch}

	\begin{proposition}
		Let $(X,A)$ be a pair of spaces and $A \subseteq X_0 \subseteq X_1 \subseteq ... \;X = \smash{\colim \limits_{i\in \N_0}U_i}$ a filtration of $X$. Then the canonical morphism
		\begin{equation*}
			\colim \limits_{i \in \N_0} H^\text{sing}(X_i,A) \xrightarrow{\pi} H^\text{sing}(X,A)
		\end{equation*}
		is an isomorphism.
	\end{proposition}
	\begin{sketch}
		Note that every compact subset $K \subseteq X$ is contained in some $X_i$, since $(K \intersection \int X_i)_i$ is an open cover of $K$. In particular for any $n$-cycle $c \in C_n^\text{sing}(X)$ the support is contained in some $X_i$, hence all $n$-simplices contributing to $c$ have their image in $X_i$ and can be identified with $n$-simplices in $X_i$. Therefore $c$ can be seen as a chain $c' \in C_n^\text{sing}(X_i)$.

		We first show that $\pi$ is surjective, so let $[[c]] \in H_n^\text{sing}(X,A)$. By the previous argument $c \in C_n^\text{sing}(X)$ can be identified with a cycle $c' \in C_n^\text{sing}(X_i)$, which gives $[[c']] \in H_n^\text{sing}(X_i,A)$ with $\pi([[c']]) = [[c]].$

		\textbf{Warning}: 
		\vspace{-1.8em}
		\begin{tab}[2cm]
			In the following the class $[c]$ will be considered as an element of $C_n^\text{sing}(X_i,A)$, $C_n^\text{sing}(X_j,A)$ and $C_n^\text{sing}(X,A)$ depending on the context. I did not find a way around this without notation getting out of hand.
		\end{tab}

		To show injectivity let $[[c]] \in H_n(X_i,A)$ be a representative of an homology class in $\colim \limits_{i \in \N_0} H_n^\text{sing}(U_i,A)$, which is mapped to $0 \in H_n^\text{sing}(X,A)$. This means that there is an ($n+1$)-chain $[d] \in C_{n+1}(X,A)$ such that $[c] - \partial_{n+1}([d]) = 0 \in C_n^\text{sing}(X,A)$. By the previous argument there is a $j \geq i$ such that $[d]$ can be identified with $[d'] \in C_{n+1}^\text{sing}(X_j,A)$. But then $[c]-\partial_{n+1}[d'] = 0 \in C_n^\text{sing}(X_j,A)$, so $[[c]] = 0 \in H_n^\text{sing}(X_j,A)$, which represents $0 \in \colim \limits_{i \in \N_0} H_n^\text{sing}(X,A)$.
	\end{sketch}

	% \begin{proposition}
	% 	Singular homology preserves filtered colimits, ie. the canonical morphism 
	% 	\begin{align*}
	% 		\fcolim \limits_{i \in I} H_\bullet^\text{sing}(X_i,A_i) --> H_\bullet^\text{sing}(\fcolim \limits_{i \in I} (X_i,A_i)) = H_\bullet^\text{sing}(X,A)
	% 	\end{align*}
	% 	is an isomorphism.

	% 	In particular, given a filtration $A \subseteq U_0 \subseteq U_1 \subseteq ... \;X = \smash{\colim \limits_{i\in \N_0}U_i}$ of $X$ by open subsets $U_i$ we have $$\colim_i H_\bullet^\text{sing}(U_i) \isom H_\bullet^\text{sing}(X).$$
	% \end{proposition}
	% \begin{sketch}
	% 	Surjective:
	% 	\begin{tab}
	% 		Let $[[c]] \in H_n^\text{sing}(X,A)$, i.e. $[c] \in C_n^\text{sing}(X,A)$ and $c \in C_n^\text{sing}(X)$.
	% 	\end{tab}
	% \end{sketch}

	\begin{theorem}[Eilenberg Zilber]
		Let $X,Y$ be topological spaces. Then there is a chain homotopy equivalence
		\begin{equation*}
			C_\bullet^\text{sing}(X \times Y) \xrightarrow{\simeq} C_\bullet^\text{sing}(X) \tensor C_\bullet^\text{sing}(Y),
		\end{equation*}
		which is natural in $X$ and $Y$.
	\end{theorem}

	\begin{corollary}[Künneth Formula]
		Let $X,Y$ be topological spaces. For any $n$ there is a short exact sequence
		\begin{align*}
			0 --> \Oplus \limits_{p+q=n}H_p^\text{sing}(X) \tensor H_q^\text{sing}(Y) --> H_n^\text{sing}(X \times Y) --> \Oplus \limits_{p+q=n} \Tor(H_p^\text{sing}(X),H_q^\text{sing}(Y)) --> 0,
		\end{align*}
		which is natural in $X$ and $Y$ \TODO{and compatible with boundaries?}
	\end{corollary}

	\newpage
	\subsection{Applications of Singular Homology}

	\begin{proposition}
		\vspace{-1.5em}
		\begin{enumerate}[(i)]
			\item{
				For an embedding $f:\D^k \longrightincl \S^n$ we have $\widetilde{H}_\bullet^\text{sing}(\S^n \setminus f(\D^k),x_0) = 0$.
			}
			\item{
				For $k<n$ and an embedding $g: \S^k \longrightincl \S^n$ it holds that $\widetilde{H}_\bullet^\text{sing}(\S^n \setminus \S^k) = \Z_{\bullet-n+k+1}$.
			}
		\end{enumerate}
	\end{proposition}
	\begin{sketch}
		\begin{enumerate}[(i)]
			\item{
				By induction on $k$.
				\begin{enumerate}
					\item[(IB)]{
						For $k=0$ we calculate $\widetilde{H}_\bullet^\text{sing}(\S^n \setminus f(\D^0),x_0) \isom \widetilde{H}_\bullet^\text{sing}(\R^n,y_0) = 0$.
					}
					\item[(IH)]{
						\TODO{TODO}
					}
					\item[(IS)]{
						For $k-1 \leadsto k$ note that $\D^k \isom [0,1]^k$.

						\textit{Assume} there is an $m \in \N$ and a nonzero element $[[x]] \in \widetilde{H}_m^\text{sing}(\S^n \setminus f([0,1]^k),x_0)$
						\begin{tab}[1.3cm]
							Let\vspace{-2.2em}
							\begin{align*}
								A_1 &= \S^n \setminus f([0,1]^{k-1}\times[0,\tfrac{1}{2}])\\
								A_2 &= \S^n \setminus f([0,1]^{k-1}\times[\tfrac{1}{2},1])\\
								A_1 \union A_2 &= \S^n \setminus f([0,1]^{k-1}\times\{\tfrac{1}{2}\})\\
								A_1 \intersection A_2 &= \S^n \setminus f([0,1]^k).
							\end{align*}
							Mayer Vietories for the cover $A_1, A_2$ of $A_1 \union A_2$ yields by induction hypothesis (IH) an isomorphism 
							\begin{equation*}
								\widetilde{H}_m^\text{sing}(A_1 \intersection A_2,x_0) \isom \widetilde{H}_m^\text{sing}(A_1,x_0) \oplus \widetilde{H}_m^\text{sing}(A_2,x_0)
							\end{equation*}
							and $[[x]]$ maps to a nonzero element in at least one of the homology groups of $A_1$ or $A_2$.
							Inductively we obtain a sequence of intervals 
							\begin{equation*}
								[0,1] = I \supseteq I_1 \supseteq I_2 \supseteq ... 
							\end{equation*}
							with the property that $\widetilde{H}_m^\text{sing}(\S^n \setminus f(I^{k-1}\times I_i),x_0)$ is nonzero.  Note that
							\begin{align*}
								\Union \limits_{i \in \N} \S^n\setminus f(I^{k-1} \times I_i) &= \S^n \setminus \Intersection \limits_{i \in \N} f(I^{k-1} \times I_i)\\ 
								&= \S^n \setminus f(\Intersection \limits_{i \in \N} I^{k-1} \times I_i)\\
								&= \S^n \setminus f(I^{k-1} \times \{t\})
							\end{align*}
							Since singular homology preserves filtrations and by induction hypothesis (IH) we get
							\begin{equation*}
								0 \neq \colim \limits_{i \in \N} \widetilde{H}_m^\text{sing}(\S^n \setminus f(I^{k-1}\times I_i),x_0) \isom \widetilde{H}_m^\text{sing}(\S^n \setminus f(I^{k-1}\times \{t\}),x_0) = 0.
							\end{equation*}
							A contradiction.
						\end{tab}
					}
				\end{enumerate}
			}
			\item{
				By induction on $k$.
				\begin{enumerate}
					\item[(IB)]{
						For $k=0$ note that $\S^n \setminus f(\S^0) \isom \R \times \S^{n-1} \simeq \S^{n-1}$, so $\widetilde{H}_\bullet^\text{sing}(\S^n \setminus \S^0) \isom \widetilde{H}_\bullet^\text{sing}(\S^{n-1}) = \Z_{\bullet-n+0+1}$.
					}
					\item[(IH)]{
						\TODO{TODO}
					}
					\item[(IS)]{
						For $k-1 \leadsto k$ consider $\S^k$ as the pushout of upper and lower hemispheres $\D^k_{+}$ and $\D^k_{-}$. Let
						\begin{align*}
							B_1 &= \S^n \setminus f(\D^k_{+})\\
							B_1 &= \S^n \setminus f(\D^k_{+})\\
							B_1 \union B_2 &= \S^n \setminus f(\S^{k-1})\\
							B_1 \intersection B_2 &= \S^n \setminus f(\S^k).
						\end{align*}
						Mayer Vietories for the cover $B_1,B_2$ of $B_1 \union B_2$ in combination with (i) yield an isomorphism
						\begin{equation*}
							\widetilde{H}_\bullet^\text{sing}(\S^n \setminus f(\S^{k-1})) \isom \widetilde{H}_{\bullet-1}^\text{sing}(\S^n \setminus f(\S^{k})),
						\end{equation*}
						so by induction hypothesis we get
						\begin{equation*}
							\widetilde{H}_{\bullet}^\text{sing}(\S^n \setminus f(\S^{k})) \isom \widetilde{H}_{\bullet+1}^\text{sing}(\S^n \setminus f(\S^{k-1})) \isom \Z_{(\bullet-n+(k-1)+1)+1}.
						\end{equation*}
					}
				\end{enumerate}
			}\vspace{-2.5em}
		\end{enumerate}
	\end{sketch}

	\begin{corollary}[Jordan Curve Theorem]
		Any embedding $f:\S^{n-1} \longrightincl \S^n$ splits $\S^n$ into precisely two pathconnected components.
	\end{corollary}
	\begin{proof}
		~\vspace{-2.55em}
		\begin{align*}
			\# \largepi_0(\S^n \setminus f(\S^{n-1})) &= \rk \Z[\largepi_0(\S^n \setminus f(\S^{n-1}))]\\
			&= \rk H_0(\S^n \setminus f(\S^{n-1}))\\
			&= \rk \left (\widetilde{H}_0(\S^n \setminus f(\S^{n-1})) \oplus \Z\right)\\
			&= \rk \left (\Z_{0-n+(n-1)+1} \oplus \Z\right) = 2
		\end{align*}
		~\vspace{-2em}
	\end{proof}

	\begin{corollary}[Invariance of Domain]
		If $U$ is an open subset of $\R^n$ and $f: U \longrightincl \R^n$ is any embedding, then $f(U)$ is open.
	\end{corollary}
	\begin{proof}
		Let $x \in U$ and $D$ be a closed ball around $x$ contained in $U$. Show that $f(\interior D)$ is open in $\S^n$. By JCT $\S^n \setminus f(\boundary D)$ has exactly two path components, namely $f(\interior D)$ and $\S^n \setminus f(D)$. These path components are open, so in particular $f(\interior D)$ is an open neighborhood of $f(x)$ in $f(U)$.
	\end{proof}

	\begin{corollary}
		If $M$ is a compact topological $n$-manifold and $N$ is a connected topological $n$-manifold, then every embedding $f: M \longrightincl N$ is a homeomorphism.

		In particular every embedding $\S^n \longrightincl \S^n$ is a homeomorphism.
	\end{corollary}
	\begin{proof}
		$f(M)$ is a closed subset of $N$ as a compact subset of a hausdorff space. Since $N$ is connected we find that $f$ is a homeomorphism, if $f(M)$ is open. Let $x \in M$ and $U$ be an euclidean neighborhood. Moreover $f(x)$ is contained in an euclidean neighborhood $V$, so we find an open neighborhood $W = U \intersection f^{-1}(V)$ of $x$, which by invariance of domain is mapped to an open set $f(W)$ containing $x$. Hence $f(M)$ is open.
	\end{proof}

	\begin{corollary}[Nonexistence of Embeddings]
		\vspace{-1.5em}
		\begin{enumerate}[(i)]
			\item{
				For $m > n$ there is no embedding $\S^m \longrightincl \S^n$.
			}
			\item{
				For $m \geq n$ there is no embedding $\S^m \longrightincl \R^n$.
			}
			\item{
				For $m > n$ there is no embedding $\R^m \longrightincl \R^n$.
			}
		\end{enumerate}
	\end{corollary}
	\begin{sketch}
		\begin{enumerate}[(i)]
			\item{
				By the previous corollary $\S^m \isom \S^n$, so by invariance of dimension $m = n$.
			}
			\item{
				The composite $\S^m \longrightincl \R^n \longrightincl \S^n$ gives an embedding of $\S^m$ into $\S^n$. Even in the case $m=n$ the second morphism is not surjective, so we don't get an homeomorphism.
			}
			\item{
				The last morphism in the composite $\S^n \longrightincl \R^m \longrightincl \R^n \longrightincl \S^n$ is not surjective.
			}
			\vspace{-1.8em}
		\end{enumerate}
	\end{sketch}

	\newpage
	\subsection{The Universal Coefficient Theorem}

	Given a $\Z$-graded abelian group $M$, considered as a chain complex with zero differential, we can extend the definition of singular homology to be concentrated in $M$. For this purpose we note that we obtain a short exact sequence of chain complexes
	\begin{align*}
		0 --> C^\text{sing}(A) \tensor M --> C^\text{sing}(X) \tensor M --> C^\text{sing}(X,A) \tensor M --> 0,
	\end{align*}
	since $C^\text{sing}(A)$ is free and hence flat, and $- \tensor M$ preserves quotients.

	\begin{definition}
		Let $M$ be a $\Z$-graded abelian group.

		The \textbf{(relative) singular chain complex functor with coefficients in $M$} is the functor
		\begin{equation*}
			\begin{array}{rcl}
				\ncat{Top}^2 & \xrightarrow{C^\text{sing}(-,- \,| M)} & \ncat{Ch}\\
				(X,A) & \longmapsto & C^\text{sing}(X,A \,| M) = C^\text{sing}(X,A) \tensor M\\
				{[(X,A)\rightarrow(Y,B)]} & \longmapsto & {[C^\text{sing}(X,A) \tensor M \rightarrow C^\text{sing}(X,A) \tensor M]}
			\end{array}
		\end{equation*}
		Postcomposing with the homology functor $H_{*}$ yields the \textbf{(relative) singular homology functor with coefficients in $M$}
		\begin{align*}
			H(-,- | M): \ncat{Top}^2 --> \ncat{Ab}^{\Z gr}.
		\end{align*}
		The corresponding boundary operator is given by the boundary map induced by the short exact sequences of the form
		\begin{align*}
			0 --> C^\text{sing}(A) \tensor M --> C^\text{sing}(X) \tensor M --> C^\text{sing}(X,A) \tensor M --> 0.
		\end{align*}
	\end{definition}

	\TODO{Eilenberg Zilber}
	\TODO{Universal Coefficient theorem}



	\newpage
	\subsection{Homological Orientation, Fundamental Classes and Degree}

	\begin{lemma}
		Let $(H_\bullet,\partial_\bullet)$ be an $R$-homology theory.

		Let $X$ be an $n$-dimensional topological manifold and $x$ be a point in $X$. Then there is an isomorphism $H_n(X,X\setminus \{x\}) \isom R$ and in particular $H_n(X,X\setminus \{x\})$ is a cyclic $R$-module.
	\end{lemma}
	\begin{proof}
		By definition of a manifold there is a chart $\phi_x: U_x \isom \R^n$ with $\phi_x(x) = 0$. By excission (of the complement of $\phi_x^{-1}(\D^n)$) we get a chain of isomorphisms
		\begin{equation*}
			H_\bullet(X,X\setminus\{x\}) \isom H_\bullet(\phi_x^{-1}(\D^n),\phi_x^{-1}(\D^n\setminus \{0\})) \isom H_\bullet(\D^n,\D^n\setminus \{0\}) \isom H_\bullet(\D^n,\S^{n-1}) \isom M_{\bullet-n}.
		\end{equation*}
		In particular we have $H_n(X,X\setminus\{x\}) \isom R$. 
	\end{proof}

	\begin{remark}
		Note that the property of $H_n(X,X\setminus \{x\})$ being cyclic is implied but does not depend on the specific isomorphism to $R$ we constructed. Hence the set of generators of this cyclic module is in no way linked to this isomorphism.
	\end{remark}

	\begin{definition}
		Let $R$ be a ring and $(H_\bullet, \partial_\bullet)$ be an $R$-homology theory. Let $X$ be an $n$-dimensional topological manifold and $x$ be in $X$. 

		A \textbf{local $R$-orientation of $X$ at $x$} is a choice of generator of the cyclic $R$-module $H_n(X,X\setminus\{x\})$.

		Let $U$ be an open neighborhood with $H_n(X,X\setminus U)$ being a nontrivial cyclic $R$-module. A \textbf{local $R$-orientation of $X$ along $U$} is a choice of generator $g_U$ of the cyclic $R$-module $H_n(X,X\setminus U)$. If $g_x$ is a local $R$-orientation a point $x$ in $U$ and $g_U$ is mapped onto $g_x$ under the induced map $H_n(X,X\setminus U) --> H_n(X,X\setminus \{x\})$, we say that $g_U$ is a \textbf{continuation} of $g_x$ onto $U$.
	\end{definition}

	\begin{lemma}[Continuation Lemma]
		Let $(H_\bullet, \partial_\bullet)$ be an $R$-homology theory, $X$ be an $n$-dimensional topological manifold and $W \subseteq X$ an open neighborhood of a point $x$ in $X$.

		Then there exists an open neighborhood $U$ of $x$ such that $U \subseteq W$ and $H_n(X,X\setminus U) \isom R$, that is to say $H_n(X,X\setminus U)$ is a nontrivial cyclic $R$-module and thus $X$ admits a local $R$-orienation along $U$.

		Furthermore, given a local $R$-orienation $g_x \in H_n(X,X\setminus\{x\})$, there is a unique continuation $g_U$ of $g_x$ onto $U$.
	\end{lemma}
	\begin{proof}
		By definition of a topological manifold there is a chart $\phi_x: U_x \isom \R^n$ with the property that $\phi_x(x)=0$ and $U_x \subseteq W$. Set $U = \phi_x^{-1}(\B^n)$. By homotopy invariance we have an induced isomorphism
		\begin{equation*}
			H_\bullet(X,X\setminus U) \isom H_\bullet(X,X\setminus\{x\})
		\end{equation*}
		since $U$ is contractible to $\{x\}$. This isomorphism gives $H_n(X,X\setminus U) \isom R$, which turns $H_n(X,X\setminus U)$ into a nontrivial cyclic $R$-module. Furtheremore under its inverse a generator $g_x$ of $H_n(X,X\setminus\{s\})$ maps uniquely to a generator $g_U$ of $H_n(X,X\setminus U)$. Finally, since $U$ is contractible to any other point $y$ in $U$, we also have an induced isomorphism
		\begin{equation*}
			H_\bullet(X,X\setminus U) \isom H_\bullet(X,X\setminus \{y\}),
		\end{equation*}
		which in degree $n$ give us generators $g_y$ of $H_n(X,X\setminus \{y\})$ as images of $g_U$.
	\end{proof}

	\begin{definition}
		Let $(H_\bullet, \partial_\bullet)$ be an $R$-homology theory and $X$ be an $n$-dimensional topological manifold.

		An \textbf{$R$-orientation} of $X$ consists of a choice of generator $g_x \in H_n(X,X\setminus\{x\})$ for each point $x$ in $X$, such that each $x$ admits an open neighborhood $U$, with $H_n(X,X\setminus U)$ being a nontrivial \TODO{cyclic $R$-module} and having a local $R$-orientation $g_U$, which is a continuation of all $g_y$ for $y$ in $U$.

		$X$ is \textbf{$R$-orientable}, if it satisfies one of the following equivalent conditions.
		\begin{enumerate}[(i)]
			\item{
				admits  an $R$-orientation.
			}
			\item{
				\TODO{oriented atlas, oriented tangent microbundle see switzer}
			}
		\end{enumerate}
	\end{definition}

	\begin{lemma}[Properties of $R$-orientations]\vspace{-1.5em}
		\begin{enumerate}[(i)]
			\item{
				Open submanifolds of $R$-orientable manifolds are $R$-orientable.
			}
			\item{
				A manifold is $R$-orientable, if and only if all its components are $R$-orientable.
			}
			\item{
				Let $X$ be a connected $R$-orientable manifold. Two $R$-orientations, which agree at one point, are equal.
			}
		\end{enumerate}
	\end{lemma}

	\begin{proposition}
		Let $(H_\bullet, \partial_\bullet)$ be an $R$-homology theory and $X$ be an $n$-dimensional, closed, connected, topological manifold. 

		Then $X$ is $R$-orientable if and only if $H_n(X) \isom R$.
	\end{proposition}
	\begin{proof}
		\TODO{via Mayer Vietoris?}
	\end{proof}

	\begin{definition}
		Let $(H_\bullet, \partial_\bullet)$ be an $R$-homology theory and $X$ be an $n$-dimensional, closed, connected, $R$-orientable, topological manifold. A generator $[X]$ of $H_n(X)$ is called a \textbf{fundamental class} of $X$.
	\end{definition}
\end{document}