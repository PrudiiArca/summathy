\documentclass{article}

% --- text layout
\usepackage[parfill]{parskip}
\usepackage{geometry}
\geometry{a4paper, margin=3cm}
\usepackage{romanbar}
\usepackage{placeins} %FloatBarrier
\usepackage[utf8]{inputenc}

\usepackage{xcolor}
\definecolor{darkred}{HTML}{990000}
\definecolor{darkgreen}{HTML}{009900}

\usepackage{sectsty}
\sectionfont{\color{darkred}}
\subsectionfont{\color{darkred}}
\subsubsectionfont{\color{darkred}}
% \usepackage{algpseudocode} --> better pseudo code?
%\usepackage[pagewise]{lineno} % row numbers

% --- syncing, referencing, citing
\usepackage{pdfsync} 			% synchronization LaTeX <--> PDF
\usepackage[hidelinks]{hyperref} % hidelinks avoids red boxes around links
\usepackage{csquotes}			% enquote
\usepackage{cleveref}
\usepackage{pdfpages}

\usepackage[backend=bibtex,style=alphabetic,url=false]{biblatex}
\addbibresource{sources.bib}
% define bibliography filter like
%\defbibfilter{computability-pca}{
%	keyword=computability or keyword=pca
%}

% --- lists and tables
\usepackage[shortlabels]{enumitem}
\usepackage{tabularx}
\usepackage{tabu}
\usepackage{makecell}

% --- graphics
\usepackage{graphicx}
\usepackage{xcolor}

% --- math
\usepackage{amsmath}
\usepackage{amssymb}
\usepackage{fixmath} % whyyy? "bug": italic PI and SIGMA
\usepackage{mathtools}
\usepackage{mathrsfs} %mathscr

% --- own packages
\usepackage{../notation}
\usepackage{../diagrams}
\usepackage{../environments}

\title{$\Sum\text{math}(y)$\\Generalized Homology Theories}
\author{Jonas Linssen}

\begin{document}
	\maketitle
	\tableofcontents

	\newpage
	\section{Generalized Homology}
	\subsection{Definition and Exact Sequences}

	\begin{definition}
		A \textbf{generalized} (\textbf{unreduced, relative}) \textbf{homology theory} on $\ncat{Top}^2$ is a functor $H_\bullet: \ncat{Top}^2 --> \ncat{Mod}_R^{\Z\text{gr}}$ together with a natural transformation $\partial_\bullet: H_\bullet ==> H_{\bullet-1} \circ R$, the so called \textbf{boundary operator} or \textbf{connecting homomorphism}, which satisfy the following axioms.
	\begin{enumerate}[$\bullet$]
		\item{
			\textit{homotopy invariance}\\
			$H_\bullet$ factors via $\ncat{hTop}^2$.
		}
		\item{
			\textit{exactness}\\
			For any object $(X,A)$ in $\ncat{Top}^2$ the sequence
			\begin{equation*}
				...
				\begin{diagram}
					\fivebyone[verywide]
						{H_{\bullet+1}(X,A)}
						{H_\bullet(A,\emptyset)}
						{H_\bullet(X,\emptyset)}
						{H_\bullet(X,A)}
						{H_{\bullet-1}(A,\emptyset)}

					\arrow{ww}{w}{\partial_{\bullet+1}}[above]
					\arrow{w}{c}{H_\bullet(i)}[above]
					\arrow{c}{e}{H_\bullet(j)}[above]
					\arrow{e}{ee}{\partial_\bullet}[above]
				\end{diagram}
				...
			\end{equation*}
				is exact, where $i: (A,\emptyset) \longrightincl (X,\emptyset)$ and $j: (X,\emptyset) \longrightincl (X,A)$ are the inclusion maps.

		}
		\item{
			\textit{excission}\\
			For every object $(X,A)$ in $\ncat{Top}^2$ and subset $U \subseteq \closure U \subseteq \interior A$, the inclusion $(X\setminus U, A \setminus U) \subseteq (X, A)$ induces an isomorphism $$H_\bullet(X\setminus U, A \setminus U) \isom H_\bullet(X,A).$$
		}
	\end{enumerate}
	One might further assume
	\begin{enumerate}[$\bullet$]
		\item{
			\textit{additivity}\\
			$H_\bullet$ preserves coproducts. Specifically, for any family $(X_i,A_i)_i$ in $\ncat{Top}^2$ the canonical morphism
			\begin{equation*}
				\bigoplus \limits_{i \in I}H_\bullet(X_i,A_i) \longrightarrow H_\bullet(\Coproduct \limits_{i \in I} (X_i,Y_i))
			\end{equation*}
			is an isomorphism.
		}
		\item{
			\textit{dimension}\\
			For all $n \neq 0$ it holds that $H_n(*,\emptyset) = 0$.

			In this case we call $(H_\bullet,\partial_\bullet)$ an \textbf{ordinary} (\textbf{unreduced, relative}) \textbf{homology theory}.
		}
	\end{enumerate}
	\end{definition}

	In the following we will abbreviate $H_\bullet(X,\emptyset)$ to $H_\bullet(X)$.

	\begin{lemma}[First Calculations]
		Let $(H_\bullet, \partial_\bullet)$ be a generalized homology theory.
		\begin{enumerate}[(i)]
			\item{
				If $f:(X_1,A_1) \xrightarrow{\simeq} (X_2,A_2)$ is a homotopy equivalence, then $H_\bullet(X_1,A_1) \isom H_\bullet(X_2,A_2)$.

				In particular for $n \in \N$ we have
				\begin{equation*}
					H_\bullet(*) \isom H_\bullet(\D^n) \isom H_\bullet(\R^n) \hspace{1cm}\text{and}\hspace{1cm} \H_\bullet(\S^{n-1}) \isom H_\bullet(\D^n \setminus \{0\}).
				\end{equation*}
			}
			\item{
				If $A \subseteq X$ is a homotopy equivalence, then $H_\bullet(X,A) \isom 0$, in particular $H_\bullet(X,X) \isom 0$.
			}
			\item{
				If $H_\bullet$ is additive, it holds that $H_\bullet(X_\text{disc}) = \Directsum \limits_X H_\bullet(*)$.
			}
		\end{enumerate}
	\end{lemma}
	\begin{proof}
		\begin{enumerate}[(i)]
			\item{
				By homotopy invariance $f$ becomes an isomorphism in $\ncat{hTop}$, which is then mapped to a isomorphism in homology.
			}
			\item{
				Consider the long exact sequence
				\begin{equation*}
					\begin{diagram}
						\fivebyone[verywide]
							{H_{\bullet}(A)}{H_{\bullet}(X)}{H_\bullet(X,A)}{H_{\bullet-1}(A)}{H_{\bullet-1}(X)}
						\arrow{ww}{w}{}
						\arrow{w}{c}{}
						\arrow{c}{e}{\partial_\bullet}[above]
						\arrow{e}{ee}{}
					\end{diagram}
				\end{equation*}
				and note that as in (i) we have that $H_\bullet(A) --> H_\bullet(X)$ is an isomorphism. Hence by exactness we have $H_\bullet(X,A) \isom 0$.
			}
		\end{enumerate}
	\end{proof}

	\begin{proposition}[Mayer Vitoris]
		Let $(H_\bullet,\partial_\bullet)$ be a generalized homology theory and $B \subseteq A_1,A_2 \subseteq X$ in $\ncat{Top}$, such that $X \subseteq \interior(A_1) \union \interior(A_2)$. Then the following sequence is exact.
		\begin{equation*}
			\begin{diagram}
				\threebythree[hyperwide]
					{}{}{...H_{\bullet+1}(X,B)}
					{H_\bullet(A_1 \intersection A_2,B)}{H_\bullet(A_1,B)\directsum H_\bullet(A_2,B)}{H_\bullet(X,B)}
					{H_{\bullet-1}(A_1 \intersection A_2, B)...}{}{}

				\arrow{w}{c}{H_\bullet(i_1)\directsum H_\bullet(i_2)}[above]
				\arrow{c}{e}{H_\bullet(j_1) - H_\bullet(j_2)}[above]
				
				\draw[->,rounded corners] (ne.east) -| node[auto,pos=.7]{$\partial_{\bullet+1}$}($(ne)+(1.5,-.75)$) -| ($(w)+(-2,0)$) |- (w.west);
				\draw[->,rounded corners] (e.east) -| node[auto,pos=.7]{$\partial_\bullet$}($(e)+(1.5,-.75)$) -| ($(sw)+(-2,0)$) |- (sw.west);
			\end{diagram}
		\end{equation*}
	\end{proposition}

	\begin{proposition}[Triple Sequence]
		Let $(H_\bullet,\partial_\bullet)$ be a generalized homology theory and $B \subseteq A \subseteq X$ in $\ncat{Top}$. Given
		\begin{equation*}
			\Delta_\bullet :=
			\begin{diagram}
				\threebyone[wide]
					{H_\bullet(X,A)}{H_{\bullet-1}(A)}{H_{\bullet-1}(A,B)}
				\arrow{w}{c}{\partial_\bullet}[above]
				\arrow{c}{e}{}
			\end{diagram}
		\end{equation*}
		the following sequence is exact.
		\begin{equation*}
			...
			\begin{diagram}
				\fivebyone[verywide]
					{H_{\bullet+1}(X,A)}{H_\bullet(A,B)}{H_\bullet(X,B)}{H_\bullet(X,A)}{H_{\bullet-1}(A,B)}

				\arrow{ww}{w}{\Delta_{\bullet+1}}[above]
				\arrow{w}{c}{}[above]
				\arrow{c}{e}{}[above]
				\arrow{e}{ee}{\Delta_\bullet}[above]
			\end{diagram}
			...
		\end{equation*}
	\end{proposition}

	\newpage
	\subsection{Reduced Homology}

	\begin{lemma}
		Let $(H_\bullet, \partial_\bullet)$ be a generalized homology theory and $x$ be a point in a topological space $X$. Then for any $n \in \Z$ the sequence
		\begin{equation*}
			\begin{diagram}
				\threebyone[wide]
					{H_n(\{x\})}{H_n(X)}{H_n(X,\{x\})}
				\arrow{w}{c}{}
				\arrow{c}{e}{}
			\end{diagram}
		\end{equation*}
		is split exact with section $H_n(c):H_n(X) --> H_n(\{x\})$, where $c: X --> \{x\}$ denotes the constant map.

		In particular it holds that $H_\bullet(X,\{x_1\}) \isom H_\bullet(X,\{x_2\})$ and we get $H_\bullet(X) \isom H_\bullet(*) \oplus H_\bullet(X,*)$.
	\end{lemma}

	\begin{definition}
		The \textbf{module of coefficients} of a generalized homology theory $(H_\bullet, \partial_\bullet)$ is the graded module $M_\bullet := H_\bullet(*)$.

		The \textbf{reduced homology} of $H_\bullet$ is the functor \TODO{involve kernel}
		\begin{equation*}
			\begin{array}{rcl}
				\ncat{Top}_* & \xrightarrow{\widetilde{H}_\bullet} & \ncat{Mod}_R^{\Z\text{gr}}\\
				(X,x_0) & \longmapsto & H_\bullet(X,\{x_0\})\\
				f:(X,x_0)\rightarrow (Y,y_0) & \longmapsto & H_\bullet(f)
			\end{array}
		\end{equation*}
	\end{definition}

	\begin{remark}
		\TODO{mapping cone and reduced suspension}
	\end{remark}

	\begin{lemma}
		Let $(H_\bullet,\partial_\bullet)$ be a generalized homology theory.
		\begin{enumerate}[(i)]
			\item{
				$\widetilde{H}_\bullet$ is homotopy invariant, i.e. factors via $\ncat{hTop}_{*}$.
			}
			\item{
				Given $i_A:A \longrightincl X$ in $\ncat{Top}$ and $x_0 \in A$ it holds that $H_\bullet(X,A) \isom \widetilde{H}_\bullet(\Cone(i_A), x_0)$.
			}
			\item{
				Given $x_0 \in A$, the long exact sequence of $(X,A)$ induces a long exact sequence.
				\begin{equation*}
					\begin{diagram}
						\threebythree[ultrawide]
							{}{}{...\widetilde{H}_{\bullet+1}(\Cone(i_A),x_0)}
							{\widetilde{H}_\bullet(A,x_0)}{\widetilde{H}_\bullet(X,x_0)}{\widetilde{H}_\bullet(\Cone(i_A),x_0)}
							{\widetilde{H}_{\bullet-1}(A,x_0)...}{}{}

						\arrow{w}{c}{}
						\arrow{c}{e}{}

						\draw[->,rounded corners] (ne.east) -| node[auto,pos=.7]{${\partial}_{\bullet+1}$}($(ne)+(2,-.75)$) -| ($(w)+(-1.5,0)$) |- (w.west);
						\draw[->,rounded corners] (e.east) -| node[auto,pos=.7]{${\partial}_\bullet$}($(e)+(2,-.75)$) -| ($(sw)+(-1.5,0)$) |- (sw.west);
					\end{diagram}
				\end{equation*}
			}
			\item{
				\TODO{suspension}
			}
		\end{enumerate}
	\end{lemma}

	\begin{lemma}[Reduced Mayer Vitoris]
		Let $(H_\bullet,\partial_\bullet)$ be a generalized homology theory and $A_1,A_2 \subseteq X$ in $\ncat{Top}$, such that $X \subseteq \interior(A_1) \union \interior(A_2)$ and $x_0 \in A_1 \intersection A_2$. Then the following sequence is exact.
		\begin{equation*}
			\begin{diagram}
				\threebythree[hyperwide]
					{}{}{...\widetilde{H}_{\bullet+1}(X,x_0)}
					{\widetilde{H}_\bullet(A_1 \intersection A_2,x_0)}{\widetilde{H}_\bullet(A_1,x_0)\directsum \widetilde{H}_\bullet(A_2,x_0)}{\widetilde{H}_\bullet(X,x_0)}
					{\widetilde{H}_{\bullet-1}(A_1 \intersection A_2,x_0)...}{}{}

				\arrow{w}{c}{H_\bullet(i_1)\directsum H_\bullet(i_2)}[above]
				\arrow{c}{e}{H_\bullet(j_1) - H_\bullet(j_2)}[above]
				
				\draw[->,rounded corners] (ne.east) -| node[auto,pos=.7]{$\partial_{\bullet+1}$}($(ne)+(1.5,-.75)$) -| ($(w)+(-2,0)$) |- (w.west);
				\draw[->,rounded corners] (e.east) -| node[auto,pos=.7]{$\partial_\bullet$}($(e)+(1.5,-.75)$) -| ($(sw)+(-2,0)$) |- (sw.west);
			\end{diagram}
		\end{equation*}
	\end{lemma}

	\newpage
	\subsection{Examples and Applications}

	\begin{lemma}[Homology of the Sphere]
		Let $(H_\bullet, \partial_\bullet)$ be a generalized homology theory satisfying additivity and $n \in \N$. There are isomorphisms
		\begin{equation*}
			\widetilde{H}_\bullet(\S^n,*) \isom H_\bullet(\D^n,\S^{n-1}) \isom \widetilde{H}_{\bullet-1}(\S^{n-1},*),
		\end{equation*}
		which imply
		\begin{equation*}
			H_\bullet(\S^n) \isom M_\bullet \oplus M_{\bullet-n} \hspace{1cm}\text{and}\hspace{1cm} \widetilde{H}_\bullet(\S^n,*) \isom M_{\bullet-n}.
		\end{equation*}
	\end{lemma}
	\begin{proof}
		Consider the canoncial decomposition $\S^n = \S^n_{+} \union \S^n_{-}$ into upper and lower halfs and let $x_0$ denote the north pole in $\S^n$. Note that by homotopy invariance $\widetilde{H}_\bullet(\S^n_{+},x_0) \isom \widetilde{H}_\bullet(\D^n,0) \isom 0$. Hence by the long exact sequence of $\widetilde{H}_\bullet$ we have
		\begin{equation*}
			\begin{diagram}
				\fourbyone[ultrawide]
					{0\isom\widetilde{H}_\bullet(\S^n_{+},x_0)}{\widetilde{H}_\bullet(\S^n,x_0)}{H_\bullet(\S^n,\S^n_{+})}{\widetilde{H}_{\bullet-1}(\S^n_{+},x_0)\isom0}
				\arrow{ww}{w}{}
				\arrow{w}{e}{}
				\arrow{e}{ee}{\partial_\bullet}[above]
			\end{diagram},
		\end{equation*}
		which gives an isomorphism $\widetilde{H}_\bullet(\S^n,x_0) \isom H_\bullet(\S^n,\S^n_{+})$. By applying excision of $x_0$ and using homotopy invariance we get a chain of isomorphisms
		\begin{equation*}
			\widetilde{H}_\bullet(\S^n,x_0) \isom H_\bullet(\S^n,\S^n_{+}) \isom H_\bullet(\S^n \setminus \{x_0\}, \S^n_{+} \setminus \{x_0\}) \isom H_\bullet(\D^n,\S^{n-1}).
		\end{equation*}
		Denote by $x_1$ the north pole in $\S^{n-1}$. Again the long exact sequence of $\widetilde{H}_\bullet$ gives
		\begin{equation*}
			\begin{diagram}
				\fourbyone[ultrawide]
					{0 \isom \widetilde{H}_\bullet(\D^n,x_1)}{H_\bullet(\D^n,\S^{n-1})}{\widetilde{H}_{\bullet-1}(\S^{n-1},x_1)}{\widetilde{H}_{\bullet-1}(\D^n,x_1) \isom 0}
				\arrow{ww}{w}{}
				\arrow{w}{e}{\partial_\bullet}[above]
				\arrow{e}{ee}{}
			\end{diagram}
		\end{equation*}
		and we get an isomorphism $H_\bullet(\D^n,\S^{n-1}) \isom \widetilde{H}_{\bullet-1}(\S^{n-1})$. We can now explicitly calculate the homology of $\S^n$ (with $n \in \N_0$ now) by induction.
		\begin{enumerate}
			\item[(IB)]{
				For $n=0$ we have by additivity
				\begin{equation*}
					H_\bullet(\S^0) = H_\bullet(* \sqcup *) \isom H_\bullet(*) \oplus H_\bullet(*) = M_\bullet \oplus M_\bullet
				\end{equation*}
				and further \TODO{why}
				\begin{equation*}
					\widetilde{H}_\bullet(\S^n,*) = M_\bullet.
				\end{equation*}
			}
			\item[(IH)]{
				For $k<n$ it holds that $\widetilde{H}_\bullet(\S^k,*) \isom M_{\bullet-k}$.
			}
			\item[(IS)]{
				For $n$ we calculate
				\begin{equation*}
					\widetilde{H}_\bullet(\S^n,*) \isom \widetilde{H}_{\bullet-1}(\S^{n-1},*) \underset{(IH)}{\isom} M_{\bullet-1-(n-1)} = M_{\bullet-n}
				\end{equation*}
				and finally
				\begin{equation*}
					H_\bullet(\S^n) \isom M_\bullet \oplus H_\bullet(\S^n,*) \isom M_\bullet \oplus M_{\bullet-n}.
				\end{equation*}
			}\vspace{-3.5em}
		\end{enumerate}
	\end{proof}

	\begin{theorem}[Invariance of Dimension]
		Let $(H_\bullet, \partial_\bullet)$ be an ordinary additive homology theory, ie. satisfying additivity and dimension.

		Let $U \subseteq \R^m$ and $V \subseteq \R^n$ be open subsets. If there is a homeomorphism $f:U-->V$, then $m = n$.
	\end{theorem}
	\begin{proof}
		Note that for a point $x \in U$ we can calculate by homotopy invariance
		\begin{equation*}
			H_\bullet(U,U\setminus\{x\}) \isom H_\bullet(\B^m(x,r),\B^m(x,r)\setminus\{x\}) \isom H_\bullet(\D^m,\S^{m-1}) \isom \widetilde{H}_\bullet(\S^m) \isom M_{\bullet-m}
		\end{equation*}
		and similarly for $y \in V$. By the dimension axiom we then have that 
		\begin{equation*}
			M_{\bullet-m} \isom H_\bullet(U,U\setminus\{x\}) \isom H_\bullet(V,V\setminus\{f(x)\}) \isom M_{\bullet-n}
		\end{equation*}
		implies $m=n$.
	\end{proof}

	\begin{lemma}[Homology of Wedge Sums of Circles]
		\TODO{Wedge sums of circles}
	\end{lemma}

	\begin{lemma}[Homology of the Oriented Surfaces]
		\TODO{Torus and higher genus}
	\end{lemma}

	\newpage
	\subsection{Homological Orientation, Fundamental Classes and Degree}

	\begin{definition}
		Let $R$ be a (commutative unital) ring. An $R$-\textbf{homology theory} is an ordinary additive homology theory, whose module of coefficients is $M_0 = R$.
	\end{definition}

	\begin{lemma}
		Let $(H_\bullet,\partial_\bullet)$ be an $R$-homology theory.

		Let $X$ be an $n$-dimensional topological manifold and $x$ be a point in $X$. Then there is an isomorphism $H_n(X,X\setminus \{x\}) \isom R$ and in particular $H_n(X,X\setminus \{x\})$ is a cyclic $R$-module.
	\end{lemma}
	\begin{proof}
		By definition of a manifold there is a chart $\phi_x: U_x \isom \R^n$ with $\phi_x(x) = 0$. By excission (of the complement of $\phi_x^{-1}(\D^n)$) we get a chain of isomorphisms
		\begin{equation*}
			H_\bullet(X,X\setminus\{x\}) \isom H_\bullet(\phi_x^{-1}(\D^n),\phi_x^{-1}(\D^n\setminus \{0\})) \isom H_\bullet(\D^n,\D^n\setminus \{0\}) \isom H_\bullet(\D^n,\S^{n-1}) \isom M_{\bullet-n}.
		\end{equation*}
		In particular we have $H_n(X,X\setminus\{x\}) \isom R$. 
	\end{proof}

	\begin{remark}
		Note that the property of $H_n(X,X\setminus \{x\})$ being cyclic is implied but does not depend on the specific isomorphism to $R$ we constructed. Hence the set of generators of this cyclic module is in no way linked to this isomorphism.
	\end{remark}

	\begin{definition}
		Let $R$ be a ring and $(H_\bullet, \partial_\bullet)$ be an $R$-homology theory. Let $X$ be an $n$-dimensional topological manifold and $x$ be in $X$. 

		A \textbf{local $R$-orientation of $X$ at $x$} is a choice of generator of the cyclic $R$-module $H_n(X,X\setminus\{x\})$.

		Let $U$ be an open neighborhood with $H_n(X,X\setminus U)$ being a nontrivial cyclic $R$-module. A \textbf{local $R$-orientation of $X$ along $U$} is a choice of generator $g_U$ of the cyclic $R$-module $H_n(X,X\setminus U)$. If $g_x$ is a local $R$-orientation a point $x$ in $U$ and $g_U$ is mapped onto $g_x$ under the induced map $H_n(X,X\setminus U) --> H_n(X,X\setminus \{x\})$, we say that $g_U$ is a \textbf{continuation} of $g_x$ onto $U$.
	\end{definition}

	\begin{lemma}[Continuation Lemma]
		Let $(H_\bullet, \partial_\bullet)$ be an $R$-homology theory, $X$ be an $n$-dimensional topological manifold and $W \subseteq X$ an open neighborhood of a point $x$ in $X$.

		Then there exists an open neighborhood $U$ of $x$ such that $U \subseteq W$ and $H_n(X,X\setminus U) \isom R$, that is to say $H_n(X,X\setminus U)$ is a nontrivial cyclic $R$-module and thus $X$ admits a local $R$-orienation along $U$.

		Furthermore, given a local $R$-orienation $g_x \in H_n(X,X\setminus\{x\})$, there is a unique continuation $g_U$ of $g_x$ onto $U$.
	\end{lemma}
	\begin{proof}
		By definition of a topological manifold there is a chart $\phi_x: U_x \isom \R^n$ with the property that $\phi_x(x)=0$ and $U_x \subseteq W$. Set $U = \phi_x^{-1}(\B^n)$. By homotopy invariance we have an induced isomorphism
		\begin{equation*}
			H_\bullet(X,X\setminus U) \isom H_\bullet(X,X\setminus\{x\})
		\end{equation*}
		since $U$ is contractible to $\{x\}$. This isomorphism gives $H_n(X,X\setminus U) \isom R$, which turns $H_n(X,X\setminus U)$ into a nontrivial cyclic $R$-module. Furtheremore under its inverse a generator $g_x$ of $H_n(X,X\setminus\{s\})$ maps uniquely to a generator $g_U$ of $H_n(X,X\setminus U)$. Finally, since $U$ is contractible to any other point $y$ in $U$, we also have an induced isomorphism
		\begin{equation*}
			H_\bullet(X,X\setminus U) \isom H_\bullet(X,X\setminus \{y\}),
		\end{equation*}
		which in degree $n$ give us generators $g_y$ of $H_n(X,X\setminus \{y\})$ as images of $g_U$.
	\end{proof}

	\begin{definition}
		Let $(H_\bullet, \partial_\bullet)$ be an $R$-homology theory and $X$ be an $n$-dimensional topological manifold.

		An \textbf{$R$-orientation} of $X$ consists of a choice of generator $g_x \in H_n(X,X\setminus\{x\})$ for each point $x$ in $X$, such that each $x$ admits an open neighborhood $U$, with $H_n(X,X\setminus U)$ being a nontrivial \TODO{cyclic $R$-module} and having a local $R$-orientation $g_U$, which is a continuation of all $g_y$ for $y$ in $U$.

		$X$ is \textbf{$R$-orientable}, if it satisfies one of the following equivalent conditions.
		\begin{enumerate}[(i)]
			\item{
				admits  an $R$-orientation.
			}
			\item{
				\TODO{oriented atlas}
			}
		\end{enumerate}
	\end{definition}

	\begin{lemma}[Properties of $R$-orientations]\vspace{-1.5em}
		\begin{enumerate}[(i)]
			\item{
				Open submanifolds of $R$-orientable manifolds are $R$-orientable.
			}
			\item{
				A manifold is $R$-orientable, if and only if all its components are $R$-orientable.
			}
			\item{
				Let $X$ be a connected $R$-orientable manifold. Two $R$-orientations, which agree at one point, are equal.
			}
		\end{enumerate}
	\end{lemma}

	\begin{proposition}
		Let $(H_\bullet, \partial_\bullet)$ be an $R$-homology theory and $X$ be an $n$-dimensional, closed, connected, topological manifold. 

		Then $X$ is $R$-orientable if and only if $H_n(X) \isom R$.
	\end{proposition}
	\begin{proof}
		\TODO{via Mayer Vietoris?}
	\end{proof}

	\begin{definition}
		Let $(H_\bullet, \partial_\bullet)$ be an $R$-homology theory and $X$ be an $n$-dimensional, closed, connected, $R$-orientable, topological manifold. A generator $[X]$ of $H_n(X)$ is called a \textbf{fundamental class} of $X$.
	\end{definition}

	Recall that any morphism of $R$-modules $f:R --> R$ is given by multiplication with $r = f(1)$.

	\begin{definition}
		Let $(H_\bullet,\partial_\bullet)$ be an $R$-homology theory and fix an isomorphism $H_n(\S^n) \isom R$. The $R$-degree of a continuous function $f:\S^n --> \S^n$ is the element $\deg_R(f) := \tilde{f}(1)$ in $R$ for the morphism $\tilde f$ defined by the composition
		\begin{equation*}
			R \isom H_n(\S^n) \xrightarrow{H_n(f)} H_n(\S^n)\isom R
		\end{equation*}
	\end{definition}

	\begin{remark}
		By construction $\deg_R(gf) = \deg_R(g)\deg_R(f)$ and $\deg_R(\id) = 1$ in $R$. Thus we have a morphism of monoids
		\begin{equation*}
			\deg_R: \End_\ncat{Top}(\S^n) \longrightarrow (R,\cdot).
		\end{equation*}
	\end{remark}


	\newpage
	\section{Singular Homology}
	\subsection{Simplicial Spaces}

	Recall the \textbf{simplex category} $\Delta$ consisting of the total ordered sets $[n] = \{0,...,n\}$ and order preserving maps $[m] --> [n]$. In what comes the following two kinds of morphisms play a special role:

	\begin{minipage}{.65\textwidth}
		The \textbf{$i$-th boundary} of the $n$-th simplex is the map
		\begin{equation*}
			\delta_i^n: [n-1] \longrightarrow [n],\; \delta_i^n(k) = \left\{\begin{array}{ll}
				k & k<i\\
				k+1 & k\geq i
			\end{array}\right..
		\end{equation*}
	\end{minipage}
	\begin{minipage}{.35\textwidth}
		\vspace{.8cm}
		\begin{center}
		\begin{tikzpicture}
			% \node (x) at (-1.5,1.25) {$[1]$};
			% \node (y) at (1.5,1.25) {$[2]$};
			% \draw[->] (x) -- (y);
			%
			\node (a0) at (-2,0) {$0$};
			\node (a1) at (-1,0) {$1$};
			\draw[->] (a0) -- (a1);

			\node at (0,.25) {$\delta^2_1$};
			\node at (0,0) {$\longmapsto$};

			\node (b0) at (1,-.4) {$0$};
			\node[gray] (b1) at (2,-.4) {$1$};
			\node (b2) at (1.5,.4) {$2$};
			\draw[->,gray] (b0) -- (b1);
			\draw[->] (b0) -- (b2);
			\draw[->,gray] (b1) -- (b2);
		\end{tikzpicture}
		\end{center}
	\end{minipage}
	\begin{minipage}{.65\textwidth}
		The \textbf{$i$-th degeneration} of the $n$-th simplex is the map
		\begin{equation*}
			\sigma_i^n: [n] \longrightarrow [n-1],\; \sigma_i^n(k) = \left\{\begin{array}{ll}
				k & k\leq i\\
				k-1 & k > i
			\end{array}\right..
		\end{equation*}
	\end{minipage}
	\begin{minipage}{.35\textwidth}
		\vspace{.6cm}
		\begin{center}
		\begin{tikzpicture}
			\node (a0) at (1,-.4) {$0$};
			\node (a1) at (2,-.4) {$1$};
			\node (a2) at (1.5,.4) {$2$};
			\draw[->] (a0) -- (a1);
			\draw[->] (a0) -- (a2);
			\draw[->,thick] (a1) -- (a2);

			\node at (0,.25) {$\sigma^3_1$};
			\node at (0,0) {$\longmapsto$};

			\node (b0) at (-2.2,-.6) {$0$};
			\node (b1) at (-.8,-.6) {$1$};
			\node (b2) at (-1.5,0) {$2$};
			\node (b3) at (-1.5,.8) {$3$};
			\draw[->] (b0) -- (b1);
			\draw[->] (b0) -- (b2);
			\draw[->] (b0) -- (b3);
			\draw[->] (b1) -- (b2);
			\draw[->] (b1) -- (b3);
			\draw[->,thick] (b2) -- (b3);
		\end{tikzpicture}
		\end{center}
	\end{minipage}

	\begin{lemma}
		Given $n \in \N$ and $i \leq j \leq n$ the boundary morphisms satisfy the equality
		\begin{equation*}
			\delta_{j+1}^{n+1}\delta_i^{n+1} = \delta_i^{n+1}\delta_j^{n}.
		\end{equation*}
	\end{lemma}
	\begin{sketch}
		If $i=j$ consider
		\begin{equation*}
			\hspace{-3em}
			\begin{tikzpicture}[diagram]
				\matrix[matrix of nodes,
		ampersand replacement=\&,column sep=1em,row sep=1em]{
					\node (m-0-0) {\ensuremath{\bullet}}; \&
					\node (m-0-1) {\ensuremath{\bullet}}; \&
					\node (m-0-2) {\ensuremath{\bullet}}; \&
					\node (m-0-3) {\ensuremath{\bullet}}; \&
					\node (m-0-4) {\ensuremath{\bullet}}; \&
					\node (m-0-5) {}; \&
					\node (m-0-6) {}; \\
	%
					\node (m-1-0) {\ensuremath{\bullet}}; \&
					\node (m-1-1) {\ensuremath{\bullet}}; \&
					\node (m-1-2) {\ensuremath{\bullet}}; \&
					\node (m-1-3) {\ensuremath{\bullet}}; \&
					\node (m-1-4) {\ensuremath{\bullet}}; \&
					\node (m-1-5) {\ensuremath{\bullet}}; \&
					\node (m-1-6) {}; \\
	%
					\node (m-2-0) {\ensuremath{\bullet}}; \&
					\node (m-2-1) {\ensuremath{\bullet}}; \&
					\node (m-2-2) {\ensuremath{\bullet}}; \&
					\node (m-2-3) {\ensuremath{\bullet}}; \&
					\node (m-2-4) {\ensuremath{\bullet}}; \&
					\node (m-2-5) {\ensuremath{\bullet}}; \&
					\node (m-2-6) {\ensuremath{\bullet}}; \\
				};
				\node at ($(m-0-2)+(0,1em)$) {\ensuremath{i}};
				\node at ($.5*(m-0-0) + .5*(m-1-0) + (-2em,0)$) {\ensuremath{\delta_i^n}};
				\node at ($.5*(m-1-0) + .5*(m-2-0) + (-2em,0)$) {\ensuremath{\delta_{i+1}^{n+1}}};
				\draw[->,thick] (m-0-0) -- (m-1-0) -- (m-2-0);
				\draw[->,thick] (m-0-1) -- (m-1-1) -- (m-2-1);
				\draw[->,thick] (m-0-2) -- (m-1-3) -- (m-2-4);
				\draw[->,thick] (m-0-3) -- (m-1-4) -- (m-2-5);
				\draw[->,thick] (m-0-4) -- (m-1-5) -- (m-2-6);
				\draw[->,gray] (m-1-2) -- (m-2-2);
			\end{tikzpicture}
			\;=\;
			\begin{tikzpicture}[diagram]
				\matrix[matrix of nodes,
		ampersand replacement=\&,column sep=1em,row sep=1em]{
					\node (m-0-0) {\ensuremath{\bullet}}; \&
					\node (m-0-1) {\ensuremath{\bullet}}; \&
					\node (m-0-2) {\ensuremath{\bullet}}; \&
					\node (m-0-3) {\ensuremath{\bullet}}; \&
					\node (m-0-4) {\ensuremath{\bullet}}; \&
					\node (m-0-5) {}; \&
					\node (m-0-6) {}; \\
	%
					\node (m-1-0) {\ensuremath{\bullet}}; \&
					\node (m-1-1) {\ensuremath{\bullet}}; \&
					\node (m-1-2) {\ensuremath{\bullet}}; \&
					\node (m-1-3) {\ensuremath{\bullet}}; \&
					\node (m-1-4) {\ensuremath{\bullet}}; \&
					\node (m-1-5) {\ensuremath{\bullet}}; \&
					\node (m-1-6) {}; \\
	%
					\node (m-2-0) {\ensuremath{\bullet}}; \&
					\node (m-2-1) {\ensuremath{\bullet}}; \&
					\node (m-2-2) {\ensuremath{\bullet}}; \&
					\node (m-2-3) {\ensuremath{\bullet}}; \&
					\node (m-2-4) {\ensuremath{\bullet}}; \&
					\node (m-2-5) {\ensuremath{\bullet}}; \&
					\node (m-2-6) {\ensuremath{\bullet}}; \\
				};
				\node at ($(m-0-2)+(0,1em)$) {\ensuremath{i}};
				\node at ($.5*(m-0-0) + .5*(m-1-0) + (-2em,0)$) {\ensuremath{\delta_i^n}};
				\node at ($.5*(m-1-0) + .5*(m-2-0) + (-2em,0)$) {\ensuremath{\delta_{i}^{n+1}}};
				\draw[->,thick] (m-0-0) -- (m-1-0) -- (m-2-0);
				\draw[->,thick] (m-0-1) -- (m-1-1) -- (m-2-1);
				\draw[->,thick] (m-0-2) -- (m-1-3) -- (m-2-4);
				\draw[->,thick] (m-0-3) -- (m-1-4) -- (m-2-5);
				\draw[->,thick] (m-0-4) -- (m-1-5) -- (m-2-6);
				\draw[->,gray] (m-1-2) -- (m-2-3);
			\end{tikzpicture}
		\end{equation*}
		If $i < j$ consider
		\begin{equation*}
			\hspace{-3em}
			\begin{tikzpicture}[diagram]
				\matrix[matrix of nodes,
		ampersand replacement=\&,column sep=1em,row sep=1em]{
					\node (m-0-0) {\ensuremath{\bullet}}; \&
					\node (m-0-1) {\ensuremath{\bullet}}; \&
					\node (m-0-2) {\ensuremath{\bullet}}; \&
					\node (m-0-3) {\ensuremath{\bullet}}; \&
					\node (m-0-4) {\ensuremath{\bullet}}; \&
					\node (m-0-5) {\ensuremath{\bullet}}; \&
					\node (m-0-6) {}; \&
					\node (m-0-7) {}; \\
	%
					\node (m-1-0) {\ensuremath{\bullet}}; \&
					\node (m-1-1) {\ensuremath{\bullet}}; \&
					\node (m-1-2) {\ensuremath{\bullet}}; \&
					\node (m-1-3) {\ensuremath{\bullet}}; \&
					\node (m-1-4) {\ensuremath{\bullet}}; \&
					\node (m-1-5) {\ensuremath{\bullet}}; \&
					\node (m-1-6) {\ensuremath{\bullet}}; \&
					\node (m-1-7) {}; \\
	%
					\node (m-2-0) {\ensuremath{\bullet}}; \&
					\node (m-2-1) {\ensuremath{\bullet}}; \&
					\node (m-2-2) {\ensuremath{\bullet}}; \&
					\node (m-2-3) {\ensuremath{\bullet}}; \&
					\node (m-2-4) {\ensuremath{\bullet}}; \&
					\node (m-2-5) {\ensuremath{\bullet}}; \&
					\node (m-2-6) {\ensuremath{\bullet}}; \&
					\node (m-2-7) {\ensuremath{\bullet}}; \\
				};
				\node at ($(m-0-2)+(0,1em)$) {\ensuremath{i}};
				\node at ($(m-0-3)+(0,1em)$) {\ensuremath{j}};
				\node at ($(m-0-4)+(0,1em)$) {\ensuremath{j+1}};
				\node at ($.5*(m-0-0) + .5*(m-1-0) + (-2em,0)$) {\ensuremath{\delta_i^n}};
				\node at ($.5*(m-1-0) + .5*(m-2-0) + (-2em,0)$) {\ensuremath{\delta_{j+1}^{n+1}}};
				\draw[->,thick] (m-0-0) -- (m-1-0) -- (m-2-0);
				\draw[->,thick] (m-0-1) -- (m-1-1) -- (m-2-1);
				\draw[->,thick] (m-0-2) -- (m-1-3) -- (m-2-3);
				\draw[->,thick] (m-0-3) -- (m-1-4) -- (m-2-5);
				\draw[->,thick] (m-0-4) -- (m-1-5) -- (m-2-6);
				\draw[->,thick] (m-0-5) -- (m-1-6) -- (m-2-7);
				\draw[->,gray] (m-1-2) -- (m-2-2);
			\end{tikzpicture}
			\;=\;
			\begin{tikzpicture}[diagram]
				\matrix[matrix of nodes,
		ampersand replacement=\&,column sep=1em,row sep=1em]{
					\node (m-0-0) {\ensuremath{\bullet}}; \&
					\node (m-0-1) {\ensuremath{\bullet}}; \&
					\node (m-0-2) {\ensuremath{\bullet}}; \&
					\node (m-0-3) {\ensuremath{\bullet}}; \&
					\node (m-0-4) {\ensuremath{\bullet}}; \&
					\node (m-0-5) {\ensuremath{\bullet}}; \&
					\node (m-0-6) {}; \&
					\node (m-0-7) {}; \\
	%
					\node (m-1-0) {\ensuremath{\bullet}}; \&
					\node (m-1-1) {\ensuremath{\bullet}}; \&
					\node (m-1-2) {\ensuremath{\bullet}}; \&
					\node (m-1-3) {\ensuremath{\bullet}}; \&
					\node (m-1-4) {\ensuremath{\bullet}}; \&
					\node (m-1-5) {\ensuremath{\bullet}}; \&
					\node (m-1-6) {\ensuremath{\bullet}}; \&
					\node (m-1-7) {}; \\
	%
					\node (m-2-0) {\ensuremath{\bullet}}; \&
					\node (m-2-1) {\ensuremath{\bullet}}; \&
					\node (m-2-2) {\ensuremath{\bullet}}; \&
					\node (m-2-3) {\ensuremath{\bullet}}; \&
					\node (m-2-4) {\ensuremath{\bullet}}; \&
					\node (m-2-5) {\ensuremath{\bullet}}; \&
					\node (m-2-6) {\ensuremath{\bullet}}; \&
					\node (m-2-7) {\ensuremath{\bullet}}; \\
				};
				\node at ($(m-0-2)+(0,1em)$) {\ensuremath{i}};
				\node at ($(m-0-3)+(0,1em)$) {\ensuremath{j}};
				\node at ($(m-0-4)+(0,1em)$) {\ensuremath{j+1}};
				\node at ($.5*(m-0-0) + .5*(m-1-0) + (-2em,0)$) {\ensuremath{\delta_j^n}};
				\node at ($.5*(m-1-0) + .5*(m-2-0) + (-2em,0)$) {\ensuremath{\delta_{i}^{n+1}}};
				\draw[->,thick] (m-0-0) -- (m-1-0) -- (m-2-0);
				\draw[->,thick] (m-0-1) -- (m-1-1) -- (m-2-1);
				\draw[->,thick] (m-0-2) -- (m-1-2) -- (m-2-3);
				\draw[->,thick] (m-0-3) -- (m-1-4) -- (m-2-5);
				\draw[->,thick] (m-0-4) -- (m-1-5) -- (m-2-6);
				\draw[->,thick] (m-0-5) -- (m-1-6) -- (m-2-7);
				\draw[->,gray] (m-1-3) -- (m-2-4);
			\end{tikzpicture}
			\vspace{-2.4em}
		\end{equation*}
	\end{sketch}

	Now recall that a \textbf{simplicial object} on a category $\cat{C}$ is a just a presheaf $\Delta^{op} --> \cat{C}$ and one denotes by $\ncat{s}\cat{C} = |[\Delta^{op},\cat{C}|]$ the category of simplicial objects. In the case of $\cat{C} = \ncat{Set}$ or $\cat{C} = \ncat{Top}$ simplicial objects are usually called \textbf{simplicial sets} or \textbf{simplicial spaces} respectively.

	%Denote by $\R^\infty$ the real vector space of countable infinite dimension, with standard basis $e_0,e_1,...$.

	\begin{definition}
		The \textbf{geometric realization} of $\Delta$ is the functor
		\begin{equation*}
			\begin{array}{rcl}
				\Delta & \overset{\mathbb{\Delta}}{\longrightarrow} & \ncat{Top}\\
				{[n]} & \longmapsto & \mathbb{\Delta}_n := \convex \{e_0,...,e_n\} \subseteq \R^{n+1}\\
				{[\rho:[m]\rightarrow[n]]} & \longmapsto & {\left[\begin{array}{rcl}\mathbb{\Delta}_m &\rightarrow& \mathbb{\Delta}_n\\e_i &:\mapsto& e_{\rho(i)}\end{array}\right]}
			\end{array}
		\end{equation*}
		where $e_0,...,e_n$ denote the standard basis vectors and
		\begin{equation*}
			\smash{\convex \{e_0,...,e_n\} = \{(t_0,...,t_n) \mid 0 \leq t_i \text{ and } \Sum \limits_{i=0}^n t_i = 1\}}.
		\end{equation*}
		The action on morphisms is defined by linear (and hence continuous) extension.
	\end{definition}

	\begin{remark}
		The \textit{convex hull} $\{0,e_0,...,e_n\}$ in $\R^n$ lends itself to two different kinds of parameterizations, namely
		\begin{equation*}
			\smash{\convex \{e_0,...,e_n\} = \{(t_0,...,t_n) \mid 0 \leq t_i \text{ and } \Sum \limits_{i=0}^n t_i = 1\}}
		\end{equation*}
		and
		\begin{equation*}
			\convex \{e_0,...,e_n\} = \{(\varphi_1,...,\varphi_n) \mid 0 \leq \varphi_i \leq 1\}.
		\end{equation*}
		We can convert between both parametrizations by letting
		\begin{equation*}
			\begin{array}{rcl}
				(t_0,\,...\,,t_n) & \longmapsto & (t_0,\,t_0+t_1,\,...\,,\,\Sum \limits_{i=0}^{n-1} t_i)\\
				(\varphi_1,\, \varphi_2 - \varphi_1,\,...\,,\,\varphi_n-\varphi_{n-1},\,1-\varphi_n) & \reflectbox{$\longmapsto$} & (\varphi_1,\, ...\,,\, \varphi_n)
			\end{array}
		\end{equation*}
		\TODO{coordinates and their $d_i^n, s_i^n$}
	\end{remark}

	\begin{definition}
		\TODO{subdivision of the cylinder}
	\end{definition}

	\begin{definition}
		The \textbf{barycenter} of the standard $n$-simplex $\mathbb{\Delta}_n$ is the point $b_n = (\frac{1}{n+1},...,\frac{1}{n+1}) \in \R^{n+1}$.
	\end{definition}

	\newpage
	\subsection{Singular Homology}

	Let $M_\bullet$ be a $\Z$-graded $R$-module. In this section we explicitly construct a generalized homology theory with $M_\bullet$ as its module of coefficients. 

	We start by defining a functor $H:\ncat{Top} --> \ncat{Ab}^{\Z\text{gr}}$ as the composite
	\begin{equation*}
		\begin{diagram}
			\fivebyone[wide]
				{\ncat{Top}}{\ncat{sSet}}{\ncat{sAb}}{\ncat{Ch}}{\ncat{Ab}^{\Z\text{gr}}}

			\arrow{ww}{w}{\ncat{sing}}[above]
			\arrow{w}{c}{\Z[-]_{*}}[above]
			\arrow{c}{e}{C_\bullet}[above]
			\arrow{e}{ee}{H_{*}}[above]
		\end{diagram},
	\end{equation*}
	of the following functors.

	The first functor is the \TODO{somewhat} canonical way to turn a topological space into a simplicial set.
	\begin{equation*}
		\begin{array}{rcl}
			\ncat{Top} & \xrightarrow{\ncat{sing}} & \ncat{sSet}\\
			X & \longmapsto & \ncat{Top}(\mathbb{\Delta}-,X)\\
			{[X\overset{f}{\rightarrow} Y]} & \longmapsto & {[\ncat{Top}(\mathbb{\Delta}-,X) \overset{f_{*}}{\Rightarrow} \ncat{Top}(\mathbb{\Delta}-,Y)]}
		\end{array}
	\end{equation*}
	Note that by construction we have the identifications
	\begin{align*}
		\ncat{sing}(X)([0]) &= \ncat{Top}(\mathbb{\Delta}_0,X) \isom \ncat{Top}(*,X) = \{\text{points in }X\}\\
		\ncat{sing}(X)([1]) &= \ncat{Top}(\mathbb{\Delta}_1,X) \isom \ncat{Top}([0,1],X) = \{\text{paths in }X\}\\
		\ncat{sing}(X)([2]) &= \ncat{Top}(\mathbb{\Delta}_2,X) \isom \ncat{Top}(\S^1,X) = \{\text{loops in }X\text{ (without basepoint)}\}.
	\end{align*}

	The \textit{free abelian group functor} $\Z[-]: \ncat{Set} --> \ncat{Ab}$ extends by postcomposition to a functor
	\begin{equation*}
		\begin{array}{rcl}
			\ncat{sSet} & \xrightarrow{\Z[-]_*} & \ncat{sAb}\\
			{[\Delta^{op}\rightarrow\ncat{Set}]} & \longmapsto & {[\Delta^{op}\rightarrow\ncat{Set}\xrightarrow{\Z[-]}\ncat{Ab}]}\\
			{[\alpha:g_1\Rightarrow g_2]} & \longmapsto& {[\Z[\alpha]:\Z[g_1] \Rightarrow \Z[g_2]]}
		\end{array}
	\end{equation*}

	The next functor is a bit more involed, as it has to turn a simplicial object $\fun{A}:\Delta^{op} --> \ncat{Ab}$ into a chain complex. First recall that the boundary morphisms $\delta_i^n:[n-1]-->[n]$ in $\Delta$ give us morphisms $d_i^n:= \fun{A}(\delta_i^n):\fun{A}([n])-->\fun{A}([n-1])$. Let $C_\bullet(\fun{A})$ be the chain complex given by
	% \begin{equation*}
	% 	C_n(\fun{A}) = \left\{\begin{array}{ll} \fun{A}([n]) & n \geq 0\\ 0 & \text{else}\end{array}\right. \hspace{1cm} \partial_n = \left\{\begin{array}{ll}\Sum \limits_{i=0}^n(-1)^i d_i^n & n > 0\\ 0 & \text{else} \end{array}\right.
	% \end{equation*}
	\begin{equation*}
		\begin{diagram}
			\fivebyone[wide]
				{...\fun{A}([2])}{\fun{A}([1])}{\fun{A}([0])}{0}{0...}

			\arrow{ww}{w}{\partial_2}[above]
			\arrow{w}{c}{\partial_1}[above]
			\arrow{c}{e}{\partial_0}[above]
			\arrow{e}{ee}{0}[above]
		\end{diagram},
	\end{equation*}
	where
	\begin{equation*}
		\partial_n = \Sum \limits_{i=0}^n(-1)^i d_i^n
	\end{equation*}
	for $n \in \N$. By definition of a natural transformation $\beta:\fun{A} ==> \fun{B}$ the components $\beta_{[n]}$ make the diagram
	\begin{equation*}
		\begin{diagram}
			\twobytwo[large]
				{\fun{A}([n])}{\fun{A}([n-1])}
				{\fun{A}([n])}{\fun{A}([n-1])}

			\arrow{nw}{ne}{\fun{A}(\delta_i^n)}[above]
			\arrow{sw}{se}{\fun{B}(\delta_i^n)}[below]

			\arrow{nw}{sw}{\beta_{[n]}}[left]
			\arrow{ne}{se}{\beta_{[n-1]}}[right]
		\end{diagram}
		\hspace{1cm}\text{and thus}\hspace{1cm}
		\begin{diagram}
			\twobytwo[large]
				{\fun{A}([n])}{\fun{A}([n-1])}
				{\fun{A}([n])}{\fun{A}([n-1])}

			\arrow{nw}{ne}{\partial_n}[above]
			\arrow{sw}{se}{\partial_n}[below]

			\arrow{nw}{sw}{\beta_{[n]}}[left]
			\arrow{ne}{se}{\beta_{[n-1]}}[right]
		\end{diagram}
	\end{equation*}
	commute, so $C_\bullet(\beta)$ is just the extension of $\beta$ by zeroes in negative degrees.

	To show that this indeed defines a chain complex we use that by Lemma \TODO{cite} the identity
	\begin{equation*}
		d_j^nd_i^{n+1} = d_i^nd_{j+1}^{n+1}
	\end{equation*}
	holds for $i \leq j \leq n$. With this we compute
	\begin{align*}
		\partial_n\partial_{n+1} &= \Sum \limits_{i=0}^n(-1)^i d_i^n \left(\Sum \limits_{j=0}^{n+1}(-1)^j d_j^{n+1}\right)
		= \Sum \limits_{i=0}^n \Sum \limits_{j=0}^{n+1}(-1)^{i+j}d_i^nd_j^{n+1}\\
		&=\sum\limits_{i=0}^n\Sum\limits_{j=0}^{i}(-1)^{j+i}d_i^nd_j^{n+1} + \Sum\limits_{i=0}^n\Sum\limits_{j=i+1}^{n+1}(-1)^{i+j}d_i^nd_j^{n+1}\\
		&=\sum\limits_{i=0}^n\Sum\limits_{j=0}^{i}(-1)^{j+i}d_i^nd_j^{n+1} + \Sum\limits_{i=0}^n\Sum\limits_{j=i}^{n}(-1)^{i+j+1}d_i^nd_{j+1}^{n+1}\\
		&=\sum\limits_{i=0}^n\Sum\limits_{j=0}^{i}(-1)^{j+i}d_i^nd_j^{n+1} - \Sum\limits_{i=0}^n\Sum\limits_{j=i}^{n}(-1)^{i+j}d_j^nd_i^{n+1}\\
		&=\sum\limits_{i=0}^n\Sum\limits_{j=0}^{i}(-1)^{j+i}d_i^nd_j^{n+1} - \sum\limits_{j=0}^n\Sum\limits_{i=j}^{n}(-1)^{j+i}d_i^nd_j^{n+1} = 0
	\end{align*}



	\subsection{Homological Orientation and Degree}
\end{document}