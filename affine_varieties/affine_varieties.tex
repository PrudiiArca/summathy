\documentclass{article}

% --- text layout
\usepackage[parfill]{parskip}
\usepackage{geometry}
\geometry{a4paper, margin=3cm}
\usepackage{romanbar}
\usepackage{placeins} %FloatBarrier
\usepackage[utf8]{inputenc}

\usepackage{xcolor}
\definecolor{darkred}{HTML}{990000}
\definecolor{darkgreen}{HTML}{009900}

\usepackage{sectsty}
\sectionfont{\color{darkred}}
\subsectionfont{\color{darkred}}
\subsubsectionfont{\color{darkred}}
% \usepackage{algpseudocode} --> better pseudo code?
%\usepackage[pagewise]{lineno} % row numbers

% --- syncing, referencing, citing
\usepackage{pdfsync} 			% synchronization LaTeX <--> PDF
\usepackage[hidelinks]{hyperref} % hidelinks avoids red boxes around links
\usepackage{csquotes}			% enquote
\usepackage{cleveref}
\usepackage{pdfpages}

\usepackage[backend=bibtex,style=alphabetic,url=false]{biblatex}
\addbibresource{sources.bib}
% define bibliography filter like
%\defbibfilter{computability-pca}{
%	keyword=computability or keyword=pca
%}

% --- lists and tables
\usepackage[shortlabels]{enumitem}
\usepackage{tabularx}
\usepackage{tabu}
\usepackage{makecell}

% --- graphics
\usepackage{graphicx}
\usepackage{xcolor}

% --- math
\usepackage{amsmath}
\usepackage{amssymb}
\usepackage{fixmath} % whyyy? "bug": italic PI and SIGMA
\usepackage{mathtools}
\usepackage{mathrsfs} %mathscr

% --- own packages
\usepackage{../notation}
\usepackage{../diagrams}
\usepackage{../environments}

\title{
	$\Sum \text{math}(y)$\\
	Affine Varieties
}
\author{Jonas Linßen}

\begin{document}
	\maketitle
	\tableofcontents

	\newpage
	\section{Affine Varieties}
	\subsection{Affine Varieties and Hilbert's Nullstellensatz}

	\begin{convention}
		Let $K$ denote an algebraically closed field and $n \in \N$. We call $K^n$ the \textit{affine space} of dimension $n$.
	\end{convention}

	\begin{definition}[Affine Algebraic Set]
		An \textbf{affine algebraic set} $A \subseteq K^n$ is a set of the form $A = Z(F)$, where
		\begin{equation*}
			Z(F) = \{x \in K^n \mid \forall f \in F: f(x) = 0\}
		\end{equation*}
		is the \textbf{zero locus} of some subset $F \subseteq K[X_1, \dots, X_n]$.
	\end{definition}

	\begin{remark}
		%\begin{enumerate}[(i)]
		%	\item{
				Since $K$ is algebraically closed, every point $(x_1, \dots, x_n) \in K^n$ can be represented as an affine algebraic set of the form $Z(\{(X_i - x_i) \mid i = 1, .., n\})$.
		%	}
		%\end{enumerate}
	\end{remark}

	\begin{definition}[Zariski Topology on $K^n$]
		Defining $\{Z(F) \mid F \subseteq K[X_1, \dots, X_n]\}$ to be the closed sets we obtain the so called \textbf{Zariski-topology} on $K^n$.
	\end{definition}
	\begin{sketch}
		Check that the following equations hold:

		$\emptyset = Z(1)$, $K^n = Z({0})$, $\Intersection \limits_{i \in I} Z(F_i) = Z(\Union \limits_{i \in I} F_i)$, $Z(F) \union Z(G) = Z(\{f \cdot g \mid f \in F, g \in G\})$.
	\end{sketch}

	\begin{definition}[Affine Variety]
		An affine algebraic set $V \subseteq K^n$ is called \textbf{affine variety}, if it is \textit{irreducible} with respect to the Zariski topology, i.e. if it cannot be written as a union of closed proper subsets.
	\end{definition}

	\begin{definition}
		Let $S \subseteq K^n$ be a subset. The \textbf{vanishing ideal} of $S$ is the ideal 
		\begin{equation*}
			I(S) = \{f \in K[X_1, \dots, X_n] \mid \forall s \in S: f(s) = 0\} \trianglelefteq K[X_1, \dots, X_n].
		\end{equation*}
		The \textbf{coordinate ring} of $S$ is defined as $K[S] = K[X_1, \dots, X_n] / I(S)$.
	\end{definition}

	\begin{theorem}[Hilbert's Nullstellensatz]
		If $A \subseteq K^n$ is an affine algebraic set, we have
		\begin{equation*}
			Z(I(A)) = A.
		\end{equation*}

		Conversely, if $\mathfrak{a} \trianglelefteq K[X_1, \dots, X_n]$ is an ideal, then
		\begin{equation*}
			I(Z(\mathfrak{a})) = \sqrt{\mathfrak{a}},
		\end{equation*}
		where $\sqrt{\mathfrak{a}}$ denotes the \textit{radical} of $\mathfrak{a}$.
	\end{theorem}

	\begin{corollary}[Ideal Correspondence]
		$I$ and $Z$ exhibit order reversing bijections
		\begin{equation*}
			\begin{diagram}
				\threebythree[wide]
					{\{\text{radical ideals}\}}{}{\{\text{aff. alg. sets}\}}
					{\{\text{prime ideals}\}}{}{\{\text{aff. varieties}\}}
					{\{\text{maximal ideals}\}}{}{\{\text{points}\}}
			\end{diagram}
		\end{equation*}
	\end{corollary}

\end{document}