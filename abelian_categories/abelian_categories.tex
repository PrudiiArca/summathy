\documentclass{article}

% --- text layout
\usepackage[parfill]{parskip}
\usepackage{geometry}
\geometry{a4paper, margin=3cm}
\usepackage{romanbar}
\usepackage{placeins} %FloatBarrier
\usepackage[utf8]{inputenc}

\usepackage{xcolor}
\definecolor{darkred}{HTML}{990000}
\definecolor{darkgreen}{HTML}{009900}

\usepackage{sectsty}
\sectionfont{\color{darkred}}
\subsectionfont{\color{darkred}}
\subsubsectionfont{\color{darkred}}
% \usepackage{algpseudocode} --> better pseudo code?
%\usepackage[pagewise]{lineno} % row numbers

% --- syncing, referencing, citing
\usepackage{pdfsync} 			% synchronization LaTeX <--> PDF
\usepackage[hidelinks]{hyperref} % hidelinks avoids red boxes around links
\usepackage{csquotes}			% enquote
\usepackage{cleveref}
\usepackage{pdfpages}

\usepackage[backend=bibtex,style=alphabetic,url=false]{biblatex}
\addbibresource{sources.bib}
% define bibliography filter like
%\defbibfilter{computability-pca}{
%	keyword=computability or keyword=pca
%}

% --- lists and tables
\usepackage[shortlabels]{enumitem}
\usepackage{tabularx}
\usepackage{tabu}
\usepackage{makecell}

% --- graphics
\usepackage{graphicx}
\usepackage{xcolor}

% --- math
\usepackage{amsmath}
\usepackage{amssymb}
\usepackage{fixmath} % whyyy? "bug": italic PI and SIGMA
\usepackage{mathtools}
\usepackage{mathrsfs} %mathscr

% --- own packages
\usepackage{../notation}
\usepackage{../diagrams}
\usepackage{../environments}

\title{
	$\Sum \text{math}(y)$\\
	Abelian Categories
}
\author{Jonas Linßen}

\begin{document}
	\maketitle
	\tableofcontents

	\newpage
	\section{Abelian Categories}
	\subsection{Pointed Categories}

	\begin{definition}[Pointed Category]
		A category $\cat{C}$ is \textbf{pointed}, if it has a \textit{zero object}, i.e. an object $0$, which is both initial and terminal. Recall that for given objects $X,Y$ in $\cat{C}$ the unique morphism $0_{XY}:X-->0-->Y$ is called the \textit{zero morphism} from $X$ to $Y$.
	\end{definition}

	\begin{definition}[Kernel, Cokernel]
		Let $\cat{C}$ be a pointed category and $f:X-->Y$ be a morphism in $\cat{C}$.

		The \textit{equalizer} $\ker f := \eq (f,0_{XY})$ is called the \textbf{kernel} of $f$, if it exists.\\
		Dually, the \textit{coequalizer} $\coker f := \coeq(f,0_{XY})$ is called \textbf{cokernel} of $f$, if it exists.
	\end{definition}

	\begin{lemma}[Properties of Kernels and Cokernels]
		Let $\cat{C}$ be a pointed category. Then
		\begin{enumerate}[(i)]
			\item{
				The kernel of $0:X-->Y$ is given by $1_X$ (up to isomorphism).\\
				Dually, the cokernel of $0:X-->Y$ is given by $1_Y$ (up to isomorphism)
			}
			\item{
				For ever monomorphism $f:X\longrightmono Y$ is holds that $\ker f = 0_{0X}$.\\
				Dually, for every epimorphism $f:X \longrightepi Y$ it holds that $\coker f = 0_{Y0}$
			}
			\item{
				For any morphisms $f:X-->Y$ and monomorphism $g:Y \longrightmono Z$ we have $\ker gf = \ker f$.\\
				Dually, for any epmorphisms $f:X \longrightepi Y$ and morphism $g:Y \longrightmono Z$ we have the identity $\coker gf = \coker g$.
			}
		\end{enumerate}
	\end{lemma}
	\begin{sketch}
		\begin{minipage}{\textwidth}
		\begin{center}
		\begin{tabular}{l|l}
			\makecell{
				(i) Consider\\
				\begin{diagram}
					\threebytwo
						{}{T}{}
						{X}{X}{Y}
					\arrow[dashed]{n}{sw}{}
					\arrow{n}{s}{t}[right]
					\arrow{sw}{s}{1_X}[below]
					\arrow[higher]{s}{se}{0_{XY}}[above]
					\arrow[lower]{s}{se}{0_{XY}}[below]
				\end{diagram}
			}
			&
			\makecell{
				(ii) Consider\\
				\begin{diagram}
					\threebytwo
						{}{T}{}
						{0}{X}{Y}
					\arrow[dashed]{n}{sw}{}
					\arrow{n}{s}{t}[right]
					\arrow{sw}{s}{0_{0X}}[below]
					\arrow[higher,mono]{s}{se}{f}[above]
					\arrow[lower]{s}{se}{0_{XY}}[below]
				\end{diagram}\vspace{1em}\\
				and calculate $ft = 0t = 0 = f0$.
			}\\
			\hline
			\makecell{
				(iii) Consider\\
				\begin{diagram}
					\threebytwo
						{}{T}{}
						{K}{X}{Z}
					\arrow[dashed]{n}{sw}{}
					\arrow{n}{s}{t}[right]
					\arrow{sw}{s}{\ker gf}[below]
					\arrow[higher]{s}{se}{gf}[above]
					\arrow[lower]{s}{se}{0_{XZ}}[below]
				\end{diagram}\vspace{1em}\\
				and observe that\\
				$gft = 0t = g0t$ implies $ft = 0t$.
			}
		\end{tabular}
		\end{center}
		\end{minipage}
	\end{sketch}

	Note that a kernel is always mono and a cokernel is always epi. This leads to the following definition.

	\begin{definition}[Normal Mono, Normal Epi]
		Let $\cat{C}$ be a pointed category.

		A monomorphism is \textbf{normal}, if it is a kernel.\\
		Dually, an epimorphism is \textbf{normal}, if it is a cokernel.
	\end{definition}

	\begin{lemma}[Characterization of Normal Morphisms]
		Let $\cat{C}$ be a pointed category, $f:X-->Y$ be a morphism in $\cat{C}$ and assume that the following kernels and cokernels exist.

		Then $f$ is a normal monomorphism, if and only if $f = \ker \coker f$.\\
		Dually, $f$ is a normal epimorphism, if and only if $f = \coker \ker f$.
	\end{lemma}
	\begin{proof}
		\enquote{$<==$}:\\
		Holds by definition.

		\enquote{$==>$}:\\
		Let $t:T-->Y$ with \textcolor{blue}{$ct = 0t = 0$}. Since \textcolor{orange}{$gf=0f$} $=0=$ \textcolor{darkgreen}{$g0=gf$} there is a unique $z:C-->Z$ such that $zc = g$. Now \textcolor{blue}{$zct = z0t$} $=$ \textcolor{orange}{$0t=gt$} $=zct$, hence there is a unique $x:T-->X$ satisfying $fx=t$.
		\begin{equation*}
			\begin{matrix}
			\begin{diagram}
				\threebythree
					{}{T}{}
					{X}{Y}{C}[gray]
					{}{Z}{}

				\arrow[dashed,orange]{n}{w}{\exists!x}[above left]
				\arrow[orange]{n}{c}{t}[right]

				\arrow[higher,orange]{w}{c}{f}[above]
				\arrow[lower,gray]{w}{c}{0}[below]
				\arrow[higher,gray]{c}{e}{c}[above]
				\arrow[lower,gray]{c}{e}{0}[below]

				\arrow[lefter,orange]{c}{s}{0}[left]
				\arrow[righter,orange]{c}{s}{g}[right]
				\arrow[dashed,gray]{e}{s}{z}[below right]
			\end{diagram}&
			\begin{diagram}
				\threebythree
					{}{T}[gray]{}
					{X}{Y}{C}
					{}{Z}{}

				\arrow[dashed,gray]{n}{w}{x}[above left]
				\arrow[gray]{n}{c}{t}[right]

				\arrow[higher,darkgreen]{w}{c}{f}[above]
				\arrow[lower,darkgreen]{w}{c}{0}[below]
				\arrow[higher,darkgreen]{c}{e}{c}[above]
				\arrow[lower,gray]{c}{e}{0}[below]

				\arrow[lefter,gray]{c}{s}{0}[left]
				\arrow[righter,darkgreen]{c}{s}{g}[right]
				\arrow[dashed,darkgreen]{e}{s}{\exists!z}[below right]
			\end{diagram}&
			\begin{diagram}
				\threebythree
					{}{T}{}
					{X}{Y}{C}
					{}{Z}[gray]{}

				\arrow[dashed,blue]{n}{w}{\exists!x}[above left]
				\arrow[blue]{n}{c}{t}[right]

				\arrow[higher,blue]{w}{c}{f}[above]
				\arrow[lower,gray]{w}{c}{0}[below]
				\arrow[higher,blue]{c}{e}{c}[above]
				\arrow[lower,blue]{c}{e}{0}[below]

				\arrow[lefter,gray]{c}{s}{0}[left]
				\arrow[righter,gray]{c}{s}{g}[right]
				\arrow[dashed,gray]{e}{s}{z}[below right]
			\end{diagram}\\
			{\color{orange}f\text{ normal mono}}
			&{\color{darkgreen}c=\coker f}
			&{\color{blue}f = \ker c = \ker \coker f}
			\end{matrix}\vspace{-2em}
		\end{equation*}
	\end{proof}

	\subsection{Ab-enriched, Additive and Abelian Categories}

	\begin{definition}[$\ncat{Ab}$--enriched Category]
		A locally small category $\cat{A}$ is \textbf{$\ncat{Ab}$-enriched}, if every homset $\cat{A}(A,B)$ is an abelian group and the composition $\circ_{ABC}: \cat{A}(B,C) \times \cat{A}(A,B) --> \cat{A}(A,C)$ is a group homomorphism in both variables.
	\end{definition}

	\begin{definition}[Biproduct]
		Let $\cat{A}$ be an $\ncat{Ab}$-enriched category and $A,B$ be two objects. The \textbf{biproduct} $A \oplus B$ of $A$ and $B$ is an object $C$, which satisfies one of the following equivalent conditions.

		\begin{enumerate}[(i)]
			\item{
				There are morphisms
				\begin{equation*}
					\begin{diagram}
						\threebyone
							{A}{C}{B}
						\arrow[higher]{c}{w}{\pi_A}[above]
						\arrow[lower]{w}{c}{\iota_A}[below]
						\arrow[higher]{c}{e}{\pi_B}[above]
						\arrow[lower]{e}{c}{\iota_A}[below]
					\end{diagram}
				\end{equation*}
				% $$
				% 	\begin{matrix}
				% 		\pi_A:C-->A,&\iota_A:A-->C,\\
				% 		\pi_B:C-->B,&\iota_B:B-->C,
				% 	\end{matrix}
				% $$
				which satisfy the equations
				\begin{align*}
					\pi_A\iota_A, = 1_A, \hspace{1cm}\pi_B\iota_A = 0,\\
					\pi_B\iota_B, = 1_B, \hspace{1cm}\pi_A\iota_B = 0,\\
					\iota_A\pi_A + \iota_B\pi_B = 1_C\hspace{.7cm}
				\end{align*}
			}
			\item{
				$C$ is the product $A \times B$.
			}
			\item{
				$C$ is the coproduct $A + B$.
			}
		\end{enumerate}
	\end{definition}
	We have to show that (i) - (iii) are indeed equivalent
	\begin{sketch}
		\enquote{(i) $==>$ (ii)}:\\
		For $t_A:T-->A,t_B:T-->B$ note that $t := \iota_At_A + \iota_Bt_B$ satisfies the universal property of the product.

		\enquote{(ii) $==>$ (i)}:\\
		Use the first four equations to define $\iota_A$ and $\iota_B$ via the universal property. Observe that $\pi_A(\iota_A\pi_A + \iota_B\pi_B) = \pi_A$ and $\pi_B(\iota_A\pi_A + \iota_B\pi_B) = \pi_B$ and use the universal property to derive the fifth equation.

		\enquote{(i) $<==>$ (iii)}: By duality.
	\end{sketch}

	\begin{definition}[Additive category, Abelian category]
		A locally small category $\cat{A}$ is \textbf{additive}, if
		\begin{enumerate}[$\circ$]
			\item{it is pointed, i.e. has a zero object,}
			\item{is $\ncat{Ab}$-enriched}
			\item{and has all binary biproducts.}
		\end{enumerate}

		An \textbf{abelian} category is an additive category, which
		\begin{enumerate}[$\circ$]
			\item{has all kernels and cokernels}
			\item{and for which ever mono- or epimorphism is normal.}
		\end{enumerate}
	\end{definition}

	\begin{proposition}[Characterization of Equalizers and Coequalizers]
		Let $\cat{A}$ be an additive category and $f_1,f_2$ be parallel morphisms.

		Then, if the equalizer $\eq(f_1,f_2)$, the kernel $\ker(f_1-f_2)$ or the kernel $\ker(f_2-f_1)$ exists, so do the others and they are isomorphic.\\
		Dually, if the coequalizer $\coeq(f_1,f_2)$, the cokernel $\coker(f_1-f_2)$ or the cokernel $\coker(f_2-f_1)$ exists, so do the others and they are isomorphic.
	\end{proposition}
	\begin{sketch}
		TODO
	\end{sketch}

	\begin{corollary}[(Co-)Completeness of Abelian Categories]
		Any abelian category is complete and cocomplete.
	\end{corollary}
	\begin{sketch}
		TODO
	\end{sketch}

	\begin{theorem}[Uniqueness of Additive Structure]
		The $\ncat{Ab}$-enrichment of an abelian group is unique (up to group isomorphism).
	\end{theorem}
	\begin{sketch}
		TODO
	\end{sketch}

	\begin{corollary}[Characterization of Abelian Categories]
		A locally small category $\cat{A}$ is an abelian category if and only if it is a complete and cocomplete category with finite biproducts, where every monomorphism and every epimorphism is normal.
	\end{corollary}

	\begin{proposition}[Characterization of Monos, Epis and Isos]
		Let $\cat{A}$ be an abelian category and $f:A-->B$ be a morphism.
		
		~\\
		Then the following are equivalent.
		\begin{enumerate}[(i)]
			\item{
				The morphism $f$ is a monomorphism.
			}
			\item{
				The kernel of $f$ satisfies $\ker f = 0$.
			}
			\item{
				For any morphism $t:T-->A$ it holds that $ft = 0$ implies $t = 0$.
			}
		\end{enumerate}
		Dually, the following are equivalent.
		\begin{enumerate}[(i)]
			\item{
				The morphism $f$ is an epimorphism.
			}
			\item{
				The cokernel of $f$ satisfies $\coker f = 0$.
			}
			\item{
				For any morphism $t:B-->T$ it holds that $tf = 0$ implies $t = 0$.
			}
		\end{enumerate}

		~\\
		Moreover it holds that $f$ is an isomorphism, if and only if it is both mono and epi.
	\end{proposition}
	\begin{sketch}
		TODO
	\end{sketch}

	\begin{theorem}[Epi-Mono Factorization]
		Let $\cat{A}$ be an abelian category. Every morphism $f:A-->B$ admits a factorization $f=ip$, where $i = \ker \coker f$ and $p = \coker \ker f$, which is unique (up to isomorphism).
	\end{theorem}
	\begin{sketch}
		TODO
	\end{sketch}

	\begin{definition}[Image]
		Let $\cat{C}$ be a pointed category and $f:X-->Y$ be a morphism.

		The \textbf{image} of $f$ is defined to be $\im f := \ker \coker f$, if the corresponding kernel and cokernel exist.
	\end{definition}

	\begin{theorem}[Isomorphism Theorem]
		TODO
	\end{theorem}

	\begin{definition}[Quotient]
		TODO
	\end{definition}

	\TODO{pullbacks and pushouts?}

	\subsection{Additive Functors}

	\begin{definition}[Additive Functor]
		TODO
	\end{definition}

	\begin{lemma}[Representables Are Additive]
		TODO
	\end{lemma}

	\begin{lemma}[$\Add(A,B)$ is $\ncat{Ab}$-enriched]
		TODO
	\end{lemma}

	\begin{theorem}[Additive Yoneda Lemma]
		TODO
	\end{theorem}

	\begin{proposition}[Characterization of Additive Functors]
		TODO
	\end{proposition}

	\subsection{Exact Sequences and Exact Functors}

	\begin{definition}[Exact Sequence]
		TODO
	\end{definition}

	\begin{lemma}[Properties of Exact Sequences]
		TODO
	\end{lemma}

	\begin{definition}[Split Exact Sequence]
		TODO
	\end{definition}

	\begin{lemma}[Characterization of Split Exact Sequences]
		TODO
	\end{lemma}

	\begin{definition}[Exact Functor]
		TODO
	\end{definition}

	\begin{lemma}[Characterization of Exact Functors]
		TODO
	\end{lemma}

	\newpage
	\section{The Freyd Mitchel Embedding Theorem}

	\newpage
	\section{(Co-)Homology and Derived Functors}
	\subsection{(Co-)Chain Complexes and (Co-)Homology}
	\subsection{Derived Functors}
\end{document}