\documentclass{article}

% --- text layout
\usepackage[parfill]{parskip}
\usepackage{geometry}
\geometry{a4paper, margin=3cm}
\usepackage{romanbar}
\usepackage{placeins} %FloatBarrier
\usepackage[utf8]{inputenc}

\usepackage{xcolor}
\definecolor{darkred}{HTML}{990000}
\definecolor{darkgreen}{HTML}{009900}

\usepackage{sectsty}
\sectionfont{\color{darkred}}
\subsectionfont{\color{darkred}}
\subsubsectionfont{\color{darkred}}
% \usepackage{algpseudocode} --> better pseudo code?
%\usepackage[pagewise]{lineno} % row numbers

% --- syncing, referencing, citing
\usepackage{pdfsync} 			% synchronization LaTeX <--> PDF
\usepackage[hidelinks]{hyperref} % hidelinks avoids red boxes around links
\usepackage{csquotes}			% enquote
\usepackage{cleveref}
\usepackage{pdfpages}

\usepackage[backend=bibtex,style=alphabetic,url=false]{biblatex}
\addbibresource{sources.bib}
% define bibliography filter like
%\defbibfilter{computability-pca}{
%	keyword=computability or keyword=pca
%}

% --- lists and tables
\usepackage[shortlabels]{enumitem}
\usepackage{tabularx}
\usepackage{tabu}
\usepackage{makecell}

% --- graphics
\usepackage{graphicx}
\usepackage{xcolor}

% --- math
\usepackage{amsmath}
\usepackage{amssymb}
\usepackage{fixmath} % whyyy? "bug": italic PI and SIGMA
\usepackage{mathtools}
\usepackage{mathrsfs} %mathscr

% --- own packages
\usepackage{../notation}
\usepackage{../diagrams}
\usepackage{../environments}

\title{
	$\Sum \text{math}(y)$\\
	Manifolds and Bundles
}
\author{Jonas Linssen}

\begin{document}
	\maketitle
	\tableofcontents

	\newpage
	\section{Topological Manifolds}
	\subsection{Definition and Basic Properties}

	\begin{definition}
		A \textbf{locally euclidean space} of dimension $n$ is a topological space $X$, such that every point $x$ has an \TODO{open} \textbf{euclidean neighborhood} $U_x$, which is homeomorphic to $\R^n$, or equivalently to $\B^n$, via a (\textbf{coordinate}) \textbf{chart} $\phi_x:U_x \overset{\isom}{-->} \R^n$. Note that euclidean neighborhoods are not unique and in fact form a basis of the topology of $X$. \TODO{make $n$ local}

		A cover of $X$ by euclidean neighborhoods in combination with their charts is called an \textbf{atlas} for $X$.
	\end{definition}

	\begin{remark}
		\TODO{invariance of domain implies euclidean neighborhood is open}
	\end{remark}

	\begin{lemma}[Properties of Locally Euclidean Spaces]
		\vspace{-1.5em}
		\begin{enumerate}[(i)]
			\item{
				The property of being locally euclidean is preserved under local homeomorphisms, i.e. if $M$ is etalé over $X$, then $X$ is locally euclidean. In particular, if $X$ is homeomorphic to $M$, then it is locally euclidean.
			}
			\item{
				Locally euclidean spaces / $n$-manifolds are locally hausdorff (in particular Frechét), locally compact, locally connected, locally path connected, locally contractible, locally metrizable and first countable.
			}
			\item{
				In an locally euclidean space $X$ all connected components are open, in particular $\pi_0(X)$ is discrete. Moreover $X$ is path-connected, if and only if it is connected.
			}
		\end{enumerate}
	\end{lemma}

	\begin{definition}
		A \textbf{topological $n$-manifold} is a a second countable, locally euclidean Hausdorff space.

		The category of topological manifolds (of arbitrary dimension) is the corresponding full subcategory of $\ncat{Top}$ and denoted by $\ncat{TopMf}$.
	\end{definition}

	\begin{lemma}[Properties of $\ncat{TopMf}$]
		\vspace{-1.5em}
		\begin{enumerate}[(i)]
			\item{
				The cartesian product of an $m$-manifold and an $n$-manifold is again an $(m+n)$-manifold
			}
			\item{
				Every open subset of a $n$-manifold is a submanifold.
			}
			\item{
				The countable disjoint union of $n$-manifolds is again an $n$-manifold. \TODO{make n local}
			}
			\item{
				\TODO{connected sum}
			}
		\end{enumerate}
	\end{lemma}

	\begin{remark}
		\TODO{paracompact iff metrizable}

		\TODO{second countable iff $\sigma$-compact}

		\TODO{second countable iff paracompact and separable}

		\TODO{paracompact + countable components implies second countable}

		\TODO{compact locally euclidean hausdorff implies paracompact and second countable}
	\end{remark}

	\begin{proposition}[Bootstrap Lemma]
		\TODO{Bredon Lem. 7.9}
		% Let $M$ be a topological manifold write $P_M(A)$ for \enquote{The statement $P_M$ is true for the closed set $A$}.

		% Consider the following properties.
		% \begin{enumerate}[(1)]
		% 	\item{
		% 		If $A$ is a compact and convex subset of an euclidean subset $U \subseteq M$, then $P_M(A)$ holds.
		% 	}
		% 	\item{
		% 		If $A,B$ are closed subsets such that $P_M(A)$, $P_M(B)$ and $P_M(A \intersection B)$ hold, then $P_M(A \union B)$ is true.
		% 	}
		% 	\item{
		% 		If $A_1 \supseteq A_2 \subseteq A_2$
		% 	}
		% \end{enumerate}
	\end{proposition}

	\newpage
	\subsection{Embedding Theorems}

	\begin{lemma}
		Every compact $n$-manifold $M$ embedds into $\R^N$ for some $N > n$ and the inequality is necessarily strict.
	\end{lemma}
	\begin{sketch}
		Let $U_1,...,U_k$ be a finite cover of $M$ by euclidean opens and consider the maps
		\begin{equation*}
			f_i: M \longrightarrow M / (M \setminus U_i)\isom \S^n \longrightarrow \R^{n+1}.
		\end{equation*}
		Then $(f_1,...,f_k)$ defines an embedding of $M$ into $\R^{k(n+1)}$.
	\end{sketch}

	\begin{theorem}
		Every $n$-manifold embedds into $\R^{2n+1}$.

		\TODO{https://mathoverflow.net/questions/34658/is-there-a-whitney-embedding-theorem-for-non-smooth-manifolds}
	\end{theorem}

	\subsection{Classification of Manifolds}

	\begin{lemma}
		The $0$-manifolds are precisely the countable discrete spaces.
	\end{lemma}

	\begin{proposition}
		$\R$ and $\S^1$ are the only $1$-manifolds.
	\end{proposition}

	\begin{remark}
		\TODO{classification of 2-manifolds (next section) and 3-manifolds (Perelman) possible. For 4 unknown? For $\geq 5$ impossible! Problem: test of being simply connected undecidable. Classification of $\geq 5$-manifolds exists.}
	\end{remark}

	\subsection{Classification of Surfaces}

	\newpage
	\section{Bundles}
	\subsection{Fiber Bundles}
	\begin{definition}
		\begin{minipage}[t]{\linewidth-4cm}
			Let $F$ be a topological space. A \textbf{fiber bundle} with \textbf{fiber} $F$ consists of a \textbf{base space} $B$, a \textbf{total space} $E$ and a surjective \textbf{bundle projection} or \textbf{submersion} $p:E-->>X$ subject to the following local triviality condition.

			\vspace{1em}
			Each point $x$ of $B$ has an open neighborhood $U_x \subseteq B$ and a homeomorphism $t:F \times U_x \isom p^{-1}(U_x)$, such that the diagram on the right commutes. The pair $(U_x,t)$ is called a \textbf{local trivialization} of $p$ at $x$.
		\end{minipage}
		\begin{minipage}[t]{4cm}
			\begin{equation*}
				\begin{diagram}
					\twobytwo[wide]
						{F \times U_x}{p^{-1}(U_x)}
						{}{B}

					\arrow{nw}{ne}{\begin{array}{c}\isom\\t\end{array}}
					\arrow{nw}{se}{\pi_{U_x}}[below left]
					\arrow{ne}{se}{p}[right]
				\end{diagram}
			\end{equation*}
		\end{minipage}

		Note that $F \isom \pi_{U_x}^{-1}(\{x\}) \isom p^{-1}(\{x\})$ is indeed the fiber over each point $x$ in the base space $B$.

		The space $F \times B$ is called the \textbf{trivial fiber bundle} over $B$ with fiber $F$.
	\end{definition}

	\begin{lemma}[Properties of Fiber Bundles]
		Let $p:E-->>B$ be a fiber bundle with fiber $F$.
		\begin{enumerate}[(i)]
			\item{
				The submersion $p$ is open and $B$ carries the quotient topology of $E$ by the equivalence relation \TODO{?}
			}
			\item{
				Any fiber bundle over a contractible CW-complex \TODO{space?} is a trivial bundle
			}
			\item{
				\TODO{fiber bundle over manifold is manifold}
			}
		\end{enumerate}
	\end{lemma}

	\begin{definition}
		\begin{minipage}{\linewidth-4cm}
			A \textbf{bundle morphism} of fiber bundles $p:E-->>B$ and $p':E'-->>B'$ with fibers $F$ and $F'$ respectively is given by a pair of continous functions $\tilde f:B --> B'$ and $f:E-->E'$ making the diagram on the right commute. Note that by $p$ being epi, the morphism $\tilde f$ is uniquely determined by $f$.

			\vspace{1em}
			The category of fiber bundles, denoted $\ncat{Bdl}$, hence may be identified as the full subcategory of the arrow category $\ncat{Top}^2$ on the submersions. The full subcategory of all fiber bundles with fiber $F$ will be denoted $\ncat{Bdl}_F$.
		\end{minipage}
		\begin{minipage}{4cm}
			\begin{equation*}
				\begin{diagram}
					\twobytwo
						{E}{E'}
						{B}{B'}

					\arrow[epi]{nw}{sw}{p}[left]
					\arrow[epi]{ne}{se}{p'}[right]
					\arrow{nw}{ne}{f}[above]
					\arrow{sw}{se}{\tilde f}[below]
				\end{diagram}
			\end{equation*}
		\end{minipage}
	\end{definition}

	\begin{lemma}[Properties of  $\ncat{Bdl}$ and {$\ncat{Bdl}_F$}]
		\TODO{??}
	\end{lemma}

	\TODO{pullback bundle, bundle map over fixed $B$}

	\begin{definition}
		Let $p:E-->>B$ a fiber bundle. A \textbf{local section} on $U \subseteq B$ is a morphism $s:U --> E$ with $ps = 1_U$. If $U=B$ the section $s$ is a \textbf{global section}.

		Considering the open subsets of $B$ as a poset category, one obtains the \textbf{sheaf of local sections}
		\begin{equation*}
			\begin{array}{rcl}
				\Sigma: B^{op} & \longrightarrow & \ncat{Set}\\
				U & \longmapsto & \Sigma U := \{\text{local sections of } p\text{ on }U\}\\
				{[U \supseteq V]} & \longmapsto & {\left[\begin{array}{rcl}\Sigma U \mapsto \Sigma V\\s \mapsto s\vert_V\end{array}\right]}
			\end{array}
		\end{equation*}
	\end{definition}

	\begin{definition}
		\TODO{$G$-bundle}
	\end{definition}

	\newpage
	\subsection{Vector Bundles}
	\subsection{Tangent (Micro)-bundles}
	\subsection{Covering Spaces}
\end{document}